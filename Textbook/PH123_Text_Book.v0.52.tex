
\documentclass{sebase}
%%%%%%%%%%%%%%%%%%%%%%%%%%%%%%%%%%%%%%%%%%%%%%%%%%%%%%%%%%%%%%%%%%%%%%%%%%%%%%%%%%%%%%%%%%%%%%%%%%%%%%%%%%%%%%%%%%%%%%%%%%%%%%%%%%%%%%%%%%%%%%%%%%%%%%%%%%%%%%%%%%%%%%%%%%%%%%%%%%%%%%%%%%%%%%%%%%%%%%%%%%%%%%%%%%%%%%%%%%%%%%%%%%%%%%%%%%%%%%%%%%%%%%%%%%%%
\usepackage{amssymb}
\usepackage{LECTURENOTES}

%TCIDATA{OutputFilter=LATEX.DLL}
%TCIDATA{Version=5.50.0.2953}
%TCIDATA{<META NAME="SaveForMode" CONTENT="1">}
%TCIDATA{BibliographyScheme=Manual}
%TCIDATA{Created=Tuesday, April 16, 2019 16:21:17}
%TCIDATA{LastRevised=Tuesday, July 16, 2019 12:46:41}
%TCIDATA{<META NAME="GraphicsSave" CONTENT="32">}
%TCIDATA{<META NAME="DocumentShell" CONTENT="Style Editor\LECTURENOTES">}
%TCIDATA{Language=American English}
%TCIDATA{CSTFile=LECTURENOTES.cst}

\input{tcilatex}
\begin{document}


%TCIMACRO{%
%\TeXButton{*}{\pagenumbering{roman}
%\bgroup
%\parindent=0pt
%\thispagestyle{empty}
%\null}}%
%BeginExpansion
\pagenumbering{roman}
\bgroup
\parindent=0pt
\thispagestyle{empty}
\null%
%EndExpansion
\vspace{1in}\vspace{0.5in}

{\Huge Physics Lectures PH123 }

{\Large Fundamentals of Physics II}\vspace{1in}\vspace{0.5in}

{\Large Wave Motion, Thermal Physics, and Optics}

{\Huge \vspace{0.1in}}

\vspace{0.5in}

\vspace{1in}

{\LARGE R. Todd Lines}

%TCIMACRO{%
%\TeXButton{*}{\vfill
%}}%
%BeginExpansion
\vfill
%
%EndExpansion
\newpage

%TCIMACRO{\TeXButton{Bottom of the Page}{\mbox{}\vfill}}%
%BeginExpansion
\mbox{}\vfill%
%EndExpansion

Created in Scientific Workplace 
%TCIMACRO{\TeXButton{TM}{$^{\text{\tiny TM}}$}}%
%BeginExpansion
$^{\text{\tiny TM}}$%
%EndExpansion

Copyright 
%TCIMACRO{\TeXButton{Copyright}{\copyright} }%
%BeginExpansion
\copyright
%EndExpansion
2016 by Author

\newpage

%TCIMACRO{%
%\TeXButton{*}{\thispagestyle{empty}
%\enlargethispage{1in}}}%
%BeginExpansion
\thispagestyle{empty}
\enlargethispage{1in}%
%EndExpansion

\QTP{preface}
Preface

\section{Preface}

This set of notes is intended to be an aid to the student. It is what I
intend to present in class. There are likely errors and mistakes, so use
these notes, but don't expect perfection. If there are things that are
confusing, please talk to me or ask questions in class.

\section{Forward}

\section{Acknowledgments}

I consulted many authors and colleagues in creating these notes. Principle
among the authors was Serway and Jewett and Haliday and Resnic. Colleagues
of special note are Kevin Kelly, Brian Pyper, Steve Tercotte, and Ryan
Nielson. I especially appreciate the PH123 students of the Fall of 2006.
They were the first to test run this sets of notes, and I appreciate their
feedback (and patience).

\emph{BYU-I} \hfill \emph{R. Todd Lines}.

\newpage

\QTP{contentsection}
Contents

%TCIMACRO{\TeXButton{1.0 Line Spacing}{\setlength{\parskip}{0pt}}}%
%BeginExpansion
\setlength{\parskip}{0pt}%
%EndExpansion

\setlength{\parskip}{0pt}%EndExpansion

%TCIMACRO{\TeXButton{TOC}{\TableOfContents}}%
%BeginExpansion
\TableOfContents%
%EndExpansion

%TCIMACRO{\TeXButton{1.5 Line Spacing}{\setlength{\parskip}{9pt}}}%
%BeginExpansion
\setlength{\parskip}{9pt}%
%EndExpansion

\pagenumbering{arabic}

\setcounter{page}{1}

\setcounter{chapter}{0}

\chapter{Introduction to the Course, Simple Harmonic Motion}

%TCIMACRO{%
%\TeXButton{Fundamental Concepts}{\hspace{-1.3in}{\Large Fundamental Concepts\vspace{0.25in}}}}%
%BeginExpansion
\hspace{-1.3in}{\Large Fundamental Concepts\vspace{0.25in}}%
%EndExpansion

\begin{itemize}
\item Simple Harmonic Motion

\item Frequency and Period

\item Mathematical description of Simple Harmonic Motion

\item Relationship between Simple Harmonic Motion and circular motion.
\end{itemize}

\section{Oscillation and Simple Harmonic Motion}

Last semester, in PH121 (or Dynamics) you studied how things move. We
identified the object we were studying as the \emph{mover mass }and the
objects that exerted forces on the mover as the \emph{environmental objects.}
You learned about forces and torques which get mover objects moving. You
also learned and practiced a lot of math. This semester, we will start with
a very special type of motion. It is the motion that results from
oscillation. We call this very special type of motion, \emph{simple harmonic
motion}.

Simple harmonic motion (SHM) means a motion that repeats in a special,
simple way. Some characteristics of this type of motion are as follows:

1.\qquad The motion repeats in a regular way, like a grandfather clock
pendulum swings back and forth in a set amount of time. This set amount of
time is called a period.

2.\qquad The mover object moves about a center position and is symmetric
about that position. This center position is the bottom of the swing for our
grandfather clock. The idea of symmetry just means that the pendulum reaches
the same height on both sides as it swings

3.\qquad The mover object's motion can be traced out in a position vs. time
graph. For SHM, the shape of the position vs. time graph is a sinusoid.

4.\qquad The velocity of the mover object constantly changes. It is zero at
the extreme points (the points farthest from the center, or the largest
positive and the largest negative displacements). That is where it turns
around and goes back the other way. It stops there, but just for a split
second. The velocity is largest at the equilibrium position (the center
position). If we plotted a velocity vs. time graph, the velocity would also
be a sinusoid or snake-like shape.

5.\qquad The acceleration also constantly changes. And it is also a sinusoid.

The center position is called the Equilibrium position, and for convenience
we often define it as the origin of our coordinate system.\FRAME{dtbpF}{%
2.5261in}{1.9484in}{0pt}{}{}{Figure}{\special{language "Scientific
Word";type "GRAPHIC";maintain-aspect-ratio TRUE;display "USEDEF";valid_file
"T";width 2.5261in;height 1.9484in;depth 0pt;original-width
1.2834in;original-height 0.9833in;cropleft "0";croptop "1";cropright
"1";cropbottom "0";tempfilename 'PQXXQWOK.wmf';tempfile-properties "XPR";}}

\begin{definition}
Equilibrium Position: The position of the mover mass (not the spring, which
is an environmental object acting on the mover mass) when the spring is
neither stretched nor compressed.
\end{definition}

Let's draw a picture of simple harmonic motion. First we need a device to do
this. Suppose you go to the supermarket. But instead of putting ramen on the
belt at the checkout counter, you strap on a device like this\footnote{%
Please don't really try this at the local supermarkets. They don't seem to
have any sense of humor about such things at all.}\FRAME{dhF}{4.3474in}{%
1.5904in}{0pt}{}{}{Figure}{\special{language "Scientific Word";type
"GRAPHIC";maintain-aspect-ratio TRUE;display "USEDEF";valid_file "T";width
4.3474in;height 1.5904in;depth 0pt;original-width 4.2964in;original-height
1.5549in;cropleft "0";croptop "1";cropright "1";cropbottom "0";tempfilename
'PQXXQWOL.wmf';tempfile-properties "XPR";}}You can see the mover mass on the
spring. But we have placed a pen on the mass and that pen is tracing out a
pattern as the belt moves. You will recognize this pattern as a trig
function. The result might look something like this if you removed the belt.%
\FRAME{dhF}{3.9418in}{1.4226in}{0pt}{}{}{Figure}{\special{language
"Scientific Word";type "GRAPHIC";maintain-aspect-ratio TRUE;display
"USEDEF";valid_file "T";width 3.9418in;height 1.4226in;depth
0pt;original-width 3.8934in;original-height 1.388in;cropleft "0";croptop
"1";cropright "1";cropbottom "0";tempfilename
'PQXXQWOM.wmf';tempfile-properties "XPR";}}Of course, what we have made is a
position vs. time graph. We remember these from PH121. This gives us a
record of the past motion of the mass. $\cos x$\FRAME{dtbpFX}{3.9583in}{%
1.8542in}{0pt}{}{}{Plot}{\special{language "Scientific Word";type
"MAPLEPLOT";width 3.9583in;height 1.8542in;depth 0pt;display
"USEDEF";plot_snapshots TRUE;mustRecompute FALSE;lastEngine "MuPAD";xmin
"0";xmax "4.7";xviewmin "0";xviewmax "4.7";yviewmin "-10";yviewmax
"10";viewset"XY";rangeset"X";plottype 4;labeloverrides 3;x-label "t
(s)";y-label "x (cm)";axesFont "Times New
Roman,12,0000000000,useDefault,normal";numpoints 100;plotstyle
"patch";axesstyle "normal";axestips FALSE;xis \TEXUX{t};var1name
\TEXUX{$t$};function \TEXUX{$5\cos \left( 2t+0\right) $};linecolor
"blue";linestyle 1;pointstyle "point";linethickness 1;lineAttributes
"Solid";var1range "0,4.7";num-x-gridlines 100;curveColor
"[flat::RGB:0x000000ff]";curveStyle "Line";VCamFile
'PQXXR23J.xvz';valid_file "T";tempfilename
'PQXXQWON.wmf';tempfile-properties "XPR";}}where in this graph, $x_{\max }=5%
\unit{cm}.$ Having the power of mathematics, we know we can write an
equation that would describe this curve. From your Trigonometry (trig)
experience, we can guess that it might look something like%
\[
x\left( t\right) =x_{\max }\cos \left( \theta \right) 
\]%
Notice that the instantaneous position, $x\left( t\right) $ is, indeed, a
function of time. Back in trigonometry we would have said a cosine function
was a function of an angle, $\theta .$ But we know this is a position vs.
time graph, so our angle must be different for different times. Let's write 
\[
\theta \left( t\right) =\omega t 
\]%
as strange sort of an angle. This \textquotedblleft angle\textquotedblright\
is a function of time, and let's say how fast our \textquotedblleft
angle\textquotedblright\ is changing is given by 
\[
\frac{d\theta }{dt}=\omega 
\]%
This is a sort of speed for how fast our angle is changing. So our equation
must be 
\[
x\left( t\right) =x_{\max }\cos \left( \omega t\right) 
\]%
In the next figure $x\left( t\right) =x_{\max }\cos \left( \omega t\right) $
is plotted with two different values for $\omega .$ \FRAME{dtbpFX}{3.9583in}{%
1.3578in}{0pt}{}{}{Plot}{\special{language "Scientific Word";type
"MAPLEPLOT";width 3.9583in;height 1.3578in;depth 0pt;display
"USEDEF";plot_snapshots TRUE;mustRecompute FALSE;lastEngine "MuPAD";xmin
"0";xmax "4.7";xviewmin "0";xviewmax "4.7";yviewmin "-10";yviewmax
"10";viewset"XY";rangeset"X";plottype 4;labeloverrides 3;x-label "t
(s)";y-label "x (cm)";axesFont "Times New
Roman,12,0000000000,useDefault,normal";numpoints 100;plotstyle
"patch";axesstyle "normal";axestips FALSE;xis \TEXUX{t};var1name
\TEXUX{$t$};function \TEXUX{$5\cos \left( 2t+0\right) $};linecolor
"blue";linestyle 1;pointstyle "point";linethickness 1;lineAttributes
"Solid";var1range "0,4.7";num-x-gridlines 100;curveColor
"[flat::RGB:0x000000ff]";curveStyle "Line";function \TEXUX{$5\cos \left(
4t+0\right) $};linecolor "maroon";linestyle 2;pointstyle
"point";linethickness 1;lineAttributes "Dash";var1range
"0,4.7";num-x-gridlines 100;curveColor "[flat::RGB:0x00800000]";curveStyle
"Line";VCamFile 'PQXXR23I.xvz';valid_file "T";tempfilename
'PQXXQWOO.wmf';tempfile-properties "XPR";}}We can see what $\omega $ does
for us. It stretches out or compresses our curve. In the blue (dashed) curve
our angle changes quickly. In the red (solid) curve the angle changes
slowly. Note that we are not plotting position vs. angle. Both plots would
be the same if we did! We are plotting position vs. time. And for the blue
curve our mass is oscillating much more quickly than it is for the red
curve. Let's try this with a coordinate system and the values

I\ have used $x_{\max }$ where most books use the letter $A$ for \emph{%
amplitude }to emphasize that the amplitude is the maximum displacement of
the mover mass from the mover mass equilibrium position. In our coordinate
system, the mass is going back and forth in the $x$ direction. The amplitude
means the maximum displacement, so for our coordinate system $x_{\max }$ is
a good way to write amplitude. Here is the same graph but with $x_{\max }=9%
\unit{cm}$\FRAME{dtbpFX}{3.9583in}{1.3578in}{0pt}{}{}{Plot}{\special%
{language "Scientific Word";type "MAPLEPLOT";width 3.9583in;height
1.3578in;depth 0pt;display "USEDEF";plot_snapshots TRUE;mustRecompute
FALSE;lastEngine "MuPAD";xmin "0";xmax "4.7";xviewmin "0";xviewmax
"4.7";yviewmin "-10";yviewmax "10";viewset"XY";rangeset"X";plottype
4;labeloverrides 3;x-label "t (s)";y-label "x (cm)";axesFont "Times New
Roman,12,0000000000,useDefault,normal";numpoints 100;plotstyle
"patch";axesstyle "normal";axestips FALSE;xis \TEXUX{t};var1name
\TEXUX{$t$};function \TEXUX{$9\cos \left( 2t+0\right) $};linecolor
"blue";linestyle 1;pointstyle "point";linethickness 1;lineAttributes
"Solid";var1range "0,4.7";num-x-gridlines 100;curveColor
"[flat::RGB:0x000000ff]";curveStyle "Line";function \TEXUX{$9\cos \left(
4t+0\right) $};linecolor "maroon";linestyle 2;pointstyle
"point";linethickness 1;lineAttributes "Dash";var1range
"0,4.7";num-x-gridlines 100;curveColor "[flat::RGB:0x00800000]";curveStyle
"Line";VCamFile 'PQXXR23H.xvz';valid_file "T";tempfilename
'PQXXQWOP.wmf';tempfile-properties "XPR";}}The two graphs for the two
different $\omega $ values are not more stretched out in time, but now they
are taller along the position axis. This means that the mass is moving
farther as it oscillates.

You will remember from your trigonometry class that the \emph{period,} $T,$
tells us the time it takes for the oscillation to go through a complete
cycle. You can probably guess that how long it takes to oscillate and how
often it oscillates would be related. How often the oscillator completes a
cycle is called the frequency. The longer the period, the lower the
frequency. 
\[
f=\frac{1}{T} 
\]%
Think of cars passing you on your way to class. Period is like how long you
wait in between cars. Frequency is like how often cars pass. If you wait
less time between cars, the cars pass more frequently. And that is just what
our equation says! We can see from our graphs that our stretching quantity $%
\omega ,$ must be related to the frequency of our oscillation. If the
frequency is high, then $\omega $ must be large so that we reach different
\textquotedblleft angles\textquotedblright\ faster. But for the cosine
function to work we need angle units. We will choose radians for our units
and we will write our stretching quantity as 
\[
\omega =2\pi f 
\]%
$f=\frac{1}{T}$This works, if $\omega $ is bigger then our oscillation
happens more frequently. The $2\pi $ has units of radians. So $\omega $ has
units of $\unit{rad}\unit{Hz}$ or more commonly $\unit{rad}/\unit{s}.$ That
matches our derivative above. Let's give a name to the quantity $\omega .$
Since it has radians in it we might guess that it has something to do with
circular motion (more on this later) and it has frequency in it. So we will
call $\omega $ the \emph{angular frequency}.

\section{Velocity and Acceleration}

So far we have guessed the descriptive equation for SHM. 
\begin{equation}
x\left( t\right) =x_{\max }\cos \left( \omega t\right)
\end{equation}%
This gives us the position of the particle at any given time. Since knowing
the position as a function of time is a good description of the motion of
our mover mass, we can call this equation the \emph{equation of motion} for
our mass. We will see soon that this equation is correct. But let's pretend
that we are earlier researchers and we just know the equation makes the
right shape. Still, we can learn much from knowing this equation. Since we
know how to take derivatives, and we know the derivative of position with
respect to time is the velocity, we can see that the velocity of the mass at
any given time is given by

\begin{equation}
v\left( t\right) =\frac{dx\left( t\right) }{dt}=-\omega x_{\max }\sin \left(
\omega t\right)
\end{equation}%
From our fond memories of our trigonometry class we know the maximum of a
sine function is always $1.\strut $\FRAME{dtbpFX}{4.4996in}{1.0421in}{0pt}{}{%
}{Plot}{\special{language "Scientific Word";type "MAPLEPLOT";width
4.4996in;height 1.0421in;depth 0pt;display "USEDEF";plot_snapshots
TRUE;mustRecompute FALSE;lastEngine "MuPAD";xmin "-10.001000";xmax
"10.001000";xviewmin "-10.001000";xviewmax "10.001000";yviewmin
"-1.000187";yviewmax "1.000187";viewset"XY";rangeset"X";plottype
4;labeloverrides 3;x-label "x";y-label "sin(x)";axesFont "Times New
Roman,12,0000000000,useDefault,normal";numpoints 100;plotstyle
"patch";axesstyle "normal";axestips FALSE;xis \TEXUX{x};var1name
\TEXUX{$x$};function \TEXUX{$\sin x$};linecolor "blue";linestyle
1;pointstyle "point";linethickness 1;lineAttributes "Solid";var1range
"-10.001000,10.001000";num-x-gridlines 100;curveColor
"[flat::RGB:0x000000ff]";curveStyle "Line";function \TEXUX{$1$};linecolor
"black";linestyle 2;pointstyle "point";linethickness 1;lineAttributes
"Dash";var1range "-10.001000,10.001000";num-x-gridlines 100;curveColor
"[flat::RGB:0000000000]";curveStyle "Line";function \TEXUX{$-1$};linecolor
"black";linestyle 2;pointstyle "point";linethickness 1;lineAttributes
"Dash";var1range "-10.001000,10.001000";num-x-gridlines 100;curveColor
"[flat::RGB:0000000000]";curveStyle "Line";VCamFile
'PQXXR23G.xvz';valid_file "T";tempfilename
'PQXXQWOQ.wmf';tempfile-properties "XPR";}}Notice that the $\omega $ and the 
$x_{\max }$ aren't changing for our oscillation mover mass. They are
constant. Then if we want the maximum speed we can simply set the $\sin
\left( \omega t\right) =1.$ Then the maximum speed will be 
\begin{equation}
v_{\max }=\omega x_{\max }\left( 1\right)
\end{equation}%
Notice that this does not tell us when the speed is maximum. Just what the
maximum speed is. We will often use this trick of knowing the maximum of
sine is one.

We can also find the acceleration. We just take another derivative.%
\begin{eqnarray*}
a\left( t\right) &=&\frac{dv}{dt} \\
&=&\frac{d^{2}x\left( t\right) }{dt^{2}} \\
&=&-\omega ^{2}x_{\max }\cos \left( \omega t\right)
\end{eqnarray*}%
This is the acceleration of the mass attached to the spring. It's also true
that the maximum for a cosine function is $1,$ so the maximum acceleration
would be

\begin{equation}
a_{\max }=\omega ^{2}x_{\max }\left( 1\right)
\end{equation}

These are significant results, so let's summarize. For a simple harmonic
oscillator, the instantaneous position, speed, and acceleration are given by%
\begin{eqnarray}
x\left( t\right) &=&x_{\max }\cos \left( \omega t\right) \\
v\left( t\right) &=&-\omega x_{\max }\sin \left( \omega t\right)  \nonumber
\\
a\left( t\right) &=&-\omega ^{2}x_{\max }\cos \left( \omega t\right) 
\nonumber
\end{eqnarray}

Let's plot $x\left( t\right) ,$ $v\left( t\right) ,$ and $a\left( t\right) $
for a specific case\FRAME{dtbpFX}{3.6357in}{2.4163in}{0pt}{}{}{Plot}{\special%
{language "Scientific Word";type "MAPLEPLOT";width 3.6357in;height
2.4163in;depth 0pt;display "USEDEF";plot_snapshots TRUE;mustRecompute
FALSE;lastEngine "MuPAD";xmin "-5";xmax "5";xviewmin "-5";xviewmax
"5";yviewmin "-20";yviewmax "20";viewset"XY";rangeset"X";plottype
4;labeloverrides 3;x-label "t";y-label "x";axesFont "Times New
Roman,12,0000000000,useDefault,normal";numpoints 100;plotstyle
"patch";axesstyle "normal";axestips FALSE;xis \TEXUX{t};var1name
\TEXUX{$t$};function \TEXUX{$5\cos \left( 2t+0\right) $};linecolor
"blue";linestyle 1;pointstyle "point";linethickness 3;lineAttributes
"Solid";var1range "-5,5";num-x-gridlines 100;curveColor
"[flat::RGB:0x000000ff]";curveStyle "Line";function \TEXUX{$-2\ast 5\sin
\left( 5t+0\right) $};linecolor "green";linestyle 2;pointstyle
"point";linethickness 3;lineAttributes "Dash";var1range
"-5,5";num-x-gridlines 100;curveColor "[flat::RGB:0x00008000]";curveStyle
"Line";function \TEXUX{$-\left( 2\right) ^{2}\left( 5\right) \cos \left(
2t+0\right) $};linecolor "red";linestyle 4;pointstyle "point";linethickness
3;lineAttributes "DotDash";var1range "-5,5";num-x-gridlines 100;curveColor
"[flat::RGB:0x00ff0000]";curveStyle "Line";VCamFile
'PQXXR23F.xvz';valid_file "T";tempfilename
'PQXXQWOR.wmf';tempfile-properties "XPR";}}Red (solid) is the displacement,
green (dashed) is the velocity, and blue (dot-dashed) is the acceleration.
Note that each as a different maximum amplitude. Also note that they don't
rise and fall at the same time. We will describe this as being \emph{not in
phase}.\FRAME{dtbpF}{2.9888in}{2.8409in}{0pt}{}{}{Figure}{\special{language
"Scientific Word";type "GRAPHIC";maintain-aspect-ratio TRUE;display
"USEDEF";valid_file "T";width 2.9888in;height 2.8409in;depth
0pt;original-width 6.5639in;original-height 6.2396in;cropleft "0";croptop
"1";cropright "1";cropbottom "0";tempfilename
'PQXXQWOS.wmf';tempfile-properties "XPR";}}The acceleration is $90\unit{%
%TCIMACRO{\U{b0}}%
%BeginExpansion
{{}^\circ}%
%EndExpansion
}$ \emph{out of phase} from the velocity. Let's think about why this would
be. Suppose we attach a mass to a spring and allow the mass to slide on a
frictionless surface.\FRAME{dtbpF}{2.4111in}{1.7461in}{0pt}{}{}{Figure}{%
\special{language "Scientific Word";type "GRAPHIC";maintain-aspect-ratio
TRUE;display "USEDEF";valid_file "T";width 2.4111in;height 1.7461in;depth
0pt;original-width 7.043in;original-height 5.0903in;cropleft "0";croptop
"1";cropright "1";cropbottom "0";tempfilename
'PQXXQWOT.wmf';tempfile-properties "XPR";}}Let's start by stretching the
spring by pulling the mass to the right and releasing. This is the situation
in the left hand part of the last figure.

\FRAME{dtbpF}{1.4114in}{1.1692in}{0pt}{}{}{Figure}{\special{language
"Scientific Word";type "GRAPHIC";maintain-aspect-ratio TRUE;display
"USEDEF";valid_file "T";width 1.4114in;height 1.1692in;depth
0pt;original-width 2.1084in;original-height 1.7417in;cropleft "0";croptop
"1";cropright "1";cropbottom "0";tempfilename
'PQXXQWOU.wmf';tempfile-properties "XPR";}}The spring is pulling strongly on
the mass. We could draw a free body diagram for this situation. We will need
the spring force \FRAME{dtbpF}{1.5592in}{1.0957in}{0pt}{}{}{Figure}{\special%
{language "Scientific Word";type "GRAPHIC";maintain-aspect-ratio
TRUE;display "USEDEF";valid_file "T";width 1.5592in;height 1.0957in;depth
0pt;original-width 2.1577in;original-height 1.5091in;cropleft "0";croptop
"1";cropright "1";cropbottom "0";tempfilename
'PQXXQWOV.wmf';tempfile-properties "XPR";}}Back in PH121 we said that
Hooke's Law is not something that is always true, but by \textquotedblleft
law\textquotedblright\ we mean a mathematical representation (and equation)
that comes from our mental model of how the universe works. In this case,
Hooke's law is an equation that comes from Hooke's model of how springs
work. It is a good model for most springs as long as we don't stretch them
too far. You remember Hooke's law from PH 121. It tells us that the spring
force is proportional to how far we stretch or compress the spring $\left(
\Delta x\right) $ and how stiff the spring is $\left( k\right) .$

\begin{eqnarray}
F_{s} &=&S=-k\Delta x \\
&=&-k\left( x-x_{o}\right)  \nonumber
\end{eqnarray}%
where $x_{o}$ is the \emph{equilibrium position of the mass. }If we assume
the equilibrium position is at $x=0$ then we can write our spring force as%
\begin{equation}
S=-kx
\end{equation}%
If we write out Newton's second law we get 
\begin{eqnarray*}
F_{net_{x}} &=&ma_{x}=-S_{ms} \\
F_{net_{y}} &=&ma_{y}=N_{mT}-W_{mE}
\end{eqnarray*}%
We can see that $a_{y}$ should be zero because the mass won't lift off the
table in the $y$-direction. but in the $x$-direction 
\[
a_{x}=\frac{S_{ms}}{m} 
\]%
and knowing that 
\[
S_{ms}=-kx 
\]%
from Hooke's law, we can see that 
\[
a_{x}=-\frac{kx}{m} 
\]%
and note that since $x$ is large at this point, the acceleration must be big
at this point. since the net force is big, the acceleration is big. Of
course we realize that way back in our Newton's second law formulation it
might have been better to say 
\[
-F_{net_{x}}=-ma_{x}=-S_{ms} 
\]%
but it worked out fine to say we didn't know the direction of $a_{x}.$ In
the end, the minus sign tells us that $a_{x}$ must be to the left.

But suppose the mass was to the left of $x=0$ \FRAME{dtbpF}{2.0643in}{%
0.9582in}{0in}{}{}{Figure}{\special{language "Scientific Word";type
"GRAPHIC";maintain-aspect-ratio TRUE;display "USEDEF";valid_file "T";width
2.0643in;height 0.9582in;depth 0in;original-width 2.0254in;original-height
0.9254in;cropleft "0";croptop "1";cropright "1";cropbottom "0";tempfilename
'PQXXQWOW.wmf';tempfile-properties "XPR";}}Now $x<0$ so it is negative. That
makes 
\[
a_{x}=-\frac{kx}{m} 
\]%
positive. The acceleration always seems to point toward $x=0,$ the
equilibrium position for the mass. This is an important part of the
definition of simple harmonic motion, having an acceleration that always
points toward the equilibrium position. We call a force that makes the
acceleration point toward the equilibrium position \emph{restoring force}.

\begin{definition}
Restoring force:\ A force that is always directed toward the equilibrium
position
\end{definition}

Now lets look at the situation a short time later \FRAME{dtbpF}{1.4096in}{%
1.1087in}{0pt}{}{}{Figure}{\special{language "Scientific Word";type
"GRAPHIC";maintain-aspect-ratio TRUE;display "USEDEF";valid_file "T";width
1.4096in;height 1.1087in;depth 0pt;original-width 1.375in;original-height
1.075in;cropleft "0";croptop "1";cropright "1";cropbottom "0";tempfilename
'PQXXQWOX.wmf';tempfile-properties "XPR";}}We can see that in this position, 
$x=0$ so 
\[
a_{x}=\frac{k\left( 0\right) }{m}=0 
\]%
so at this position the acceleration is zero. That is because right at $x=0$
the spring is neither pushing nor pulling. There is no net force so no
acceleration when $x=0.$

\section{The Idea of Phase}

We said before that $x\left( t\right) ,$ $v\left( t\right) ,$ and $a\left(
t\right) $ are \textquotedblleft out of phase.\textquotedblright\ Let's look
at the idea of \textquotedblleft phase\textquotedblright\ more carefully.
Suppose you return to the grocery store and start your SHM\ device.\FRAME{dhF%
}{3.2292in}{1.1813in}{0pt}{}{}{Figure}{\special{language "Scientific
Word";type "GRAPHIC";maintain-aspect-ratio TRUE;display "USEDEF";valid_file
"T";width 3.2292in;height 1.1813in;depth 0pt;original-width
4.2964in;original-height 1.5549in;cropleft "0";croptop "1";cropright
"1";cropbottom "0";tempfilename 'PQXXQWOY.wmf';tempfile-properties "XPR";}}%
But this time you work with a lab partner, and the lab partner tries to
start the stopwatch when you let go of the mass. But, due to having a slow
reaction time, your partner starts the clock too late. The resulting graph
looks like this\FRAME{dtbpFX}{3.9583in}{1.8542in}{0pt}{}{}{Plot}{\special%
{language "Scientific Word";type "MAPLEPLOT";width 3.9583in;height
1.8542in;depth 0pt;display "USEDEF";plot_snapshots TRUE;mustRecompute
FALSE;lastEngine "MuPAD";xmin "0";xmax "4.7";xviewmin "0";xviewmax
"4.7";yviewmin "-10";yviewmax "10";viewset"XY";rangeset"X";plottype
4;labeloverrides 3;x-label "t";y-label "x";axesFont "Times New
Roman,12,0000000000,useDefault,normal";numpoints 100;plotstyle
"patch";axesstyle "normal";axestips FALSE;xis \TEXUX{t};var1name
\TEXUX{$t$};function \TEXUX{$5\cos \left( 2t+\pi /3\right) $};linecolor
"blue";linestyle 1;pointstyle "point";linethickness 1;lineAttributes
"Solid";var1range "0,4.7";num-x-gridlines 100;curveColor
"[flat::RGB:0x000000ff]";curveStyle "Line";VCamFile
'PQXXR23E.xvz';valid_file "T";tempfilename
'PQXXQWOZ.wmf';tempfile-properties "XPR";}}But we know that the only
difference between the two situations is that the lab partner was slow, so
the graph is shifted on the time axes. We expect we can use the same
mathematical model for SHM, but we must need to change something.

We know from our algebra classes what a shift looks like. Take the
expression 
\[
y=x^{2} 
\]%
We can plot this to get a parabola\FRAME{dtbpFX}{2.6455in}{1.7642in}{0pt}{}{%
}{Plot}{\special{language "Scientific Word";type "MAPLEPLOT";width
2.6455in;height 1.7642in;depth 0pt;display "USEDEF";plot_snapshots
TRUE;mustRecompute FALSE;lastEngine "MuPAD";xmin "-5";xmax "5";xviewmin
"-5.0010000010002";xviewmax "5.0010000010002";yviewmin "0";yviewmax
"25.0024997474245";plottype 4;axesFont "Times New
Roman,12,0000000000,useDefault,normal";numpoints 100;plotstyle
"patch";axesstyle "normal";axestips FALSE;xis \TEXUX{x};var1name
\TEXUX{$x$};function \TEXUX{$x^{2}$};linecolor "blue";linestyle 1;pointstyle
"point";linethickness 3;lineAttributes "Solid";var1range
"-5,5";num-x-gridlines 100;curveColor "[flat::RGB:0x000000ff]";curveStyle
"Line";VCamFile 'PQXXR23D.xvz';valid_file "T";tempfilename
'PQXXQWP0.wmf';tempfile-properties "XPR";}}A shifted parabola would be
expressed as 
\[
y=\left( x-\phi _{o}\right) ^{2} 
\]%
where $\phi _{o}$ is the amount of the shift. Suppose $\phi _{o}=2,$ then 
\[
y=\left( x-2\right) ^{2} 
\]%
and our shifted parabola looks like this\FRAME{dtbpFX}{2.6455in}{1.7642in}{%
0pt}{}{}{Plot}{\special{language "Scientific Word";type "MAPLEPLOT";width
2.6455in;height 1.7642in;depth 0pt;display "USEDEF";plot_snapshots
TRUE;mustRecompute FALSE;lastEngine "MuPAD";xmin "-5";xmax "5";xviewmin
"-5.0010000010002";xviewmax "5.0010000010002";yviewmin
"-0.003981639427977";yviewmax "49.0048999130736";plottype 4;axesFont "Times
New Roman,12,0000000000,useDefault,normal";numpoints 100;plotstyle
"patch";axesstyle "normal";axestips FALSE;xis \TEXUX{x};var1name
\TEXUX{$x$};function \TEXUX{$\left( x-2\right) ^{2}$};linecolor
"blue";linestyle 1;pointstyle "point";linethickness 3;lineAttributes
"Solid";var1range "-5,5";num-x-gridlines 100;curveColor
"[flat::RGB:0x000000ff]";curveStyle "Line";VCamFile
'PQXXR23C.xvz';valid_file "T";tempfilename
'PQXXQWP1.wmf';tempfile-properties "XPR";}}Recall that a shift like 
\[
y=\left( x-\phi _{o}\right) ^{2} 
\]%
will move the parabola to the right while 
\[
y=\left( x+\phi _{o}\right) ^{2} 
\]%
will move the parabola to the left.

We can use this idea to write our SHM expression for our situation with the
slow lab partner. 
\[
x\left( t\right) =x_{\max }\cos \left( \omega \left( t\pm \tau _{o}\right)
\right) 
\]%
In our case, the lab partner was late by $\tau _{o}=\allowbreak
4.\,\allowbreak 712\,4\unit{s}.$ This is not usually how we express the
shift, however. We usually distribute the $\omega $ so our equation looks
like%
\begin{eqnarray*}
x\left( t\right) &=&x_{\max }\cos \left( \omega t\pm \omega \tau _{o}\right)
\\
&=&x_{\max }\cos \left( \omega t\pm \omega \tau _{o}\right)
\end{eqnarray*}%
We usually use the symbol $\phi _{o}=\omega \tau _{o}$ so we will write our
SHM expression for position as a function of time as%
\[
x\left( t\right) =x_{\max }\cos \left( \omega t\pm \phi _{o}\right) 
\]

We could call $\phi _{o}$ the \emph{slow lab partner constant, }but that is
long and not very kind. So let's call $\phi _{o}$ by the name \emph{phase
constant}. It is also customary to call the entire expression in
parenthesis, $\left( \omega t\pm \phi _{o}\right) ,$ the phase of the cosine
function. This is especially used in the fields of Optics and
Electrodynamics.

From what we did before, we know that $x\left( t\right) ,$ $v\left( t\right)
,$ and $a\left( t\right) $ are out of phase, so there must be a phase
constant involved somehow. Let's look for it in what follows.

\section{Initial Conditions}

Usually we need to know how we start our oscillator to solve a problem.
Let's see how this works.\FRAME{dtbpF}{1.5833in}{1.2187in}{0pt}{}{}{Figure}{%
\special{language "Scientific Word";type "GRAPHIC";maintain-aspect-ratio
TRUE;display "USEDEF";valid_file "T";width 1.5833in;height 1.2187in;depth
0pt;original-width 1.585in;original-height 1.2134in;cropleft "0";croptop
"1";cropright "1";cropbottom "0";tempfilename
'PQXXQWP2.wmf';tempfile-properties "XPR";}}

Suppose we start the motion of a mass attached to a spring (a harmonic
oscillator) by pulling the mass to $x=x_{\max }$ and releasing it at $t=0.$
Let's see if we can find the phase. Our initial conditions require 
\begin{eqnarray*}
x\left( 0\right) &=&x_{\max } \\
v\left( 0\right) &=&0
\end{eqnarray*}

Using our formula for $x\left( t\right) $ and $v\left( t\right) $ we have%
\begin{eqnarray*}
x\left( 0\right) &=&x_{\max }=x_{\max }\cos \left( 0+\phi _{o}\right) \\
v\left( 0\right) &=&0=-v_{\max }\sin \left( 0+\phi _{o}\right)
\end{eqnarray*}%
If we choose $\phi _{o}=0,$ these conditions are met.

Notice that we needed to know the starting time and the position and the
velocity at that time. These are what we call \emph{initial conditions}. It
is still true that $x\left( t\right) $ and $v\left( t\right) $ are out of
phase. But we fond $\phi _{o}=0.$ There is another phase term hiding in our
expression for $v\left( t\right) $ and to find it we need a small trig
identity.%
\begin{eqnarray*}
\sin \left( \alpha +\phi _{o}\right) &=&\cos \left( \alpha -\left( \frac{\pi 
}{2}-\phi _{o}\right) \right) \\
&=&\cos \left( \alpha -\frac{\pi }{2}+\phi _{o}\right)
\end{eqnarray*}
so we could write 
\begin{eqnarray*}
v\left( t\right) &=&-\omega x_{\max }\sin \left( \omega t+\phi _{o}\right) \\
&=&-\omega x_{\max }\cos \left( \alpha -\frac{\pi }{2}+\phi _{o}\right)
\end{eqnarray*}%
and we can see that there was a phase shift of $-\pi /2$ hiding in our sine
function. So $v\left( t\right) $ really must be out of phase with $x\left(
t\right) .$

\subsection{A second example \protect\FRAME{dtbpF}{1.7039in}{1.4422in}{0pt}{%
}{}{Figure}{\special{language "Scientific Word";type
"GRAPHIC";maintain-aspect-ratio TRUE;display "USEDEF";valid_file "T";width
1.7039in;height 1.4422in;depth 0pt;original-width 1.7074in;original-height
1.4413in;cropleft "0";croptop "1";cropright "1";cropbottom "0";tempfilename
'PQXXQWP3.wmf';tempfile-properties "XPR";}}}

Using the same equipment, let's start with 
\begin{eqnarray*}
x\left( 0\right) &=&0 \\
v\left( 0\right) &=&v_{i}
\end{eqnarray*}%
that is, the mover mass is already moving, and we start watching just as it
passes the equilibrium point.%
\begin{eqnarray*}
x\left( 0\right) &=&0=x_{\max }\cos \left( 0+\phi _{o}\right) \\
v\left( 0\right) &=&v_{i}=-v_{\max }\sin \left( 0+\phi _{o}\right)
\end{eqnarray*}

from the first equation we have 
\[
0=x_{\max }\cos \left( \phi _{o}\right) 
\]%
and that gives us 
\[
\phi _{o}=\cos ^{-1}\left( 0\right) 
\]%
so 
\[
\phi _{o}=\pm \frac{\pi }{2} 
\]%
really this is 
\[
\phi _{o}=\pm \frac{\pi }{2}\pm n\pi \qquad n=0,1,2,\cdots 
\]%
This gives the red dots in the next plot. \FRAME{dtbpFX}{4.4996in}{1.1736in}{%
0pt}{}{}{Plot}{\special{language "Scientific Word";type "MAPLEPLOT";width
4.4996in;height 1.1736in;depth 0pt;display "USEDEF";plot_snapshots
TRUE;mustRecompute FALSE;lastEngine "MuPAD";xmin "-15.001000";xmax
"15.001000";xviewmin "-15.001000";xviewmax "15.001000";yviewmin
"-1";yviewmax "1";viewset"XY";rangeset"X";plottype 4;axesFont "Times New
Roman,12,0000000000,useDefault,normal";numpoints 100;plotstyle
"patch";axesstyle "normal";axestips FALSE;xis \TEXUX{x};var1name
\TEXUX{$x$};function \TEXUX{$\cos \left( x\right) $};linecolor
"black";linestyle 1;pointstyle "point";linethickness 1;lineAttributes
"Solid";var1range "-15.001000,15.001000";num-x-gridlines 100;curveColor
"[flat::RGB:0000000000]";curveStyle "Line";function \TEXUX{$\left[
\MATRIX{2,6}{c}\VR{,,c,,,}{,,c,,,}{,,,,,}\HR{,,,,,,}\CELL{\frac{\pi
}{2}+0}\CELL{0}\CELL{\frac{\pi }{2}+1\pi }\CELL{0}\CELL{\frac{\pi }{2}+2\pi
}\CELL{0}\CELL{\frac{\pi }{2}+3\pi }\CELL{0}\CELL{\frac{\pi }{2}+4\pi
}\CELL{0}\CELL{\frac{\pi }{2}+5\pi }\CELL{0}\right] $};linecolor
"blue";linestyle 1;pointplot TRUE;pointstyle "point";linethickness
1;lineAttributes "Solid";curveColor "[flat::RGB:0x000000ff]";curveStyle
"Point";function \TEXUX{$\left[
\MATRIX{2,6}{c}\VR{,,c,,,}{,,c,,,}{,,,,,}\HR{,,,,,,}\CELL{-\frac{\pi
}{2}+0}\CELL{0}\CELL{-\frac{\pi }{2}-1\pi }\CELL{0}\CELL{\frac{\pi }{2}+2\pi
}\CELL{0}\CELL{-\frac{\pi }{2}-3\pi }\CELL{0}\CELL{-\frac{\pi }{2}-4\pi
}\CELL{0}\CELL{-\frac{\pi }{2}-5\pi }\CELL{0}\right] $};linecolor
"blue";linestyle 1;pointplot TRUE;pointstyle "point";linethickness
1;lineAttributes "Solid";curveColor "[flat::RGB:0x000000ff]";curveStyle
"Point";VCamFile 'PQXXR23B.xvz';valid_file "T";tempfilename
'PQXXQWP4.wmf';tempfile-properties "XPR";}}Notice that at each of these
locations $\cos \left( \phi \right) $ is zero. but let's make an agreement
that we will choose the smallest value for $\phi _{o}$ that makes $\cos
\left( \phi _{o}\right) =0.$ That is our%
\[
\phi _{o}=\pm \frac{\pi }{2} 
\]%
but positive and negative $\pi /2$ are the same \textquotedblleft
smallness.\textquotedblright\ We don't know the sign. Using our initial
velocity condition will let us determine the sign. Let's try it 
\begin{eqnarray*}
v_{i} &=&-v_{\max }\sin \left( \frac{\pi }{2}\right) \\
v_{i} &=&\pm v_{\max } \\
v_{i} &=&\pm \omega x_{\max }
\end{eqnarray*}

From this we can see 
\[
x_{\max }=\pm \frac{v_{i}}{\omega } 
\]%
We defined the initial velocity as positive, and we insist on having
positive amplitudes, we choose 
\[
\phi _{o}=-\frac{\pi }{2} 
\]

Our solutions are%
\begin{eqnarray*}
x\left( t\right) &=&\frac{v_{i}}{\omega }\cos \left( \omega t-\frac{\pi }{2}%
\right) \\
v\left( t\right) &=&v_{i}\sin \left( \omega t-\frac{\pi }{2}\right)
\end{eqnarray*}

\begin{remark}
Generally to have a complete solution, you must find all the constants based
on the initial conditions. This would mean we need $x_{\max },$ $\omega ,$
and $\phi _{o}$ to have a complete solution.
\end{remark}

\subsection{Example}

A particle moving along the $x$ axis in simple harmonic motion starts from
its equilibrium position, the origin, at $t=0$ and moves to the right. The
amplitude of its motion is $2.00\unit{cm},$ and the frequency is $1.50\unit{%
Hz}.$

a) show that the position of the particle is given by%
\[
x=\left( 2.00\unit{cm}\right) \sin \left( 3.00\pi t\right) 
\]%
determine

b) the maximum speed and the earliest time $(t>0)$ at which the particle has
this speed,

c) the maximum acceleration and the earliest time $(t>0)$ at which the
particle has this acceleration, and

d) the total distance traveled between $t=0$ and $t=1.00\unit{s}$

\FRAME{dtbpF}{1.8775in}{0.8484in}{0pt}{}{}{Figure}{\special{language
"Scientific Word";type "GRAPHIC";maintain-aspect-ratio TRUE;display
"USEDEF";valid_file "T";width 1.8775in;height 0.8484in;depth
0pt;original-width 1.8403in;original-height 0.8164in;cropleft "0";croptop
"1";cropright "1";cropbottom "0";tempfilename
'PQXXQWP5.wmf';tempfile-properties "XPR";}}

Basic Equations

\begin{eqnarray*}
x\left( t\right) &=&x_{\max }\cos \left( \omega t+\phi _{o}\right) \\
v\left( t\right) &=&-\omega x_{\max }\sin \left( \omega t+\phi _{o}\right) \\
a\left( t\right) &=&-\omega ^{2}A\cos \left( \omega t+\phi _{o}\right)
\end{eqnarray*}

\[
\omega =2\pi f 
\]

\[
v_{m}=\omega x_{m} 
\]

\[
a_{m}=\omega ^{2}x_{m} 
\]

\[
T=\frac{1}{f} 
\]

Variables

\[
\begin{tabular}{lll}
$t$ & time, initial time =0 & $t_{o}=0$ \\ 
$x$ & Position, Initial position =0 & $x\left( 0\right) =0$ \\ 
$v$ &  &  \\ 
$a$ &  &  \\ 
$x_{m}$ & x amplitude & $x_{m}=2.00\unit{cm}$ \\ 
$v_{m}$ & v amplitude &  \\ 
$a_{m}$ & a amplitude &  \\ 
$\omega $ & angular frequency &  \\ 
$\phi _{o}$ & phase constant &  \\ 
$f$ & frequency & $f=1.50\unit{Hz}$%
\end{tabular}%
\]

Symbolic Solution

Part (a)

We can start by recognizing that we know $\omega $ because we know the
frequency. 
\[
\omega =2\pi f 
\]%
We also know the amplitude $A=x_{\max }$ which is given. Knowing that at $%
t=0 $%
\[
x\left( 0\right) =0=x_{\max }\cos \left( 0+\phi _{o}\right) 
\]%
we can guess that 
\[
\phi _{o}=\pm \frac{\pi }{2} 
\]

Using%
\[
v\left( 0\right) =-\omega x_{\max }\sin \left( 0\pm \frac{\pi }{2}\right) 
\]%
and demanding that amplitudes be positive values, and noting that at $t=0$
the velocity is positive from the initial conditions:%
\[
\phi =-\frac{\pi }{2} 
\]%
We also note from out trig identity that we used above%
\[
\cos \left( \theta -\frac{\pi }{2}\right) =\sin \left( \theta \right) 
\]%
that we have%
\begin{eqnarray*}
x\left( t\right) &=&x_{\max }\cos \left( 2\pi ft-\frac{\pi }{2}\right) \\
&=&x_{\max }\sin \left( 2\pi ft\right)
\end{eqnarray*}%
which is what we were to show.

Part (b)

We have a formula for 
\begin{eqnarray*}
v_{\max } &=&\omega x_{\max } \\
&=&2\pi fx_{\max }
\end{eqnarray*}%
to find when this happens, take%
\[
v\left( t\right) =v_{\max }=-\omega x_{\max }\sin \left( 2\pi ft-\frac{\pi }{%
2}\right) 
\]%
and recognize that $\sin \left( \theta \right) =1$ is at a maximum when $%
\theta =\pi /2$ so%
\[
\frac{\pi }{2}=2\pi ft-\frac{\pi }{2} 
\]%
\[
\pi =2\pi ft 
\]%
\[
\frac{1}{2f}=t 
\]%
\[
\frac{1}{2\left( 1.50\unit{Hz}\right) }=t 
\]

Part (c) Like with the velocity we must the formula

\[
a\left( t\right) =-\omega ^{2}A\cos \left( \omega t+\phi _{o}\right) 
\]%
but recognize that the maximum is achieved when $\cos \left( \omega t+\phi
_{o}\right) $ or when $\omega t+\phi _{o}=0$

\begin{eqnarray*}
t &=&\frac{\phi _{o}}{\omega } \\
&=&\frac{-\frac{\pi }{2}}{2\pi f} \\
&=&\frac{-1}{4f}
\end{eqnarray*}%
The formula for $a_{\max }$ is 
\begin{eqnarray*}
a_{\max } &=&-\omega ^{2}x_{\max } \\
&=&-(2\pi f)^{2}x_{m}
\end{eqnarray*}

Part (d)

We know the period is 
\[
T=\frac{1}{f} 
\]%
We should find the number of periods in $t=1.00\unit{s}$ and find the
distance traveled in one period, 
\[
\frac{t}{T} 
\]%
and multiply them together. In one period the distance traveled is 
\[
d=4x_{m} 
\]

\[
d_{tot}=d\ast \frac{t}{T}=4fx_{m}t 
\]

Numerical Solutions

Part (a)

\begin{eqnarray*}
x\left( t\right) &=&x_{\max }\sin \left( 2\pi ft\right) \\
&=&\left( 2.00\unit{cm}\right) \sin \left( 3.00\pi t\right)
\end{eqnarray*}

Part (b)

\begin{eqnarray*}
v_{m} &=&2\pi \left( 1.50\unit{Hz}\right) \left( 2.00\unit{cm}\right) \\
&=&0.188\,50\frac{\unit{m}}{\unit{s}}
\end{eqnarray*}

\[
\frac{1}{2f}=t 
\]%
\begin{eqnarray*}
\frac{1}{2\left( 1.50\unit{Hz}\right) } &=&t \\
&=&\allowbreak 0.333\,\unit{s}
\end{eqnarray*}

Part (c)

\begin{eqnarray*}
t &=&\frac{-1}{4f} \\
&=&-0.166\,67\unit{s}
\end{eqnarray*}

\begin{eqnarray*}
a_{\max } &=&(2\pi f)^{2}x_{m} \\
&=&\left( 2\pi 1.5\unit{Hz}\right) ^{2}(2.00\unit{cm}) \\
&=&\allowbreak 1.\,\allowbreak 776\,5\frac{\unit{m}}{\unit{s}^{2}}
\end{eqnarray*}

Part (d)

\begin{eqnarray*}
d_{tot} &=&4fx_{m}t \\
&=&8.00\unit{cm}\ast 1.50\unit{Hz}\ast 1.00\unit{s} \\
&=&\allowbreak 0.12\unit{m}
\end{eqnarray*}

\section{Comparing Simple Harmonic Motion with Uniform Circular Motion}

That circular motion and SHM are related should not be a surprise once we
found the solutions to the equations of motion were trig functions. Recall
that the trig functions are defined on a unit circle

\FRAME{dtbpF}{2.2866in}{2.3186in}{0in}{}{}{Figure}{\special{language
"Scientific Word";type "GRAPHIC";maintain-aspect-ratio TRUE;display
"USEDEF";valid_file "T";width 2.2866in;height 2.3186in;depth
0in;original-width 2.2468in;original-height 2.2788in;cropleft "0";croptop
"1";cropright "1";cropbottom "0";tempfilename
'PQXXQWP6.wmf';tempfile-properties "XPR";}}%
\begin{eqnarray}
\tan \theta &=&\frac{x}{y} \\
\cos \theta &=&\frac{x}{h} \\
\sin \theta &=&\frac{y}{h}
\end{eqnarray}

Let's relate this to our equations of motion.\FRAME{dtbpF}{1.7876in}{1.7487in%
}{0pt}{}{}{Figure}{\special{language "Scientific Word";type
"GRAPHIC";maintain-aspect-ratio TRUE;display "USEDEF";valid_file "T";width
1.7876in;height 1.7487in;depth 0pt;original-width 1.7504in;original-height
1.7115in;cropleft "0";croptop "1";cropright "1";cropbottom "0";tempfilename
'PQXXQWP7.wmf';tempfile-properties "XPR";}}Look at the projection $x$ of the
point $P$ on the $x$ axis. This is just the $x$-component of the position!
Lets follow this projection as $P$ travels around the circle. We find it
ranges from $-x_{\max }$ to $x_{\max }.$ If we watch closely we find it's
velocity is zero at the extreme points and is a maximum in the middle. This
projection is given as the $\cos $ of the vector from the origin to $P.$ It
is just taking the $x$-component! This model, indeed fits our SHO solution.

Now lets define a projection of $P$ onto the $y$ axis. Again we have SHM,
but this time the projection is a sine function. Because 
\begin{equation}
\cos \left( \theta -\frac{\pi }{2}\right) =\sin \left( \theta \right)
\end{equation}%
we can see that this is just a SHO that is $90\unit{%
%TCIMACRO{\U{b0}}%
%BeginExpansion
{{}^\circ}%
%EndExpansion
}=\frac{\pi }{2}\unit{rad}$ out of phase. It is probably worth recalling
that a projection of one vector on another can be expressed by a dot
product. We could express our length $x$ as 
\[
x=\overrightarrow{\mathbf{r}}\cdot \mathbf{\hat{\imath}}=r\cos \theta 
\]%
where $r$ is the radius of the circle. \FRAME{dtbpF}{2.3687in}{2.3186in}{0in%
}{}{}{Figure}{\special{language "Scientific Word";type
"GRAPHIC";maintain-aspect-ratio TRUE;display "USEDEF";valid_file "T";width
2.3687in;height 2.3186in;depth 0in;original-width 2.3281in;original-height
2.2788in;cropleft "0";croptop "1";cropright "1";cropbottom "0";tempfilename
'PQXXQWP8.wmf';tempfile-properties "XPR";}}We can see that when $\theta =0$
we have $x=r$ and this will be the largest $x$ value, so $r=x_{\max }.$

So by projecting circular motion onto the $x$-axis 
\[
x\left( t\right) =x_{\max }\cos \left( \theta \right) 
\]%
But $\theta $ changes in time. We can recall from our PH121 or Dynamics
experience that the angular speed 
\[
\omega =\frac{\Delta \theta }{\Delta t} 
\]%
or, if we agree to start from $\theta _{i}=0$ and $t_{i}=0,$ 
\[
\omega =\frac{\theta }{t} 
\]%
so 
\[
\theta =\omega t 
\]%
then we 
\[
x\left( t\right) =x_{\max }\cos \left( \omega t\right) 
\]%
have found SHM. Now we can see why we used \textquotedblleft $\omega $%
\textquotedblright\ for both angular speed and angular frequency. Really
they are very related. Both tell us something about how fast a cyclic event
happens. One is how fast the point $P$ goes around the circle, and the other
is how often the projection goes back and forth. It makes sense that these
have to be the same rate.\FRAME{dtbpFU}{2.5694in}{2.5201in}{0pt}{\Qcb{The
projection of circular motion onto the $x$-axis gives simple harmonic motion.%
}}{}{Figure}{\special{language "Scientific Word";type
"GRAPHIC";maintain-aspect-ratio TRUE;display "USEDEF";valid_file "T";width
2.5694in;height 2.5201in;depth 0pt;original-width 2.5287in;original-height
2.4785in;cropleft "0";croptop "1";cropright "1";cropbottom "0";tempfilename
'PQXXQWP9.wmf';tempfile-properties "XPR";}}

It is also true that we can project onto the $y$-axis.\FRAME{dtbpF}{2.2753in%
}{2.2269in}{0in}{}{}{Figure}{\special{language "Scientific Word";type
"GRAPHIC";maintain-aspect-ratio TRUE;display "USEDEF";valid_file "T";width
2.2753in;height 2.2269in;depth 0in;original-width 2.2355in;original-height
2.1862in;cropleft "0";croptop "1";cropright "1";cropbottom "0";tempfilename
'PQXXQWPA.wmf';tempfile-properties "XPR";}}and we would find that we can
describe this projection as 
\[
y\left( t\right) =y_{\max }\sin \left( \omega t\right) 
\]

We will choose the cosine function, but from our trig experience it should
be clear that these projections are equivalent, just $90\unit{%
%TCIMACRO{\U{b0}}%
%BeginExpansion
{{}^\circ}%
%EndExpansion
}$ out of phase

\[
\cos \left( \theta \pm \frac{\pi }{2}\right) =\mp \sin \theta 
\]

So%
\[
x\left( t\right) =x_{\max }\cos \left( \omega t+\phi _{o}\right) 
\]%
could be a sine function if $\phi _{o}=\pm \pi /2.$ I this way we have
incorporated both possibilities into one function.

Note that this is just the function we guessed from our observation!

\begin{remark}
We see that uniform circular motion can be thought of as the combination of
two SHOs, with a phase difference of $\pi /2\unit{rad}.$
\end{remark}

The angular velocity is given by 
\begin{equation}
\omega =\frac{v}{r}
\end{equation}%
\FRAME{dtbpF}{2.5997in}{2.6184in}{0pt}{}{}{Figure}{\special{language
"Scientific Word";type "GRAPHIC";maintain-aspect-ratio TRUE;display
"USEDEF";valid_file "T";width 2.5997in;height 2.6184in;depth
0pt;original-width 2.6201in;original-height 2.6388in;cropleft "0";croptop
"1";cropright "1";cropbottom "0";tempfilename
'PQXXQWPB.wmf';tempfile-properties "XPR";}}A particle traveling on the $x$%
-axis in SHM will travel from $x_{\max }\ $to $-x_{\max }$ and from $%
-x_{\max }\ $to $x_{\max }$ (one complete period, $T$) while the particle
traveling with $P$ makes one complete revolution. Thus, the angular
frequency $\omega $ of the SHO and the angular velocity of the particle at $%
P $ are the same. (Now we know why we used the same symbol). The magnitude
of the tangential velocity is then%
\begin{equation}
v_{t}=\omega r=\omega x_{\max }
\end{equation}%
and the projection of this velocity onto the $x$-axis is 
\begin{equation}
v_{tx}=-\omega x_{\max }\sin \left( \omega t+\phi _{o}\right)
\end{equation}%
Which is just what we expected from our observation!\FRAME{dtbpF}{2.4794in}{%
2.5079in}{0pt}{}{}{Figure}{\special{language "Scientific Word";type
"GRAPHIC";maintain-aspect-ratio TRUE;display "USEDEF";valid_file "T";width
2.4794in;height 2.5079in;depth 0pt;original-width 2.4388in;original-height
2.4664in;cropleft "0";croptop "1";cropright "1";cropbottom "0";tempfilename
'PQXXQWPC.wmf';tempfile-properties "XPR";}}

The centripetal acceleration of a particle at $P$ is given by 
\begin{equation}
a_{c}=\frac{v_{t}^{2}}{r}=\frac{v_{t}^{2}}{x_{\max }}=\frac{\omega
^{2}x_{\max }^{2}}{x_{\max }}=\omega ^{2}x_{\max }
\end{equation}%
The direction of the acceleration is inward toward the origin. Of course, we
just want the $x$-component of this, so again we make a projection. If we
project this onto the $x$-axis we have%
\begin{equation}
a_{x}=-\omega ^{2}x_{\max }\cos \left( \omega t+\phi \right)
\end{equation}%
Again this is just want we expected from our observation.

So now we have shown that our set of equations 
\begin{eqnarray}
x\left( t\right) &=&x_{\max }\cos \left( \omega t\right) \\
v\left( t\right) &=&-\omega x_{\max }\sin \left( \omega t\right)  \nonumber
\\
a\left( t\right) &=&-\omega ^{2}x_{\max }\cos \left( \omega t\right) 
\nonumber
\end{eqnarray}%
is correct for our harmonic oscillator.

\section{Our first problem type: Simple Harmonic Motion}

You have probably thought by now that we have a new problem type. We can
call it the \emph{simple harmonic motion }problem type or SHM. The equations
we have so far for this problem type are 
\begin{eqnarray}
x\left( t\right) &=&x_{\max }\cos \left( \omega t+\phi _{o}\right) \\
v\left( t\right) &=&-\omega x_{\max }\sin \left( \omega t+\phi _{o}\right) 
\nonumber \\
a\left( t\right) &=&-\omega ^{2}x_{\max }\cos \left( \omega t+\phi
_{o}\right)  \nonumber
\end{eqnarray}%
\[
\omega =2\pi f 
\]

\[
v_{m}=\omega x_{m} 
\]

\[
a_{m}=\omega ^{2}x_{m} 
\]%
\[
T=\frac{1}{f} 
\]

Next lecture, we will study the energy in a harmonic oscillator.

\chapter{Energy and Dynamics of SHM}

Back in PH121 we started with position, velocity, and acceleration to
describe motion. We have done that again for a simple harmonic oscillator.
In PH121, after basic motion, we found that we could use the idea of a force
and Newton's second law to find the acceleration for our description of
motion. Then we changed view points and used the idea of energy to find
motion. The energy picture was easier in some ways. Let's try to develop and
energy picture for simple harmonic oscillators. Of course the goal in
physics is to describe the motion of the object, so we will say that we are
looking for the \emph{dynamics} of simple harmonic oscillators when we look
for the position, velocity, and acceleration as a function of time.

%TCIMACRO{%
%\TeXButton{Fundamental Concepts}{\hspace{-1.3in}{\Large Fundamental Concepts\vspace{0.25in}}}}%
%BeginExpansion
\hspace{-1.3in}{\Large Fundamental Concepts\vspace{0.25in}}%
%EndExpansion

\begin{itemize}
\item Energy and SHM

\item Equation of motion

\item Vertical oscillations
\end{itemize}

\section{Oscillators and Energy}

You might have noticed that we are calling mover objects that experience
simple harmonic motion (SHM) by the name \emph{simple harmonic oscillators
(SHO). }Let's consider such a SHO. Because our SHO is moving, we know there
must be energy associated with it. To understand the energy involved, let's
start with kinetic energy. Recall from PH121 that 
\begin{equation}
K=\frac{1}{2}mv^{2}
\end{equation}%
and we recall that for a spring, we have the spring potential energy given by%
\begin{equation}
U_{s}=\frac{1}{2}kx^{2}
\end{equation}%
Let's try a problem.

Find the maximum velocity of the mass in terms of the Amplitude and the
angular frequency. We want to use our problem solving steps:

\textbf{Type of problem:} This is an energy and a SHO problem:

\textbf{Drawing:} \FRAME{dtbpF}{1.8775in}{0.8484in}{0pt}{}{}{Figure}{\special%
{language "Scientific Word";type "GRAPHIC";maintain-aspect-ratio
TRUE;display "USEDEF";valid_file "T";width 1.8775in;height 0.8484in;depth
0pt;original-width 1.8403in;original-height 0.8164in;cropleft "0";croptop
"1";cropright "1";cropbottom "0";tempfilename
'PQXXQWPD.wmf';tempfile-properties "XPR";}}

Variables: 
\[
\begin{tabular}{ll}
$E$ & energy \\ 
$K$ & Kinetic energy \\ 
$x$ & Position \\ 
$v$ & velocity or speed \\ 
$m$ & mass of the object \\ 
$k$ & spring stiffness constant%
\end{tabular}%
\]

\textbf{Basic Equations:}

\begin{eqnarray*}
K &=&\frac{1}{2}mv^{2} \\
U_{s} &=&\frac{1}{2}kx^{2}
\end{eqnarray*}

Symbolic Answer

You might say, this is an easy problem, we know from last time that 
\[
v_{\max }=x_{\max }\omega 
\]%
but let's find this again using the ideas of energy. In PH121 we often found
that using energy made problems easier, so this might be worth a little more
work now. The total energy is 
\[
E=K+U_{s}+U_{g} 
\]%
Let's say that our oscillator is moving horizontally, and that we define the
center of mass of our object so that it's $y$-position is right at $y=0$ so
our object has zero gravitational potential energy. 
\[
E=K+U_{s}+0 
\]%
And let's say we have no friction, so that 
\begin{eqnarray*}
E_{i} &=&E_{f} \\
K_{i}+U_{si} &=&K_{f}+U_{sf} \\
\frac{1}{2}mv_{i}^{2}+\frac{1}{2}kx_{i}^{2} &=&\frac{1}{2}mv_{f}^{2}+\frac{1%
}{2}kx_{f}^{2}
\end{eqnarray*}%
and let's start with our initial condition being where the spring is
stretched to $x_{\max }.$ Then at that moment, $v_{i}=0.$ And let's take our
final position when the mass is moving as fast as possible (because that is
what we are looking for, $v_{\max }$). We know from our PH121 experience
that this will be right when $U_{sf}=0$ (so all the energy is kinetic). Then 
\[
0+\frac{1}{2}kx_{\max }^{2}=\frac{1}{2}mv_{\max }^{2}+0 
\]%
or%
\[
\frac{1}{2}mv_{\max }^{2}=\frac{1}{2}kx_{\max }^{2} 
\]%
We can solve for $v_{\max }$ 
\[
v_{\max }^{2}=\frac{kx_{\max }^{2}}{m} 
\]%
\[
v_{\max }=x_{\max }\sqrt{\frac{k}{m}} 
\]%
But is this what we wanted? We expected that 
\[
v_{\max }=x_{\max }\omega 
\]%
And this is what we got so long as 
\[
\omega =\sqrt{\frac{k}{m}} 
\]
And this is always true for a mass on a spring. We don't need a numeric
answer, This is reasonable (just what we expect) and the units check.

Let's look at kinetic and potential energy as a function of time. For our
Simple Harmonic Oscillator (SHO) we know the velocity as a function of time,%
\[
v\left( t\right) =-\omega x_{\max }\sin \left( \omega t+\phi \right) 
\]%
so the kinetic energy as a function of time must be%
\begin{eqnarray*}
K &=&\frac{1}{2}m\left( -\omega x_{\max }\sin \left( \omega t+\phi \right)
\right) ^{2} \\
&=&\frac{1}{2}m\omega ^{2}x_{\max }^{2}\sin ^{2}\left( \omega t+\phi \right)
\end{eqnarray*}%
and now we know that $\omega =\sqrt{k/m},$ so we can write this as 
\begin{eqnarray*}
K &=& \\
&=&\frac{1}{2}m\left( \sqrt{\frac{k}{m}}\right) ^{2}x_{\max }^{2}\sin
^{2}\left( \omega t+\phi \right) \\
&=&\frac{1}{2}m\frac{k}{m}x_{\max }^{2}\sin ^{2}\left( \omega t+\phi \right)
\\
&=&\frac{1}{2}kx_{\max }^{2}\sin ^{2}\left( \omega t+\phi \right)
\end{eqnarray*}%
As the spring is stretched or compressed we store energy as spring potential
energy. The potential energy due to a spring acting on our SHO (mover mass)
is given by (from your PH121 class) 
\begin{equation}
U_{s}=\frac{1}{2}kx^{2}
\end{equation}%
For our SHO we also know the position as a function of time 
\[
x\left( t\right) =x_{\max }\cos \left( \omega t+\phi _{o}\right) 
\]%
so the potential energy as a function of time must be%
\[
U_{s}=\frac{1}{2}kx_{\max }^{2}\cos ^{2}\left( \omega t+\phi \right) 
\]%
\FRAME{dtbpF}{2.1473in}{0.9911in}{0pt}{}{}{Figure}{\special{language
"Scientific Word";type "GRAPHIC";maintain-aspect-ratio TRUE;display
"USEDEF";valid_file "T";width 2.1473in;height 0.9911in;depth
0pt;original-width 2.1084in;original-height 0.9582in;cropleft "0";croptop
"1";cropright "1";cropbottom "0";tempfilename
'PQXXQWPE.wmf';tempfile-properties "XPR";}}Let's say that our oscillator is
moving horizontally, and that we define the center of mass of our object so
that it's $y$-position is right at $y=0$ so our object has zero
gravitational potential energy%
\[
U_{g}=mgy=0 
\]%
so the total mechanical energy is given by%
\[
E=K+U_{s} 
\]%
which we can write out as%
\begin{eqnarray*}
E &=&K+U_{s} \\
&=&\frac{1}{2}kx_{\max }^{2}\sin ^{2}\left( \omega t+\phi \right) +\frac{1}{2%
}kx_{\max }^{2}\cos ^{2}\left( \omega t+\phi \right) \\
&=&\frac{1}{2}kx_{\max }^{2}\left( \sin ^{2}\left( \omega t+\phi \right)
+\cos ^{2}\left( \omega t+\phi \right) \right) \\
&=&\frac{1}{2}kx_{\max }^{2}
\end{eqnarray*}

It tells us that if there are no loss mechanisms (e.g. no friction) then the
energy in a harmonic oscillator never changes. And we remember that we would
say that energy is conserved for such a situation, which is not surprising
because we have already used conservation of energy for springs in PH121 and
in the previous example. But let's give a new name to a quantity that does
not change. Let's call it a \emph{constant of motion}. So we can make a
statement about the total energy for our SHO.

\begin{remark}
The total mechanical energy of an ideal SHO is a constant of motion
\end{remark}

If we plot the amount of kinetic and potential energy for an oscillator we
might find something like this:\FRAME{dtbpF}{4.7539in}{3.5699in}{0pt}{}{}{%
Figure}{\special{language "Scientific Word";type
"GRAPHIC";maintain-aspect-ratio TRUE;display "USEDEF";valid_file "T";width
4.7539in;height 3.5699in;depth 0pt;original-width 10.0258in;original-height
7.5247in;cropleft "0";croptop "1";cropright "1";cropbottom "0";tempfilename
'PQXXQWPF.wmf';tempfile-properties "XPR";}}Note that the kinetic and
potential energy are out of phase with each other. If we plot them on the
same scale ( for the case $\phi =0$) we have\FRAME{dtbpF}{3.2655in}{2.2364in%
}{0pt}{}{}{Figure}{\special{language "Scientific Word";type
"GRAPHIC";maintain-aspect-ratio TRUE;display "USEDEF";valid_file "T";width
3.2655in;height 2.2364in;depth 0pt;original-width 3.2206in;original-height
2.1958in;cropleft "0";croptop "1";cropright "1";cropbottom "0";tempfilename
'PQXXQXPG.wmf';tempfile-properties "XPR";}}

Let's try another problem using energy of a SHO.

\section{Mathematical Representation of Simple Harmonic Motion}

We have a mathematical representation of simple harmonic motion from looking
at our graph of position vs. time. But as a problem, let's use math and what
we know from PH121 to show that our equation 
\[
x\left( t\right) =x_{\max }\cos \left( \omega t\right) 
\]%
must be right.

Recall from PH 121 
\begin{equation}
a=\frac{dv}{dt}=\frac{d^{2}x}{dt^{2}}
\end{equation}

and let's assume we have SHO moving only in the $x$-direction. \FRAME{dtbpF}{%
2.2649in}{1.0585in}{0pt}{}{}{Figure}{\special{language "Scientific
Word";type "GRAPHIC";maintain-aspect-ratio TRUE;display "USEDEF";valid_file
"T";width 2.2649in;height 1.0585in;depth 0pt;original-width
2.2252in;original-height 1.0248in;cropleft "0";croptop "1";cropright
"1";cropbottom "0";tempfilename 'PQXXQXPH.wmf';tempfile-properties "XPR";}}%
Further assume the surface the mass rests on is frictionless. Also let's say
that the equilibrium position $x_{em}=0.$ Then we can write Newton's second
law as%
\begin{eqnarray}
F_{net_{x}} &=&ma_{x}=-kx \\
m\frac{d^{2}x}{dt^{2}} &=&-kx  \nonumber
\end{eqnarray}%
We have a new kind of equation. If you are taking this freshman class as
a... well... freshman, you may not have seen this kind of equation before.
It is called a differential equation. The solution of this equation is a
function or functions that will describe the motion of our mass-spring
system as a function of time. It says that the way the object moves is an
equation where the second derivative is almost the same as the original
function. The only difference is some constants that are multiplied.

It is this function that we want, so let's see how we can find it.

Start by getting all the constants on the same side of the equation. 
\[
\frac{d^{2}x}{dt^{2}}=-\frac{k}{m}x 
\]%
It would be tidier if we defined a quantity $\omega $ as%
\begin{equation}
\omega ^{2}=\frac{k}{m}
\end{equation}%
why define $\omega ^{2}?$ Think of our previous example. We found that $%
\omega =\sqrt{k/m}.$ This is just the square of what we found in our
example. Then we can write our differential equation as%
\begin{equation}
\frac{d^{2}x}{dt^{2}}=-\omega ^{2}x
\end{equation}%
We need a function who's second derivative is the negative of itself with
just a constant out front. From Math 112 we know a few of these%
\begin{eqnarray*}
x\left( t\right) &=&A\cos \left( \omega t+\phi _{o}\right) \\
x\left( t\right) &=&A\sin \left( \omega t+\phi _{o}\right)
\end{eqnarray*}%
where $A,$ $\omega ,$ and $\phi _{o}$ are constants that we must find. Let's
choose the cosine function and explicitly take it's derivatives to see if
this function does solve our differential equation%
\begin{eqnarray}
x\left( t\right) &=&A\cos \left( \omega t+\phi _{o}\right) \\
\frac{dx\left( t\right) }{dt} &=&-\omega A\sin \left( \omega t+\phi
_{o}\right)  \nonumber \\
\frac{d^{2}x\left( t\right) }{dt^{2}} &=&-\omega ^{2}A\cos \left( \omega
t+\phi _{o}\right)  \nonumber
\end{eqnarray}%
Let's substitute these expressions into our differential equation for the
motion%
\begin{eqnarray*}
\frac{d^{2}x}{dt^{2}} &=&-\omega ^{2}x \\
-\omega ^{2}A\cos \left( \omega t+\phi _{o}\right) &=&-\omega ^{2}A\cos
\left( \omega t+\phi _{o}\right)
\end{eqnarray*}%
As long as the constant $\omega ^{2}$ is our $\omega ^{2}=k/m$ we have a
solution. We could have found that 
\[
\omega ^{2}=\frac{k}{m} 
\]%
by solving this differential equation, but it might not have been as
meaningful that way. We can identify $\omega $ as the angular frequency.%
\[
\omega =2\pi f 
\]

Thus%
\begin{equation}
\omega =\sqrt{\frac{k}{m}}=2\pi f
\end{equation}%
This says that if the spring is stiffer, we get a higher frequency, or if
the mass is larger, we get a lower frequency.

We still don't have a complete solution, because we don't know $A$ and $\phi
_{o}.$ We recognize $\phi _{o}$ as the initial phase angle. We will have to
find this by knowing the initial conditions of the motion. $A$ is the
amplitude. That must be the maximum displacement $x_{\max }.$ Let's look at
a specific case%
\begin{equation}
\begin{tabular}{|l|}
\hline
$x_{\max }=5$ \\ \hline
$\phi _{o}=0$ \\ \hline
$\omega =2$ \\ \hline
\end{tabular}%
\end{equation}

\FRAME{dtbpFX}{2.7812in}{1.8542in}{0pt}{}{}{Plot}{\special{language
"Scientific Word";type "MAPLEPLOT";width 2.7812in;height 1.8542in;depth
0pt;display "USEDEF";plot_snapshots TRUE;mustRecompute FALSE;lastEngine
"MuPAD";xmin "-5";xmax "5";xviewmin "-5";xviewmax "5";yviewmin
"-10";yviewmax "10";viewset"XY";rangeset"X";plottype 4;labeloverrides
3;x-label "t";y-label "x";axesFont "Times New
Roman,12,0000000000,useDefault,normal";numpoints 100;plotstyle
"patch";axesstyle "normal";axestips FALSE;xis \TEXUX{t};var1name
\TEXUX{$t$};function \TEXUX{$5\cos \left( 2t+0\right) $};linecolor
"blue";linestyle 1;pointstyle "point";linethickness 1;lineAttributes
"Solid";var1range "-5,5";num-x-gridlines 100;curveColor
"[flat::RGB:0x000000ff]";curveStyle "Line";VCamFile
'PQXXR23A.xvz';valid_file "T";tempfilename
'PQXXQXPI.wmf';tempfile-properties "XPR";}}We can easily see that the
amplitude $A$ corresponds to the maximum displacement $x_{\max }.$ (how
would you prove this?). We know from trigonometry that a cosine function has
a period $T.$

\FRAME{fhF}{2.9101in}{1.9484in}{0pt}{}{}{Figure}{\special{language
"Scientific Word";type "GRAPHIC";maintain-aspect-ratio TRUE;display
"USEDEF";valid_file "T";width 2.9101in;height 1.9484in;depth
0pt;original-width 4.4616in;original-height 2.9793in;cropleft "0";croptop
"1";cropright "1";cropbottom "0";tempfilename
'PQXXQXPJ.wmf';tempfile-properties "XPR";}}The period is related to the
frequency%
\begin{equation}
T=\frac{1}{f}=\frac{2\pi }{\omega }
\end{equation}%
We can write the period and frequency in terms of our mass and spring
constant%
\begin{eqnarray}
T &=&2\pi \sqrt{\frac{m}{k}} \\
f &=&\frac{1}{2\pi }\sqrt{\frac{k}{m}}
\end{eqnarray}

\subsection{Hanging Springs}

Let's do a problem. We now know we have forces involved with our SHOs. So
let's do a force problem. In class so far, I have always hung our masses and
springs, but we have used horizontal systems for calculations. Let's find
the equation of motion for a mass hanging from a spring.

\FRAME{dtbpF}{3.429in}{1.9977in}{0in}{}{}{Figure}{\special{language
"Scientific Word";type "GRAPHIC";maintain-aspect-ratio TRUE;display
"USEDEF";valid_file "T";width 3.429in;height 1.9977in;depth
0in;original-width 3.3831in;original-height 1.9579in;cropleft "0";croptop
"1";cropright "1";cropbottom "0";tempfilename
'PQXXQXPK.wmf';tempfile-properties "XPR";}}

From observation, we would guess that we can just choose the new equilibrium
point to be $y=0$ and use the same equation%
\[
y\left( t\right) =y_{\max }\cos (\omega t-\phi _{o}) 
\]%
let's see if that is true. Start with a free body diagram for the mass (the
hanging mass is our mover, the spring is part of the environment). There are
two forces acting on the mass. The force due to gravity, and the force due
to the spring.%
\[
\Sigma F_{y}=ma_{y}=S_{ms}-W_{mE} 
\]%
We know the form of these forces%
\begin{eqnarray*}
S_{ms} &=&-k\Delta y \\
W_{mE} &=&-mg
\end{eqnarray*}%
but we need to carefully choose our origin. Let's try the top of the spring
where it attaches to the stand. If the mass just hangs there we would expect
the spring to stretch to an equilibrium length $y_{eq}$ and any other motion
would either shorten or lengthen the spring. We can write our force equation
as 
\begin{eqnarray*}
ma_{y} &=&S_{ms}-W_{mE} \\
&=&k\left( y_{em}-y\right) -mg
\end{eqnarray*}%
where $y$ can be positive or negative. If $y=0$ and the mass is just sitting
there, not oscillating, there is no acceleration. Then the stretched length
is just $y_{em}.$ In this case%
\begin{eqnarray*}
m\left( 0\right) &=&k\left( y_{em}-0\right) -mg \\
ky_{em} &=&mg
\end{eqnarray*}%
But now let's let our mass move again. We can substitute our stationary mass
answer this into the previous equation for a moving mass%
\begin{eqnarray*}
ma &=&k\left( y_{em}-y\right) -mg \\
&=&ky_{em}-ky-mg \\
&=&mg-ky-mg \\
&=&-ky
\end{eqnarray*}%
This gives a net force for the system of 
\[
F_{net}=-ky 
\]%
It is as though the system were horizontal with no gravitational force and
only a spring force. We can see that we are justified in claiming that we
could simply choose the origin at the distance $y_{em}$ from the top of the
spring, and we can use the equation 
\[
y\left( t\right) =y_{\max }\cos (\omega t-\phi _{o}) 
\]%
as our equation of motion.

We have used energy to describe simple harmonic motion, and we have found
our equation of motion using a differential equation for mass-spring
systems. And we have simple harmonic motion in the $y$-direction including
the weight force due to gravity for mass-spring systems. In our next
lecture, we will study simple harmonic motion for different systems,
pendula, and other things. It turns out that SHM is a good model for many
different systems.

\chapter{More Oscillators, Forces and Friction}

You have probably wondered if anyone actually uses mass-spring systems. And
we do. They were more common in the past where springs were used to store
energy ($U_{s\text{ }}$) to run clocks and toys and machines. But simple
harmonic motion can describe other systems as well. You have probably seen
an old fashioned clock with a pendulum. A pendulum almost experiences simple
harmonic motion. Let's see how this works.

%TCIMACRO{%
%\TeXButton{Fundamental Concepts}{\hspace{-1.3in}{\Large Fundamental Concepts\vspace{0.25in}}}}%
%BeginExpansion
\hspace{-1.3in}{\Large Fundamental Concepts\vspace{0.25in}}%
%EndExpansion

\begin{itemize}
\item Pendula

\item Damping

\item Driving

\item Resonance
\end{itemize}

\section{The Simple Pendulum}

\FRAME{dtbpF}{1.7305in}{1.6737in}{0pt}{}{}{Figure}{\special{language
"Scientific Word";type "GRAPHIC";maintain-aspect-ratio TRUE;display
"USEDEF";valid_file "T";width 1.7305in;height 1.6737in;depth
0pt;original-width 1.7349in;original-height 1.6773in;cropleft "0";croptop
"1";cropright "1";cropbottom "0";tempfilename
'PQXXQXPL.wmf';tempfile-properties "XPR";}}A simple pendulum is a mass on a
string. The mass is called a \textquotedblleft bob.\textquotedblright\
Usually we study the motion of the pendulum bob, so let's consider the
pendulum the mover object. A simple pendulum bob exhibits periodic motion,
but not exactly simple harmonic motion. The forces on the bob are $%
\overrightarrow{\mathbf{W}}$, and $\overrightarrow{\mathbf{T}}$ the tension
on the string. The tangential component of $W$ is always directed toward $%
\theta =0.$ This is a restoring force!

Let's call the path the bob takes \textquotedblleft $s.$\textquotedblright\
Then from Jr. High geometry we recall\footnote{%
Really I did not recall this, I\ have to look it up every time, but $s$ is
called the \emph{arclength}.}%
\begin{equation}
s=L\theta
\end{equation}%
We will use a the cylindrical or $rtz$ coordinate system. The radial axis is
directed along the string. The tangential direction is along the circular
path the bob takes and is always tangent to the path. In this coordinate
system, we can solve for the part of the force directed along the path. This
is the restoring part of the net force. Remember from Newton's second law 
\begin{eqnarray*}
F_{t} &=&ma_{t} \\
F_{r} &=&ma_{r}
\end{eqnarray*}%
and 
\[
a_{t}=\frac{d^{2}s}{dt^{2}} 
\]%
Let's write out Newton's second with the sum of the forces part.%
\begin{eqnarray*}
ma_{t} &=&-W\cos \left( 90-\theta \right) \\
-ma_{r} &=&-T+W\sin \left( 90-\theta \right)
\end{eqnarray*}%
then, using a trig identity (but only a small one) 
\begin{eqnarray*}
a_{t} &=&-\frac{W}{m}\sin \left( \theta \right) \\
&=&-g\sin \theta
\end{eqnarray*}

We have two expressions for $a_{t.}$ We can set them equal 
\begin{equation}
\frac{d^{2}s}{dt^{2}}=-g\sin \theta  \label{pendulum1}
\end{equation}%
Remember that $s=L\theta $ so we could write the left hand side as%
\[
\frac{d^{2}s}{dt^{2}}=\frac{d^{2}}{dt^{2}}\left( L\theta \right) =L\frac{%
d^{2}}{dt^{2}}\left( \theta \right) 
\]%
then equation (\ref{pendulum1}) becomes%
\[
L\frac{d^{2}}{dt^{2}}\left( \theta \right) =-g\sin \theta 
\]%
or 
\[
\frac{d^{2}\theta }{dt^{2}}=-\frac{g}{L}\sin \theta 
\]%
This is a differential equation much like our differential equation for a
harmonic oscillator, 
\[
\frac{d^{2}x}{dt^{2}}=-\omega ^{2}x 
\]%
except it has a sine function in it. But, if we take $\theta $ as a very
small angle, then 
\begin{equation}
\sin \left( \theta \right) \approx \theta  \label{small angle approximation}
\end{equation}%
This approximation has a name, it is called the \textquotedblleft small
angle approximation.\textquotedblright

In this approximation%
\[
\frac{d^{2}\theta }{dt^{2}}=-\frac{g}{L}\theta 
\]%
and we have a differential equation we recognize! If we compare to 
\[
\frac{d^{2}x}{dt^{2}}=-\omega ^{2}x 
\]

we see that it is a match if 
\begin{equation}
\omega ^{2}=\frac{g}{L}
\end{equation}%
we have all the same solutions for $\theta $ that we found last time for $x.$
Since $\omega $ changed, the frequency and period will now be in terms of $g$
and $L.$%
\begin{equation}
T=\frac{2\pi }{\omega }=2\pi \sqrt{\frac{L}{g}}
\end{equation}

\begin{remark}
\textbf{For a pendulum that oscillates only over small angles, the period
and frequency depend only on }$L$ \textbf{and }$g$!
\end{remark}

The analysis we just did for the pendulum we can do for other simple
harmonic oscillators (or near simple harmonic oscillators). Let's try a few.

\subsection{Physical Pendulum}

Usually when we build a pendulum, we assume the string is so small compared
to the bob, that we can ignore it's mass. What if this is not true? Suppose
we build a pendulum by making a large solid object swing from one point. Can
we describe it's motion?

\FRAME{dhF}{1.2998in}{1.2842in}{0pt}{}{}{Figure}{\special{language
"Scientific Word";type "GRAPHIC";maintain-aspect-ratio TRUE;display
"USEDEF";valid_file "T";width 1.2998in;height 1.2842in;depth
0pt;original-width 1.2644in;original-height 1.2488in;cropleft "0";croptop
"1";cropright "1";cropbottom "0";tempfilename
'PQXXQXPM.wmf';tempfile-properties "XPR";}}Let's pull it to the right\FRAME{%
dhF}{1.6769in}{2.0652in}{0pt}{}{}{Figure}{\special{language "Scientific
Word";type "GRAPHIC";maintain-aspect-ratio TRUE;display "USEDEF";valid_file
"T";width 1.6769in;height 2.0652in;depth 0pt;original-width
1.6405in;original-height 2.0263in;cropleft "0";croptop "1";cropright
"1";cropbottom "0";tempfilename 'PQXXQXPN.wmf';tempfile-properties "XPR";}}%
then we consider that because of $F_{g}$ we will have a torque about an axis
through $O.$ 
\begin{eqnarray*}
\tau &=&\mathbf{r\times F} \\
&=&\mathbf{d}\times \mathbf{F}_{g} \\
&=&-F_{g}d\sin \theta \\
&=&-mgd\sin \theta
\end{eqnarray*}%
Remember that an extended body has a moment of inertia, $\mathbb{I}.$
Remember also that angular acceleration is given by

\[
\alpha =\frac{dw}{dt}=\frac{d^{2}\theta }{dt^{2}} 
\]

From the rotational form of Newton's second law.%
\[
\Sigma \tau =\mathbb{I}\alpha 
\]

we can write%
\[
-mgd\sin \theta =\mathbb{I}\frac{d^{2}\theta }{dt^{2}} 
\]%
\[
\frac{d^{2}\theta }{dt^{2}}=-\frac{mgd}{\mathbb{I}}\sin \left( \theta
\right) 
\]%
and again if we let $\theta $ be small so that $\sin \left( \theta \right)
\approx \theta $%
\[
\frac{d^{2}\theta }{dt^{2}}=-\frac{mgd}{\mathbb{I}}\theta 
\]%
which we can compare to 
\[
\frac{d^{2}x}{dt^{2}}=-\omega ^{2}x 
\]%
and we can see that we have the same differential equation if 
\begin{equation}
\omega ^{2}=\frac{mgd}{\mathbb{I}}
\end{equation}

In this case%
\begin{equation}
T=2\pi \sqrt{\frac{\mathbb{I}}{mgd}}
\end{equation}

\subsection{Torsional Pendulum}

Physics majors and those taking PH220 in the future, you will see a
torsional pendulum.

\FRAME{dhF}{1.1597in}{1.7443in}{0pt}{}{}{Figure}{\special{language
"Scientific Word";type "GRAPHIC";maintain-aspect-ratio TRUE;display
"USEDEF";valid_file "T";width 1.1597in;height 1.7443in;depth
0pt;original-width 1.126in;original-height 1.7071in;cropleft "0";croptop
"1";cropright "1";cropbottom "0";tempfilename
'PQXXQXPO.wmf';tempfile-properties "XPR";}}A torsional pendulum is made by
suspending a rigid body from a wire. The body rotates.\FRAME{dhF}{4.5844in}{%
1.8273in}{0pt}{}{}{Figure}{\special{language "Scientific Word";type
"GRAPHIC";maintain-aspect-ratio TRUE;display "USEDEF";valid_file "T";width
4.5844in;height 1.8273in;depth 0pt;original-width 4.5325in;original-height
1.7902in;cropleft "0";croptop "1";cropright "1";cropbottom "0";tempfilename
'PQXXQXPP.wmf';tempfile-properties "XPR";}}The twisted wire exerts a
restoring torque on the body that is proportional to the angular position
(sound familiar)%
\[
\mathbf{\tau }=-\kappa \theta 
\]%
This looks like a greek version of $F=-kx$! Again%
\begin{eqnarray*}
\Sigma \tau &=&\mathbb{I}\alpha \\
&=&\mathbb{I}\frac{d^{2}\theta }{dt^{2}}
\end{eqnarray*}%
so%
\begin{eqnarray*}
\mathbb{I}\frac{d^{2}\theta }{dt^{2}} &=&-\kappa \theta \\
\frac{d^{2}\theta }{dt^{2}} &=&-\frac{\kappa }{\mathbb{I}}\theta
\end{eqnarray*}%
Once again we have our favorite differential equation so long as 
\begin{equation}
\omega ^{2}=\frac{\kappa }{\mathbb{I}}
\end{equation}%
which makes the period of the oscillation%
\begin{equation}
T=2\pi \sqrt{\frac{\mathbb{I}}{\kappa }}
\end{equation}

\section{Damped Oscillations}

You remember friction from PH121. So far we have only allowed frictionless
oscillators to make the math easy. But what if we do have friction? To
investigate this,suppose we add in another force%
\begin{equation}
\mathbf{F}_{d}=-b\mathbf{v}  \label{Damping force}
\end{equation}%
This force is proportional to the velocity. This a dissipative
(friction-like) force typical of what we find when we moves objects through
viscus fluids. \FRAME{dtbpF}{2.127in}{1.6383in}{0pt}{}{}{Figure}{\special%
{language "Scientific Word";type "GRAPHIC";maintain-aspect-ratio
TRUE;display "USEDEF";valid_file "T";width 2.127in;height 1.6383in;depth
0pt;original-width 2.1385in;original-height 1.6409in;cropleft "0";croptop
"1";cropright "1";cropbottom "0";tempfilename
'PQXXQXPQ.wmf';tempfile-properties "XPR";}}We call $b$ the damping
coefficient. Now, from Newton's second law,%
\[
\Sigma F=-kx-bv_{x}=ma 
\]%
We can write the acceleration and velocity as derivatives of the position
just like we have done before 
\[
-kx-b\frac{dx}{dt}=m\frac{d^{2}x}{dt^{2}} 
\]%
This is another differential equation! But it is harder to guess its
solution, and finding that solution is a subject for a differential
equations class like M316 or PH332, so we won't learn how to find the
solution here, but we can use the results from our trusted colleagues in the
math department\footnote{%
That is, until you finish M316, then you will know how to do this kind of
problem yourself!}.%
\begin{equation}
x\left( t\right) =x_{\max }e^{-\frac{b}{2m}t}\cos \left( \omega t+\phi
_{o}\right)  \label{dampped solution}
\end{equation}%
which looks simple enough, but now we $\ $have the added complication that $%
\omega $ is more complex 
\begin{equation}
\omega =\sqrt{\frac{k}{m}-\left( \frac{b}{2m}\right) ^{2}}
\end{equation}%
so we get quite a mess if we write equation (\ref{dampped solution}) with
this new $\omega .$%
\begin{equation}
x\left( t\right) =x_{\max }e^{-\frac{b}{2m}t}\cos \left( \left( \sqrt{\frac{k%
}{m}-\left( \frac{b}{2m}\right) ^{2}}\right) t+\phi _{o}\right)
\end{equation}

To see what this solution means, we should study three cases:

\begin{enumerate}
\item the damping force is small: ($bv_{\max }<kx_{\max })$ The system
oscillates, but the amplitude is smaller as as time goes on. We call this
\textquotedblleft underdamped.\textquotedblright

\item the damping force is large: ($bv_{\max }>kx_{\max })$The system does
not oscillate. we call this \textquotedblleft overdamped.\textquotedblright\
We can also say that $\frac{b}{2m}>\omega _{o}$ (after we define $\omega
_{o} $ below)

\item The system is \textquotedblleft critically damped\textquotedblright\
(see below).
\end{enumerate}

Let's look at an example. Suppose we have an oscillator with the following
characteristics:

\begin{enumerate}
\item 
\[
\begin{tabular}{|l|}
\hline
$x_{\max }=5\unit{cm}$ \\ \hline
$b=0.005\frac{\unit{kg}}{\unit{s}}$ \\ \hline
$k=0.2\frac{\unit{N}}{\unit{m}}$ \\ \hline
$m=.5\unit{kg}$ \\ \hline
$\phi _{o}=0$ \\ \hline
\end{tabular}%
\]

Graphing the equation of motion $x\left( t\right) ,$ we we get a graph that
looks like this\FRAME{dtbpFX}{4.4996in}{3in}{0pt}{}{}{Plot}{\special%
{language "Scientific Word";type "MAPLEPLOT";width 4.4996in;height 3in;depth
0pt;display "USEDEF";plot_snapshots TRUE;mustRecompute FALSE;lastEngine
"MuPAD";xmin "0";xmax "500";xviewmin "0";xviewmax "500";yviewmin
"-0.050771";yviewmax "0.05";viewset"XY";rangeset"X";plottype
4;labeloverrides 3;x-label "t(sec)";y-label "y(m)";axesFont "Times New
Roman,12,0000000000,useDefault,normal";numpoints 100;plotstyle
"patch";axesstyle "normal";axestips FALSE;xis \TEXUX{t};var1name
\TEXUX{$t$};function \TEXUX{$\left( 0.05\right) e^{-\frac{0.005}{2\left(
0.5\right) }t}$};linecolor "gray";linestyle 1;pointstyle
"point";linethickness 1;lineAttributes "Solid";var1range
"0,500";num-x-gridlines 100;curveColor "[flat::RGB:0x00c0c0c0]";curveStyle
"Line";rangeset"X";function \TEXUX{$-\left( 0.05\right)
e^{-\frac{0.005}{2\left( 0.5\right) }t}$};linecolor "yellow";linestyle
1;pointstyle "point";linethickness 1;lineAttributes "Solid";var1range
"0,500";num-x-gridlines 100;curveColor "[flat::RGB:0x00808000]";curveStyle
"Line";function \TEXUX{$\left( 0.05\right) e^{-\frac{0.005}{2\left(
0.5\right) }t}\cos \left( \left( \left( \frac{0.2}{0.5}-\left(
\frac{0.005}{2\left( 0.05\right) }\right) ^{2}\right) ^{\frac{1}{2}}\right)
t\right) $};linecolor "blue";linestyle 1;pointstyle "point";linethickness
1;lineAttributes "Solid";var1range "0,500";num-x-gridlines 900;curveColor
"[flat::RGB:0x000000ff]";curveStyle "Line";rangeset"X";VCamFile
'PQXXR239.xvz';valid_file "T";tempfilename
'PQXXQXPR.wmf';tempfile-properties "XPR";}}The gray lines are given by%
\[
\pm x_{\max }e^{-\frac{b}{2m}t} 
\]%
Notice that this quantity is as a collection of terms multiplies the cosine
part. It is marked in curly braces blow 
\[
x\left( t\right) =\left\{ x_{\max }e^{-\frac{b}{2m}t}\right\} \cos \left(
\left( \sqrt{\frac{k}{m}-\left( \frac{b}{2m}\right) ^{2}}\right) t+\phi
_{o}\right) 
\]%
We know that the part of the equation that multiplies the cosine part is the
amplitude. But now the amplitude is more than just $x_{\max }.$ And notice
that the amplitude $\left\{ x_{\max }e^{-\frac{b}{2m}t}\right\} $ changes
with time. The gray lines in the figure show how the amplitude changes. We
call this the \emph{envelope} of the curve. The oscillation fits within the
gray lines (like an old fashioned letter fits inside a paper envelope).
\end{enumerate}

Now let's change $b$ to a larger value%
\[
\begin{tabular}{|l|}
\hline
$x_{\max }=5\unit{cm}$ \\ \hline
$b=0.05\frac{\unit{kg}}{\unit{s}}$ \\ \hline
$k=0.2\frac{\unit{N}}{\unit{m}}$ \\ \hline
$m=.5\unit{kg}$ \\ \hline
$\phi _{o}=0$ \\ \hline
\end{tabular}%
\]%
\FRAME{dtbpFX}{4.4996in}{3in}{0pt}{}{}{Plot}{\special{language "Scientific
Word";type "MAPLEPLOT";width 4.4996in;height 3in;depth 0pt;display
"USEDEF";plot_snapshots TRUE;mustRecompute FALSE;lastEngine "MuPAD";xmin
"0";xmax "500";xviewmin "0";xviewmax "500";yviewmin "-0.050771";yviewmax
"0.05";viewset"XY";rangeset"X";plottype 4;labeloverrides 1;x-label
"t(sec)";axesFont "Times New
Roman,12,0000000000,useDefault,normal";numpoints 100;plotstyle
"patch";axesstyle "normal";axestips FALSE;xis \TEXUX{t};var1name
\TEXUX{$t$};function \TEXUX{$\left( 0.05\right) e^{-\frac{0.05}{2\left(
0.5\right) }t}$};linecolor "gray";linestyle 1;pointstyle
"point";linethickness 1;lineAttributes "Solid";var1range
"0,500";num-x-gridlines 100;curveColor "[flat::RGB:0x00c0c0c0]";curveStyle
"Line";rangeset"X";function \TEXUX{$-\left( 0.05\right)
e^{-\frac{0.05}{2\left( 0.5\right) }t}$};linecolor "yellow";linestyle
1;pointstyle "point";linethickness 1;lineAttributes "Solid";var1range
"0,500";num-x-gridlines 100;curveColor "[flat::RGB:0x00808000]";curveStyle
"Line";function \TEXUX{$\left( 0.05\right) e^{-\frac{0.05}{2\left(
0.5\right) }t}\cos \left( \left( \left( \frac{0.2}{0.5}-\left(
\frac{0.05}{2\left( 0.05\right) }\right) ^{2}\right) ^{\frac{1}{2}}\right)
t\right) $};linecolor "blue";linestyle 1;pointstyle "point";linethickness
1;lineAttributes "Solid";var1range "0,500";num-x-gridlines 900;curveColor
"[flat::RGB:0x000000ff]";curveStyle "Line";rangeset"X";VCamFile
'PQXXR238.xvz';valid_file "T";tempfilename
'PQXXQXPS.wmf';tempfile-properties "XPR";}}

we see we have less oscillation. The envelope has become more restrictive,
making the oscillation die out more quickly. This is a little bit like going
over a bump in your car. The car may go up and down a few times, but not
many. Now let's increase $b$ even more.%
\begin{equation}
\begin{tabular}{|l|}
\hline
$x_{\max }=5\unit{cm}$ \\ \hline
$b=0.5\frac{\unit{kg}}{\unit{s}}$ \\ \hline
$k=0.2\frac{\unit{N}}{\unit{m}}$ \\ \hline
$m=.5\unit{kg}$ \\ \hline
$\phi _{o}=0$ \\ \hline
\end{tabular}%
\end{equation}%
\FRAME{dtbpFX}{4.4996in}{3in}{0pt}{}{}{Plot}{\special{language "Scientific
Word";type "MAPLEPLOT";width 4.4996in;height 3in;depth 0pt;display
"USEDEF";plot_snapshots TRUE;mustRecompute FALSE;lastEngine "MuPAD";xmin
"0";xmax "50";xviewmin "0";xviewmax "50";yviewmin "-0.050005";yviewmax
"0.05";viewset"XY";rangeset"X";plottype 4;labeloverrides 3;x-label
"t(sec)";y-label "y(m)";axesFont "Times New
Roman,12,0000000000,useDefault,normal";numpoints 100;plotstyle
"patch";axesstyle "normal";axestips FALSE;xis \TEXUX{t};var1name
\TEXUX{$t$};function \TEXUX{$\left( 0.05\right) e^{-\frac{1}{2\left(
0.5\right) }t}$};linecolor "gray";linestyle 1;pointstyle
"point";linethickness 1;lineAttributes "Solid";var1range
"0,50";num-x-gridlines 100;curveColor "[flat::RGB:0x00c0c0c0]";curveStyle
"Line";rangeset"X";function \TEXUX{$-\left( 0.05\right) e^{-\frac{1}{2\left(
0.5\right) }t}$};linecolor "yellow";linestyle 1;pointstyle
"point";linethickness 1;lineAttributes "Solid";var1range
"0,50";num-x-gridlines 100;curveColor "[flat::RGB:0x00808000]";curveStyle
"Line";VCamFile 'PQXXR237.xvz';valid_file "T";tempfilename
'PQXXQXPT.wmf';tempfile-properties "XPR";}}What happened?

When the damping force gets bigger, the oscillation eventually stops. Only
the exponential decay is observed. This happens when%
\[
\frac{b}{2m}=\sqrt{\frac{k}{m}} 
\]%
When that is true, 
\[
\omega =\sqrt{\frac{k}{m}-\left( \frac{b}{2m}\right) ^{2}}=0 
\]%
We call this situation \emph{critically damped}. We are just on the edge of
oscillation. We define%
\[
\omega _{o}=\sqrt{\frac{k}{m}} 
\]%
as the \emph{natural frequency} of the system. Then the value of $b$ that
gives us critically damped behavior is%
\[
b_{c}=2m\omega _{o} 
\]%
When $\frac{b}{2\pi }\geq \omega _{o}$ the solution in equation (\ref%
{dampped solution}) is not valid! You will find out more about this
situation in your advanced mechanics classes.

\section{Forced Oscillations}

We found in the last section that if we added a force like

\[
\mathbf{F}_{d}=-b\mathbf{v} 
\]%
our oscillation died out. An example would be a small child on a swing. You
give them a push, but eventually they stop swinging.

\begin{enumerate}
\item Suppose we want to keep the child going? Let's apply a periodic force
like%
\[
F\left( t\right) =F_{o}\sin \left( \omega _{f}t\right) 
\]%
where $\omega _{f}$ is the angular frequency of this new driving force and
where $F_{o}$ is a constant.%
\[
\Sigma F=F_{o}\sin \left( \omega _{f}t\right) -kx-bv_{x}=ma 
\]

When this system starts out, the solution is very messy. It is so messy that
we will not give it in this class! But after a while, a steady-state is
reached. In this state, the energy added by our driving force $F_{o}\sin
\left( \omega _{f}t\right) $ is equal to the energy lost by the drag force,
and we have 
\[
x\left( t\right) =A\cos \left( \omega _{f}t+\phi \right) 
\]%
our old friend! BUT NOW the amplitude is given by 
\[
A=\frac{\frac{F_{o}}{m}}{\sqrt{\left( \omega _{f}^{2}-\omega _{o}^{2}\right)
^{2}+\left( \frac{b\omega _{f}}{m}\right) ^{2}}} 
\]%
where%
\begin{equation}
\omega _{o}=\sqrt{\frac{k}{m}}
\end{equation}%
is the natural frequency as before. So now our solution looks more like our
original SHM solution (except for the wild formula for $A$). In fact, it
operates very like our SHM solution. But what does the new version of $A$
mean?

Lets look at $A$ for some values of $\omega _{f}.$ I will pick some nice
numbers for the other values.%
\[
\begin{tabular}{|l|}
\hline
$F_{o}=2\unit{N}$ \\ \hline
$b=0.5\frac{\unit{kg}}{\unit{s}}$ \\ \hline
$k=0.5\frac{\unit{N}}{\unit{m}}$ \\ \hline
$m=0.5\unit{kg}$ \\ \hline
$\phi _{o}=0$ \\ \hline
\end{tabular}%
\]%
\FRAME{dtbpFX}{4.4996in}{3in}{0pt}{}{}{Plot}{\special{language "Scientific
Word";type "MAPLEPLOT";width 4.4996in;height 3in;depth 0pt;display
"USEDEF";plot_snapshots TRUE;mustRecompute FALSE;lastEngine "MuPAD";xmin
"0";xmax "3";xviewmin "0";xviewmax "3";yviewmin "0";yviewmax
"250.7930";viewset"XY";rangeset"X";plottype 4;labeloverrides 3;x-label
"omega_f (rad/sec)";y-label "y(m)";axesFont "Times New
Roman,12,0000000000,useDefault,normal";numpoints 100;plotstyle
"patch";axesstyle "normal";axestips FALSE;xis \TEXUX{v58147};var1name
\TEXUX{$\omega $};function \TEXUX{$\allowbreak
\frac{20.0}{\sqrt{0.0001\omega ^{2}+\left( \omega ^{2}-1.0\right)
^{2}}}$};linecolor "green";linestyle 1;pointstyle "point";linethickness
1;lineAttributes "Solid";var1range "0,3";num-x-gridlines 100;curveColor
"[flat::RGB:0x00008000]";curveStyle "Line";VCamFile
'PQXXR236.xvz';valid_file "T";tempfilename
'PQXXQXPU.wmf';tempfile-properties "XPR";}}now let's calculate $\omega _{o}$ 
\begin{eqnarray*}
\omega _{o} &=&\sqrt{\frac{0.5\frac{\unit{N}}{\unit{m}}}{0.5\unit{kg}}} \\
&=&\allowbreak \frac{1.0}{\unit{s}}
\end{eqnarray*}%
Notice that right at $\omega _{f}=\omega _{o}$ our solution gets very big.
This is called \emph{resonance}. To see why this happens, think of the
velocity for a simple harmonic oscillator%
\[
\frac{dx\left( t\right) }{dt}=-\omega A\sin \left( \omega t+\phi _{o}\right) 
\]%
note that our driving force is%
\[
F\left( t\right) =F_{o}\sin \left( \omega t\right) 
\]%
Remember that work is given by%
\[
w=\int \overrightarrow{\mathbf{F}}\cdot d\overrightarrow{\mathbf{x}} 
\]%
or just 
\[
w=\overrightarrow{\mathbf{F}}\cdot \Delta \overrightarrow{\mathbf{x}} 
\]%
if our force is constant. The rate at which work is done (power) is 
\begin{eqnarray}
\mathcal{P} &=&\frac{\overrightarrow{\mathbf{F}}\cdot \Delta \overrightarrow{%
\mathbf{x}}}{\Delta t}=\overrightarrow{\mathbf{F}}\cdot \overrightarrow{%
\mathbf{v}} \\
&=&-F_{o}\omega A\sin \left( \omega t\right) \sin \left( \omega t+\phi
_{o}\right)
\end{eqnarray}%
if $F$ and $v$ are in phase $(\phi _{o}=0)$, the power will be at a maximum!
Think of pushing the child on the swing. You push in phase with the
oscillation of the child. When you do this the amplitude of the swing get's
bigger.

We can plot $A$ for several values of $b$\FRAME{dtbpFUX}{4.4996in}{3in}{0pt}{%
\Qcb{Green: b=0.005kg/s; Blue: b=0.05kg/s; Red b=0.01 kg/s}}{}{Plot}{\special%
{language "Scientific Word";type "MAPLEPLOT";width 4.4996in;height 3in;depth
0pt;display "USEDEF";plot_snapshots TRUE;mustRecompute FALSE;lastEngine
"MuPAD";xmin "0";xmax "3";xviewmin "0";xviewmax "3";yviewmin "0";yviewmax
"250.7930";viewset"XY";rangeset"X";plottype 4;labeloverrides 3;x-label
"omega(rad/sec)";y-label "y(m)";axesFont "Times New
Roman,12,0000000000,useDefault,normal";numpoints 100;plotstyle
"patch";axesstyle "normal";axestips FALSE;xis \TEXUX{v58147};var1name
\TEXUX{$\omega $};function \TEXUX{$\allowbreak
\frac{20.0}{\sqrt{0.0001\omega ^{2}+\left( \omega ^{2}-1.0\right)
^{2}}}$};linecolor "green";linestyle 1;pointstyle "point";linethickness
1;lineAttributes "Solid";var1range "0,3";num-x-gridlines 100;curveColor
"[flat::RGB:0x00008000]";curveStyle "Line";function
\TEXUX{$\frac{20.0}{\sqrt{1\omega ^{2}+\left( \omega ^{2}-1.0\right)
^{2}}}$};linecolor "blue";linestyle 1;pointstyle "point";linethickness
1;lineAttributes "Solid";var1range "0,3";num-x-gridlines 100;curveColor
"[flat::RGB:0x000000ff]";curveStyle "Line";function
\TEXUX{$\frac{20.0}{\sqrt{0.01\omega ^{2}+\left( \omega ^{2}-1.0\right)
^{2}}}$};linecolor "red";linestyle 1;pointstyle "point";linethickness
1;lineAttributes "Solid";var1range "0,3";num-x-gridlines 100;curveColor
"[flat::RGB:0x00ff0000]";curveStyle "Line";VCamFile
'PQXXR235.xvz';valid_file "T";tempfilename
'PQXXQXPV.wmf';tempfile-properties "XPR";}}As $b\rightarrow 0$ we see that
our resonance peak gets larger. In real systems $b$ can never be zero, but
sometimes it can get small. As $b\rightarrow $large, the resonance dies down
and our $A$ gets small.

Resonance can be great, it can make a musical instrument sound louder (more
on this later). But it can also be bad. Here is a picture of the Tacoma
Narrows Bridge.\FRAME{dhFU}{2.9528in}{2.2316in}{0pt}{\Qcb{Fall of the Tacoma
Narrows Bridge (Image in the Public Domain)}}{}{Figure}{\special{language
"Scientific Word";type "GRAPHIC";maintain-aspect-ratio TRUE;display
"USEDEF";valid_file "T";width 2.9528in;height 2.2316in;depth
0pt;original-width 4.1252in;original-height 3.1107in;cropleft "0";croptop
"1";cropright "1";cropbottom "0";tempfilename
'PQXXQXPW.bmp';tempfile-properties "XPR";}}The wind gusts formed a periodic
driving force that allowed a driving harmonic oscillation to form. Since the
bridge was resonant with the gust frequency, the amplitude grew until the
bridge materials broke. Resonance can be a bad thing for structures.
\end{enumerate}

\chapter{What is a Wave?}

We studied oscillation with mass-spring systems and pendula. Oscillation is
a motion of a mass, but with another object (a spring or string)
participating in the motion. The air in our classroom is an example of this
type of motion in a way, but instead of one object on a spring, there are
millions of objects, the air molecules. The molecules move randomly, but
with a specific distribution of speeds. But what would happen if all the
molecules moved together in the same direction (more or less) and at the
same speed (more or less)? This is what we call wind! Of course we can have
bulk motion of millions of objects, like wind. We are going to study an even
more specific motion of millions of object that is not random like thermal
motion. This specific motion of the objects we will call a \emph{wave.}

%TCIMACRO{%
%\TeXButton{Fundamental Concepts}{\hspace{-1.3in}{\LARGE Fundamental Concepts\vspace{0.25in}}}}%
%BeginExpansion
\hspace{-1.3in}{\LARGE Fundamental Concepts\vspace{0.25in}}%
%EndExpansion

\begin{enumerate}
\item A wave requires a disturbance, and a medium that can transfer energy

\item Waves are categorized as longitudinal or transverse (or a combination
of the two).
\end{enumerate}

\section{What is a Wave?}

Waves are organized motions in a group of objects. We will give a name to
the group of objects, we will call a \emph{medium}.%
%TCIMACRO{%
%\TeXButton{Spring Demo}{\marginpar {
%\hspace{-0.5in}
%\begin{minipage}[t]{1in}
%\small{Spring Demo}
%\end{minipage}
%}}}%
%BeginExpansion
\marginpar {
\hspace{-0.5in}
\begin{minipage}[t]{1in}
\small{Spring Demo}
\end{minipage}
}%
%EndExpansion

\subsection{Criteria for being a wave}

Waves involve energy transfer, but in the case of waves the energy is
transferred through space without transfer of matter. Winds transfers both
energy and matter. So waves are different than wind. In a wave the objects
don't move far from where they start. Think of an oscillation. The mass does
move, but never gets too far away from its equilibrium position. Waves are
very like this. This is a very specific kind of motion. To be a wave, the
motion must have the following characteristics"

%TCIMACRO{%
%\TeXButton{Spring Demo-marked part}{\marginpar {
%\hspace{-0.5in}
%\begin{minipage}[t]{1in}
%\small{Spring Demo-marked part}
%\end{minipage}
%}}}%
%BeginExpansion
\marginpar {
\hspace{-0.5in}
\begin{minipage}[t]{1in}
\small{Spring Demo-marked part}
\end{minipage}
}%
%EndExpansion

\begin{enumerate}
\item some source of disturbance

\item a medium (group of objects) that can be disturbed

\item some physical mechanism by which the objects of the medium can
influence each other
\end{enumerate}

A wave made by a rock thrown into a pond will go out in all directions away
from the place where the rock started the wave. This is a normal way that
waves are formed. A disturbance starts the wave (the rock disturbs the
water) and the energy from the disturbance moves away from the disturbance
as a wave. But if we have a wave in a rope or string, the wave can't go in
all directions because the string does not go in all directions.

Let's take on this one-dimensional case of a wave on a rope or string first.
In the limit that the string mass is negligible we represent a
one-dimensional wave mathematically as a function of two variables, position
and time, $y\left( x,t\right) .$

\subsection{Longitudinal vs. transverse}

We divide the various kinds of waves that occur into two basic types:

\begin{definition}
transverse wave: a traveling wave or pulse that causes the elements of the
disturbed medium to move perpendicular to the direction of propagation
\end{definition}

\begin{definition}
Longitudinal wave: a traveling wave or pulse that causes the elements of the
medium to move parallel to the direction of propagation
\end{definition}

\FRAME{dhF}{2.6705in}{1.7218in}{0pt}{}{}{Figure}{\special{language
"Scientific Word";type "GRAPHIC";maintain-aspect-ratio TRUE;display
"USEDEF";valid_file "T";width 2.6705in;height 1.7218in;depth
0pt;original-width 3.781in;original-height 2.4284in;cropleft "0";croptop
"1";cropright "1";cropbottom "0";tempfilename
'PQXXQXPX.wmf';tempfile-properties "XPR";}}

\subsection{Examples of waves:}

Let's look at some specific cases of wave motion.

\subsubsection{A pulse on a rope:}

\FRAME{dhF}{1.3837in}{1.8273in}{0in}{}{}{Figure}{\special{language
"Scientific Word";type "GRAPHIC";maintain-aspect-ratio TRUE;display
"USEDEF";valid_file "T";width 1.3837in;height 1.8273in;depth
0in;original-width 1.3482in;original-height 1.7902in;cropleft "0";croptop
"1";cropright "1";cropbottom "0";tempfilename
'PQXXQXPY.wmf';tempfile-properties "XPR";}}

In the picture above, you see wave that has just one peak traveling to the
right. We call such a wave a \emph{pulse}. Notice how the piece of the rope
marked $P$ moves up and down, but the wave is moving to the right. This
pulse is a transverse wave because the parts of the medium (observe point $P$%
) move perpendicular to the direction the wave is moving.

\subsubsection{An ocean wave:}

Of course, some waves are a combination of these two basic types. You may
have noticed that in Physics we tend to define basic types of things, and
then use these basic types to define more complex objects. Water waves, for
example, are transverse at the surface of the water, but are longitudinal
throughout the water.

\FRAME{dtbpF}{1.6622in}{0.7835in}{0pt}{}{}{Figure}{\special{language
"Scientific Word";type "GRAPHIC";maintain-aspect-ratio TRUE;display
"USEDEF";valid_file "T";width 1.6622in;height 0.7835in;depth
0pt;original-width 5.5901in;original-height 2.6212in;cropleft "0";croptop
"1";cropright "1";cropbottom "0";tempfilename
'PQXXQXPZ.wmf';tempfile-properties "XPR";}}

\subsubsection{Earthquake waves:}

Earthquakes produce both transverse and longitudinal waves. The two types of
waves even travel at different speeds! $P$ waves are longitudinal and travel
faster, $S$ waves are transverse and slower.

\section{Wave speed}

We can perform an experiment with a rope or a long spring. Make a wave on
the rope or spring. Then pull the rope or spring tighter and make another
wave. We see that the wave on the tighter spring travels faster.

It is harder to do, but we can also experiment with two different ropes, one
light and one heavy. We would find that the heaver the rope, the slower the
wave. We can express this as 
\[
v=\sqrt{\frac{T_{s}}{\mu }} 
\]%
where $T_{s}$ is the tension in the rope, and $\mu $ is the linear mass
density%
\[
\mu =\frac{m}{L} 
\]%
where $m$ is the mass of the rope, and $L$ is the length.

The term $\mu $ might need an analogy to make it seem helpful. So suppose I
have an iron bar that has a mass of $200\unit{kg}$ and is $2\unit{m}$ long.
Further suppose I want to know how much mass there would be in a $20\unit{cm}
$ section cut of the end of the rod. How would I find out?

This is not very hard, We could say that there are $200\unit{kg}$ spread out
over $2\unit{m},$ so each meter of rod has $100\unit{kg}$ of mass, that is,
there is $100\unit{kg}/\unit{m}$ of mass per unit length. Then to find how
much mass there is in a $0.20\unit{m}$ section of the rod I\ take 
\[
m=100\frac{\unit{kg}}{\unit{m}}\times 0.20\unit{m}=20.0\unit{kg} 
\]
The $100\unit{kg}/\unit{m}$ is $\mu .$ It is how much mass there is in a
unit length segment of something In this example, it is a unit length of
iron bar, but for waves on string, we want the mass per unit length of
string.

If you are buying stock steel bar, you might be able to buy it with a mass
per unit length. If the mass per unit length is higher then the bar is more
massive. The same is true with string. The larger $\mu ,$ the more massive
equal string segments will be.

We should note that in forming this relationship, we have used our standard
introductory physics assumption that the mass of the rope can be neglected.
Let's consider what would happen if this were not true. Say we make a wave
in a heavy cable that is suspended. The mass at the lower end of the cable
pulls down on the upper part of the cable. The tension will actually change
along the length of the cable, and so will the wave speed. Such a situation
can't be represented by a single wave speed. But for our class, we will
assume that any such changes are small enough to be ignored.

\section{Example: Sound waves}

Sound is a wave. The medium is air particles. The transfer of energy is done
by collision. \FRAME{dhF}{1.0369in}{1.1744in}{0pt}{}{}{Figure}{\special%
{language "Scientific Word";type "GRAPHIC";maintain-aspect-ratio
TRUE;display "USEDEF";valid_file "T";width 1.0369in;height 1.1744in;depth
0pt;original-width 2.5581in;original-height 2.9006in;cropleft "0";croptop
"1";cropright "1";cropbottom "0";tempfilename
'PQXXQXQ0.wmf';tempfile-properties "XPR";}}The wave will be a longitudinal
wave. Let's see how it forms. We can take a tube with a piston in it. \FRAME{%
dhF}{3.2707in}{0.7809in}{0pt}{}{}{Figure}{\special{language "Scientific
Word";type "GRAPHIC";maintain-aspect-ratio TRUE;display "USEDEF";valid_file
"T";width 3.2707in;height 0.7809in;depth 0pt;original-width
3.2258in;original-height 0.7498in;cropleft "0";croptop "1";cropright
"1";cropbottom "0";tempfilename 'PQXXQXQ1.wmf';tempfile-properties "XPR";}}%
As we exert a force on the piston, the air molecules are compressed into a
group. In the next figure, each dot represents a group of air molecules. In
the top picture, the air molecules are not displaced. But when the piston
moves, the air molecules receive energy by collision. They bunch up. We see
this in the second picture. \FRAME{dhF}{2.9352in}{1.5082in}{0pt}{}{}{Figure}{%
\special{language "Scientific Word";type "GRAPHIC";maintain-aspect-ratio
TRUE;display "USEDEF";valid_file "T";width 2.9352in;height 1.5082in;depth
0pt;original-width 2.8911in;original-height 1.471in;cropleft "0";croptop
"1";cropright "1";cropbottom "0";tempfilename
'PQXXQXQ2.wmf';tempfile-properties "XPR";}}The graph below the two pictures
shows how much displacement each molecule group experiences.

Suppose we now pull the piston back. This would allow the molecules to
bounce back to the left, but the molecules that they have collided with will
receive some energy and go to the right. This is shown in the next figure.
Color coded dots are displayed above the before and after picture so you can
see where the molecule groups started.\FRAME{dhF}{2.9334in}{1.5402in}{0pt}{}{%
}{Figure}{\special{language "Scientific Word";type
"GRAPHIC";maintain-aspect-ratio TRUE;display "USEDEF";valid_file "T";width
2.9334in;height 1.5402in;depth 0pt;original-width 3.6979in;original-height
1.9285in;cropleft "0";croptop "1";cropright "1";cropbottom "0";tempfilename
'PQXXQXQ3.wmf';tempfile-properties "XPR";}}If we pull the piston back
further, the molecules can pass their original positions.\FRAME{dhF}{2.917in%
}{1.567in}{0pt}{}{}{Figure}{\special{language "Scientific Word";type
"GRAPHIC";maintain-aspect-ratio TRUE;display "USEDEF";valid_file "T";width
2.917in;height 1.567in;depth 0pt;original-width 3.2811in;original-height
1.7495in;cropleft "0";croptop "1";cropright "1";cropbottom "0";tempfilename
'PQXXQXQ4.wmf';tempfile-properties "XPR";}}Then we can push inward again and
compress the gas.\FRAME{dhF}{2.8228in}{1.5791in}{0pt}{}{}{Figure}{\special%
{language "Scientific Word";type "GRAPHIC";maintain-aspect-ratio
TRUE;display "USEDEF";valid_file "T";width 2.8228in;height 1.5791in;depth
0pt;original-width 3.7671in;original-height 2.0954in;cropleft "0";croptop
"1";cropright "1";cropbottom "0";tempfilename
'PQXXQXQ5.wmf';tempfile-properties "XPR";}}This may seem like a senseless
thing to do, but it is really what a speaker does to produce sound. In
particular, a speaker is a harmonic oscillator. The simple harmonic motion
of the speaker is the disturbance that makes the sound wave.\FRAME{dhF}{%
2.6135in}{2.1344in}{0pt}{}{}{Figure}{\special{language "Scientific
Word";type "GRAPHIC";maintain-aspect-ratio TRUE;display "USEDEF";valid_file
"T";width 2.6135in;height 2.1344in;depth 0pt;original-width
2.5719in;original-height 2.0954in;cropleft "0";croptop "1";cropright
"1";cropbottom "0";tempfilename 'PQXXQXQ6.wmf';tempfile-properties "XPR";}}

\section{One dimensional waves}

To mathematically describe a wave we will define a function of both time and
position.%
\begin{equation}
y\left( x,t\right)
\end{equation}%
Note, that this is new in our physics experience. Before we usually were
concerned about only one variable at a time. For oscillation we had just $%
y\left( t\right) $ for example. But now we will be concerned about two
variables, $x,$ and $t.$

let's take a specific example\footnote{%
This is not an important wave function, just one I picked because it makes a
nice graphic example.}%
\begin{equation}
y=\frac{2}{\left( x-3.0t\right) ^{2}+1}
\end{equation}

Let's plot this for $t=0$

\FRAME{dtbpFX}{2.514in}{1.676in}{0pt}{}{}{Plot}{\special{language
"Scientific Word";type "MAPLEPLOT";width 2.514in;height 1.676in;depth
0pt;display "USEDEF";plot_snapshots TRUE;mustRecompute FALSE;lastEngine
"MuPAD";xmin "-50";xmax "50";xviewmin "-50";xviewmax "50";yviewmin
"0";yviewmax "2";viewset"XY";rangeset"X";plottype 4;axesFont "Times New
Roman,12,0000000000,useDefault,normal";numpoints 100;plotstyle
"patch";axesstyle "normal";axestips FALSE;xis \TEXUX{x};var1name
\TEXUX{$x$};function \TEXUX{$\frac{2}{\left( x\right) ^{2}+1}$};linecolor
"blue";linestyle 1;pointstyle "point";linethickness 3;lineAttributes
"Solid";var1range "-50,50";num-x-gridlines 100;curveColor
"[flat::RGB:0x000000ff]";curveStyle "Line";VCamFile
'PQXXR234.xvz';valid_file "T";tempfilename
'PQXXQXQ7.wmf';tempfile-properties "XPR";}}what will this look like for $%
t=10 $?

\FRAME{dtbpFX}{2.7709in}{1.8472in}{0pt}{}{}{Plot}{\special{language
"Scientific Word";type "MAPLEPLOT";width 2.7709in;height 1.8472in;depth
0pt;display "USEDEF";plot_snapshots TRUE;mustRecompute FALSE;lastEngine
"MuPAD";xmin "-50";xmax "50";xviewmin "-50";xviewmax "50";yviewmin
"0";yviewmax "2";viewset"XY";rangeset"X";plottype 4;axesFont "Times New
Roman,12,0000000000,useDefault,normal";numpoints 100;plotstyle
"patch";axesstyle "normal";axestips FALSE;xis \TEXUX{x};var1name
\TEXUX{$x$};function \TEXUX{$\frac{2}{\left( x-30\right) ^{2}+1}$};linecolor
"blue";linestyle 1;pointstyle "point";linethickness 3;lineAttributes
"Solid";var1range "-50,50";num-x-gridlines 100;curveColor
"[flat::RGB:0x000000ff]";curveStyle "Line";VCamFile
'PQXXR233.xvz';valid_file "T";tempfilename
'PQXXQXQ8.wmf';tempfile-properties "XPR";}}The pulse travels along the $x$%
-axis as a function of time. Note that there is a value for $y$ for every $x$
position and that these $y$ values change for different times. That is what
we meen by saying we have a function of both $x$ and $t.$

We denote the speed of the pulse as $v,$ then we can define a function 
\begin{equation}
y\left( x,t\right) =y\left( x-vt,0\right)
\end{equation}%
that describes a pulse as it travels. An element of the medium (rope,
string, etc.) at position $x$ at some time $t,$ will have the displacement
that an element had earlier at $x-vt$ when $t=0.$

We will give $y\left( x-vt,0\right) $ a special name, the \emph{wave function%
}. It represents the $y$ position, the transverse position in our example,
of any element located at a position $x$ at any time $t$.

Notice that wave functions depend on two variables, $x,$ and $t.$ It is hard
to draw a wave so that this dual dependence is clear. Often we draw two
different graphs of the same wave so we can see independently the position
and time dependence. So far we have used one of these graphs. A graph of our
wave at a specific time, $t_{o}.$ This gives $y\left( x,t_{o}\right) $. This
representation of a wave is very like a photograph of the wave taken with a
digital camera. It gives a picture of the entire wave, but only for one
time, the time at which the photograph was taken. Of course we could take a
series of photographs, but still each would be a picture of the wave at just
one time. Here is such a series of graphs at times $t_{1}$ through $t_{4}.$%
\FRAME{dtbpF}{2.4993in}{3.7472in}{0in}{}{}{Figure}{\special{language
"Scientific Word";type "GRAPHIC";maintain-aspect-ratio TRUE;display
"USEDEF";valid_file "T";width 2.4993in;height 3.7472in;depth
0in;original-width 2.4587in;original-height 3.6997in;cropleft "0";croptop
"1";cropright "1";cropbottom "0";tempfilename
'PQXXQXQ9.wmf';tempfile-properties "XPR";}}

The second representation is to observe the wave at just one point in the
medium, but for many times. This is very like taking a video camera and
using it to record the displacement of just one part of the medium for many
times. You could envision marking just one part of a rope, and then using
the video recorder to make a movie of the motion of that single part of the
rope. We could then go frame by frame through the video, and plot the
displacement of our marked part of the rope as a function of time. Such a
graph is sometimes called a history graph of the wave. Here is such a graph
for the position $x_{1}.$ \FRAME{dtbpF}{2.5918in}{1.7279in}{0in}{}{}{Figure}{%
\special{language "Scientific Word";type "GRAPHIC";maintain-aspect-ratio
TRUE;display "USEDEF";valid_file "T";width 2.5918in;height 1.7279in;depth
0in;original-width 2.5503in;original-height 1.6916in;cropleft "0";croptop
"1";cropright "1";cropbottom "0";tempfilename
'PQXXQXQA.wmf';tempfile-properties "XPR";}}To go from the time graphs to the
history graph you observe what happens at the location $x_{1}$ for each of
the times. Then plot those $y$ positions on the $y$ vs. $t$ graph. Then you
must connect the points. This takes some thinking to make sure you connect
them right (or a whole lot of points). But with practice, this is not hard
and both view points are valuable ways to look at a wave.

\chapter{Sinusoidal Waves in one and Two Dimensions}

%TCIMACRO{%
%\TeXButton{Fundamental Concepts}{\hspace{-1.3in}{\Large Fundamental Concepts\vspace{0.25in}}}}%
%BeginExpansion
\hspace{-1.3in}{\Large Fundamental Concepts\vspace{0.25in}}%
%EndExpansion

\begin{itemize}
\item It takes two variables to describe a wave, position and time

\item Wave graphs have many named parts, amplitude, period, wavelength
crest, trough, etc.

\item There is a spatial analog to temporal frequency called spacial
frequency and it is represented by the factor $k$ in our equations

\item We can also express sound waves in terms of pressure changes

\item We don't hear all frequencies equally well

\item Waves from point sources are spherical

\item Light waves are waves in the electromagnetic field
\end{itemize}

\section{Sinusoidal Waves}

\FRAME{dhF}{2.5875in}{2.1127in}{0pt}{}{\Qlb{sine wave}}{Figure}{\special%
{language "Scientific Word";type "GRAPHIC";maintain-aspect-ratio
TRUE;display "USEDEF";valid_file "T";width 2.5875in;height 2.1127in;depth
0pt;original-width 2.2667in;original-height 1.8455in;cropleft "0";croptop
"1";cropright "1";cropbottom "0";tempfilename
'PQXXQXQB.wmf';tempfile-properties "XPR";}}A sinusoidal graph should be
familiar from our study of oscillation. Simple harmonic oscillators are
described by the function%
\begin{equation}
y\left( t\right) =y_{\max }\cos \left( \omega t+\phi \right) \qquad \text{SHM%
}
\end{equation}%
but this only gave us a vertical displacement. Now our sinusoidal function
must also be a function of position along the wave.%
\begin{equation}
y\left( t\right) =y_{\max }\cos \left( ax-\omega t+\phi \right) \qquad \text{%
waves}
\end{equation}%
but before we study the nature of this function, lets see what we can learn
from the graph of a sinusoidal wave. We will need both our two views, the
camera snapshot and the video of a point. Look at figure \ref{sine wave}.
This is two camera snap shots superimposed. The red curve shows the wave ($y$
position for each value of $x)$ at $t=0.$ At some some later time $t,$ the
wave pattern has moved to the right as shown by the blue curve. The shift is
by an amount $\Delta x=v\Delta t.$

\section{Parts of a wave}

The peak of a wave is called the crest. For a sine wave we have a series of
crests. We define the wavelength as the distance between any to identical
points (e.g. crests) on adjacent waves in a snapshot view.

Notice that this is very similar to the definition of the period, $T,$ when
we graphed SHM on a $y,t$ set of axis. In fact, this similarity is even more
apparent if we plot a sinusoidal wave using our two wave pictures. In the
next figure, the snap-shot comes first. We can see that there will be
crests. The distance between the crests is given the name \emph{wavelength}.
This is not the entire length of the whole wave. But is is a characteristic
length of part of the wave that is easy to identify. The next figure shows
all this using our snapshot and history graph for a sine wave. \FRAME{dtbpF}{%
2.2831in}{4.0136in}{0pt}{}{}{Figure}{\special{language "Scientific
Word";type "GRAPHIC";maintain-aspect-ratio TRUE;display "USEDEF";valid_file
"T";width 2.2831in;height 4.0136in;depth 0pt;original-width
4.34in;original-height 7.6662in;cropleft "0";croptop "1";cropright
"1";cropbottom "0";tempfilename 'PQXXQXQC.wmf';tempfile-properties "XPR";}}%
Note that there are crests in the history graph view as well. That is
because one marked part of the medium is being displaced as a function of
time. Think of a marked piece of the rope going up and down, or think of
floating in the ocean at one point, you travel up and down (and a little bit
back and forth) as the waves go by. But now the horizontal axis is time.
There will be a characteristic time between crests. That time is called the 
\emph{period}. This is just like oscillatory motion--because it is
oscillatory motion! Like the wavelength is not the length of the whole wave,
the period is not the time the whole wave exists. It is just the time it
takes the part of the medium we are watching to go through one complete
cycle. Notice that this video picture is exactly the same as a plot of the
motion of a simple harmonic oscillator! For a sinusoidal wave, each part of
the medium experiences simple harmonic motion.

We remember frequency from simple harmonic motion. But now we have a wave,
and the wave is moving. We can extend our view of frequency by defining it
as follows:

\emph{The frequency of a periodic wave is the number of crests (or any other
point of the wave) that pass a given point in a unit time interval.}

\FRAME{dtbpFX}{2.6109in}{1.7417in}{0pt}{}{\Qlb{frequency}}{Plot}{\special%
{language "Scientific Word";type "MAPLEPLOT";width 2.6109in;height
1.7417in;depth 0pt;display "USEDEF";plot_snapshots TRUE;mustRecompute
FALSE;lastEngine "MuPAD";xmin "0";xmax "10";xviewmin "0";xviewmax
"10";yviewmin "-1.000187";yviewmax
"1.000187";viewset"XY";rangeset"X";plottype 4;axesFont "Times New
Roman,12,0000000000,useDefault,normal";numpoints 100;plotstyle
"patch";axesstyle "normal";axestips FALSE;xis \TEXUX{v58130};var1name
\TEXUX{$\theta $};function \TEXUX{$\sin \left( \theta \right) $};linecolor
"blue";linestyle 1;pointstyle "point";linethickness 1;lineAttributes
"Solid";var1range "0,10";num-x-gridlines 100;curveColor
"[flat::RGB:0x000000ff]";curveStyle "Line";function \TEXUX{$\sin \left(
2\theta \right) $};linecolor "red";linestyle 1;pointstyle
"point";linethickness 1;lineAttributes "Solid";var1range
"0,10";num-x-gridlines 100;curveColor "[flat::RGB:0x00ff0000]";curveStyle
"Line";VCamFile 'PQXXR232.xvz';valid_file "T";tempfilename
'PQXXQXQD.wmf';tempfile-properties "XPR";}}In the previous figure, the blue
curve has twice the frequency as the red curve. Notice how it has two crests
for every red crest. The maximum displacement of the wave is called the 
\emph{amplitude }just as it was for simple harmonic oscillators.

\section{Wavenumber and wave speed}

Consider again a sinusoidal wave. 
\begin{equation}
y\left( x,t\right) =y_{\max }\cos \left( kx-\omega t+\phi _{o}\right)
\end{equation}%
We have drawn the wave in the snapshot picture mode\FRAME{dtbpFX}{2.6109in}{%
1.7417in}{0pt}{}{}{Plot}{\special{language "Scientific Word";type
"MAPLEPLOT";width 2.6109in;height 1.7417in;depth 0pt;display
"USEDEF";plot_snapshots TRUE;mustRecompute FALSE;lastEngine "MuPAD";xmin
"0";xmax "10";xviewmin "0";xviewmax "10";yviewmin "-1.000187";yviewmax
"1.000187";viewset"XY";rangeset"X";plottype 4;axesFont "Times New
Roman,12,0000000000,useDefault,normal";numpoints 100;plotstyle
"patch";axesstyle "normal";axestips FALSE;xis \TEXUX{v58130};var1name
\TEXUX{$\theta $};function \TEXUX{$\sin \left( \theta \right) $};linecolor
"blue";linestyle 1;pointstyle "point";linethickness 1;lineAttributes
"Solid";var1range "0,10";num-x-gridlines 100;curveColor
"[flat::RGB:0x000000ff]";curveStyle "Line";VCamFile
'PQXXR231.xvz';valid_file "T";tempfilename
'PQXXQXQE.wmf';tempfile-properties "XPR";}}To make this graph, we set $t=0$
and plot the resulting function%
\begin{equation}
y\left( x,0\right) =y_{\max }\sin \left( kx+0\right)
\end{equation}%
$y_{\max }$ is the amplitude. I want to investigate the meaning of the
constant $k.$ Lets find $k$ like we did for SHM when we found $\omega $.
Consider the point $x=0$. At this point 
\begin{equation}
y\left( 0,0\right) =y_{\max }\sin \left( k\left( 0\right) \right) =0
\end{equation}%
The next time $y=0$ is when $x=\frac{\lambda }{2}$ \FRAME{dtbpF}{2.8781in}{%
2.2373in}{0pt}{}{}{Figure}{\special{language "Scientific Word";type
"GRAPHIC";maintain-aspect-ratio TRUE;display "USEDEF";valid_file "T";width
2.8781in;height 2.2373in;depth 0pt;original-width 4.4858in;original-height
3.4809in;cropleft "0";croptop "1";cropright "1";cropbottom "0";tempfilename
'PQXXQXQF.wmf';tempfile-properties "XPR";}}then%
\begin{equation}
y\left( \frac{\lambda }{2},0\right) =y_{\max }\sin \left( k\frac{\lambda }{2}%
\right) =0
\end{equation}%
From our trigonometry experience, we know that this is true when 
\begin{equation}
k\frac{\lambda }{2}=\pi
\end{equation}%
solving for $k$ gives%
\begin{equation}
k=\frac{2\pi }{\lambda }
\end{equation}%
Then we now have a feeling for what $k$ means. It is $2\pi $ over the
spacing between the crests. The $2\pi $ must have units of radians attached.
Then%
\begin{equation}
y\left( x,0\right) =A\sin \left( \frac{2\pi }{\lambda }x+0\right)
\end{equation}

We have a special name for the quantity $k.$ It is called the \emph{wave
number}$.$ 
\begin{equation}
k\equiv \frac{2\pi }{\lambda }
\end{equation}

Both the name and the symbol are somewhat unfortunate. Neither gives much
insight into the meaning of this quantity. But from what we have done in
studying oscillaiton, we can understand this new quantity. For a harmonic
oscillator, we know that 
\[
y\left( t\right) =y_{\max }\sin \left( \omega t\right) 
\]%
where 
\[
\omega =2\pi f=\frac{2\pi }{T} 
\]%
$T$ is how far, \emph{in time}, the crests are apart, and the inverse of
this, $\frac{1}{T}$ is the frequency. The frequency tells us how often we
encounter a crest as we march along in time. So $\frac{1}{T}$ must be how
many crests we have in a unit amount of time.

Now think of the relationship between the snapshot and the video
representation for a sinusoidal wave. We have a new quantity 
\[
k=\frac{2\pi }{\lambda } 
\]%
where $\lambda $ is how far, \emph{in distance}, the crests are apart. This
implies that $\frac{1}{\lambda }$ plays the same role in the snap shot graph
that $f$ plays in the video graph. It must tell us how many crests we have,
but this time it is how many crests in a given amount of distance. We found
above that $k$ told us something about how often the zeros (well, every
other zero) will occur. But the crests must occur at the same rate. So $k$
tells us how often we encounter a crest in our snapshot graph. This is not
too strange. It would be meaningful, for example, to say that a farmer had
plowed his or her field to have two furrows per meter.\FRAME{dtbpF}{2.9853in%
}{1.6198in}{0in}{}{}{Figure}{\special{language "Scientific Word";type
"GRAPHIC";maintain-aspect-ratio TRUE;display "USEDEF";valid_file "T";width
2.9853in;height 1.6198in;depth 0in;original-width 2.9421in;original-height
1.5835in;cropleft "0";croptop "1";cropright "1";cropbottom "0";tempfilename
'PQXXQXQG.wmf';tempfile-properties "XPR";}}

The frequency in the video graph is how often we encounter a crest, $\frac{1%
}{\lambda }$ is how often we encounter a crest in the snap shot graph. Thus $%
\frac{1}{\lambda }$ is playing the same role for a snap shot graph as
frequency plays for a history graph. We could call $\frac{1}{\lambda }$a 
\emph{spatial frequency. }It is how often we encounter a crest as we march
along in position, or how many crests we have in a unit amount of distance.
And $\lambda $ could be called a \emph{spatial period}. Both $1/T$ and $%
1/\lambda $ answer the question \textquotedblleft how often something
happens in a unit of something\textquotedblright\ but one asks the question
in time and the other in position along the wave.

My mental image for this is the set of groves on the edge of a highway.
There is a distance between them, like a wavelength, and how often I
encounter one as I move a distance along the road is the spatial frequency.
You could say that there are, maybe, $3/\unit{m}$. That is a spatial
frequency! It is how many of something happens in a unit distance. We use
this concept in optics to test how well an optical system resolves details
in a photograph. The next figure is a test image. A good camera will resolve
all spatial frequencies equally well. Notice the test image has sets of bars
with different spatial frequencies. By forming an image of this pattern, you
can see which spacial frequencies are faithfully represented by the optical
system.\FRAME{dtbpFU}{2.6974in}{2.1819in}{0pt}{\Qcb{Resolution test target
based on the USAF\ 1951 Resolution Test Pattern (not drawn to exact
specifications).}}{}{Figure}{\special{language "Scientific Word";type
"GRAPHIC";maintain-aspect-ratio TRUE;display "USEDEF";valid_file "T";width
2.6974in;height 2.1819in;depth 0pt;original-width 2.655in;original-height
2.143in;cropleft "0";croptop "1";cropright "1";cropbottom "0";tempfilename
'PQXXQXQH.wmf';tempfile-properties "XPR";}}In class you will see that our
projector does not represent all spatial frequencies equally well! You can
also see this now in the copy you are reading. If you are reading on-line or
an electronic copy, your screen resolution will limit the representation of
some spatial frequencies. Look for the smallest set of three bars where you
can still tell for sure that there are three bars without zooming. A printed
version that has been printed on a laser printer will usually allow you to
see even smaller sets of three bars clearly.

Let's place $k$ in the full equation for the sine wave for any time, $t$.%
\begin{equation}
y\left( t\right) =y_{\max }\cos \left( kx-\omega t+\phi _{o}\right)
\end{equation}%
We would like this to look like our wave function equation 
\[
y\left( x,t\right) =y\left( x-vt,0\right) 
\]%
With a little algebra we can do this%
\begin{eqnarray*}
y\left( t\right) &=&y_{\max }\cos \left( kx-\omega t+\phi _{o}\right) \\
&=&y_{\max }\cos \left( \frac{2\pi }{\lambda }x-\frac{2\pi }{T}t+\phi
_{o}\right) \\
&=&y_{\max }\cos \left( \frac{2\pi }{\lambda }\left( x-\frac{\lambda }{T}%
t\right) +\phi _{o}\right)
\end{eqnarray*}%
This is in the form of a wave function so long as%
\begin{equation}
v=\frac{\lambda }{T}
\end{equation}%
then 
\begin{equation}
y\left( x,t\right) =y_{\max }\sin \left( \frac{2\pi }{\lambda }\left(
x-vt\right) \right)
\end{equation}%
We can see that the wave travels one wavelength in one period. The simple
relationship 
\begin{equation}
v=\frac{\lambda }{T}
\end{equation}%
is of tremendous importance.

\subsubsection{Wave speed forms}

We found%
\begin{equation}
v=\frac{\lambda }{T}
\end{equation}%
but it is easy to see that 
\begin{equation}
v=\frac{2\pi \lambda }{2\pi T}=\frac{\omega }{k}
\end{equation}%
and 
\begin{equation}
v=\lambda f
\end{equation}%
This last formula is, perhaps, the most common form encountered in our study
of light.

\subsubsection{Phase}

You may be wondering about the phase constant we learned about in our study
of SHM. We have ignored it up to now. But of course we can shift our sine
just like we did for our plots of position vs. time for oscillation. Only
now with a wave we have two graphs, a history and snapshot graphs, so we
could shift along the $x$ in a snapshot graph or along the $t$ axes in a
history graph. So the sine wave has the form.%
\begin{equation}
y\left( x,t\right) =A\sin \left( kx-\omega t+\phi _{o}\right)
\end{equation}%
were $\phi _{o}$ will need to be determined by initial conditions just like
in SHM problems and those initial conditions will include initial positions
as well as initial times.

Let's consider that we have two views of a wave, the snapshot and history
view. Each of these looks like sinusoids for a sinusoidal wave. Let's
consider a specific wave, 
\[
y\left( x,t\right) =5\sin \left( 3\pi x-\frac{\pi }{5}t+\frac{\pi }{2}%
\right) 
\]%
we look at a snapshot graph at $t=0$ 
\[
y\left( x,0\right) =5\sin \left( 3\pi x-\frac{\pi }{5}\left( 0\right) +\frac{%
\pi }{2}\right) 
\]%
\FRAME{dtbpFX}{4.4996in}{0.7706in}{0pt}{}{}{Plot}{\special{language
"Scientific Word";type "MAPLEPLOT";width 4.4996in;height 0.7706in;depth
0pt;display "USEDEF";plot_snapshots TRUE;mustRecompute FALSE;lastEngine
"MuPAD";xmin "-3";xmax "3";xviewmin "-3";xviewmax "3";yviewmin
"-5.000998";yviewmax "5";viewset"XY";rangeset"X";plottype 4;axesFont "Times
New Roman,12,0000000000,useDefault,normal";numpoints 100;plotstyle
"patch";axesstyle "normal";axestips FALSE;xis \TEXUX{x};var1name
\TEXUX{$x$};function \TEXUX{$5\sin \left( 3\pi x-0+\frac{\pi }{2}\right)
$};linecolor "blue";linestyle 1;pointstyle "point";linethickness
3;lineAttributes "Solid";var1range "-3,3";num-x-gridlines 100;curveColor
"[flat::RGB:0x000000ff]";curveStyle "Line";VCamFile
'PQXXR230.xvz';valid_file "T";tempfilename
'PQXXQXQI.wmf';tempfile-properties "XPR";}}and another at $t=2\unit{s}$%
\[
y\left( x,2\unit{s}\right) =5\sin \left( 3\pi x-\frac{\pi }{5}\left( 2\unit{s%
}\right) +\frac{\pi }{2}\right) 
\]%
\FRAME{dtbpFX}{4.4996in}{0.7706in}{0pt}{}{}{Plot}{\special{language
"Scientific Word";type "MAPLEPLOT";width 4.4996in;height 0.7706in;depth
0pt;display "USEDEF";plot_snapshots TRUE;mustRecompute FALSE;lastEngine
"MuPAD";xmin "-3";xmax "3";xviewmin "-3";xviewmax "3";yviewmin
"-5.000998";yviewmax "5";viewset"XY";rangeset"X";plottype 4;axesFont "Times
New Roman,12,0000000000,useDefault,normal";numpoints 100;plotstyle
"patch";axesstyle "normal";axestips FALSE;xis \TEXUX{x};var1name
\TEXUX{$x$};function \TEXUX{$5\sin \left( 3\pi x-\frac{\pi }{5}\left(
2\right) +\frac{\pi }{2}\right) $};linecolor "blue";linestyle 1;pointstyle
"point";linethickness 3;lineAttributes "Solid";var1range
"-3,3";num-x-gridlines 100;curveColor "[flat::RGB:0x000000ff]";curveStyle
"Line";VCamFile 'PQXXR22Z.xvz';valid_file "T";tempfilename
'PQXXQXQJ.wmf';tempfile-properties "XPR";}}Comparing the two, we could view
the latter as having a different phase constant that is the sum of what we
have called the phase constant, $\phi _{o}$ plus $\omega \Delta t,$ which
tells us how different the times are between the two graphs. This time
difference is what is shifting the graph 
\[
\phi _{total}=\omega \Delta t+\phi _{o}=-\frac{\pi }{5}\left( 2\unit{s}%
\right) +\frac{\pi }{2} 
\]%
that is, within the snapshot view, the time dependent part of the argument
of the sign acts like an additional phase constant.

Likewise, in the history view, we can plot our wave at $x=0$%
\[
y\left( 0,t\right) =5\sin \left( 3\pi \left( 0\right) -\frac{\pi }{5}t+\frac{%
\pi }{2}\right) 
\]%
\FRAME{dtbpFX}{4.4996in}{0.7706in}{0pt}{}{}{Plot}{\special{language
"Scientific Word";type "MAPLEPLOT";width 4.4996in;height 0.7706in;depth
0pt;display "USEDEF";plot_snapshots TRUE;mustRecompute FALSE;lastEngine
"MuPAD";xmin "-8";xmax "8";xviewmin "-8";xviewmax "8";yviewmin
"-5.000998";yviewmax "5";viewset"XY";rangeset"X";plottype 4;labeloverrides
1;x-label "t";axesFont "Times New
Roman,12,0000000000,useDefault,normal";numpoints 100;plotstyle
"patch";axesstyle "normal";axestips FALSE;xis \TEXUX{t};var1name
\TEXUX{$t$};function \TEXUX{$5\sin \left( 3\pi \left( 0\right) -\frac{\pi
}{5}t+\frac{\pi }{2}\right) $};linecolor "blue";linestyle 1;pointstyle
"point";linethickness 3;lineAttributes "Solid";var1range
"-8,8";num-x-gridlines 100;curveColor "[flat::RGB:0x000000ff]";curveStyle
"Line";VCamFile 'PQXXR22Y.xvz';valid_file "T";tempfilename
'PQXXQXQK.wmf';tempfile-properties "XPR";}}and at $x=1.5\unit{m}$%
\[
y\left( 1.5\unit{m},t\right) =5\sin \left( 3\pi \left( 1.5\unit{m}\right) -%
\frac{\pi }{5}t+\frac{\pi }{2}\right) 
\]%
\FRAME{dtbpFX}{4.4996in}{0.7706in}{0pt}{}{}{Plot}{\special{language
"Scientific Word";type "MAPLEPLOT";width 4.4996in;height 0.7706in;depth
0pt;display "USEDEF";plot_snapshots TRUE;mustRecompute FALSE;lastEngine
"MuPAD";xmin "-8";xmax "8";xviewmin "-8";xviewmax "8";yviewmin
"-5.000998";yviewmax "5";viewset"XY";rangeset"X";plottype 4;labeloverrides
1;x-label "t";axesFont "Times New
Roman,12,0000000000,useDefault,normal";numpoints 100;plotstyle
"patch";axesstyle "normal";axestips FALSE;xis \TEXUX{t};var1name
\TEXUX{$t$};function \TEXUX{$5\sin \left( 3\pi \left( 1.5\right) -\frac{\pi
}{5}t+\frac{\pi }{2}\right) $};linecolor "blue";linestyle 1;pointstyle
"point";linethickness 3;lineAttributes "Solid";var1range
"-8,8";num-x-gridlines 100;curveColor "[flat::RGB:0x000000ff]";curveStyle
"Line";VCamFile 'PQXXR22X.xvz';valid_file "T";tempfilename
'PQXXQXQL.wmf';tempfile-properties "XPR";}}Within the history view, the $kx$
part of the argument acts like a phase constant.%
\[
\phi _{total}=k\Delta x+\phi _{o}=3\pi \left( 1.5\unit{m}\right) +\frac{\pi 
}{2} 
\]

Of course neither $kx$ nor $\omega t$ are constant, But within individual
views of the wave we have set them constant to form our snapshot and history
representations. We can see that any part of the argument of the sine, $%
kx-\omega t+\phi _{o}$ could contribute to a phase shift, depending on the
view we are taking.

Because of this, it is customary to call the entire argument of the sine
function, $\phi =kx-\omega t+\phi _{o}$ the \emph{phase of the wave}. Where $%
\phi _{o}$ is the phase constant, $\phi $ is the phase. Of course then, $%
\phi $ must be a function of $x$ and $t,$ so we have a different value for $%
\phi \left( x,t\right) $ for every point on the wave for every time. This
can be a little confusing, $\phi $ and $\phi _{o}$ look a lot the same, but
they are different.

\section{The Speed of Waves}

We have already considered the speed of waves on strings%
\[
v=\sqrt{\frac{T}{\mu }} 
\]%
we should see where this comes from so we know it's limitations. We would
expect that there may be things that change the speed of sound waves in air
as well (like a tension or a massiveness of the medium). Let's start with a
formal derivation of our string speed formula, then take on sound waves in
air.

\subsection{Derivation of the string wave speed formula}

Let's work a problem together. Let's find an expression for the speed of the
wave as it travels along a string.\FRAME{dhF}{2.7397in}{1.4364in}{0in}{}{}{%
Figure}{\special{language "Scientific Word";type
"GRAPHIC";maintain-aspect-ratio TRUE;display "USEDEF";valid_file "T";width
2.7397in;height 1.4364in;depth 0in;original-width 2.6974in;original-height
1.4019in;cropleft "0";croptop "1";cropright "1";cropbottom "0";tempfilename
'PQXXQXQM.wmf';tempfile-properties "XPR";}}

What are the forces acting on an element of string?

\begin{enumerate}
\item Tension on the RHS of the element from the rest of the string on the
right, $T_{r}$

\item Tension on the LHS of the element from the rest of the string on the
left, $T_{l}$

\item $F_{g}$
\end{enumerate}

Lets assume that the element of string, $\Delta s,$ at the crest is
approximately an arc of a circle with radius $R$.

There is a force pulling left on the left end of the element that is tangent
to the arc, there is a force pulling right at the right end of the element
which is also tangent to the arc. These forces produce centripetal
accelerations%
\begin{equation}
a=\frac{v^{2}}{R}
\end{equation}%
The horizontal components of the forces cancel $\left( T\cos \theta \right)
. $ The vertical component, $\left( T\sin \left( \theta \right) \right) $ is
directed toward the center of the arc. If the rope is not moving in the $x$
direction, then $T_{l}=T_{r}.$

Then, the radial force $F_{r}$ will have matching components from each side
of the element that together are $2T\sin \left( \theta \right) .$ Notice
that to make sure the components are the same we must assume that the string
is uniform and that it is not too massive (almost massless string
approximation!). Since the element is small,%
\begin{equation}
F_{r}=2T\sin \left( \theta \right) \approx 2T\theta
\end{equation}

The element has a mass $m.$ 
\begin{equation}
m=\mu \Delta s
\end{equation}

where $\mu $ is the mas per unit length. Using the arc length formula%
\begin{equation}
\Delta s=R\left( 2\theta \right)
\end{equation}%
so%
\begin{equation}
m=\mu \Delta s=2\mu R\theta
\end{equation}%
and finally we use the formula for the radial acceleration%
\begin{equation}
F_{r}=ma=\left( 2\mu R\theta \right) \frac{v^{2}}{R}
\end{equation}%
Combining these two expressions for $F_{r}$%
\begin{equation}
2T\theta =\left( 2\mu R\theta \right) \frac{v^{2}}{R}
\end{equation}%
\begin{equation}
T=\left( \mu R\right) \frac{v^{2}}{R}
\end{equation}%
\begin{equation}
\frac{T}{\mu }=v^{2}
\end{equation}%
and we find that 
\begin{equation}
v=\sqrt{\frac{T}{\mu }}
\end{equation}%
which is just what we stated before, only now we see just how much
approximation there was in building this formula. We might guess that it
would not work so well for large cables supporting bridges or with cables
that change size along their length.

\subsection{Speed of Sound Waves}

The speed of sound in air is around $340\unit{m}/\unit{s}.$ The speed
changes when we change media, and even when we are in the same media but the
temperature changes. For sound in air, a good approximation is%
\begin{equation}
v=v_{o}\sqrt{1+\frac{T_{c}}{T_{o}}}
\end{equation}%
where $v_{o}=331\frac{\unit{m}}{\unit{s}}$ and $T_{o}=273\unit{K}$ ($0\unit{%
%TCIMACRO{\U{2103}}%
%BeginExpansion
{}^{\circ}{\rm C}%
%EndExpansion
}).$\footnote{$v=v_{o}\sqrt{1+\frac{T_{C}}{T_{o}}}=v_{o}\sqrt{\frac{T_{o}}{%
T_{o}}+\frac{T_{C}}{T_{o}}}=v_{o}\sqrt{\frac{T_{o}+T_{C}}{T_{o}}}=v_{o}\sqrt{%
\frac{T_{K}}{T_{o}}}$}

Why temperature? We will find out later in this course that the density and
pressure of air change with temperature. But for now we can see that
changing the air density is a little like changing the linear mass density
of a string. It is bound to have an effect on the wave speed in the air.

\section{Waves in two and three dimensions}

So far we have written expressions for waves, but our experience tells us
that waves don't usually come as one dimensional phenomena. In the next
figure, we see the disturbance (a drop) creating a water wave. \FRAME{dtbpFU%
}{2.3709in}{1.5389in}{0pt}{\Qcb{Picture of a water drop (Jon Paul Johnson,
used by permision)}}{}{Figure}{\special{language "Scientific Word";type
"GRAPHIC";maintain-aspect-ratio TRUE;display "USEDEF";valid_file "T";width
2.3709in;height 1.5389in;depth 0pt;original-width 2.3315in;original-height
1.5031in;cropleft "0";croptop "1";cropright "1";cropbottom "0";tempfilename
'Circular_wave_from_a_drip0.wmf';tempfile-properties "XNPR";}}You can see
the wave forming, but the wave is clearly not one dimensional. It appears
nearly circular. In fact, it is closer to hemispherical, and this limit is
only true because the disturbance is at the air-water boundary. Most waves
in a uniform medium will be roughly spherical. As such a wave travels away
from the source, the energy traveling gets more spread out. This causes the
amplitude to decrease. Think of a sound wave, it gets quieter the farther
you are from the source. We change our equation to account for this by
making the amplitude a function of the distance, $r,$ from the source%
\begin{equation}
y\left( r,t\right) =A\left( r\right) \sin \left( kr-\omega t+\phi _{o}\right)
\end{equation}%
Of course, if we look at a very large wave, but we only look at part of the
wave, we see that our part looks flatter as the wave expands. \FRAME{dhF}{%
2.4881in}{1.6189in}{0pt}{}{}{Figure}{\special{language "Scientific
Word";type "GRAPHIC";maintain-aspect-ratio TRUE;display "USEDEF";valid_file
"T";width 2.4881in;height 1.6189in;depth 0pt;original-width
2.4474in;original-height 1.5826in;cropleft "0";croptop "1";cropright
"1";cropbottom "0";tempfilename 'PQXXQXQN.wmf';tempfile-properties "XPR";}}%
Very far from the source, our wave is flat enough that we can ignore the
curvature across it's wave fronts. We call such a wave a \emph{plane wave}.
There are no true plane waves in nature, but this idealization makes our
mathematical solutions simpler and many waves come close to this
approximation.

We have said that sound is a longitudinal wave with a medium of air. Really
any solid, liquid, or gas will work as a medium for sound. For our study, we
will take sound to be a longitudinal wave and treat liquids and gasses.
Solids have additional forces involved due to the tight bonding of the
atoms, and therefore are more complicated. Technically in a solid sound can
be a transverse wave as well a longitudinal wave, but we usually call
transverse waves of this nature \emph{shear waves.}

\section{Periodic Sound Waves, Pressure}

%TCIMACRO{%
%\TeXButton{Vollyball Demo}{\marginpar {
%\hspace{-0.5in}
%\begin{minipage}[t]{1in}
%\small{Vollyball Demo}
%\end{minipage}
%}}}%
%BeginExpansion
\marginpar {
\hspace{-0.5in}
\begin{minipage}[t]{1in}
\small{Vollyball Demo}
\end{minipage}
}%
%EndExpansion
Suppose I have a ball and I\ ask six people from the class to come up and
press on the ball from all directions. This is a new force situation, unlike
most we dealt with in Dynamics or PH121. Each person exerts a force on the
ball. The person uses the area of their hand to exert the force. The motion
of the ball, and even it's shape depend on both the force (magnitude and
direction) and the area involved in each push.

From our demo, it seems that the force and area of the ball could be related
to better describe the situation. Let's look at the ratio%
\[
\frac{F}{A} 
\]%
What does it represent? This ratio tells us how spread out an applied force
may be. The area is important. Think of the sides of a the eraser of a
pencil.

It is coinvent to give this concept a name. We will call it \emph{pressure.}

\begin{Note}
\textbf{Definition: Pressure} is a scalar value that describes how a force
acts over an area\mathstrut\ 
\begin{equation}
P\equiv \frac{F}{A}  \label{Pressure Definition}
\end{equation}%
\mathstrut
\end{Note}

But in a fluid, what is the force? Lets consider a ball hitting a wall. Is
there a force? During the collision, there is a force of the ball on the
wall, and a Newton's third law force of the wall molecules pushing back on
the ball. The ball will bounce back. 
%TCIMACRO{%
%\TeXButton{Ping Pong Ball Demo}{\marginpar {
%\hspace{-0.5in}
%\begin{minipage}[t]{1in}
%\small{Ping Pong Ball Demo}
%\end{minipage}
%}}}%
%BeginExpansion
\marginpar {
\hspace{-0.5in}
\begin{minipage}[t]{1in}
\small{Ping Pong Ball Demo}
\end{minipage}
}%
%EndExpansion

This force is small and only lasts during the collision. But now suppose we
have many balls, and all the balls impact the wall. Further suppose that
every time a ball bounces back is ends up headed back to the wall and
bounces again. If the balls keep coming, there will be a force on the wall
quite a bit of the time. At least, on average there is a force, anyway. This
is the force that causes air pressure. The air molecules in this room are
like the small balls. we will find that they have an amount of kinetic
energy. They impact the walls (and us) all over our surface area. The result
is air pressure.

The water pressure in a swimming pool is caused by moving water particles.
You should convince yourself that the reason the water stays in the pool is
partly because the air molecules bounce against the water surface exerting a
pressure on the waver!

Let's go back to making sounds. Suppose we push our piston as we did before
in figure \ref{Creation of a Pulse}. When we push in the piston, it creates
a region of higher pressure next to it.

When we pull back the piston the fluid expands to fill the void. We create a
rarefaction next to the piston.

Suppose we drive the piston sinusoidally. Can we describe the motion of the
particles and of the wave?

\begin{enumerate}
\item Compression: A local region of higher pressure in a fluid

\item Rarefaction: A local region of lower pressure in a fluid
\end{enumerate}

We can identify the distance between two compressions as $\lambda .$

We define $s\left( x,t\right) $ (like we defined a wave function, $y\left(
x,t\right) )$ as the displacement a particle of fluid relative to its
equilibrium position.%
\begin{equation}
s\left( x,t\right) =s_{\max }\cos \left( kx-\omega t\right)
\end{equation}%
but what is $s_{\max }?$

We remember that $s_{\max }$ is the maximum displacement of a particle of
fluid from its equilibrium position. We plotted this using a bar graph to
show displacement from the equilibrium position for our molecules. As we
push the piston in and out we will get something like this. \FRAME{dhF}{%
3.1211in}{4.9035in}{0pt}{}{}{Figure}{\special{language "Scientific
Word";type "GRAPHIC";maintain-aspect-ratio TRUE;display "USEDEF";valid_file
"T";width 3.1211in;height 4.9035in;depth 0pt;original-width
6.3676in;original-height 10.0336in;cropleft "0";croptop "1";cropright
"1";cropbottom "0";tempfilename 'PQXXQXQO.wmf';tempfile-properties "XPR";}}%
found before that we get something that looks like a sine wave, but remember
what the bars represent. They represent the displacement from original
position. \FRAME{dhF}{2.4613in}{1.849in}{0pt}{}{}{Figure}{\special{language
"Scientific Word";type "GRAPHIC";maintain-aspect-ratio TRUE;display
"USEDEF";valid_file "T";width 2.4613in;height 1.849in;depth
0pt;original-width 5.0004in;original-height 3.7498in;cropleft "0";croptop
"1";cropright "1";cropbottom "0";tempfilename
'PQXXQXQP.wmf';tempfile-properties "XPR";}}We don't usually draw bar graphs,
we usually just draw the sine wave.

\FRAME{dtbpF}{2.3315in}{1.9294in}{0in}{}{}{Figure}{\special{language
"Scientific Word";type "GRAPHIC";maintain-aspect-ratio TRUE;display
"USEDEF";valid_file "T";width 2.3315in;height 1.9294in;depth
0in;original-width 2.2917in;original-height 1.8913in;cropleft "0";croptop
"1";cropright "1";cropbottom "0";tempfilename
'PQXXQXQQ.wmf';tempfile-properties "XPR";}}

The variation of the gas pressure $\Delta P$ measured from its equilibrium
is also periodic

\FRAME{dhF}{2.6083in}{2.8072in}{0pt}{}{}{Figure}{\special{language
"Scientific Word";type "GRAPHIC";maintain-aspect-ratio TRUE;display
"USEDEF";valid_file "T";width 2.6083in;height 2.8072in;depth
0pt;original-width 6.02in;original-height 6.4809in;cropleft "0";croptop
"1";cropright "1";cropbottom "0";tempfilename
'PQXXQXQR.wmf';tempfile-properties "XPR";}}which is why we often refer to a
sound wave as a pressure wave. Think of when the wave gets to your ear. the
wave consists of a group of particles all headed for your ear drum. When
they hit, they exert a force. Pressure is a force spread over an area, 
\[
P=\frac{F}{A} 
\]%
so in a sense, we hear changes in air pressure!

\section{Loudness and frequency}

%TCIMACRO{%
%\TeXButton{Frequency Range Demo}{\marginpar {
%\hspace{-0.5in}
%\begin{minipage}[t]{1in}
%\small{Frequency Range Demo}
%\end{minipage}
%}}}%
%BeginExpansion
\marginpar {
\hspace{-0.5in}
\begin{minipage}[t]{1in}
\small{Frequency Range Demo}
\end{minipage}
}%
%EndExpansion
\FRAME{dtbpFU}{2.3328in}{2.3647in}{0pt}{\Qcb{Robinson-Dadson equal loudness
curves (Image in the Public Domain courtesy Lindosland)}}{}{Figure}{\special%
{language "Scientific Word";type "GRAPHIC";maintain-aspect-ratio
TRUE;display "USEDEF";valid_file "T";width 2.3328in;height 2.3647in;depth
0pt;original-width 2.3487in;original-height 2.3815in;cropleft "0";croptop
"1";cropright "1";cropbottom "0";tempfilename
'PQXXQXQS.wmf';tempfile-properties "XPR";}}

Our ears are not designed to be quantitative scientific instruments (though
they are truly amazing in their range and ability). Sounds with the same
intensity at different frequencies do not appear to us to have the same
loudness. The frequency response graph above shows how this relationship
works for test subjects\footnote{%
Ones that have not gone to Guns n Roses concerts}.

\section{Boundaries}

Suppose two pulses travel in the same medium, say, on a rope, and they
approach a different rope with a different linear mass density. If the new
rope is heavier, we expect the wave speed to slow down. So as one pulse
reaches the boundary, it will go slower. This allows the second pulse to
catch-up before it too slows down at the boundary.

Now suppose a sinusoidal wave approaches the boundary. We can envision the
crests like pulses, and we expect the first crest to slow down when it
reaches the boundary, letting the other crests catch up. Once the wave
passes the boundary, the crests will be closer together. The wavelength
changes as we move to the slower medium.

But does the frequency change? We know that 
\[
v=\lambda f 
\]%
so%
\[
f=\frac{v}{\lambda } 
\]%
both the speed and the wavelength have changed, but did they change
proportionately so $f$ is constant? This must be so. Think that the change
in wavelength is due to the relative speed of the wave in the two media. If $%
\Delta v$ is small the change in $\lambda $ will be small because the crests
are not delayed too long. If $\Delta v$ is large, the crests are delayed by
a large amount and so the change in $\lambda $ is large. We won't derive the
fact that $f$ is constant, but we can see that is is very believable that it
is true.\FRAME{dhF}{2.2727in}{1.2644in}{0pt}{}{}{Figure}{\special{language
"Scientific Word";type "GRAPHIC";maintain-aspect-ratio TRUE;display
"USEDEF";valid_file "T";width 2.2727in;height 1.2644in;depth
0pt;original-width 3.4618in;original-height 1.9147in;cropleft "0";croptop
"1";cropright "1";cropbottom "0";tempfilename
'PQXXQXQT.wmf';tempfile-properties "XPR";}}

This is true for all waves, even light. When a wave crosses a boundary from
a fast to a slow or a slow to a fast medium, $\lambda $ will change and $f$
will remain constant.

\section{Waves in Fields-Light}

Sound is a wave in matter, but what is light? It will really take the rest
of the course (and then some) to answer this question. But we know that
light can travel through a vacuum. Therefore, light can't be a wave in some
type of matter. We will find later in this course that there exists
something called an electromagnetic field created by charged particles. It
turns out that light seems to be a wave in this electromagnetic field. It
will take us a while to fully understand this concept--it is done in
PH220--but don't worry. Physicists knew that light was a wave for almost 80
years before the electric field was shown to be the medium. We can do a lot
just knowing light behaves like a wave.

This makes the light we see just one small part of a whole class of waves
that are possible in this electromagnetic field medium. Radio waves, and
microwaves, and x-rays are all just different types of electromagnetic
waves. The next figure show where all of these electromagnetic waves fit
ordered by wavelength (and frequency).\FRAME{dtbpFU}{5.3048in}{0.4013in}{0pt%
}{\Qcb{{\protect\small Image Courtesy NASA}}}{}{Figure}{\special{language
"Scientific Word";type "GRAPHIC";maintain-aspect-ratio TRUE;display
"USEDEF";valid_file "T";width 5.3048in;height 0.4013in;depth
0pt;original-width 7.2032in;original-height 0.518in;cropleft "0";croptop
"1";cropright "1";cropbottom "0";tempfilename
'PQXXQXQU.wmf';tempfile-properties "XPR";}}

There is something very unique about this electromagnetic field medium. The
waves in this medium travel at a constant speed no matter what frame of
reference we are in. This fact lead to the formation of the Special Theory
of Relativity and the famous equation 
\[
E=mc^{2} 
\]%
where $c$ is this speed of light%
\[
c=299792458\frac{\unit{m}}{\unit{s}} 
\]

Light does slow down when it enters a material, like glass, or even air.
This is not really because it moves slower. what happens is that light is
absorbed by the electrons in the atoms of the material substance. The
electron temporarily takes up all the energy from a bit of the light wave.
But only temporarily. It eventually has to give up the energy a light wave
is reformed coming from the atoms. But it has lost some time in the process,
so it's average speed is less.\footnote{%
There is really a superposition of what is left of the original light wave,
and the new light wave generated by the atoms. But we have not gotten to
superposition yet. So this explanation will have to do for now.} How much
less depends on how long the electrons in the atoms can hang on to the
light. Each substance is different.

We can devise a way to express how much slower light will appear to go in a
substance using the ratio%
\[
\frac{c}{v} 
\]%
the ratio of the speed of light, $c,$ to the average speed in the substance, 
$v.$ This ratio is so useful that we give it a name, the \emph{index of
refraction}.%
\[
n=\frac{c}{v} 
\]

In PH121, we used the concept of energy to do problems (specifically to make
hard problems easier!). We should expect to be able to use the concept of
energy with our waves. After all, the medium must transfer energy to make a
wave, so waves must transfer energy! We will take up the topics of energy
and waves in our next lecture.

\chapter{Power, Doppler Effect and Superposition}

%TCIMACRO{%
%\TeXButton{Fundamental Concepts}{\hspace{-1.3in}{\Large Fundamental Concepts\vspace{0.25in}}}}%
%BeginExpansion
\hspace{-1.3in}{\Large Fundamental Concepts\vspace{0.25in}}%
%EndExpansion

\begin{itemize}
\item Because waves are three dimensional, describing the power or energy
delivered per time of the wave is not enough, We describe how spatially
spread out that power is. We call this spread power \emph{intensity}

\item We don't hear a doubling of intensity as twice as loud. To
mathematically describe how our ears work we use decibels

\item When we observe waves from different reference frames, we observe a
apparent frequency shift. This is called the \emph{Doppler effect}

\item If we make more than one wave in a medium, the waves \textquotedblleft
add up\textquotedblright\ or \emph{superimpose.}
\end{itemize}

\section{Power and Intensity}

We know that energy is being transferred by the wave, whether it is a light
or sound wave. We should wonder, how fast is energy transferring. This can
mean the difference between a warm ray of sin on a cool spring day and being
burned by a laser beam. We will start by considering the rate of energy
transfer, \emph{power}. The concept of power should be familiar to us from
PH121. We can find the power as the rate at which energy is transferred.%
\[
\mathcal{P}=\frac{\Delta E}{\Delta t} 
\]%
But if we consider our spherical wave from a point source, we can see that
this description isn't good enough. In the next figure we have two
professors. \FRAME{dtbpF}{2.8426in}{2.3488in}{0pt}{}{}{Figure}{\special%
{language "Scientific Word";type "GRAPHIC";maintain-aspect-ratio
TRUE;display "USEDEF";valid_file "T";width 2.8426in;height 2.3488in;depth
0pt;original-width 2.8003in;original-height 2.3082in;cropleft "0";croptop
"1";cropright "1";cropbottom "0";tempfilename
'PQXXQXQV.wmf';tempfile-properties "XPR";}}Thinking from our experience we
would say that the sound will seem louder for professor A. This is because
the energy in the sound wave is being spread over the surface of the wave,
and that surface is getting bigger as the wave moves outward. The energy is
more spread out by the time it gets to professor B.

\subsection{Intensity}

%TCIMACRO{%
%\TeXButton{Tuning Fork Demo}{\marginpar {
%\hspace{-0.5in}
%\begin{minipage}[t]{1in}
%\small{Tuning Fork Demo}
%\end{minipage}
%}}}%
%BeginExpansion
\marginpar {
\hspace{-0.5in}
\begin{minipage}[t]{1in}
\small{Tuning Fork Demo}
\end{minipage}
}%
%EndExpansion
To be able to describe how much energy we get from our wave we need define
something new.%
\begin{equation}
\mathcal{I}\equiv \frac{\mathcal{P}}{A}
\end{equation}%
that is, the power divided by an area. But what does it mean?

Consider a point source.\FRAME{dtbpF}{2.4751in}{2.2485in}{0in}{}{}{Figure}{%
\special{language "Scientific Word";type "GRAPHIC";maintain-aspect-ratio
TRUE;display "USEDEF";valid_file "T";width 2.4751in;height 2.2485in;depth
0in;original-width 2.4336in;original-height 2.2087in;cropleft "0";croptop
"1";cropright "1";cropbottom "0";tempfilename
'PQXXQXQW.wmf';tempfile-properties "XPR";}}it sends out waves in all
directions. The wave crests will define a sphere around the points source
(the figure shows a cross section but remember it is a wave from a point
source, so we are really drawing concentric spheres like balloons inside of
balloons.). Then form our point source%
\begin{equation}
\mathcal{I}=\frac{\mathcal{P}}{4\pi r^{2}}
\end{equation}%
As the wave travels, its the power per unit area decreases with the square
of the distance (think gravity) because the area is getting larger.\FRAME{dhF%
}{2.655in}{1.8273in}{0pt}{}{}{Figure}{\special{language "Scientific
Word";type "GRAPHIC";maintain-aspect-ratio TRUE;display "USEDEF";valid_file
"T";width 2.655in;height 1.8273in;depth 0pt;original-width
2.6135in;original-height 1.7902in;cropleft "0";croptop "1";cropright
"1";cropbottom "0";tempfilename 'PQXXQXQX.wmf';tempfile-properties "XPR";}}

This quantity that tells us how spread out our power has become is called
the \emph{intensity} of the wave. Professor A would agree with us that the
wave he heard was more intense than the wave heard by Professor B. That is
because the wave was less spread out for Professor A.

Suppose we cup our hand to our ear. What are we doing? We are increasing the
area of our ear. Our ears work by transferring the energy of the sound wave
to a mechanical-electro-chemical device that creates a nerve signal.%
\footnote{%
The inner hair cells in the organ of Corti in the cochlea.} The more energy,
the stronger the signal. If we are a distance $r$ away from the source of
the sound then the intensity is 
\[
\mathcal{I}=\frac{\mathcal{P}_{source}}{A_{wave}} 
\]%
But we are collecting the sound wave with another area, the area of our
hand. The power received is 
\begin{eqnarray*}
\mathcal{P}_{received} &=&\mathcal{I}A_{hand} \\
&=&\frac{A_{hand}}{A_{wave}}\mathcal{P}_{source}
\end{eqnarray*}%
and we can see that, indeed, the larger the hand, the more power, and
therefore more energy we collect. This is the idea behind a dish antenna for
communications and the idea behind the acoustic dish microphones we see at
sporting events. In next figure, we can see that it would take an
increasingly larger dish to maintain the same power gathering capability as
we get farther from the source.\FRAME{dhF}{3.0312in}{2.2131in}{0pt}{}{}{%
Figure}{\special{language "Scientific Word";type
"GRAPHIC";maintain-aspect-ratio TRUE;display "USEDEF";valid_file "T";width
3.0312in;height 2.2131in;depth 0pt;original-width 4.9355in;original-height
3.595in;cropleft "0";croptop "1";cropright "1";cropbottom "0";tempfilename
'PQXXQXQY.wmf';tempfile-properties "XPR";}}

\subsection{Sound Levels in Decibels}

Our Design Engineer made an interesting choice in building us. We need to
hear very faint sounds, and very loud sounds too. In order to make us able
to hear the soft sounds without causing extreme discomfort when we hear the
loud, He gave up linearity. That is, we don't hear twice the sound intensity
as twice as loud.

The mathematical expression that matches our perception of loudness to the
intensity is 
\begin{equation}
\beta =10\log _{10}\left( \frac{I}{I_{o}}\right)
\end{equation}

where the quantity $I_{o}$ is a reference intensity.

%TCIMACRO{%
%\TeXButton{Sound Meter Demo}{\marginpar {
%\hspace{-0.5in}
%\begin{minipage}[t]{1in}
%\small{Sound Meter Demo}
%\end{minipage}
%}}}%
%BeginExpansion
\marginpar {
\hspace{-0.5in}
\begin{minipage}[t]{1in}
\small{Sound Meter Demo}
\end{minipage}
}%
%EndExpansion
We call $\beta $ the \emph{sound level}. The quantity $I_{o}$ we choose to
be the \emph{threshold of hearing},

\[
I_{o}=1\times 10^{-12}\frac{\unit{W}}{\unit{m}^{2}} 
\]%
the intensity that is just barely audible. Measured this way, we say that
intensity is in units of decibels (dB).The decibel, is useful because it can
describe a non-linear response in a linear way that is easy to match to our
human experience. But it is tricky because, like radians, it is
dimensionless because we are comparing two intensities $\left(
I/I_{o}\right) .$ We expect the same strange unit behavior with dB that we
see with radians.

Suppose we double the intensity by a factor of $2.$%
\begin{eqnarray*}
\beta &=&10\log _{10}\left( \frac{2I_{o}}{I_{o}}\right) \\
&=&10\log _{10}2 \\
&=&\allowbreak 3.\,\allowbreak 010\,3dB
\end{eqnarray*}%
The sound intensity level is not twice as large, but only $3dB$ larger. It
is a tiny increase. This is what we hear. A good rule to remember is that $%
3dB$ corresponds to a doubling of the intensity.

The tables that follow give some common sounds in units of dB and $\unit{W}/%
\unit{m}^{2}.$ Just for reference, I have measured a Guns n Roses concert at 
$120$ dB outside the stadium.

\[
\begin{tabular}{|l|l|}
\hline
Sound & Sound Level (dB) \\ \hline
Jet Airplane at 30m & 140 \\ \hline
Rock Concert & 120 \\ \hline
Siren at 30m & 100 \\ \hline
Car interior when Traveling 60mi/hr & 90 \\ \hline
Street Traffic & 70 \\ \hline
Talk at 30cm & 65 \\ \hline
Whisper & 20 \\ \hline
Rustle of Leaves & 10 \\ \hline
Quietest thing we can hear (Io) & 0 \\ \hline
\end{tabular}%
\]

\section{Doppler Effect}

%TCIMACRO{%
%\TeXButton{Doppler Ball Demo}{\marginpar {
%\hspace{-0.5in}
%\begin{minipage}[t]{1in}
%\small{Doppler Ball Demo}
%\end{minipage}
%}}}%
%BeginExpansion
\marginpar {
\hspace{-0.5in}
\begin{minipage}[t]{1in}
\small{Doppler Ball Demo}
\end{minipage}
}%
%EndExpansion
Let's start by considering an inertial reference frame (remember this from
Dynamics/PH121?)

Suppose we pick two reference frames, one traveling with a velocity $v_{r}$
with respect to the other. Let's also place them far away from any other
object.

\FRAME{dtbpF}{3.237in}{1.2505in}{0in}{}{}{Figure}{\special{language
"Scientific Word";type "GRAPHIC";maintain-aspect-ratio TRUE;display
"USEDEF";valid_file "T";width 3.237in;height 1.2505in;depth
0in;original-width 3.192in;original-height 1.2159in;cropleft "0";croptop
"1";cropright "1";cropbottom "0";tempfilename
'PQXXQXQZ.wmf';tempfile-properties "XPR";}}

Person $A$ sees himself as stationary and sees person $B$ traveling with
velocity $v_{x}.$ Person $B$ sees himself as stationary, and person $A$
traveling with velocity $-v_{x}.$

In looking at this situation it is the \emph{relative} speed $v_{x}$ that we
must consider. We recall that we could write the speed of guy $b$ as seen
from platform $A$ as 
\[
v_{bA}=v_{bB}+v_{BA} 
\]%
where $v_{bB}$ is guy b's speed seen from platform $B$ and $v_{BA}$ is the
speed of platform $B$ as seen from platform $A.$ In this case $v_{BA}=v_{x}.$
Now suppose we have a wave generator (a point source) creating spherical
waves. Let the point source be at rest. 
%TCIMACRO{%
%\TeXButton{BYU Demo}{\marginpar {
%\hspace{-0.5in}
%\begin{minipage}[t]{1in}
%\small{BYU Demo}
%\end{minipage}
%}}}%
%BeginExpansion
\marginpar {
\hspace{-0.5in}
\begin{minipage}[t]{1in}
\small{BYU Demo}
\end{minipage}
}%
%EndExpansion
We will call this point source an emitter and use a subscript $e$ for it. 
\FRAME{dtbpF}{3.0467in}{1.7322in}{0pt}{}{}{Figure}{\special{language
"Scientific Word";type "GRAPHIC";maintain-aspect-ratio TRUE;display
"USEDEF";valid_file "T";width 3.0467in;height 1.7322in;depth
0pt;original-width 4.0672in;original-height 2.3004in;cropleft "0";croptop
"1";cropright "1";cropbottom "0";tempfilename
'PQXXQXR0.wmf';tempfile-properties "XPR";}}Let's also assume a detector. If
the detector is stationary with respect to the emitter so the detector is in
the emitter's reference frame, it sees a frequency of the wave in the
emitter frame, $f_{we}.$ But lets have the detector move relative to the
emitter.%
%TCIMACRO{%
%\TeXButton{Move George}{\marginpar {
%\hspace{-0.5in}
%\begin{minipage}[t]{1in}
%\small{Move George}
%\end{minipage}
%}}}%
%BeginExpansion
\marginpar {
\hspace{-0.5in}
\begin{minipage}[t]{1in}
\small{Move George}
\end{minipage}
}%
%EndExpansion

\FRAME{dtbpF}{2.5495in}{2.45in}{0in}{}{}{Figure}{\special{language
"Scientific Word";type "GRAPHIC";maintain-aspect-ratio TRUE;display
"USEDEF";valid_file "T";width 2.5495in;height 2.45in;depth
0in;original-width 2.5088in;original-height 2.4085in;cropleft "0";croptop
"1";cropright "1";cropbottom "0";tempfilename
'PQXXQXR1.wmf';tempfile-properties "XPR";}}Remember, that the frequency is
the number of crests that pass by a given point in a unit time. Does the
moving detector see the same number of crests per unit time as when it was
stationary?

No, the frequency appears to be higher! How about if we let the detector
move the other way?

\FRAME{dtbpF}{2.1223in}{2.0384in}{0in}{}{}{Figure}{\special{language
"Scientific Word";type "GRAPHIC";maintain-aspect-ratio TRUE;display
"USEDEF";valid_file "T";width 2.1223in;height 2.0384in;depth
0in;original-width 2.0833in;original-height 2.0003in;cropleft "0";croptop
"1";cropright "1";cropbottom "0";tempfilename
'PQXXQXR2.wmf';tempfile-properties "XPR";}}Again the frequency seen by the
detector is different, but this time lower.

We can quantify this change. Take our usual variables $f_{we},$ $\lambda
_{we},$ and the velocity of sound $v_{sound}=v_{we}$ for the wave speed of
the sound wave because the air itself is not moving with respect to the
emitter and the wave travels through the air. When the detector moves toward
the source, it sees the wave velocity as 
\begin{equation}
v_{wd}=v_{we}+v_{de}
\end{equation}%
where $v_{wd}$ is the velocity of the wave in the detector frame, and $%
v_{de} $ is the velocity of the detector in the emitter frame. The
wavelength will not be changed, so 
\[
v_{we}=\lambda _{we}f_{we} 
\]%
can just be written as 
\[
v_{we}=\lambda f_{we} 
\]%
which tells us the frequency must change.%
\[
f_{wd}=\frac{v_{wd}}{\lambda }=\frac{v_{we}+v_{de}}{\lambda } 
\]%
We can eliminate $\lambda $ from this expression for the change in $f$ by
using $v_{we}=\lambda f_{we}$ again, this time solving for $\lambda $ we get%
\[
\lambda =\frac{v_{we}}{f_{we}} 
\]%
and substitute this into our $f_{wd}$ equation%
\[
f_{wd}=\frac{v_{we}+v_{de}}{\frac{v_{we}}{f_{we}}} 
\]%
or, after rearranging%
\begin{equation}
f_{wd}=\frac{v_{we}+v_{de}}{v_{we}}f_{we}\text{ \qquad detector moving
toward the emitter}
\end{equation}%
or recognizing that $v_{we}$ is the speed of sound in the stationary frame
of the emitter 
\begin{equation}
f_{wd}=\frac{v_{sound}+v_{de}}{v_{sound}}f_{we}\text{ \qquad detector moving
toward the emitter}
\end{equation}

%TCIMACRO{%
%\TeXButton{Change the Demo}{\marginpar {
%\hspace{-0.5in}
%\begin{minipage}[t]{1in}
%\small{Change the Demo}
%\end{minipage}
%}}}%
%BeginExpansion
\marginpar {
\hspace{-0.5in}
\begin{minipage}[t]{1in}
\small{Change the Demo}
\end{minipage}
}%
%EndExpansion
Now if the detector is going the other way%
\[
v_{wd}=v_{we}-v_{de} 
\]%
and the same reasoning gives%
\begin{equation}
f_{wd}=\frac{v_{we}-v_{de}}{v_{we}}f_{we}\text{ \qquad detector moving away
from the emitter}
\end{equation}%
or 
\begin{equation}
f_{wd}=\frac{v_{sound}-v_{de}}{v_{sound}}f_{we}\text{ \qquad detector moving
away from the emitter}
\end{equation}

From our thinking about the motion of two inertial reference frames, we
should expect a similar situation if the detector is stationary and the
source moves. In this case the detector will see a different wavelength.

\FRAME{dtbpF}{1.5307in}{1.4754in}{0pt}{}{}{Figure}{\special{language
"Scientific Word";type "GRAPHIC";maintain-aspect-ratio TRUE;display
"USEDEF";valid_file "T";width 1.5307in;height 1.4754in;depth
0pt;original-width 1.4944in;original-height 1.4399in;cropleft "0";croptop
"1";cropright "1";cropbottom "0";tempfilename
'PQXXQXR3.wmf';tempfile-properties "XPR";}}In fact, if we measure the
distance between the crests we must account for the fact that the source
moved by an amount%
\[
\Delta x_{ed}=v_{ed}T_{we}=\frac{v_{ed}}{f_{we}} 
\]%
during one period of oscillation of the emitter before making the next
crest. We can see that in this case, $v_{ed}=v_{sound}$ because this time
the air is not moving with respect to the detector, so they are in the same
reference frame. Then the wavelength will be shorter by this amount! That
is, if we take the wavelength that we would get if the emitter were at rest,
and subtract $\Delta x_{ed}$ we should have our new wavelength at the
detector. Let's start by finding the wavelength we expect if the emitter
were at rest. 
\[
\lambda _{we,rest}=\frac{v_{we}}{f_{we}}=\frac{v_{sound}}{f_{we}} 
\]
This is because if the emitter were at rest, then $v_{we}=v_{sound.}$ But in
our case the emitter is moving, so we must subtract $\Delta x_{ed}$ from this%
\[
\lambda _{wd}=\lambda _{we,rest}-\Delta x_{ed} 
\]
Then the wavelength we would see at the detector would be%
\[
\lambda _{wd}=\lambda _{we,rest}-\frac{v_{ed}}{f_{we}} 
\]

Using 
\[
\lambda _{wd}=\frac{v_{wd}}{f_{wd}} 
\]%
once more, we can write the frequency in the detector frame as 
\begin{eqnarray*}
f_{wd} &=&\frac{v_{wd}}{\lambda _{wd}}=\frac{v_{wd}}{\lambda _{we,rest}-%
\frac{v_{ed}}{f_{we}}} \\
&=&\frac{v_{wd}}{\frac{v_{sound}}{f_{we}}-\frac{v_{ed}}{f_{we}}} \\
&&\frac{v_{wd}}{v_{sound}-v_{ed}}f_{we}
\end{eqnarray*}%
or, again with a little rearranging%
\begin{equation}
f_{wd}=\frac{v_{wd}}{v_{sound}-v_{ed}}f_{we}\qquad \text{emitter moving
toward detector}
\end{equation}%
Now recall that it is the detector that is stationary this time so $%
v_{sound}=v_{wd}$%
\begin{equation}
f_{wd}=\frac{v_{sound}}{v_{sound}-v_{ed}}f_{we}\qquad \text{emitter moving
toward detector}
\end{equation}%
When the source is moving away from the detector, \FRAME{dtbpF}{1.8507in}{%
1.7841in}{0pt}{}{}{Figure}{\special{language "Scientific Word";type
"GRAPHIC";maintain-aspect-ratio TRUE;display "USEDEF";valid_file "T";width
1.8507in;height 1.7841in;depth 0pt;original-width 1.8135in;original-height
1.7469in;cropleft "0";croptop "1";cropright "1";cropbottom "0";tempfilename
'PQXXQXR4.wmf';tempfile-properties "XPR";}}we expect the wavelength to be
larger. The same reasoning gives

\subsubsection{%
\protect\begin{equation}
f^{\prime }=\frac{v_{wd}}{v_{wd}+v_{ed}}f_{we}\qquad \text{emitter moving
away from detector} 
\protect\end{equation}%
or%
\protect\begin{equation}
f^{\prime }=\frac{v_{sound}}{v_{sound}+v_{ed}}f_{we}\qquad \text{emitter
moving away from detector} 
\protect\end{equation}%
Combined Doppler Equation}

We can combine these formulae to make one expression, but to do so we need
to remember what $v_{de}$ and $v_{ed}$ mean. The first was the speed of the
detector when the emitter was not moving. The second was the speed of the
emitter when the detector was not moving. But we are experienced with
relative motion. We should ask, \textquotedblleft not moving with respect to
what?\textquotedblright\ Let's envision a reference frame that is not tied
to either the emitter or detector. In this reference frame $v_{dR}$ is the
speed of the detector, and $v_{eR}$ is the speed of the emitter. In this $R$
reference frame our first Doppler equation for a moving detector with a
stationary emitter might be written as%
\begin{equation}
f_{wd}=\frac{v_{sound}\pm v_{dR}}{v_{sound}}f_{we}\text{ \qquad detector
moving emitter stationary in R frame}
\end{equation}%
and our second equation for a moving emitter with a stationary detector
might be written as%
\begin{equation}
f^{\prime }=\frac{v_{wd}}{v_{wd}\mp v_{eR}}f_{we}\qquad \text{emitter moving
detector stationary in R frame}
\end{equation}%
We could, of course have both the detector and emitter moving in the $R$
frame. This would combine both of our previous scenarios 
\begin{equation}
f_{wd}=\frac{v_{sound}\pm v_{dR}}{v_{sound}\mp v_{eR}}f_{we}
\end{equation}%
where we use the top sign for the speed when the mover is going toward the
non-mover.

With this view, that we are neither in the $e$ frame nor the $d$ frame but
in a separate $R$ frame, we could even drop the $R$ subscript without too
much confusion%
\begin{equation}
f_{wd}=\frac{v_{sound}\pm v_{d}}{v_{sound}\mp v_{e}}f_{we}
\end{equation}%
where now $v_{d}$ is the speed of the detector in the reference frame of the
observer, and $v_{e}$ is the speed of the emitter in the reference frame of
the observer. The quantity $f_{we}$ is still the frequency of the wave as
seen by the emitter and $f_{wd}$ is still the frequency of the wave recorded
by the detector.

\subsection{Doppler effect in light}

Light is also a wave, and so we would expect a Doppler shift in light.
Indeed we do see a Doppler shift when we look at moving objects. Here is an
optical spectrum of the Sun on the top and a spectrum of a similar start
moving away from us in the middle. The final spectrum is for a star moving
toward us.\FRAME{dtbpFU}{1.5757in}{1.3785in}{0pt}{\Qcb{{\protect\small Top:
Normal `dark' spectral line positions at rest. Middle: Source moving away
from observer. Bottom: Source moving towards observer. (Public domain image
courtesy NASA: http://www.jwst.nasa.gov/education/7Page45.pdf)}}}{}{Figure}{%
\special{language "Scientific Word";type "GRAPHIC";maintain-aspect-ratio
TRUE;display "USEDEF";valid_file "T";width 1.5757in;height 1.3785in;depth
0pt;original-width 1.5385in;original-height 1.343in;cropleft "0";croptop
"1";cropright "1";cropbottom "0";tempfilename
'PQXXQXR5.wmf';tempfile-properties "XPR";}}Note that the wavelength of the
lines is shifted toward the red part of the spectrum when the glowing object
moves away from us. This is equivalent to lowering of the frequency of a
truck engine noise as it goes away from us. The larger wavelengths indicate
a lower frequency of light because 
\[
f=\frac{c}{\lambda } 
\]%
This gives us a way to determine if distant stars and galaxies are moving
toward or away from us. We look for the chemical signature pattern of lines,
then see whether they are shifted to the red (moving away from us) or blue
(moving toward us) compared to the position in their spectrum of the Sun.
This photo is of some of the most distant galaxies that are moving very fast
away from us. Their redshift is very large.\FRAME{dhFU}{2.3748in}{2.6671in}{%
0pt}{\Qcb{{\protect\small High Redshift Galaxy Cluster shown here in false
color from the Spitzer Space Telescope. (Public domain image courtesy
NASA/JPL-Caltech/S.A. Stanford (UC Davis/LLNL)}}}{}{Figure}{\special%
{language "Scientific Word";type "GRAPHIC";maintain-aspect-ratio
TRUE;display "USEDEF";valid_file "T";width 2.3748in;height 2.6671in;depth
0pt;original-width 2.3359in;original-height 2.623in;cropleft "0";croptop
"1";cropright "1";cropbottom "0";tempfilename
'PQXXQXR6.wmf';tempfile-properties "XPR";}}

Deriving the Doppler equation for light is more tricky because the speed of
light is constant and the same in every reference frame. We really tackle
this in our PH279 class. So I\ will just quote the result here.

\begin{eqnarray}
\lambda _{-} &=&\lambda _{o}\sqrt{\frac{1+\frac{v}{c}}{1-\frac{v}{c}}}\text{ 
}\qquad \text{receding source} \\
\lambda _{-} &=&\lambda _{o}\sqrt{\frac{1-\frac{v}{c}}{1+\frac{v}{c}}}\text{ 
}\qquad \text{Approaching source}
\end{eqnarray}

\section{Superposition Principle}

%TCIMACRO{%
%\TeXButton{Wave Machine Demo}{\marginpar {
%\hspace{-0.5in}
%\begin{minipage}[t]{1in}
%\small{Wave Machine Demo}
%\end{minipage}
%}}}%
%BeginExpansion
\marginpar {
\hspace{-0.5in}
\begin{minipage}[t]{1in}
\small{Wave Machine Demo}
\end{minipage}
}%
%EndExpansion

What happens if we have more than one wave propagating in a medium? If you
remember being a little child in a bath tub, you will probably remember
making waves in the water. If you made a wave with each hand, the two waves
seemed to \textquotedblleft pile up\textquotedblright\ in the middle and
make a big splash. We should expect something like this for any kind of
wave. We call the \textquotedblleft piling up\textquotedblright\ of waves 
\emph{superposition.} The word literally means putting one wave on top of
another.

\begin{definition}
Superposition: If two or more traveling waves are moving through a medium,
the resultant wave formed at any point is the algebraic sum of the values of
the individual wave forms.
\end{definition}

So if we have 
\begin{equation}
y_{1}\left( x,t\right)
\end{equation}%
and 
\begin{equation}
y_{2}\left( x,t\right)
\end{equation}%
both propagating on a string, then we would see%
\begin{equation}
y_{r}\left( x,t\right) =y_{1}\left( x,t\right) +y_{2}\left( x,t\right)
\end{equation}

This is a fantastically simple way for the universe to act!

Let's look at an example. let's add the top wave (red) to the middle
wave(green). We get the bottom wave (purple)\FRAME{dhF}{2.77in}{3.3788in}{0pt%
}{}{}{Figure}{\special{language "Scientific Word";type
"GRAPHIC";maintain-aspect-ratio TRUE;display "USEDEF";valid_file "T";width
2.77in;height 3.3788in;depth 0pt;original-width 7.2705in;original-height
8.8816in;cropleft "0";croptop "1";cropright "1";cropbottom "0";tempfilename
'PQXXQXR7.wmf';tempfile-properties "XPR";}}Of course we are adding these in
the snapshot view. So this is all done for just one instant of time.

Let's see how to do this. \FRAME{dhF}{2.7121in}{2.1715in}{0pt}{}{}{Figure}{%
\special{language "Scientific Word";type "GRAPHIC";maintain-aspect-ratio
TRUE;display "USEDEF";valid_file "T";width 2.7121in;height 2.1715in;depth
0pt;original-width 6.576in;original-height 5.2598in;cropleft "0";croptop
"1";cropright "1";cropbottom "0";tempfilename
'PQXXQXR8.wmf';tempfile-properties "XPR";}}Start at $x=-2.$ In the figure, I
drew a red bar to show the $y$ value at $x=-2$ for the red curve. Likewise,
I have a green bar sowing the value of $y$ at $x=-2$ for the green wave.
Note that this is negative. On the bottom graph, the bars have been
repeated, and we can see that the red bar minus the green bar brings us to
the value for the resulting wave at the point $x=-2.$ We need to do this at
every point along all the waves for this instant of time.

This is tedious by hand, so we won't generally do this calculation by hand.
But a computer can do it easily.

Note that this is really only true for \emph{linear} systems. Let's take the
example of a slinky. If we form two waves in the slinky, they behave
according to the superposition principle most of the time. But suppose we
make the amplitude of the individual waves large. They may travel
individually OK, but when the amplitudes add we may overstretch the slinky.
Then it would never return to it's original shape. The wave form would have
to change. Such a wave is not linear. There is a good rule of thumb for when
waves are linear.

\begin{Note}
A wave is generally linear when its amplitude is much smaller than its
wavelength.
\end{Note}

\section{Consequences of superposition}

Linear waves traveling in media can pass through each other without being
destroyed or altered!

\FRAME{dhFU}{1.7633in}{2.0817in}{0pt}{\Qcb{{\protect\small Constructive
Interference (Public Domain image by Inductiveload,
http://commons.wikimedia.org/wiki/File:Constructive\_interference.svg)}}}{}{%
Figure}{\special{language "Scientific Word";type
"GRAPHIC";maintain-aspect-ratio TRUE;display "USEDEF";valid_file "T";width
1.7633in;height 2.0817in;depth 0pt;original-width 1.7678in;original-height
2.0924in;cropleft "0";croptop "1";cropright "1";cropbottom "0";tempfilename
'PQXXQXR9.wmf';tempfile-properties "XPR";}}

Our wave on the string makes the string segments move in the $y$ direction.
Both waves do this. So putting the two waves together just makes the string
segments move more! There is a special name for what we observe

\begin{enumerate}
\item \emph{interference}: The combination of separate waves in the same
region of space to produce a resultant wave.
\end{enumerate}

What happens if one of the pulses is inverted?\FRAME{dhFU}{1.7828in}{2.2033in%
}{0pt}{\Qcb{{\protect\small Destructive Interference (Public Domain image by
Inductiveload,
http://commons.wikimedia.org/wiki/File:Destructive\_interference1.svg)}}}{}{%
Figure}{\special{language "Scientific Word";type
"GRAPHIC";maintain-aspect-ratio TRUE;display "USEDEF";valid_file "T";width
1.7828in;height 2.2033in;depth 0pt;original-width 1.5833in;original-height
1.9638in;cropleft "0";croptop "1";cropright "1";cropbottom "0";tempfilename
'PQXXQXRA.wmf';tempfile-properties "XPR";}}

When the two pulses meet, they \textquotedblleft cancel each other
out.\textquotedblright\ But do they go away? No! the energy is still there,
the string segment motions have just summed vectorially to zero, the energy
carried by each wave is still there. We have a few more definitions. The
type of interference we have just seen is the first

\begin{enumerate}
\item \emph{Destructive Interference}:\ Interference between waves when the
displacements caused by the two waves are opposite in direction

\item \emph{Constructive Interference}: interference between waves when the
displacements caused by the two waves are in the same direction
\end{enumerate}

The combination of waves is important for both scientists and engineers. In
engineering this is the hart of vibrometry. \FRAME{dhFU}{3.5639in}{2.3851in}{%
0pt}{\Qcb{{\protect\small Marshall and Cal Poly technicians wired the
NanoSail-D spacecraft to accelerometers, instruments which measure vibration
response during simulated launch conditions. Image couracy NASA, image in
the Public Domain.}}}{}{Figure}{\special{language "Scientific Word";type
"GRAPHIC";maintain-aspect-ratio TRUE;display "USEDEF";valid_file "T";width
3.5639in;height 2.3851in;depth 0pt;original-width 3.5172in;original-height
2.3454in;cropleft "0";croptop "1";cropright "1";cropbottom "0";tempfilename
'PQXXQXRB.wmf';tempfile-properties "XPR";}}Mechanical systems have moving
parts. These moving parts can be the disturbance that creates a wave. If
more than one wave crest arrives at a location in the device, the amplitude
at that location could become large. The oscillation of this part of the
device could rattle apart welds or bolts, destroying the device. Later, as
we study spectroscopy, we will see how to diagnose such a problem and hint
at how to correct it.

\section{Shock Waves}

What happens when the speed of the source is greater than the wave speed?

Remember that the wave speed depends only on the medium. Let's call the
crests of a wave the \emph{wave front}. In the picture below, a point source
is generating a wave and the red lines are the wave fronts.

When $v_{e}=v_{sound}$ the waves begin to pile up. If we allow $%
v_{e}>v_{sound}$ then the wave fronts are no longer generated within each
other. \FRAME{dtbpF}{2.3229in}{2.3151in}{0pt}{}{}{Figure}{\special{language
"Scientific Word";type "GRAPHIC";maintain-aspect-ratio TRUE;display
"USEDEF";valid_file "T";width 2.3229in;height 2.3151in;depth
0pt;original-width 2.2831in;original-height 2.2753in;cropleft "0";croptop
"1";cropright "1";cropbottom "0";tempfilename
'PQXXQXRC.wmf';tempfile-properties "XPR";}}The leading edge of the wave
fronts \textquotedblleft build up\textquotedblright\ to form a cone shape.
We recognize this as a superposition of the waves. This is constructive
interference. The half angle of this cone is called the \emph{Mach angle} 
\begin{equation}
\sin \theta =\frac{v_{sound}t}{v_{e}t}=\frac{v_{sound}}{v_{e}}
\end{equation}%
This ration $v_{sound}/v_{e}$ is called the Mach number and the conical wave
front is called a shock wave. We see them often in water

\FRAME{dhFU}{1.9558in}{1.3402in}{0pt}{\Qcb{{\protect\small Boat wakes as a
Doppler cone. Image courtacy US\ Navy. Image is in the Public Domain.}}}{}{%
Figure}{\special{language "Scientific Word";type
"GRAPHIC";maintain-aspect-ratio TRUE;display "USEDEF";valid_file "T";width
1.9558in;height 1.3402in;depth 0pt;original-width 1.9182in;original-height
1.305in;cropleft "0";croptop "1";cropright "1";cropbottom "0";tempfilename
'PQXXQXRD.wmf';tempfile-properties "XPR";}}

and hear them when jet aircraft go supersonic. 
%TCIMACRO{%
%\TeXButton{Dopler Movie}{\marginpar {
%\hspace{-0.5in}
%\begin{minipage}[t]{1in}
%\small{Dopler Movie}
%\end{minipage}
%}} }%
%BeginExpansion
\marginpar {
\hspace{-0.5in}
\begin{minipage}[t]{1in}
\small{Dopler Movie}
\end{minipage}
}
%EndExpansion
In the next figure we can see a picture of a T-38 breaking the sound
barrier. You can see the Mach cones, but notice that there are several!
Remember that a disturbance creates a wave. There are disturbances created
by the nose of the plane, the rudder, and the wings, and perhaps the cockpit
in this Schlieren photograph.\FRAME{dhFU}{1.593in}{1.2842in}{0pt}{\Qcb{Dr.
Leonard Weinstein's Schlieren photograph of a T-38 Talon at Mach 1.1,
altitude 13,700 feet, taken at NASA Langley Research Center, Wallops in
1993. Image Courtacy NASA, image is in the Public Domain.}}{}{Figure}{%
\special{language "Scientific Word";type "GRAPHIC";maintain-aspect-ratio
TRUE;display "USEDEF";valid_file "T";width 1.593in;height 1.2842in;depth
0pt;original-width 1.5575in;original-height 1.2488in;cropleft "0";croptop
"1";cropright "1";cropbottom "0";tempfilename
'PQXXQXRE.wmf';tempfile-properties "XPR";}}

Superposition is the basis of making music and we will see how this works in
our next lecture.

\chapter{Standing Waves}

Reading Assignment 21.2, 21.3

%TCIMACRO{%
%\TeXButton{Fundamental Concepts}{\hspace{-1.3in}{\Large Fundamental Concepts\vspace{0.25in}}}}%
%BeginExpansion
\hspace{-1.3in}{\Large Fundamental Concepts\vspace{0.25in}}%
%EndExpansion

\section{Mathematical Description of Superposition}

We know what superposition is, but we don't really want to add values for
millions of points in a medium to find out what a combination of waves will
look like. At the very least, we want to make a computer do that (and
programs like OpenFoam do something very akin to this!). But where we can,
we would like to combine wave functions algebraically. Let's see how this
can work.

Lets define two wave functions%
\[
y_{1}=y_{\max }\sin \left( kx-\omega t\right) 
\]%
and 
\[
y_{2}=y_{\max }\sin \left( kx-\omega t+\phi _{o}\right) 
\]%
These are two waves with the same frequency and wave number traveling the
same direction in the medium, but they at $t=0$ the $y$ values are not the
same because of the phase shift. The graph of $y_{2}$ is shifted by an
amount $\phi _{o}.$

I will pick some values for the constants

\[
\begin{tabular}{l}
$\lambda =2$ \\ 
$k=\frac{2\pi }{\lambda }$ \\ 
$\omega =1$ \\ 
$\phi _{o}=\frac{\pi }{6}$ \\ 
$t=0$ \\ 
$A=1$%
\end{tabular}%
\]

then for $y_{1}$ we have

\begin{eqnarray*}
y_{1} &=&\left( 1\right) \sin \left( \frac{2\pi }{\lambda }x-\left( 1\right)
t\right) \\
&=&\sin \left( \frac{2\pi }{2}x-\left( 1\right) t\right) \\
&=&\sin \left( \pi x-t\right)
\end{eqnarray*}%
here is a plot of the wave function, $y_{1}$\FRAME{dtbpFX}{2.9689in}{1.2816in%
}{0pt}{}{}{Plot}{\special{language "Scientific Word";type "MAPLEPLOT";width
2.9689in;height 1.2816in;depth 0pt;display "USEDEF";plot_snapshots
TRUE;mustRecompute FALSE;lastEngine "MuPAD";xmin "0";xmax
"5.001000";xviewmin "0";xviewmax "5.001000";yviewmin "-2";yviewmax
"2";viewset"XY";rangeset"X";plottype 4;axesFont "Times New
Roman,12,0000000000,useDefault,normal";numpoints 100;plotstyle
"patch";axesstyle "normal";axestips FALSE;xis \TEXUX{x};var1name
\TEXUX{$x$};function \TEXUX{$\allowbreak \sin \pi x$};linecolor
"green";linestyle 1;pointstyle "point";linethickness 3;lineAttributes
"Solid";var1range "0,5.001000";num-x-gridlines 100;curveColor
"[flat::RGB:0x00006000]";curveStyle "Line";VCamFile
'PQXXR22W.xvz';valid_file "T";tempfilename
'PQXXQXRF.wmf';tempfile-properties "XPR";}}

Now let's consider $y_{2.}$ Using the values we chose, $y_{2}$ can be
written as%
\begin{eqnarray*}
y_{2} &=&y_{\max }\sin \left( kx-\omega t+\phi _{o}\right) \\
&=&\sin \left( \pi x-t+\frac{\pi }{6}\right)
\end{eqnarray*}%
which looks like this\FRAME{dtbpFX}{3.0519in}{1.2427in}{0pt}{}{}{Plot}{%
\special{language "Scientific Word";type "MAPLEPLOT";width 3.0519in;height
1.2427in;depth 0pt;display "USEDEF";plot_snapshots TRUE;mustRecompute
FALSE;lastEngine "MuPAD";xmin "0";xmax "5.001000";xviewmin "0";xviewmax
"5.001000";yviewmin "-2";yviewmax "2";viewset"XY";rangeset"X";plottype
4;axesFont "Times New Roman,12,0000000000,useDefault,normal";numpoints
100;plotstyle "patch";axesstyle "normal";axestips FALSE;xis
\TEXUX{x};var1name \TEXUX{$x$};function \TEXUX{$\sin \left( \frac{1}{6}\pi
+\pi x\right) $};linecolor "red";linestyle 1;pointstyle
"point";linethickness 3;lineAttributes "Solid";var1range
"0,5.001000";num-x-gridlines 100;curveColor
"[flat::RGB:0x00ff0000]";curveStyle "Line";VCamFile
'PQXXR22V.xvz';valid_file "T";tempfilename
'PQXXQXRG.wmf';tempfile-properties "XPR";}}What does it look like if we add
these waves using superposition? Symbolically we have%
\begin{equation}
y_{r}=y_{\max }\sin \left( kx-\omega t\right) +y_{\max }\sin \left(
kx-\omega t+\phi _{o}\right)  \label{Superposition basic}
\end{equation}

and putting in the numbers gives%
\[
y_{r}=\sin \left( \pi x-t\right) +\sin \left( \pi x-t+\frac{\pi }{6}\right) 
\]%
which is shown in the next graph.

\FRAME{dtbpFX}{3.0727in}{1.3292in}{0pt}{}{}{Plot}{\special{language
"Scientific Word";type "MAPLEPLOT";width 3.0727in;height 1.3292in;depth
0pt;display "USEDEF";plot_snapshots TRUE;mustRecompute FALSE;lastEngine
"MuPAD";xmin "0";xmax "5.001000";xviewmin "0";xviewmax "5.001000";yviewmin
"-2";yviewmax "2";viewset"XY";rangeset"X";plottype 4;axesFont "Times New
Roman,12,0000000000,useDefault,normal";numpoints 100;plotstyle
"patch";axesstyle "normal";axestips FALSE;xis \TEXUX{x};var1name
\TEXUX{$x$};function \TEXUX{$\sin \left( \frac{1}{6}\pi +\pi x\right) +\sin
\pi x$};linecolor "green";linestyle 1;pointstyle "point";linethickness
3;lineAttributes "Solid";var1range "0,5.001000";num-x-gridlines
100;curveColor "[flat::RGB:0x00008000]";curveStyle "Line";VCamFile
'PQXXR22U.xvz';valid_file "T";tempfilename
'PQXXQXRH.wmf';tempfile-properties "XPR";}}Notice that the wave form is
taller (larger amplitude). Noticed it is shifted along the $x$ axis.

We can find out by how much by rewriting $y_{r}.$ We need a trig identity%
\[
\sin a+\sin b=2\cos \left( \frac{a-b}{2}\right) \sin \left( \frac{a+b}{2}%
\right) 
\]%
Then let $a=kx-\omega t$ and $b=kx-\omega t+\phi $%
\begin{eqnarray*}
y_{r} &=&y_{\max }\sin \left( kx-\omega t\right) +A\sin \left( kx-\omega
t+\phi _{o}\right) \\
&=&2y_{\max }\cos \left( \frac{\left( kx-\omega t\right) -\left( kx-\omega
t+\phi _{o}\right) }{2}\right) \sin \left( \frac{\left( kx-\omega t\right)
+\left( kx-\omega t+\phi _{o}\right) }{2}\right) \\
&=&2y_{\max }\cos \left( \frac{-\phi _{o}}{2}\right) \sin \left( \frac{%
2kx-2\omega t+\phi _{o}}{2}\right) \\
&=&2y_{\max }\cos \left( \frac{-\phi _{o}}{2}\right) \sin \left( kx-\omega t+%
\frac{\phi _{o}}{2}\right) \\
&=&2y_{\max }\cos \left( \frac{\phi _{o}}{2}\right) \sin \left( kx-\omega t+%
\frac{\phi _{o}}{2}\right)
\end{eqnarray*}

where we use the fact that $\cos \left( -\theta \right) =\cos \left( \theta
\right) .$

Let's look at the parts of this expression. First take the sine part.

\begin{equation}
\sin \left( kx-\omega t+\frac{\phi _{o}}{2}\right)
\end{equation}%
This part is a traveling wave with the same $k$ and $\omega $ as our
original waves, but it has a phase of $\phi _{o}/2.$ So our combined wave is
shifted by $\phi _{o}/2$ or half the phase shift of $y_{2}.$

Now let's look at other factor%
\begin{equation}
2y_{\max }\cos \left( \frac{\phi _{o}}{2}\right)
\end{equation}%
The sine part of our wave equation is multiplied by all of this factor. So
all of this part is the new amplitude. It has a maximum value when $\phi
_{o}=0$

\FRAME{dtbpFX}{2.5313in}{1.062in}{0pt}{}{}{Plot}{\special{language
"Scientific Word";type "MAPLEPLOT";width 2.5313in;height 1.062in;depth
0pt;display "USEDEF";plot_snapshots TRUE;mustRecompute FALSE;lastEngine
"MuPAD";xmin "-5";xmax "5";xviewmin "-5.0010000010002";xviewmax
"5.0010000010002";yviewmin "-1.60264739640757";yviewmax "2";plottype
4;axesFont "Times New Roman,12,0000000000,useDefault,normal";numpoints
100;plotstyle "patch";axesstyle "normal";axestips FALSE;xis
\TEXUX{x};var1name \TEXUX{$x$};function \TEXUX{$2\left( 1\right) \cos \left(
\frac{x}{2}\right) $};linecolor "blue";linestyle 1;pointstyle
"point";linethickness 1;lineAttributes "Solid";var1range
"-5,5";num-x-gridlines 100;curveColor "[flat::RGB:0x000000ff]";curveStyle
"Line";VCamFile 'PQXXR22T.xvz';valid_file "T";tempfilename
'PQXXQXRI.wmf';tempfile-properties "XPR";}}When $\phi _{o}=\pi ,$ then 
\[
2y_{\max }\cos \left( \frac{\pi }{2}\right) =0 
\]%
so when $\phi _{o}=0$ we have a new maximum amplitude of $2y_{\max }$ and
when $\phi _{o}=\pi $ we have a zero amplitude. Here is our wave for several
choices of $\phi _{o}.$

\FRAME{dtbpFX}{3.1254in}{1.177in}{0pt}{}{}{Plot}{\special{language
"Scientific Word";type "MAPLEPLOT";width 3.1254in;height 1.177in;depth
0pt;display "USEDEF";plot_snapshots TRUE;mustRecompute FALSE;lastEngine
"MuPAD";xmin "0";xmax "5.001000";xviewmin "0";xviewmax "5.001000";yviewmin
"-2";yviewmax "2";viewset"XY";rangeset"X";plottype 4;axesFont "Times New
Roman,12,0000000000,useDefault,normal";numpoints 100;plotstyle
"patch";axesstyle "normal";axestips FALSE;xis \TEXUX{x};var1name
\TEXUX{$x$};function \TEXUX{$\allowbreak 2\sin \left( \frac{1}{2}\left(
0\right) +\pi x\right) \cos \frac{1}{2}\left( 0\right) $};linecolor
"black";linestyle 1;pointstyle "point";linethickness 2;lineAttributes
"Solid";var1range "0,5.001000";num-x-gridlines 100;curveColor
"[flat::RGB:0000000000]";curveStyle "Line";function \TEXUX{$\allowbreak
2\sin \left( \frac{1}{2}\frac{\pi }{9}+\pi x\right) \cos
\frac{1}{2}\frac{\pi }{9}$};linecolor "blue";linestyle 1;pointstyle
"point";linethickness 2;lineAttributes "Solid";var1range
"0,5.001000";num-x-gridlines 100;curveColor
"[flat::RGB:0x00000080]";curveStyle "Line";function \TEXUX{$\allowbreak
2\sin \left( \frac{1}{2}\frac{2\pi }{9}+\pi x\right) \cos
\frac{1}{2}\frac{2\pi }{9}$};linecolor "blue";linestyle 1;pointstyle
"point";linethickness 2;lineAttributes "Solid";var1range
"0,5.001000";num-x-gridlines 100;curveColor
"[flat::RGB:0x000000ff]";curveStyle "Line";function \TEXUX{$\allowbreak
2\sin \left( \frac{1}{2}\frac{3\pi }{9}+\pi x\right) \cos
\frac{1}{2}\frac{3\pi }{9}$};linecolor "green";linestyle 1;pointstyle
"point";linethickness 2;lineAttributes "Solid";var1range
"0,5.001000";num-x-gridlines 100;curveColor
"[flat::RGB:0x00008000]";curveStyle "Line";function \TEXUX{$2\sin \left(
\frac{1}{2}\frac{4\pi }{9}+\pi x\right) \cos \frac{1}{2}\frac{4\pi
}{9}$};linecolor "cyan";linestyle 1;pointstyle "point";linethickness
2;lineAttributes "Solid";var1range "0,5.001000";num-x-gridlines
100;curveColor "[flat::RGB:0x00008080]";curveStyle "Line";function
\TEXUX{$\allowbreak 2\sin \left( \frac{1}{2}\frac{5\pi }{9}+\pi x\right)
\cos \frac{1}{2}\frac{5\pi }{9}$};linecolor "cyan";linestyle 1;pointstyle
"point";linethickness 2;lineAttributes "Solid";var1range
"0,5.001000";num-x-gridlines 100;curveColor
"[flat::RGB:0x000080c0]";curveStyle "Line";function \TEXUX{$2\sin \left(
\frac{1}{2}\frac{6\pi }{9}+\pi x\right) \cos \frac{1}{2}\frac{6\pi
}{9}$};linecolor "green";linestyle 1;pointstyle "point";linethickness
2;lineAttributes "Solid";var1range "0,5.001000";num-x-gridlines
100;curveColor "[flat::RGB:0x0000ff00]";curveStyle "Line";function
\TEXUX{$2\sin \left( \frac{1}{2}\frac{7\pi }{9}+\pi x\right) \cos
\frac{1}{2}\frac{7\pi }{9}$};linecolor "black";linestyle 1;pointstyle
"point";linethickness 2;lineAttributes "Solid";var1range
"0,5.001000";num-x-gridlines 100;curveColor
"[flat::RGB:0x00400000]";curveStyle "Line";function \TEXUX{$2\sin \left(
\frac{1}{2}\frac{8\pi }{9}+\pi x\right) \cos \frac{1}{2}\frac{8\pi
}{9}$};linecolor "black";linestyle 1;pointstyle "point";linethickness
2;lineAttributes "Solid";var1range "0,5.001000";num-x-gridlines
100;curveColor "[flat::RGB:0x00404040]";curveStyle "Line";function
\TEXUX{$2\sin \left( \frac{1}{2}\frac{9\pi }{9}+\pi x\right) \cos
\frac{1}{2}\frac{9\pi }{9}$};linecolor "gray";linestyle 1;pointstyle
"point";linethickness 2;lineAttributes "Solid";var1range
"0,5.001000";num-x-gridlines 100;curveColor
"[flat::RGB:0x00cd99ff]";curveStyle "Line";VCamFile
'PQXXR22S.xvz';valid_file "T";tempfilename
'PQXXQXRJ.wmf';tempfile-properties "XPR";}}

\section{Reflection and Transmission}

In our examples so far, we have not explained how we got two waves into a
medium. One way is to simply reflect one wave back on top of itself.

In class we will made pulses on a spring with one end of the rope fixed
(held by a class member). What happened when the pulse reached the end of
the rope?

\FRAME{dhF}{3.6063in}{4.0456in}{0pt}{}{}{Figure}{\special{language
"Scientific Word";type "GRAPHIC";maintain-aspect-ratio TRUE;display
"USEDEF";valid_file "T";width 3.6063in;height 4.0456in;depth
0pt;original-width 3.5587in;original-height 3.9963in;cropleft "0";croptop
"1";cropright "1";cropbottom "0";tempfilename
'PQXXQXRK.wmf';tempfile-properties "XPR";}}

\subsection{Case I: Fixed rope end.}

There is a big change in the medium at the end of the rope. The rope ends.
There is a person or (as in the next figure) some thing holding the rope in
place. This change in medium causes a reflection.

In the fixed end case, the pulse is inverted. Why?

In the next figure I\ have envisioned a rope made of small rope segments.

\FRAME{dtbpF}{3.0519in}{1.7711in}{0pt}{}{}{Figure}{\special{language
"Scientific Word";type "GRAPHIC";maintain-aspect-ratio TRUE;display
"USEDEF";valid_file "T";width 3.0519in;height 1.7711in;depth
0pt;original-width 9.6911in;original-height 5.6109in;cropleft "0";croptop
"1";cropright "1";cropbottom "0";tempfilename
'PQXXQXRL.wmf';tempfile-properties "XPR";}}The end of the rope pushes up on
the support (person in class, or nail in the figure). By Newton's third law
there must be a normal force downward on the rope end. Compressing the
molecules in the support stores potential energy in those compressed atoms.
They will release that potential energy and create kinetic energy in the
rope end. That kinetic energy will have a rope end velocity 
\[
K=\frac{1}{2}mv^{2} 
\]%
and that velocity will be downward. This will pull the rope down, inverting
the wave.

But what happens if the rope end is not fixed?

The rope end rises, and therefore there is no force exerted. The pulse (or
at least part of the pulse energy) is still reflected, but there is no
inversion because there was not downward force or no stored potential energy
in the support!\FRAME{dtbpF}{2.2771in}{1.3958in}{0pt}{}{}{Figure}{\special%
{language "Scientific Word";type "GRAPHIC";maintain-aspect-ratio
TRUE;display "USEDEF";valid_file "T";width 2.2771in;height 1.3958in;depth
0pt;original-width 9.1687in;original-height 5.6109in;cropleft "0";croptop
"1";cropright "1";cropbottom "0";tempfilename
'PQXXQXRM.wmf';tempfile-properties "XPR";}}

\subsection{Case III: Partially attached rope end}

Now lets tie the rope to another rope that is larger, more dense, than the
rope we have been using, what will happen?

The light end of the rope exerts a force on the heavy beginning of the new
rope. The atoms in the heavy rope will be pulled and compressed. The heavy
rope will move, but the heaviness of the rope will prevent them from going
very far. The fibers on the end of the heavy rope will build up potential
energy because their bonds will be compressed. They will push downward on
the light end of the rope. Once again the potential energy from the
compressed heavy rope atoms will transfer to kinetic energy in the light
rope end. Once again that light rope end will move downward. \FRAME{dtbpF}{%
2.6532in}{1.4857in}{0pt}{}{}{Figure}{\special{language "Scientific
Word";type "GRAPHIC";maintain-aspect-ratio TRUE;display "USEDEF";valid_file
"T";width 2.6532in;height 1.4857in;depth 0pt;original-width
3.8in;original-height 2.1162in;cropleft "0";croptop "1";cropright
"1";cropbottom "0";tempfilename 'PQXXQXRN.wmf';tempfile-properties "XPR";}}

We expect part of the energy to reflect back along the light rope, and this
pulse will be inverted. Notice that the heavy rope did move upward a little,
so there will also be a pulse on the heavy rope. Since this pulse formed
from the light rope pulling up the heavy rope, and the light rope atoms were
not compressed, this pulse will not be inverted. \FRAME{dtbpF}{2.8764in}{%
2.2572in}{0pt}{}{}{Figure}{\special{language "Scientific Word";type
"GRAPHIC";maintain-aspect-ratio TRUE;display "USEDEF";valid_file "T";width
2.8764in;height 2.2572in;depth 0pt;original-width 2.8331in;original-height
2.2174in;cropleft "0";croptop "1";cropright "1";cropbottom "0";tempfilename
'PQXXQXRO.wmf';tempfile-properties "XPR";}}

\section{Mathematical description of standing waves}

Now that we have a way to make two waves to superimpose, we can study the
special case of standing waves. 
%TCIMACRO{%
%\TeXButton{Standing Wave Demo}{\marginpar {
%\hspace{-0.5in}
%\begin{minipage}[t]{1in}
%\small{Standing Wave Demo}
%\end{minipage}
%}}}%
%BeginExpansion
\marginpar {
\hspace{-0.5in}
\begin{minipage}[t]{1in}
\small{Standing Wave Demo}
\end{minipage}
}%
%EndExpansion

A standing wave pattern is the result of the superposition of two traveling
waves with the same frequency going in opposite directions%
\begin{eqnarray*}
y_{1} &=&y_{\max }\sin \left( kx-\omega t\right) \\
y_{2} &=&y_{\max }\sin \left( kx+\omega t\right)
\end{eqnarray*}

The sum is 
\[
y_{r}=y_{1}+y_{2}=y_{\max }\sin \left( kx-\omega t\right) +A\sin \left(
kx+\omega t\right) 
\]%
To gain insight into what these two waves produce, we use another of our
favorite trig identities%
\[
\sin \left( a\pm b\right) =\sin \left( a\right) \cos \left( b\right) \pm
\cos \left( a\right) \sin \left( b\right) 
\]%
to get%
\begin{eqnarray*}
y_{r} &=&y_{\max }\sin \left( kx-\omega t\right) +y_{\max }\sin \left(
kx+\omega t\right) \\
&=&y_{\max }\sin \left( kx\right) \cos \left( \omega t\right) -y_{\max }\cos
\left( kx\right) \sin \left( \omega t\right) +y_{\max }\sin \left( kx\right)
\cos \left( \omega t\right) +y_{\max }\cos \left( kx\right) \sin \left(
\omega t\right) \\
&=&2y_{\max }\sin \left( kx\right) \cos \left( \omega t\right) \\
&=&\left( 2y_{\max }\sin \left( kx\right) \right) \cos \left( \omega t\right)
\end{eqnarray*}%
This looks like the harmonic oscillator equation%
\[
y=y_{\max }\cos \left( \omega t+\phi _{o}\right) 
\]%
with $\phi _{o}=0$ and with another complicated amplitude%
\[
2y_{\max }\sin \left( kx\right) 
\]%
That is, we have a set of harmonic oscillators who's amplitude is different
for each value of $x.$

\FRAME{dhF}{2.5296in}{2.2883in}{0in}{}{}{Figure}{\special{language
"Scientific Word";type "GRAPHIC";maintain-aspect-ratio TRUE;display
"USEDEF";valid_file "T";width 2.5296in;height 2.2883in;depth
0in;original-width 2.4889in;original-height 2.2485in;cropleft "0";croptop
"1";cropright "1";cropbottom "0";tempfilename
'PQXXQXRP.wmf';tempfile-properties "XPR";}}We can identify spots along the $%
x $ axis where the amplitude is always zero! we will call these spots \emph{%
nodes.} These happen when $\sin \left( kx\right) =0$ or when 
\[
kx=n\pi 
\]

By using 
\[
k=\frac{2\pi }{\lambda } 
\]%
we have%
\begin{eqnarray*}
\frac{2\pi }{\lambda }x &=&n\pi \\
\frac{2}{\lambda }x &=&n \\
x &=&n\frac{\lambda }{2}
\end{eqnarray*}

We can also find the places along $x$ where the amplitude will be largest.
this occurs when $\sin \left( kx\right) =1$ or when%
\[
kx=n\frac{\pi }{2} 
\]%
or%
\begin{eqnarray*}
\frac{2\pi }{\lambda }x &=&n\frac{\pi }{2} \\
x &=&n\frac{\lambda }{4}
\end{eqnarray*}%
these are called \emph{antinodes}.

We can also create standing waves with sound or even light waves!

\subsection{Standing Waves in a String Fixed at Both Ends}

\FRAME{dhF}{3.2846in}{2.4699in}{0in}{}{}{Figure}{\special{language
"Scientific Word";type "GRAPHIC";maintain-aspect-ratio TRUE;display
"USEDEF";valid_file "T";width 3.2846in;height 2.4699in;depth
0in;original-width 3.2396in;original-height 2.4284in;cropleft "0";croptop
"1";cropright "1";cropbottom "0";tempfilename
'PQXXQXRQ.wmf';tempfile-properties "XPR";}}If we attach a string to
something on both ends, we find something interesting in the standing wave
pattern. Not all possible standing waves can be realized. Some frequencies
are preferred, and some never show up. The standing wave pattern is \emph{%
quantized}. The patterns that are allowed are called \emph{normal modes}. We
will see this any time a wave confined by boundary conditions (light in a
resonant cavity, radio waves in a wave guide, electrons in an atom, etc.).

The figure shows some normal modes for a string.

We find which modes are allowed by first imposing the boundary condition
that each end must be a node. We start with 
\[
y=2y_{\max }\sin \left( kx\right) \cos \left( \omega t\right) 
\]%
and recognize that we have one condition met because%
\[
y=0 
\]%
when 
\[
x=0 
\]%
we need $y=0$ when $x=L.$ That happens when%
\[
kL=n\pi 
\]%
I will write this as%
\[
k_{n}L=n\pi 
\]%
to indicate there are many values of $k$ that can work. Solving this for $%
\lambda _{n}$ gives%
\begin{eqnarray*}
\frac{2\pi }{\lambda _{n}}L &=&n\pi \\
\frac{2L}{n} &=&\lambda _{n}
\end{eqnarray*}%
Which says there are many wavelengths that will work. Let's see how this
works, the first mode will have 
\[
\lambda _{1}=2L 
\]%
where $L$ is the length of the string. Looking at the figure, this works.

The second mode has three nodes (one on each end and one in the middle).
This gives%
\[
\lambda _{2}=L 
\]%
We can keep going, the third mode will have five nodes%
\[
\lambda _{3}=\frac{3L}{2} 
\]%
and so forth to give%
\[
\lambda _{n}=\frac{2L}{n} 
\]

We use our old friend%
\[
v=f\lambda 
\]%
to find the frequencies of the modes%
\[
f=\frac{v}{\lambda } 
\]

\[
f_{1}=\frac{v}{\lambda _{1}}=\frac{v}{2L} 
\]%
or, in general%
\begin{eqnarray*}
f_{n} &=&\frac{v}{\lambda _{n}}=n\frac{v}{2L} \\
&=&\frac{n}{2L}v \\
&=&\frac{n}{2L}\sqrt{\frac{T}{\mu }}
\end{eqnarray*}%
The lowest frequency has a special name, the \emph{fundamental frequency}.
The higher frequencies are integer multiples of the fundamental. When this
happens we say that the frequencies form a \emph{harmonic series, }and the
modes are called \emph{harmonics}. Since only certain frequencies work, we
say that the frequencies of waves on the string that make standing waves are 
\emph{quantized}! Quantization means that some values are allowed. This idea
is the basis behind Quantum mechanics (which views light and even matter as
waves).

\subsection{Musical Strings}

So how do we get different notes on a guitar or Piano?%
\begin{equation}
f_{n}=\frac{n}{2L}\sqrt{\frac{T}{\mu }}
\end{equation}

A guitar uses tension to change the frequency or pitch (tuning) and length
of string (your fingers pressing on the strings) to change notes. A Piano
uses both tension and length of string (and mass per unit length as well!).
What do you expect and organ will do?

\chapter{Light and Sound Standing waves}

%TCIMACRO{%
%\TeXButton{Fundamental Concepts}{\hspace{-1.3in}{\Large Fundamental Concepts\vspace{0.25in}}}}%
%BeginExpansion
\hspace{-1.3in}{\Large Fundamental Concepts\vspace{0.25in}}%
%EndExpansion

\begin{itemize}
\item Sound Standing waves (music)
\end{itemize}

\section{Sound Standing waves.}

We have spent some time studying standing waves on strings. But we also have
studied sound waves, could we make sound standing waves?

For our attempt, let's take a pipe as shown in the next figures. The pipe
will control the direction of the sound wave. The sound wave will (mostly)
go down the pipe. But will there be a reflection at the end of the pipe like
there was for the string?

It turns out that there will be a reflection. Let's consider our sound wave
as a change in pressure to see why. For the string wave, we had a reflected
pulse because the person (or guitar bridge, or whatever is holding the
string end) exerted a force on the string. If a sound wave travels down our
pipe, the pressure will change as the wave goes down the pipe. Remember that
pressure is a force spread over an area. When the sound wave gets to the end
of the pipe, the rest of the room's air is waiting. And that air mass pushes
back on the air molecules that are trying to leave the pipe. The air mass of
the room won't change it's pressure much. That resistance to pressure change
(the force due to the rest of the room air molecules colliding with our wave
molecules, pushing them back) will send our tube molecules back down the
tube. And that starts a reflection. Thin about this for a moment. We need to
picture more than we have before. The room is full of air, and it turns out
that the air molecules aren't sitting still. So when we make our wave start
but hitting molecules with a piston, the molecules at the end of the tube
will hit room air molecules that are already moving.

\FRAME{dtbpF}{3.5129in}{2.4154in}{0in}{}{}{Figure}{\special{language
"Scientific Word";type "GRAPHIC";maintain-aspect-ratio TRUE;display
"USEDEF";valid_file "T";width 3.5129in;height 2.4154in;depth
0in;original-width 3.4662in;original-height 2.3748in;cropleft "0";croptop
"1";cropright "1";cropbottom "0";tempfilename
'PQY2AO0Y.wmf';tempfile-properties "XPR";}}Think of pool balls. If the cue
ball and another ball, say, the 6 ball, collide, usually the 6 ball is
stationary at the beginning. But if it is not, that will change the outcome
of our conservation of momentum problem. if the 6 ball is initially moving
toward the cue ball, the cue ball will bounce back the way it came. That
will happen with some of the molecules at the end of the pipe. But some will
keep going. \FRAME{dtbpF}{3.7144in}{1.7538in}{0in}{}{}{Figure}{\special%
{language "Scientific Word";type "GRAPHIC";maintain-aspect-ratio
TRUE;display "USEDEF";valid_file "T";width 3.7144in;height 1.7538in;depth
0in;original-width 3.6668in;original-height 1.7158in;cropleft "0";croptop
"1";cropright "1";cropbottom "0";tempfilename
'PQY2UK0Z.wmf';tempfile-properties "XPR";}}We will split the energy from the
wave into to waves, much like we did with waves on strings when we came to
an interface. \FRAME{dtbpF}{2.8764in}{2.2572in}{0pt}{}{}{Figure}{\special%
{language "Scientific Word";type "GRAPHIC";maintain-aspect-ratio
TRUE;display "USEDEF";valid_file "T";width 2.8764in;height 2.2572in;depth
0pt;original-width 2.8331in;original-height 2.2174in;cropleft "0";croptop
"1";cropright "1";cropbottom "0";tempfilename
'PQY34610.wmf';tempfile-properties "XPR";}}

Now that we have a reflected wave in the tube, we can end up with two waves
so long as the piston keeps working at making new waves. And with two waves
in the same medium, we have the possibility of having standing waves!

If we have a pipe open at both ends, we can see that air molecules are free
to move in and out of the ends of the pipes. If the air molecules can move,
the ends must not be nodes. In our string case, the part of the string that
experienced destructive interference and did not move was called a node and
we always had a node on the end of the string. But the pipe is different
than the string case! We can see that air molecules will move out and then
bounce back in at the end of the pipe. Still, we expect that there must be a
node somewhere. We can reasonably guess that there will be a node in the
middle of the pipe. Of course, the pressure on both ends must be atmospheric
pressure. So, remembering that pressure and displacement are $90\unit{%
%TCIMACRO{\U{b0}}%
%BeginExpansion
{{}^\circ}%
%EndExpansion
}$ out of phase for sound waves, we can guess that there are pressure nodes
on both ends. But there is a displacement anti-node at the ends of the pipe. 
\FRAME{dhF}{3.7403in}{2.2892in}{0pt}{}{}{Figure}{\special{language
"Scientific Word";type "GRAPHIC";maintain-aspect-ratio TRUE;display
"USEDEF";valid_file "T";width 3.7403in;height 2.2892in;depth
0pt;original-width 5.5054in;original-height 3.3589in;cropleft "0";croptop
"1";cropright "1";cropbottom "0";tempfilename
'PQXXQXRR.wmf';tempfile-properties "XPR";}}Then for the first harmonic we
can draw a displacement node in the middle and we see that 
\begin{equation}
\lambda _{1}=2L
\end{equation}%
and 
\begin{equation}
f_{1}=\frac{v}{2L}
\end{equation}%
The next mode fits a whole wavelength%
\begin{eqnarray}
\lambda _{2} &=&L \\
f_{2} &=&\frac{v}{L}
\end{eqnarray}%
\FRAME{dhF}{3.0597in}{2.7786in}{0pt}{}{}{Figure}{\special{language
"Scientific Word";type "GRAPHIC";maintain-aspect-ratio TRUE;display
"USEDEF";valid_file "T";width 3.0597in;height 2.7786in;depth
0pt;original-width 4.4348in;original-height 4.0248in;cropleft "0";croptop
"1";cropright "1";cropbottom "0";tempfilename
'PQXXQXRS.wmf';tempfile-properties "XPR";}}but the next mode fits a
wavelength and a half%
\begin{eqnarray}
\lambda _{3} &=&\frac{3}{2}L \\
f_{3} &=&\frac{2v}{3L}
\end{eqnarray}%
If we keep going 
\begin{eqnarray}
\lambda _{n} &=&\frac{2}{n}L \\
f_{n} &=&n\frac{v}{2L}\qquad n=1,2,3,4\ldots
\end{eqnarray}%
This is the same mathematical form that we achieved for a standing wave on a
string!

\section{Pipes closed on one end}

%TCIMACRO{%
%\TeXButton{Boom Whacker Demo}{\marginpar {
%\hspace{-0.5in}
%\begin{minipage}[t]{1in}
%\small{Boom Whacker Demo}
%\end{minipage}
%}}}%
%BeginExpansion
\marginpar {
\hspace{-0.5in}
\begin{minipage}[t]{1in}
\small{Boom Whacker Demo}
\end{minipage}
}%
%EndExpansion
But what happens if we put a cap on one end of the pipe? The air molecules
cannot move longitudinally once they hit the cap. And where the molecules
don't move, that is a node. So our capped end must be a displacement node.

%TCIMACRO{%
%\TeXButton{Question 123.24.2}{\marginpar {
%\hspace{-0.5in}
%\begin{minipage}[t]{1in}
%\small{Question 123.24.2}
%\end{minipage}
%}} }%
%BeginExpansion
\marginpar {
\hspace{-0.5in}
\begin{minipage}[t]{1in}
\small{Question 123.24.2}
\end{minipage}
}
%EndExpansion
The open end is still a displacement antinode. Our standing wave will have a
node on the capped side, and an anti-node on the open side. It might look
like this \FRAME{dhF}{4.3474in}{2.8876in}{0pt}{}{}{Figure}{\special{language
"Scientific Word";type "GRAPHIC";maintain-aspect-ratio TRUE;display
"USEDEF";valid_file "T";width 4.3474in;height 2.8876in;depth
0pt;original-width 4.2964in;original-height 2.8444in;cropleft "0";croptop
"1";cropright "1";cropbottom "0";tempfilename
'PQXXQXRT.wmf';tempfile-properties "XPR";}}In the next figure we draw the
first few harmonics for the \textquotedblleft closed on one
end\textquotedblright\ case.\FRAME{dhF}{4.1796in}{2.0358in}{0pt}{}{}{Figure}{%
\special{language "Scientific Word";type "GRAPHIC";maintain-aspect-ratio
TRUE;display "USEDEF";valid_file "T";width 4.1796in;height 2.0358in;depth
0pt;original-width 6.3252in;original-height 3.0666in;cropleft "0";croptop
"1";cropright "1";cropbottom "0";tempfilename
'PQXXQXRU.wmf';tempfile-properties "XPR";}}

The first harmonic for the closed pipe are found by using 
\begin{equation}
v=\lambda f
\end{equation}%
\begin{equation}
f=\frac{v}{\lambda }
\end{equation}%
We know the speed of sound, so we have $v.$ Knowing that the first harmonic
has a node at one end and an anti node at the other end gives us the
wavelength. If the pipe is $L$ in length, then $L$ must be 
\begin{equation}
L=\frac{1}{4}\lambda _{1}
\end{equation}%
or%
\begin{equation}
\lambda _{1}=4L
\end{equation}%
then%
\begin{equation}
f_{1}=\frac{v}{\lambda _{1}}=\frac{v}{4L}
\end{equation}

The next configuration that will have a node on one end and an antinode on
the other will have 
\begin{equation}
L=\frac{3}{4}\lambda _{2}
\end{equation}%
which gives%
\begin{equation}
\lambda _{2}=\frac{4}{3L}
\end{equation}%
and%
\begin{equation}
f_{2}=\frac{v}{\lambda _{2}}=\frac{3v}{4L}
\end{equation}

If we continued, we would find%
\begin{equation}
\lambda _{n}=\frac{4}{n}L
\end{equation}%
and 
\begin{equation}
f_{n}=n\frac{v}{4L}\qquad n=1,3,5\ldots
\end{equation}

Example: organ pipe

\FRAME{dhF}{1.6907in}{1.4088in}{0in}{}{}{Figure}{\special{language
"Scientific Word";type "GRAPHIC";maintain-aspect-ratio TRUE;display
"USEDEF";valid_file "T";width 1.6907in;height 1.4088in;depth
0in;original-width 1.6544in;original-height 1.3742in;cropleft "0";croptop
"1";cropright "1";cropbottom "0";tempfilename
'PQXXQXRV.wmf';tempfile-properties "XPR";}}

%TCIMACRO{%
%\TeXButton{Organ Pipe Demo}{\marginpar {
%\hspace{-0.5in}
%\begin{minipage}[t]{1in}
%\small{Organ Pipe Demo}
%\end{minipage}
%}}}%
%BeginExpansion
\marginpar {
\hspace{-0.5in}
\begin{minipage}[t]{1in}
\small{Organ Pipe Demo}
\end{minipage}
}%
%EndExpansion
The organ pipe shown is closed at one end so we expect%
\begin{equation}
f_{n}=n\frac{v}{4L}\qquad n=1,3,5\ldots
\end{equation}%
Measuring the pipe, and assuming about $20\unit{%
%TCIMACRO{\U{2103}}%
%BeginExpansion
{}^{\circ}{\rm C}%
%EndExpansion
}$ for the room temperature we have 
\begin{equation}
\begin{tabular}{l}
$L=0.41\unit{m}$ \\ 
$R=0.06\unit{m}$ \\ 
$v=343\frac{\unit{m}}{\unit{s}}$%
\end{tabular}%
\end{equation}%
There is a detail we have ignored in our analysis, the width of the pipe
matters a little. I will include a fudge factor to account for this. With
the fudge factor, the wavelength is 
\begin{eqnarray*}
\lambda _{1} &=&4\left( L+0.6R\right) \\
&=&178.\,\allowbreak 4\unit{cm}
\end{eqnarray*}%
then our fundamental frequency is 
\begin{eqnarray}
f_{1} &=&\frac{v}{\lambda _{1}} \\
&=&192.\,\allowbreak 26
\end{eqnarray}

We can identify this note, and compare to a standard, like a tuning fork or
a piano to verify our prediction.\FRAME{dhF}{4.6345in}{1.8844in}{0pt}{}{}{%
Figure}{\special{language "Scientific Word";type
"GRAPHIC";maintain-aspect-ratio TRUE;display "USEDEF";valid_file "T";width
4.6345in;height 1.8844in;depth 0pt;original-width 6.3529in;original-height
2.5676in;cropleft "0";croptop "1";cropright "1";cropbottom "0";tempfilename
'PQXXQXRW.wmf';tempfile-properties "XPR";}}

Lasers and standing waves

\FRAME{dhF}{2.6135in}{1.6604in}{0in}{}{}{Figure}{\special{language
"Scientific Word";type "GRAPHIC";maintain-aspect-ratio TRUE;display
"USEDEF";valid_file "T";width 2.6135in;height 1.6604in;depth
0in;original-width 2.5719in;original-height 1.6233in;cropleft "0";croptop
"1";cropright "1";cropbottom "0";tempfilename
'PQXXQXRX.wmf';tempfile-properties "XPR";}}\FRAME{dhF}{3.1868in}{1.6466in}{%
0in}{}{}{Figure}{\special{language "Scientific Word";type
"GRAPHIC";maintain-aspect-ratio TRUE;display "USEDEF";valid_file "T";width
3.1868in;height 1.6466in;depth 0in;original-width 3.1427in;original-height
1.6103in;cropleft "0";croptop "1";cropright "1";cropbottom "0";tempfilename
'PQXXQXRY.wmf';tempfile-properties "XPR";}}

%TCIMACRO{%
%\TeXButton{Question 123.24.3}{\marginpar {
%\hspace{-0.5in}
%\begin{minipage}[t]{1in}
%\small{Question 123.24.3}
%\end{minipage}
%}}}%
%BeginExpansion
\marginpar {
\hspace{-0.5in}
\begin{minipage}[t]{1in}
\small{Question 123.24.3}
\end{minipage}
}%
%EndExpansion
%TCIMACRO{%
%\TeXButton{Question 123.24.4}{\marginpar {
%\hspace{-0.5in}
%\begin{minipage}[t]{1in}
%\small{Question 123.24.4}
%\end{minipage}
%}}}%
%BeginExpansion
\marginpar {
\hspace{-0.5in}
\begin{minipage}[t]{1in}
\small{Question 123.24.4}
\end{minipage}
}%
%EndExpansion

\section{Standing Waves in Rods and Membranes}

We have hinted all chapter that the analysis techniques we were building
apply to structures. We need more math and computational tools to analyze
complex structures like bridges and buildings, but we can tackle a simple
structure like a rod that is clamped. The atoms in the rod can vibrate
longitudinally \FRAME{dtbpF}{4.318in}{1.2419in}{0in}{}{}{Figure}{\special%
{language "Scientific Word";type "GRAPHIC";maintain-aspect-ratio
TRUE;display "USEDEF";valid_file "T";width 4.318in;height 1.2419in;depth
0in;original-width 4.2661in;original-height 1.2081in;cropleft "0";croptop
"1";cropright "1";cropbottom "0";tempfilename
'PQXXQXRZ.wmf';tempfile-properties "XPR";}}Since we have motion possible on
both ends and not in the middle, we surmise that this system will have
similar solutions as did the open ended pipe.

\[
f_{n}=n\frac{v}{2L} 
\]%
The fundamental looks like

$\sin \left( x\right) $\FRAME{dtbpFX}{2.3443in}{1.5638in}{0pt}{}{}{Plot}{%
\special{language "Scientific Word";type "MAPLEPLOT";width 2.3443in;height
1.5638in;depth 0pt;display "USEDEF";plot_snapshots TRUE;mustRecompute
FALSE;lastEngine "MuPAD";xmin "-1.5708";xmax "1.5708";xviewmin
"-1.5708";xviewmax "1.5708";yviewmin "-1.2";yviewmax
"1.2";viewset"XY";rangeset"X";plottype 4;plottickdisable TRUE;axesFont
"Times New Roman,12,0000000000,useDefault,normal";numpoints 100;plotstyle
"patch";axesstyle "normal";axestips FALSE;xis \TEXUX{x};var1name
\TEXUX{$x$};function \TEXUX{$\sin \left( x\right) $};linecolor
"blue";linestyle 1;pointstyle "point";linethickness 2;lineAttributes
"Solid";var1range "-1.5708,1.5708";num-x-gridlines 100;curveColor
"[flat::RGB:0x000000ff]";curveStyle "Line";function \TEXUX{$-\sin \left(
x\right) $};linecolor "blue";linestyle 1;pointstyle "point";linethickness
2;lineAttributes "Solid";var1range "-1.5708,1.5708";num-x-gridlines
100;curveColor "[flat::RGB:0x000000ff]";curveStyle "Line";VCamFile
'PQXXR22R.xvz';valid_file "T";tempfilename
'PQXXQXS0.wmf';tempfile-properties "XPR";}}

But suppose we move the clamp. The clamp forces a node where it is placed.
If we place the clap at $L/4$

\FRAME{dtbpF}{4.6025in}{1.3266in}{0in}{}{}{Figure}{\special{language
"Scientific Word";type "GRAPHIC";maintain-aspect-ratio TRUE;display
"USEDEF";valid_file "T";width 4.6025in;height 1.3266in;depth
0in;original-width 4.5506in;original-height 1.292in;cropleft "0";croptop
"1";cropright "1";cropbottom "0";tempfilename
'PQXXQXS1.wmf';tempfile-properties "XPR";}}

\FRAME{dtbpFX}{2.3443in}{1.5638in}{0pt}{}{}{Plot}{\special{language
"Scientific Word";type "MAPLEPLOT";width 2.3443in;height 1.5638in;depth
0pt;display "USEDEF";plot_snapshots TRUE;mustRecompute FALSE;lastEngine
"MuPAD";xmin "-3.1459";xmax "3.1459";xviewmin "-3.1459";xviewmax
"3.1459";yviewmin "-1.2";yviewmax "1.2";viewset"XY";rangeset"X";plottype
4;plottickdisable TRUE;axesFont "Times New
Roman,12,0000000000,useDefault,normal";numpoints 100;plotstyle
"patch";axesstyle "normal";axestips FALSE;xis \TEXUX{x};var1name
\TEXUX{$x$};function \TEXUX{$\cos \left( x\right) $};linecolor
"blue";linestyle 1;pointstyle "point";linethickness 2;lineAttributes
"Solid";var1range "-3.1459,3.1459";num-x-gridlines 100;curveColor
"[flat::RGB:0x000000ff]";curveStyle "Line";function \TEXUX{$-\cos \left(
x\right) $};linecolor "blue";linestyle 1;pointstyle "point";linethickness
2;lineAttributes "Solid";var1range "-3.1459,3.1459";num-x-gridlines
100;curveColor "[flat::RGB:0x000000ff]";curveStyle "Line";VCamFile
'PQXXR22Q.xvz';valid_file "T";tempfilename
'PQXXQXS2.wmf';tempfile-properties "XPR";}}

We can perform a similar analysis for a drum head, but it is much more
complicated. The modes are not points, but lines or curves, and the
frequencies of oscillation are not integer multiples of each other. See for
example http://physics.usask.ca/\symbol{126}%
hirose/ep225/animation/drum/anim-drum.htm.

Of course structures can also waggle on the ends. the ends can rotate
counter to each other, etc. These are more complex modes than the
longitudinal modes we have considered.

\section{Single Frequency Interference in one dimension}

We have seen interference from superimposing two waves in the same medium.
Whether we get constructive or destructive interference 
%TCIMACRO{%
%\TeXButton{Two speaker demo}{\marginpar {
%\hspace{-0.5in}
%\begin{minipage}[t]{1in}
%\small{Two speaker demo}
%\end{minipage}
%}}}%
%BeginExpansion
\marginpar {
\hspace{-0.5in}
\begin{minipage}[t]{1in}
\small{Two speaker demo}
\end{minipage}
}%
%EndExpansion
really depends on the total phase difference. Let's review this, but let's
make our two waves as general as possible.%
\begin{eqnarray*}
y_{1} &=&y_{\max }\sin \left( k_{1}x_{1}-\omega _{1}t_{1}+\phi _{1}\right) \\
y_{2} &=&y_{\max }\sin \left( k_{2}x_{2}-\omega _{2}t_{2}+\phi _{2}\right)
\end{eqnarray*}%
where we have allowed the wavelengths to be different (so then the wave
numbers, $k_{1}$ and $k_{2}$ could be different), the distances traveled by
the waves could be different ($x_{2}$ might not be equal to $x_{1}$), the
frequencies might be different, the times observed might be different, and
the phase constants might be different. We have not made the amplitudes
different (but it is likely to happen in an upcoming homework problem).

Then the resulting wave would be 
\begin{eqnarray*}
y_{r} &=&y_{\max }\sin \left( k_{2}x_{2}-\omega _{2}t_{2}+\phi _{2}\right)
+A\sin \left( k_{1}x_{1}-\omega _{1}t_{1}+\phi _{1}\right) \\
&=&2y_{\max }\cos \left( \frac{\left( k_{2}x_{2}-\omega _{2}t_{2}+\phi
_{2}\right) -\left( k_{1}x_{1}-\omega _{1}t_{1}+\phi _{1}\right) }{2}\right)
\\
&&\times \sin \left( \frac{\left( k_{2}x_{2}-\omega _{2}t_{2}+\phi
_{2}\right) +\left( k_{1}x_{1}-\omega _{1}t_{1}+\phi _{1}\right) }{2}\right)
\end{eqnarray*}%
just as before. We can rewire this as 
\begin{eqnarray*}
y_{r} &=&2y_{\max }\cos \left( \frac{1}{2}\left[ \left( k_{2}x_{2}-\omega
_{2}t_{2}+\phi _{2}\right) -\left( k_{1}x_{1}-\omega _{1}t_{1}+\phi
_{1}\right) \right] \right) \\
&&\times \sin \left( \frac{\left( k_{2}x_{2}-\omega _{2}t_{2}+\phi
_{2}\right) +\left( k_{1}x_{1}-\omega _{1}t_{1}+\phi _{1}\right) }{2}\right)
\end{eqnarray*}%
and recognize that the sine part is still a function of position and time
(just a complicated one) so it must be a wave. The cosine part must be part
of the amplitude. Let's write out this amplitude%
\[
A=2y_{\max }\cos \left( \frac{1}{2}\left[ \left( k_{2}x_{2}-\omega
_{2}t_{2}+\phi _{2}\right) -\left( k_{1}x_{1}-\omega _{1}t_{1}+\phi
_{1}\right) \right] \right) 
\]%
The part in square brackets is what we have been calling the phase
difference 
\[
\Delta \phi =\left( k_{2}x_{2}-\omega _{2}t_{2}+\phi _{2}\right) -\left(
k_{1}x_{1}-\omega _{1}t_{1}+\phi _{1}\right) 
\]%
Notice that to make constructive interference we want the amplitude to be as
big as possible. This happens when the cosine is $\pm 1.$ We can find when
this happens by recalling that cosine is $\pm 1$ when the argument of the
cosine is a multiple of $\pi .$ Notice that the amplitude has the cosine of
half the phase difference%
\[
A=2y_{\max }\cos \left( \frac{1}{2}\Delta \phi \right) 
\]%
so 
\[
\frac{1}{2}\Delta \phi =\pi m\qquad m=0,\pm 1,\pm 2,\pm 3,\cdots 
\]%
or 
\[
\Delta \phi =2\pi m\qquad m=0,\pm 1,\pm 2,\pm 3,\cdots \qquad \text{%
Constructive} 
\]%
For total destructive interference we want the amplitude to be zero. To
achieve this the cosine must be zero and this happens at odd multiples of $%
\pi /2.$ 
\[
\frac{1}{2}\Delta \phi =\frac{\pi }{2}m\qquad m=\pm 1,\pm 3,\pm 5,\cdots 
\]%
Then we could say that our phase difference 
\[
\Delta \phi =\pi m\qquad m=\pm 1,\pm 3,\pm 5,\cdots 
\]%
another way to say this is to allow $m$ to be any integer but to write the
condition as 
\[
\Delta \phi =\left( m+\frac{1}{2}\right) 2\pi \qquad m=0,\pm 1,\pm 2,\pm
3,\cdots \qquad \text{Destructive} 
\]

Let's write these out in detail%
\begin{eqnarray*}
\left[ \left( k_{2}x_{2}-\omega _{2}t_{2}+\phi _{2}\right) -\left(
k_{1}x_{1}-\omega _{1}t_{1}+\phi _{1}\right) \right] &=&2\pi m\qquad m=0,\pm
1,\pm 2,\pm 3,\cdots \\
\left[ \left( k_{2}x_{2}-\omega _{2}t_{2}+\phi _{2}\right) -\left(
k_{1}x_{1}-\omega _{1}t_{1}+\phi _{1}\right) \right] &=&\left( m+\frac{1}{2}%
\right) 2\pi \qquad m=0,\pm 1,\pm 2,\pm 3,\cdots
\end{eqnarray*}

We can see that there are at least four sources of phase difference here. We
can change the phase difference by changing wavelength ($k_{2}\neq k_{1}$),
changing frequency ($\omega _{2}\neq \omega _{1}$), having the waves travel
different distances ($x_{2}\neq x_{1}$), having different phase constants ($%
\phi _{2}\neq \phi _{1}$) or even consider somehow mixing the waves at
different times ($t_{2}\neq t_{1}$). This last one is not as interesting, we
want our waves mixed together at the same time so let's set%
\[
t_{2}=t_{1}=t 
\]%
but let's let any of the other variables be changeable. This gives us many
ways to produce constructive or destructive interference. In our next
lecture we will use this analysis to solve specific problems. The pattern
will be to use our two criteria%
\begin{eqnarray*}
\Delta \phi &=&2\pi m\qquad m=0,\pm 1,\pm 2,\pm 3,\cdots \qquad \text{%
Constructive} \\
\Delta \phi &=&\left( m+\frac{1}{2}\right) 2\pi \qquad m=0,\pm 1,\pm 2,\pm
3,\cdots \qquad \text{Destructive}
\end{eqnarray*}%
to determine if we have constructive, destructive, or partially constructive
or destructive interference.

\chapter{Single Frequency Interference, Multiple Dimensions}

%TCIMACRO{%
%\TeXButton{Fundamental Concepts}{\hspace{-1.3in}{\Large Fundamental Concepts\vspace{0.25in}}}}%
%BeginExpansion
\hspace{-1.3in}{\Large Fundamental Concepts\vspace{0.25in}}%
%EndExpansion

\begin{itemize}
\item When two waves interfere we get a new wave with a more complicated
amplitude

\item Thin films can cause reflections that create interference

\item There may be a phase shift when a wave reflects (the wave may
invert-this is a review)

\item If two waves are out of phase by half a wavelength we have total
destructive interference (review)

\item If two waves are out of phase by a full wavelength we have total
constructive interference (review)

\item Other phases provide partial constructive or partial destructive
interference

\item If two waves superimpose but the waves have different frequencies, we
have beating.
\end{itemize}

\section{Mathematical treatment of single frequency interference}

It is time to put our treatment of interference on a general mathematical
footing. You know that the phase difference, $\Delta \phi ,$ is the key
factor in determining whether we will have constructive or destructive
interference. Here is our equation from last lecture for the amplitude of
two mixed waves%
\[
A=2y_{\max }\cos \left( \frac{1}{2}\left[ \left( k_{2}x_{2}-\omega
_{2}t_{2}+\phi _{2}\right) -\left( k_{1}x_{1}-\omega _{1}t_{1}+\phi
_{1}\right) \right] \right) 
\]%
We defined 
\[
\Delta \phi =\left( k_{2}x_{2}-\omega _{2}t_{2}+\phi _{2}\right) -\left(
k_{1}x_{1}-\omega _{1}t_{1}+\phi _{1}\right) 
\]%
and found conditions for constructive and destructive interference.%
\begin{eqnarray*}
\Delta \phi &=&2\pi m\qquad m=0,\pm 1,\pm 2,\pm 3,\cdots \qquad \text{%
Constructive} \\
\Delta \phi &=&\left( m+\frac{1}{2}\right) 2\pi \qquad m=0,\pm 1,\pm 2,\pm
3,\cdots \qquad \text{Destructive}
\end{eqnarray*}

Let's try a problem using these ideas.

Suppose we have the following two waves. 
\begin{eqnarray*}
y_{1} &=&\sin \left( \pi x-\omega t+\frac{1}{6}\pi \right) \\
y_{2} &=&\sin \left( \pi x-\omega t\right)
\end{eqnarray*}%
We want to find the equation of the resultant wave and to determine if we
have constructive or destructive interference. The two waves are plotted in
the next figure.\FRAME{dtbpFX}{4.4996in}{1.2704in}{0pt}{}{}{Plot}{\special%
{language "Scientific Word";type "MAPLEPLOT";width 4.4996in;height
1.2704in;depth 0pt;display "USEDEF";plot_snapshots TRUE;mustRecompute
FALSE;lastEngine "MuPAD";xmin "0";xmax "5";xviewmin "0";xviewmax
"5";yviewmin "-2";yviewmax "2";viewset"XY";rangeset"X";plottype 4;axesFont
"Times New Roman,12,0000000000,useDefault,normal";numpoints 100;plotstyle
"patch";axesstyle "normal";axestips FALSE;xis \TEXUX{x};var1name
\TEXUX{$x$};function \TEXUX{$\sin \left( \frac{1}{6}\pi +\pi x\right)
$};linecolor "blue";linestyle 1;pointstyle "point";linethickness
1;lineAttributes "Solid";var1range "0,5";num-x-gridlines 100;curveColor
"[flat::RGB:0x000000ff]";curveStyle "Line";function \TEXUX{$\sin \left( \pi
x\right) $};linecolor "maroon";linestyle 1;pointstyle "point";linethickness
1;lineAttributes "Solid";var1range "0,5";num-x-gridlines 100;curveColor
"[flat::RGB:0x00800000]";curveStyle "Line";VCamFile
'PQY3AT4V.xvz';valid_file "T";tempfilename
'PQXXQXS3.wmf';tempfile-properties "XPR";}}We can identify that $y_{\max }=1%
\unit{m},$ $k_{1}=k_{2}=\pi ,$ $\omega _{1}=\omega _{2}=\omega ,$ $%
t_{1}=t_{2}=t,$ $\phi _{1}=\pi /6$ and $\phi _{2}=0.$ We have this for all $%
x $ so we can chose $x_{1}=x_{2}=x$ for all $x.$ The superposition yields.%
\begin{eqnarray*}
y_{r} &=&y_{\max }\sin \left( k_{2}x_{2}-\omega _{2}t_{2}+\phi _{2}\right)
+A\sin \left( k_{1}x_{1}-\omega _{1}t_{1}+\phi _{1}\right) \\
&=&2y_{\max }\cos \left( \left( \frac{1}{2}\left[ \left( k_{2}x_{2}-\omega
_{2}t_{2}+\phi _{2}\right) -\left( k_{1}x_{1}-\omega _{1}t_{1}+\phi
_{1}\right) \right] \right) \right) \\
&&\times \sin \left( \frac{\left( k_{1}x_{1}-\omega t+\phi _{2}\right)
+\left( k_{1}x_{2}-\omega t+\phi _{1}\right) }{2}\right) \\
&=&2y_{\max }\cos \left( \frac{1}{2}\left[ \left( \pi x-\omega t+0\right)
-\left( \pi x-\omega t+\frac{\pi }{6}\right) \right] \right) \\
&&\times \sin \left( \pi x-\omega t+\frac{0-\frac{\pi }{6}}{2}\right) \\
&=&2\cos \left( -\frac{\pi }{12}\right) \sin \left( \pi x-\omega t+\frac{\pi 
}{12}\right)
\end{eqnarray*}

Our specific case is graphed in the next figure.\FRAME{dtbpFX}{3.0727in}{%
1.3292in}{0pt}{}{}{Plot}{\special{language "Scientific Word";type
"MAPLEPLOT";width 3.0727in;height 1.3292in;depth 0pt;display
"USEDEF";plot_snapshots TRUE;mustRecompute FALSE;lastEngine "MuPAD";xmin
"0";xmax "5.001000";xviewmin "0";xviewmax "5.001000";yviewmin "-2";yviewmax
"2";viewset"XY";rangeset"X";plottype 4;axesFont "Times New
Roman,12,0000000000,useDefault,normal";numpoints 100;plotstyle
"patch";axesstyle "normal";axestips FALSE;xis \TEXUX{x};var1name
\TEXUX{$x$};function \TEXUX{$2\cos \left( -\frac{\pi }{12}\right) \sin
\left( \pi x+\frac{\pi }{12}\right) $};linecolor "green";linestyle
1;pointstyle "point";linethickness 2;lineAttributes "Solid";var1range
"0,5.001000";num-x-gridlines 100;curveColor
"[flat::RGB:0x00008000]";curveStyle "Line";VCamFile
'PRGHF406.xvz';valid_file "T";tempfilename
'PRGHEW00.wmf';tempfile-properties "XPR";}}Notice that the wave form is
taller (larger amplitude). Noticed it is shifted along the $x$ axis. This
graph is not surprising to us now, because we have done a case like this
before. The shirt will be $\phi _{R}=\pi /12.$ Let's test this for
constructive or destructive interference. We could use our conditions that
we developed, but we can also just evaluate the amplitude.

\[
2\cos \left( -\frac{\pi }{12}\right) =\allowbreak 1.\,\allowbreak 931\,9 
\]%
This is bigger than $1\unit{m},$ which is our individual wave amplitude. It
is a little smaller than $2\unit{m}$ which would be total constructive
interference. We would call this partial constructive interference.

Let's do another example problem, a harder one this time.

Consider two waves traveling to the right. One passes through a slower
medium. We expect that the frequency won't change ($\omega _{2}=\omega _{1}$%
), but the wavelength will shorten for the wave that enters the slower
material ($k_{2}\neq k_{1}$). The two waves approach point $x_{a}$ with the
same phase, but by point $x_{b}$ they are out of phase. \FRAME{dtbpF}{%
3.9323in}{1.5783in}{0in}{}{}{Figure}{\special{language "Scientific
Word";type "GRAPHIC";maintain-aspect-ratio TRUE;display "USEDEF";valid_file
"T";width 3.9323in;height 1.5783in;depth 0in;original-width
3.883in;original-height 1.542in;cropleft "0";croptop "1";cropright
"1";cropbottom "0";tempfilename 'PRGI7X01.wmf';tempfile-properties "XPR";}}%
If the thickness $\left( \mathfrak{t}=x_{b}-x_{a}\right) $ were larger $%
\left( \mathfrak{t}=x_{c}-x_{a}\right) $ we might be back in phase. This
picture shows twice the original thickness and we see we have constructive
interference after the slower material. \FRAME{dtbpF}{3.8484in}{1.5446in}{0in%
}{}{}{Figure}{\special{language "Scientific Word";type
"GRAPHIC";maintain-aspect-ratio TRUE;display "USEDEF";valid_file "T";width
3.8484in;height 1.5446in;depth 0in;original-width 3.8in;original-height
1.5091in;cropleft "0";croptop "1";cropright "1";cropbottom "0";tempfilename
'PRGI9N02.wmf';tempfile-properties "XPR";}} Using this setup, let's find the
smallest thickness of material that can cause constructive interference. The
thickness of the slower material will be $\left( \mathfrak{t}%
=x_{c}-x_{a}\right) $

This time we want constructive interference, so we can use our criteria%
\begin{eqnarray*}
\Delta \phi &=&2\pi m\qquad m=0,\pm 1,\pm 2,\pm 3,\cdots \qquad \text{%
Constructive} \\
\Delta \phi &=&\left( mn+\frac{1}{2}\right) 2\pi \qquad m=0,\pm 1,\pm 2,\pm
3,\cdots \qquad \text{Destructive}
\end{eqnarray*}%
choosing the first for the constructive case. We know 
\[
\Delta \phi =\left( k_{2}x_{2}-\omega _{2}t_{2}+\phi _{2}\right) -\left(
k_{1}x_{1}-\omega _{1}t_{1}+\phi _{1}\right) 
\]%
and we can put in what we know for this specific problem. We know the $%
\omega ^{\prime }s$ are the same. We want the waves to be at the same place $%
x_{b}$ at the same time. They start at point $x_{a}$ with the same phase
constant $\phi _{2}=\phi _{1}=\phi _{o}$. Both waves travel the distance $%
x_{b}-x_{a}$ as they go from position $x_{a}$ to $x_{b}$ so 
\begin{eqnarray*}
\Delta \phi &=&\left( k_{2}\left( x_{b}-x_{a}\right) -\omega t+\phi
_{0}\right) -\left( k_{1}\left( x_{b}-x_{a}\right) -\omega t+\phi _{o}\right)
\\
&=&k_{2}\left( x_{b}-x_{a}\right) -k_{1}\left( x_{b}-x_{a}\right) \\
&=&\left( x_{b}-x_{a}\right) \left( \frac{2\pi }{\lambda _{2}}-\frac{2\pi }{%
\lambda _{1}}\right)
\end{eqnarray*}

We know that $\lambda _{2}$ is different than $\lambda _{1},$ but how
different? We can find this by remembering that the frequency does not
change as we cross a material boundary. So if $f_{out}$ is the frequency
outside the slower material, then 
\[
f_{in}=f_{out} 
\]%
the frequency did not change as the wave moved into the slower material.
Using 
\[
v=f\lambda 
\]%
or 
\[
f=\frac{v}{\lambda } 
\]

we can find an expression for the wavelength inside the slower material for
wave $1$. 
\begin{eqnarray*}
f_{in} &=&f_{out} \\
\frac{v_{in}}{\lambda _{in}} &=&\frac{v_{out}}{\lambda _{out}}
\end{eqnarray*}%
Let's rewrite this for our case%
\[
\lambda _{in}=\frac{v_{in}}{v_{out}}\lambda _{out} 
\]%
and we can multiple top and bottom by $c$ so%
\begin{eqnarray*}
\lambda _{in} &=&\frac{c}{c}\frac{v_{in}}{v_{out}}\lambda _{out} \\
&=&\frac{\frac{c}{v_{out}}}{\frac{c}{v_{in}}}\lambda _{out}
\end{eqnarray*}%
Remember we defined the index of refraction earlier in our course%
\[
n=\frac{c}{v} 
\]%
and we can see we have the index of refraction twice in our last equation.%
\begin{eqnarray*}
\lambda _{in} &=&\frac{\frac{c}{v_{out}}}{\frac{c}{v_{in}}}\lambda _{out} \\
&=&\frac{n_{out}}{n_{in}}\lambda _{out} \\
&=&\frac{1}{n}\lambda _{out}
\end{eqnarray*}%
where we used the idea that $n_{out}=n_{air}\approx 1.$

Returning to our problem, wave $1$ will have $\lambda _{1}=\lambda _{in}$
and wave $2$ will have $\lambda _{2}=\lambda _{out}$%
\begin{eqnarray*}
\Delta \phi &=&\left( x_{b}-x_{a}\right) \left( \frac{2\pi }{\lambda _{out}}-%
\frac{2\pi }{\frac{1}{n}\lambda _{out}}\right) \\
&=&\frac{2\pi \left( x_{b}-x_{a}\right) }{\lambda _{out}}\left( 1-n\right)
\end{eqnarray*}%
Now we use our condition for destructive interference 
\[
\Delta \phi =2\pi m\qquad m=0,\pm 1,\pm 2,\pm 3,\cdots \qquad \text{%
Constructive} 
\]%
and place in it our phase difference for this problem%
\[
\frac{2\pi \left( x_{b}-x_{a}\right) }{\lambda _{out}}\left( 1-n\right)
=2\pi m\qquad m=0,\pm 1,\pm 2,\pm 3,\cdots 
\]%
We can see that if the thickness of the slower material is 
\[
\left( x_{b}-x_{a}\right) =\mathfrak{t}=\frac{\lambda _{out}}{\left(
1-n\right) }m 
\]%
to find the smallest thankless that will work, we set $m=1$ so%
\[
\mathfrak{t}=\frac{\lambda _{out}}{\left( 1-n\right) } 
\]
we will have constructive interference.

This is the thinnest thickness that will make constructive interference.

Let's try another problem.

\FRAME{dhF}{4.6034in}{2.2053in}{0pt}{}{}{Figure}{\special{language
"Scientific Word";type "GRAPHIC";maintain-aspect-ratio TRUE;display
"USEDEF";valid_file "T";width 4.6034in;height 2.2053in;depth
0pt;original-width 5.8254in;original-height 2.7752in;cropleft "0";croptop
"1";cropright "1";cropbottom "0";tempfilename
'PQXXQXS7.wmf';tempfile-properties "XPR";}}

The stealth fighter is coated with an anti-reflective polymer. This is part
of it's mechanism for making the plane invisible to radar. Suppose we have a
radar system with a wavelength of $3.00\unit{cm}.$ Further suppose that the
index of refraction of the anti-reflective polymer is $n=1.50$, and that the
aircraft index of refraction is very large, how thick would you make the
coating?

The radar waves all hit the plane in phase. From the figure, we see that the
radar wave will reflect off of the coating. Because the index of refraction
of the coating is large, this is like a fixed end. There will be an
inversion. Let's call this first reflected wave, wave $1$ and from what we
have just learned $\phi _{1}=\pi $ when it leaves the surface of the polymer.

But some of the wave will penetrate the polymer. This will reflect off of
the plane body. Let's call this wave $2$. The plane body has a very large
index of refraction, so once again the wave will experience an inversion. We
expect then that $\phi _{2}=\pi $. Next think of the distances traveled.
Both waves get to the polymer together. Let's set our $y_{a}$ right at the
outside edge of the polymer. But wave 1 just bounces off. It travels no
distance before it leaves $\left( y_{1}=0\right) $. Wave $2$ however traves
twice the thickness of the polymer before it leaves $\left( y_{2}=2\mathfrak{%
t}\right) $. Once again, inside the material, we know that the wavelength
will change $\left( \lambda _{2}\neq \lambda _{1}\right) $ but we expect the
frequency to stay the same $\left( \omega _{2}=\omega _{1}\right) $. Once
again we can write our phase difference as%
\[
\Delta \phi =\left( k_{2}y_{2}-\omega _{2}t_{2}+\phi _{2}\right) -\left(
k_{1}y_{1}-\omega _{1}t_{1}+\phi _{1}\right) 
\]%
and put in what we know from our problem%
\begin{eqnarray*}
\Delta \phi &=&\left( \frac{2\pi }{\lambda _{2}}\left( 2\mathfrak{t}\right)
-\omega t+\pi \right) -\left( \frac{2\pi }{\lambda _{1}}\left( 0\right)
-\omega t+\pi \right) \\
&=&\frac{2\pi }{\lambda _{2}}\left( 2\mathfrak{t}\right)
\end{eqnarray*}%
We can force destructive interference by using the second of our
interference criteria equations.%
\begin{eqnarray*}
\Delta \phi &=&2\pi m\qquad m=0,\pm 1,\pm 2,\pm 3,\cdots \qquad \text{%
Constructive} \\
\Delta \phi &=&\left( m+\frac{1}{2}\right) 2\pi \qquad m=0,\pm 1,\pm 2,\pm
3,\cdots \qquad \text{Destructive}
\end{eqnarray*}%
so that 
\[
\frac{2\pi }{\lambda _{2}}\left( 2\mathfrak{t}\right) =\left( m+\frac{1}{2}%
\right) 2\pi \qquad m=0,\pm 1,\pm 2,\pm 3,\cdots 
\]%
or 
\[
\mathfrak{t}=\frac{\lambda _{2}}{2}\left( m+\frac{1}{2}\right) 
\]%
But we have the same problem as in our last problem. The wave entered into a
material. So the wavelength changed. We can use what we found before%
\[
\lambda _{in}=\frac{1}{n}\lambda _{out} 
\]%
and identify $\lambda _{2}=\lambda _{in}$ so 
\[
\mathfrak{t}=\frac{\frac{1}{n}\lambda _{out}}{2}\left( m+\frac{1}{2}\right) 
\]%
then for the smallest thickness that will create destructive interference we
set $m=0$ 
\[
\mathfrak{t}=\frac{\lambda _{out}}{4n_{coating}} 
\]%
Putting in the numbers, our thickness would be 
\[
t=\left( \frac{1}{4}\right) \left( \frac{3.00\unit{cm}}{1.50}\right)
=\allowbreak 0.\,\allowbreak 5\unit{cm} 
\]%
Half a centimeter of coating doesn't seem too unreasonable!

Of course we could also make a plane that would be more visible to radar.
Suppose we are building a search and rescue plane. We could make a plane
coating that gave constructive interference. We would start with%
\[
\Delta \phi =\left( \frac{2\pi }{\lambda _{in}}\left( 2\mathfrak{t}\right)
+0\right) 
\]%
but this time we would make $\Delta \phi =m2\pi .$ This would give 
\[
m2\pi =\left( \frac{2\pi }{\lambda _{in}}\left( 2\mathfrak{t}\right) \right) 
\]%
\[
m2\pi =\left( \frac{2\pi }{\frac{\lambda _{air}}{n_{coating}}}\left( 2%
\mathfrak{t}\right) \right) 
\]%
which makes a coating thickness so that 
\[
\mathfrak{t}=\frac{m}{2}\left( \frac{\lambda _{air}}{n_{coating}}\right)
\qquad m=0,1,2,\cdots 
\]%
and now the coating will provide constructive interference, making it easier
to track on radar from the command center. For the thinnest possibility, set 
$m=1$ and%
\begin{eqnarray*}
t &=&m\frac{1}{2}\left( \frac{\lambda _{air}}{n_{coating}}\right) \\
&=&\frac{1}{2}\left( \frac{3.00\unit{cm}}{1.50}\right) \\
&=&\allowbreak 1\unit{cm}
\end{eqnarray*}

Note that we reasoned out these equations for the boundary conditions that
we have in our problem. If the boundary conditions change, so do the
equations. Take a soap bubble for example.\FRAME{dhFU}{3.6198in}{3.2233in}{%
0pt}{\Qcb{{\protect\small Interference from a soap bubble. (Bubble image in
the Public Domain, courtesy Marcin Der\k{e}gowski)}}}{}{Figure}{\special%
{language "Scientific Word";type "GRAPHIC";maintain-aspect-ratio
TRUE;display "USEDEF";valid_file "T";width 3.6198in;height 3.2233in;depth
0pt;original-width 3.5725in;original-height 3.1782in;cropleft "0";croptop
"1";cropright "1";cropbottom "0";tempfilename
'PQXXQXS8.wmf';tempfile-properties "XPR";}}Suppose we see the nice blue
color near the bottom, this must be constructive interference for blue if we
can see it. So the bubble thickness must be just right to make constructive
interference for blue. So $\Delta \phi =m2\pi $. Let's find the thickness.
Again start with.

\[
\Delta \phi =\left( k_{2}y_{2}-\omega _{2}t_{2}+\phi _{2}\right) -\left(
k_{1}y_{1}-\omega _{1}t_{1}+\phi _{1}\right) 
\]%
and again $\omega _{2}=\omega _{1}$ and $t_{2}=t_{1}$. We can see that one
wave (wave 1) bounces off the outside of the bubble film. For this wave $%
\lambda _{1}=\lambda _{air}$ and if we define $y_{a}=0$ at this boundary
than $y_{1}=0.$ For wave 2, the wave travels into the bubble material so it
travels a distance $y_{2}\approx 2\mathfrak{t}$ and there is a bounce, but
this bounce will be like a free end. The wave won't invert. So $\phi _{2}=0.$
Let's try to put this all into $\Delta \phi $%
\[
\Delta \phi =\left( k_{2}y_{2}-\omega t+0\right) -\left( 0-\omega t+\pi
\right) 
\]

Wave $2$ goes into the bubble film so once again%
\[
\lambda _{2}=\lambda _{in}=\frac{1}{n_{film}}\lambda _{out} 
\]%
\begin{eqnarray*}
\Delta \phi &=&\left( \frac{2\pi }{\lambda _{out}}n_{film}\left( 2\mathfrak{t%
}\right) -\omega t+0\right) -\left( 0-\omega t+\pi \right) \\
&=&\frac{2\pi }{\lambda _{out}}n_{film}\left( 2\mathfrak{t}\right) -\pi
\end{eqnarray*}%
Using our criteria for constructive interference we have 
\[
\frac{2\pi }{\lambda _{out}}n_{film}\left( 2\mathfrak{t}\right) -\pi =2\pi
m\qquad m=0,\pm 1,\pm 2,\pm 3,\cdots 
\]

Solving for the thickness 
\begin{eqnarray*}
\frac{1}{\lambda _{out}}n_{film}\left( 2\mathfrak{t}\right) -\frac{1}{2} &=&m
\\
\frac{1}{\lambda _{out}}n_{film}\left( 2\mathfrak{t}\right) &=&m+\frac{1}{2}
\\
\mathfrak{t} &=&\frac{\lambda _{out}}{2n_{film}}\left( m+\frac{1}{2}\right)
\end{eqnarray*}%
So if the thickness is 
\[
\mathfrak{t}=\frac{\lambda _{out}}{2n_{film}}\left( m+\frac{1}{2}\right)
\qquad m=0,1,2,\cdots 
\]%
There will be constructive interference. The thinnest film that will show
the color due to constructive interference would be the $m=0$ case 
\[
\mathfrak{t}=\frac{\lambda _{out}}{4n_{film}} 
\]

In each of these problems we used our criteria for constructive or
destructive interference 
\begin{eqnarray*}
\Delta \phi &=&2\pi m\qquad m=0,\pm 1,\pm 2,\pm 3,\cdots \qquad \text{%
Constructive} \\
\Delta \phi &=&\left( m+\frac{1}{2}\right) 2\pi \qquad m=0,\pm 1,\pm 2,\pm
3,\cdots \qquad \text{Destructive}
\end{eqnarray*}%
and we used what we know from the problem to determine the total phase
change using 
\[
\Delta \phi =\left( k_{2}x_{2}-\omega _{2}t_{2}+\phi _{2}\right) -\left(
k_{1}x_{1}-\omega _{1}t_{1}+\phi _{1}\right) 
\]%
We can view this as three main methods of causing a phase difference. One is
having a path difference where either $x_{2}$ and $x_{2}$ are different, or
the wavelengths are different along the path. Another is that the frequency
changes so that $\omega _{2}$ and $\omega _{1}$ are not the same. The final
is a phase shift, where either the waves start differently or there is a
change in the phase constant (like when there is a fixed end reflection).
This will work for most problems involving mixing of two waves. Let's do a
two dimensional problem.

\section{Single frequency interference in more than one dimension}

But what happens if our waves don't travel along the same line? Suppose you
are at a dance, and there are two speakers. Further suppose that you are
testing the system with a constant tone so $\omega _{2}=\omega _{1}$ (either
that or you have somewhat boring music with long, sustained tones). Suppose
the two speakers make waves in phase ($\phi _{1}=\phi _{2}$). There is no
slower medium, so we expect $k_{2}=k_{1}.$ If you are equal distance from
the two speakers, you would expect constructive interference because both $%
\Delta \phi _{o}=0$ and $\Delta x=0$ for this case.%
\begin{eqnarray*}
\Delta \phi &=&\left( k_{2}x_{2}-\omega _{2}t_{2}+\phi _{2}\right) -\left(
k_{1}x_{1}-\omega _{1}t_{1}+\phi _{1}\right) \\
&=&\left( kx_{2}-\omega t+\phi \right) -\left( kx_{1}-\omega t+\phi \right)
\\
&=&k\left( \Delta x\right)
\end{eqnarray*}%
If we pick a spot where $x_{1}=x_{2}$ we would expect 
\[
\Delta \phi =0 
\]%
and since $0$ is a multiple of $2\pi $ in our criteria, then we expect this
will give us constructive interference.\FRAME{dhF}{1.4218in}{1.9527in}{0pt}{%
}{}{Figure}{\special{language "Scientific Word";type
"GRAPHIC";maintain-aspect-ratio TRUE;display "USEDEF";valid_file "T";width
1.4218in;height 1.9527in;depth 0pt;original-width 3.0441in;original-height
4.1909in;cropleft "0";croptop "1";cropright "1";cropbottom "0";tempfilename
'PQXXQXS9.wmf';tempfile-properties "XPR";}}But there are more places where
we expect constructive interference, because we know the sound wave is
really spherical. Any time the path difference, $\Delta x=n\lambda ,$ then 
\[
\Delta \phi =\frac{2\pi }{\lambda }\left( n\lambda \right) =n2\pi 
\]%
and we will have constructive interference The next figure shows an example
where the path difference is one wavelength.

\FRAME{dhF}{1.5489in}{2.1318in}{0pt}{}{}{Figure}{\special{language
"Scientific Word";type "GRAPHIC";maintain-aspect-ratio TRUE;display
"USEDEF";valid_file "T";width 1.5489in;height 2.1318in;depth
0pt;original-width 3.8233in;original-height 5.2736in;cropleft "0";croptop
"1";cropright "1";cropbottom "0";tempfilename
'PQXXQXSA.wmf';tempfile-properties "XPR";}}But any of the spots in the next
figure will experience constructive interference. Note the loud spots are
where there are two crests or two troughs together.\FRAME{dhF}{1.6181in}{%
2.1231in}{0pt}{}{}{Figure}{\special{language "Scientific Word";type
"GRAPHIC";maintain-aspect-ratio TRUE;display "USEDEF";valid_file "T";width
1.6181in;height 2.1231in;depth 0pt;original-width 3.7256in;original-height
4.8983in;cropleft "0";croptop "1";cropright "1";cropbottom "0";tempfilename
'PQXXQXSB.wmf';tempfile-properties "XPR";}}We also expect to see destructive
interference. This should occur where path differences are multiples of $%
\lambda /2$ so that 
\begin{eqnarray*}
\Delta \phi &=&\frac{2\pi }{\lambda }\Delta x \\
&=&\frac{2\pi }{\lambda }\left( n\frac{\lambda }{2}\right) \\
&=&n\pi
\end{eqnarray*}%
The next figure shows a case where $\Delta x=\lambda /2$\FRAME{dhF}{1.6933in%
}{2.0401in}{0pt}{}{}{Figure}{\special{language "Scientific Word";type
"GRAPHIC";maintain-aspect-ratio TRUE;display "USEDEF";valid_file "T";width
1.6933in;height 2.0401in;depth 0pt;original-width 1.9882in;original-height
2.4007in;cropleft "0";croptop "1";cropright "1";cropbottom "0";tempfilename
'PQXXQXSC.wmf';tempfile-properties "XPR";}}and the next figure shows many
places where there will be destructive interference because the two waves
are out of phase by half a wavelength.\FRAME{dhF}{1.7141in}{1.9969in}{0pt}{}{%
}{Figure}{\special{language "Scientific Word";type
"GRAPHIC";maintain-aspect-ratio TRUE;display "USEDEF";valid_file "T";width
1.7141in;height 1.9969in;depth 0pt;original-width 2.5581in;original-height
2.9836in;cropleft "0";croptop "1";cropright "1";cropbottom "0";tempfilename
'PQXXQXSD.wmf';tempfile-properties "XPR";}}

When you moved from one dimension to two dimensions in PH121, we changed
from the variables $x$ and $y$ to the variable $r$ where 
\[
r=\sqrt{x^{2}+y^{2}} 
\]%
Thus for this case our phase becomes 
\[
\Delta \phi =\left( 2\pi \frac{\Delta r}{\lambda }+\Delta \phi _{o}\right) 
\]

In our dance example, suppose we have speakers that are $4\unit{m}$ apart
and we are standing $3\unit{m}$ directly in front of one of the speakers.
Further suppose that we play an $A$ just above middle $C$ which has a
frequency of $440\unit{Hz}.$ The speed of sound is $343\unit{m}/\unit{s}$.
Our speakers are connected to the same stereo with equal length wires. What
is the phase difference at this spot?

\FRAME{dhF}{3.1592in}{3.2932in}{0pt}{}{}{Figure}{\special{language
"Scientific Word";type "GRAPHIC";maintain-aspect-ratio TRUE;display
"USEDEF";valid_file "T";width 3.1592in;height 3.2932in;depth
0pt;original-width 3.1142in;original-height 3.2474in;cropleft "0";croptop
"1";cropright "1";cropbottom "0";tempfilename
'PQXXQXSE.wmf';tempfile-properties "XPR";}}From the geometry we can tell
that the path from the second speaker must be $5\unit{m}.$ So 
\begin{eqnarray*}
\Delta x &=&5\unit{m}-3\unit{m} \\
&=&2\unit{m}
\end{eqnarray*}%
We can tell that the wavelength is%
\begin{eqnarray*}
\lambda &=&\frac{v}{f} \\
&=&\frac{343\unit{m}/\unit{s}}{440\unit{Hz}} \\
&=&\allowbreak 0.779\,55\unit{m}
\end{eqnarray*}%
Since the speakers are connected to the same stereo with equal length wires, 
$\Delta \phi _{o}=0.$ Then 
\begin{eqnarray*}
\Delta \phi &=&\frac{2\pi }{\lambda }\Delta x+\Delta \phi _{o} \\
&=&\frac{2\pi }{\allowbreak 0.779\,55\unit{m}}\left( 2\unit{m}\right) +0 \\
&=&\allowbreak 5.\,\allowbreak 131\,2\pi \\
&=&2\pi +\allowbreak 3.\,\allowbreak 131\,2\pi
\end{eqnarray*}%
$\allowbreak \allowbreak $We should ask, is this constructive or destructive
interference? Well, it is neither purely constructive interference nor total
destructive interference. Our amplitude would be 
\[
2A\cos \left( \frac{1}{2}\left( \frac{2\pi }{\lambda }\Delta x+\Delta \phi
_{o}\right) \right) 
\]%
so in this case we get%
\[
2A\cos \left( \frac{1}{2}\left( 2\pi +\allowbreak 3.\,\allowbreak 131\,2\pi
\right) \right) =-0.409\,27A 
\]%
which is smaller (in magnitude) than $A,$ so it is partial destructive
interference. It would be quieter at this spot than if we just had one
speaker playing.

You might guess that this sort of analysis plays a large part in design of
concert halls. It also is important in mechanical designs. But you should
have seen a deficit in what we have learned so far. Up to this point, we
have only mixed waves that have the same frequency. Can we mix waves that
have different frequencies?

\section{Beats}

%TCIMACRO{%
%\TeXButton{Beat Demo}{\marginpar {
%\hspace{-0.5in}
%\begin{minipage}[t]{1in}
%\small{Beat Demo}
%\end{minipage}
%}}}%
%BeginExpansion
\marginpar {
\hspace{-0.5in}
\begin{minipage}[t]{1in}
\small{Beat Demo}
\end{minipage}
}%
%EndExpansion
Up till now we only superposed waves that had the same frequency. But what
happens if we take waves with different frequencies? Let's take the case
where $\phi _{2}=\phi _{1}=\phi _{o},$ $k_{2}=k_{1}=k$ and let's let the
waves mix at the same location $x_{2}=x_{1}$ so 
\[
y_{1}=y_{\max }\sin \left( kx-\omega _{1}t+\phi _{o}\right) 
\]%
\[
y_{2}=y_{\max }\sin \left( kx-\omega _{2}t+\phi _{o}\right) 
\]%
\begin{eqnarray*}
\Delta \phi &=&\left( kx-\omega _{1}t+\phi _{o}\right) -\left( kx-\omega
_{1}t+\phi _{o}\right) \\
&=&\left( \omega _{1}-\omega _{2}\right) t
\end{eqnarray*}%
We can use our criteria for constructive and destructive interference, but
before going on let's put this back into the total amplitude function for
two wave mixing. 
\begin{eqnarray*}
A &=&2y_{\max }\cos \left( \frac{1}{2}\Delta \phi \right) \\
&=&2y_{\max }\cos \left( \frac{1}{2}\left( \omega _{1}-\omega _{2}\right)
t\right)
\end{eqnarray*}%
and we can see that our amplitude will change in time. Sometimes it will be
constructive interference and sometimes destructive interference and often
in between.

We can plot both waves on the same graph and see that this will happen.%
\FRAME{dtbpFX}{4.9018in}{2.29in}{0pt}{}{}{Plot}{\special{language
"Scientific Word";type "MAPLEPLOT";width 4.9018in;height 2.29in;depth
0pt;display "USEDEF";plot_snapshots TRUE;mustRecompute FALSE;lastEngine
"MuPAD";xmin "-15";xmax "15";xviewmin "-8";xviewmax "8";yviewmin
"-2.5";yviewmax "2.5";viewset"XY";rangeset"X";plottype 4;labeloverrides
1;x-label "t";axesFont "Times New
Roman,12,0000000000,useDefault,normal";numpoints 100;plotstyle
"patch";axesstyle "normal";axestips FALSE;xis \TEXUX{x};var1name
\TEXUX{$x$};function \TEXUX{$\cos \left( 10x\right) $};linecolor
"maroon";linestyle 1;pointstyle "point";linethickness 3;lineAttributes
"Solid";var1range "-15,15";num-x-gridlines 900;curveColor
"[flat::RGB:0x00800000]";curveStyle "Line";rangeset"X";function \TEXUX{$\cos
\left( 11.x\right) $};linecolor "green";linestyle 1;pointstyle
"point";linethickness 3;lineAttributes "Solid";var1range
"-10,10";num-x-gridlines 900;curveColor "[flat::RGB:0x00008000]";curveStyle
"Line";rangeset"X";VCamFile 'PR5OPP04.xvz';valid_file "T";tempfilename
'PRGMKS03.wmf';tempfile-properties "XPR";}}Notice that there are places
where the waves are in phase, and places where they are not. The
superposition looks like this\FRAME{dtbpFX}{5.0375in}{2.3531in}{0pt}{}{}{Plot%
}{\special{language "Scientific Word";type "MAPLEPLOT";width 5.0375in;height
2.3531in;depth 0pt;display "USEDEF";plot_snapshots TRUE;mustRecompute
FALSE;lastEngine "MuPAD";xmin "-10";xmax "10";xviewmin "-8";xviewmax
"8";yviewmin "-5";yviewmax "5";viewset"XY";rangeset"X";plottype
4;labeloverrides 1;x-label "t";axesFont "Times New
Roman,12,0000000000,useDefault,normal";numpoints 100;plotstyle
"patch";axesstyle "normal";axestips FALSE;xis \TEXUX{x};var1name
\TEXUX{$x$};function \TEXUX{$\cos \left( 10.0x\right) +\cos \left(
11.x\right) $};linecolor "cyan";linestyle 1;pointstyle "point";linethickness
3;lineAttributes "Solid";var1range "-10,10";num-x-gridlines 900;curveColor
"[flat::RGB:0x00008080]";curveStyle "Line";rangeset"X";VCamFile
'PR5OPR05.xvz';valid_file "T";tempfilename
'PRGMKS04.wmf';tempfile-properties "XPR";}}

where there is constructive interference, the resulting wave amplitude is
large, where there is destructive interference, the resulting amplitude is
zero. We get a traveling wave who's amplitude varies. We can find the
amplitude function algebraically.

We can write out the entire resultant wave in our usually way. Our two waves
are%
\[
y_{1}=y_{\max }\sin \left( kx-\omega _{1}t+\phi _{o}\right) 
\]%
\[
y_{2}=y_{\max }\sin \left( kx-\omega _{2}t+\phi _{o}\right) 
\]

and the resultant 
\begin{eqnarray*}
&&y_{r}2y_{\max }\cos \left( \frac{\left( kx-\omega _{2}t+\phi _{o}\right)
-\left( kx-\omega _{1}t+\phi _{o}\right) }{2}\right) \sin \left( \frac{%
kx-\omega _{2}t+\phi _{o}+kx-\omega _{1}t+\phi _{o}}{2}\right) \\
&=&2y_{\max }\cos \left( \frac{kx-2\pi f_{2}t-\left( kx-2\pi f_{1}t\right) }{%
2}\right) \sin \left( \frac{kx-2\pi f_{2}t+kx-2\pi f_{1}t+2\phi _{o}}{2}%
\right) \\
&=&2y_{\max }\cos \left( 2\pi \frac{f_{1}-f_{2}}{2}t\right) \sin \left(
kx-2\pi \frac{f_{1}+f_{2}}{2}t+\phi _{o}\right) \\
&=&\left[ 2y_{\max }\cos \left( 2\pi \frac{f_{1}-f_{2}}{2}t\right) \right]
\sin \left( kx-2\pi \frac{f_{1}+f_{2}}{2}t+\phi _{o}\right)
\end{eqnarray*}

The sine part is a wave, it is a function of position and time. We see that
it has a frequency that is the average of $f_{1}$ and $f_{2}.$ This is the
frequency we hear. But we have another complicated amplitude term, and this
time it is a function of time just as we suspected. The amplitude has its
own frequency that is half the difference of $f_{1}$ and $f_{2}.$ 
\[
A_{\text{resultant}}=2A\cos \left( 2\pi \frac{f_{1}-f_{2}}{2}t\right) 
\]%
So the sound amplitude will vary in time for a given position in the medium.

The situation is odder still. We have a cosine function, but it is really an
envelope for the higher frequency motion of the air particles. The air
molecules move back and forth for both the crest and the trough of the
envelope function. \FRAME{dtbpFX}{5.0375in}{2.3531in}{0pt}{}{}{Plot}{\special%
{language "Scientific Word";type "MAPLEPLOT";width 5.0375in;height
2.3531in;depth 0pt;display "USEDEF";plot_snapshots TRUE;mustRecompute
FALSE;lastEngine "MuPAD";xmin "-10";xmax "10";xviewmin "-10";xviewmax
"10";yviewmin "-5";yviewmax "5";viewset"XY";rangeset"X";plottype
4;labeloverrides 1;x-label "t";axesFont "Times New
Roman,12,0000000000,useDefault,normal";numpoints 100;plotstyle
"patch";axesstyle "normal";axestips FALSE;xis \TEXUX{x};var1name
\TEXUX{$x$};function \TEXUX{$\cos \left( 10.0x\right) +\cos \left(
11.x\right) $};linecolor "green";linestyle 1;pointstyle
"point";linethickness 1;lineAttributes "Solid";var1range
"-10,10";num-x-gridlines 900;curveColor "[flat::RGB:0x00008000]";curveStyle
"Line";rangeset"X";function \TEXUX{$2\cos (\frac{10-11}{2}x)$};linecolor
"blue";linestyle 1;pointstyle "point";linethickness 3;lineAttributes
"Solid";var1range "-10,10";num-x-gridlines 100;curveColor
"[flat::RGB:0x000000ff]";curveStyle "Line";VCamFile
'PR5OPW06.xvz';valid_file "T";tempfilename
'PQXXQYSH.wmf';tempfile-properties "XPR";}}So we will hear two maxima for
every period! This frequency with which we hear the sound get loud at a
given location as the wave goes by is called the \emph{beat frequency}. The
red envelope (solid heavy line in the last figure) has a frequency of%
\[
f_{A}=\frac{f_{1}-f_{2}}{2} 
\]%
but it is just the envelope. We can see that the green (thin line) wave will
push and pull air molecules, and therefore our ear drums, with maximum
loudness at twice this frequency. So our beat frequency is 
\[
f_{beat}=\left\vert f_{1}-f_{2}\right\vert 
\]

Any time we mix waves of different frequencies we get beating. Often the
beat frequency is very fast, and our hearing system can't track the beats,
so we don't hear them. And if we mixed more than two waves the beats might
not come at perfectly regular intervals. The mixing of waves can become
quite complicated. Yet even a barbershop quartet is a mixing of at least
four waves. So complicated superpositions are common. In the next lecture we
will try to see how we could take on these complicated combined waves.

\chapter{Non Sinusoidal Waves}

%TCIMACRO{%
%\TeXButton{Fundamental Concepts}{\hspace{-1.3in}{\Large Fundamental Concepts\vspace{0.25in}}}}%
%BeginExpansion
\hspace{-1.3in}{\Large Fundamental Concepts\vspace{0.25in}}%
%EndExpansion

You have probably wondered if all waves are sinusoidal. Can the universe
really be described by such simple mathematics? The answer is both no, and
yes. There are non-sinusoidal waves, in fact, most waves are not sinusoidal.
But it turns out that we can use a very clever mathematical trick to make
any shape wave out of a superposition of many sinusoidal waves. So our
mathematics for sinusoidal waves turns out to be quite general.

\section{Music and Non-sinusoidal waves}

Let's take the example of music.

From our example of standing waves on strings, we know that a string can
support a series of standing waves with discrete frequencies--the harmonic
series. We have also discussed that usually we excite more than one standing
wave at a time. The fundamental mode tends to give us the pitch we hear, but
what are the other standing waves for?

To understand, lets take an analogy. Making cookies and cakes.

Here is the beginning of a recipe for cookies.

\FRAME{dhF}{3.704in}{2.4837in}{0pt}{}{}{Figure}{\special{language
"Scientific Word";type "GRAPHIC";maintain-aspect-ratio TRUE;display
"USEDEF";valid_file "T";width 3.704in;height 2.4837in;depth
0pt;original-width 3.6564in;original-height 2.4431in;cropleft "0";croptop
"1";cropright "1";cropbottom "0";tempfilename
'PQXXQYSI.wmf';tempfile-properties "XPR";}}The recipe is a list of
ingredients, and a symbolic instruction to mix and bake. The product is
chocolate chip cookies. Of course we need more information. We need to know
now much of each ingredient to use.\FRAME{dhF}{3.0052in}{2.3575in}{0pt}{}{}{%
Figure}{\special{language "Scientific Word";type
"GRAPHIC";maintain-aspect-ratio TRUE;display "USEDEF";valid_file "T";width
3.0052in;height 2.3575in;depth 0pt;original-width 2.962in;original-height
2.3177in;cropleft "0";croptop "1";cropright "1";cropbottom "0";tempfilename
'PQXXQYSJ.wmf';tempfile-properties "XPR";}}This graph gives us the amount of
each ingredient by mass.

Now suppose we want chocolate cake.\FRAME{dhF}{3.704in}{2.6091in}{0pt}{}{}{%
Figure}{\special{language "Scientific Word";type
"GRAPHIC";maintain-aspect-ratio TRUE;display "USEDEF";valid_file "T";width
3.704in;height 2.6091in;depth 0pt;original-width 3.6564in;original-height
2.5676in;cropleft "0";croptop "1";cropright "1";cropbottom "0";tempfilename
'PQXXQYSK.wmf';tempfile-properties "XPR";}}

The predominant taste in each of these foods is chocolate. But chocolate
cake and chocolate chip cookies don't taste exactly the same. We can easily
see that the differences in the other ingredients make the difference
between the \textquotedblleft cookie\textquotedblright\ taste and the
\textquotedblleft cake\textquotedblright\ taste that goes along with the
\textquotedblleft chocolate\textquotedblright\ taste that predominates.

The sound waves produced by musical instruments work in a similar way. Here
is a recipe for an \textquotedblleft A\textquotedblright\ note from a
clarinet.\FRAME{dhF}{3.7879in}{2.7207in}{0pt}{}{}{Figure}{\special{language
"Scientific Word";type "GRAPHIC";maintain-aspect-ratio TRUE;display
"USEDEF";valid_file "T";width 3.7879in;height 2.7207in;depth
0pt;original-width 3.7395in;original-height 2.6783in;cropleft "0";croptop
"1";cropright "1";cropbottom "0";tempfilename
'PQXXQYSL.wmf';tempfile-properties "XPR";}}and here is one for a trumpet
playing the same \textquotedblleft A\textquotedblright\ note.\FRAME{dhF}{%
3.2569in}{2.3021in}{0pt}{}{}{Figure}{\special{language "Scientific
Word";type "GRAPHIC";maintain-aspect-ratio TRUE;display "USEDEF";valid_file
"T";width 3.2569in;height 2.3021in;depth 0pt;original-width
3.2119in;original-height 2.2623in;cropleft "0";croptop "1";cropright
"1";cropbottom "0";tempfilename 'PQXXQYSM.wmf';tempfile-properties "XPR";}}

A trumpet sounds different than a clarinet, and now we see why. There are
more harmonics involved with the trumpet sound than the clarinet sound.
These extra standing waves make up the \textquotedblleft
brassiness\textquotedblright\ of the trumpet sound. As with our baking
example, we need to know how much of each standing wave we have. Each will
have a different amplitude. For our trumpet, we might get amplitudes as
shown.

\FRAME{dhF}{3.0052in}{2.3575in}{0in}{}{}{Figure}{\special{language
"Scientific Word";type "GRAPHIC";maintain-aspect-ratio TRUE;display
"USEDEF";valid_file "T";width 3.0052in;height 2.3575in;depth
0in;original-width 2.962in;original-height 2.3177in;cropleft "0";croptop
"1";cropright "1";cropbottom "0";tempfilename
'PQXXQYSN.wmf';tempfile-properties "XPR";}}Note that the second harmonic has
a larger amplitude, but we still hear the \textquotedblleft
A\textquotedblright\ as at $440\unit{Hz}.$ A fugal horn would still sound
brassy, but would have a different mix of harmonics.

We have a tool that you can download to your PC to detect the mix of
harmonics of musical instruments, or mechanical systems. In music, the
different harmonics are called \emph{partials} because they make up part of
the sound. A graph that shows which harmonics are involved is called a \emph{%
spectrum}. The next figure is the spectrum of a six holed bamboo flute. Note
that there are several harmonics involved.

\FRAME{dhF}{3.2145in}{2.3445in}{0in}{}{}{Figure}{\special{language
"Scientific Word";type "GRAPHIC";maintain-aspect-ratio TRUE;display
"USEDEF";valid_file "T";width 3.2145in;height 2.3445in;depth
0in;original-width 3.1695in;original-height 2.3039in;cropleft "0";croptop
"1";cropright "1";cropbottom "0";tempfilename
'PQXXQYSO.wmf';tempfile-properties "XPR";}}Note that our graph has two
parts. One is the instantaneous spectrum, and one is the spectrum time
history.\FRAME{dhF}{4.6267in}{1.9389in}{0in}{}{}{Figure}{\special{language
"Scientific Word";type "GRAPHIC";maintain-aspect-ratio TRUE;display
"USEDEF";valid_file "T";width 4.6267in;height 1.9389in;depth
0in;original-width 4.574in;original-height 1.9009in;cropleft "0";croptop
"1";cropright "1";cropbottom "0";tempfilename
'PQXXQYSP.wmf';tempfile-properties "XPR";}}By observing the time history, we
can see changes in the spectrum. We can also see that we don't have pure
harmonics. The graph shows some response off the specific harmonic
frequencies. This six holed flute is very \textquotedblleft
breathy\textquotedblright\ giving a lot of wind noise along with the notes,
and we see this in the spectrum. In the next picture, I played a scale on
the flute.\FRAME{dhF}{2.7674in}{1.8421in}{0pt}{}{}{Figure}{\special{language
"Scientific Word";type "GRAPHIC";maintain-aspect-ratio TRUE;display
"USEDEF";valid_file "T";width 2.7674in;height 1.8421in;depth
0pt;original-width 2.725in;original-height 1.8049in;cropleft "0";croptop
"1";cropright "1";cropbottom "0";tempfilename
'PQXXQYSQ.wmf';tempfile-properties "XPR";}}The instantaneous spectrum is not
active in this figure (since it can't show more than one note at a time) but
in the time history we see that as the fundamental frequency changes by
shorting the length of the flute (uncovering holes), we see that every
partial also goes up in frequency. The flute still has the characteristic
spectrum of a flute, but shifted to new frequencies. We can use this fact to
identify things by their vibration spectrum. In fact, that is how you
recognize voices and musics within your auditory system!

The technique of taking apart a wave into its components is very powerful.
With light waves, the spectrum is an indication of the chemical composition
of the emitter. For example, the spectrum of the sun looks something like
this\FRAME{dhFU}{4.6265in}{1.4644in}{0pt}{\Qcb{{\protect\small Solar coronal
spectrum taken during a solar eclipse. The successive curved lines are each
different wavelengths, and the dark lines are wavelengths that are absorbed.
The pattern of absorbed wavelengths allows a chemical analysis of the
corona. (Image in the Public Domain, orignally published in Bailey, Solon,
L, Popular Science Monthly, Vol 60, Nov. 1919, pp 244)}}}{}{Figure}{\special%
{language "Scientific Word";type "GRAPHIC";maintain-aspect-ratio
TRUE;display "USEDEF";valid_file "T";width 4.6265in;height 1.4644in;depth
0pt;original-width 4.574in;original-height 1.4295in;cropleft "0";croptop
"1";cropright "1";cropbottom "0";tempfilename
'PQXXQYSR.wmf';tempfile-properties "XPR";}}The lines in this graph show the
amplitude of each harmonic component of the light. Darker lines have larger
amplitudes. The harmonics come from the excitation of electrons in their
orbitals. Each orbital is a different energy state, and when the electrons
jump from orbital to orbital, they produce specific wave frequencies. By
observing the mix of dark lines in pervious figure, and comparing to
laboratory measurements from each element (see next figure) we can find the
composition of the source. This figure shows the emission spectrum for
Calcium. because it is an emission spectrum the lines are bright instead of
dark. We can even see the color of each line!\FRAME{dhFU}{4.7693in}{0.9774in%
}{0pt}{\Qcb{{\protect\small Emission spectrum of Calcium (Image in the
Public Domain, courtesy NASA)}}}{}{Figure}{\special{language "Scientific
Word";type "GRAPHIC";maintain-aspect-ratio TRUE;display "USEDEF";valid_file
"T";width 4.7693in;height 0.9774in;depth 0pt;original-width
6.2699in;original-height 1.2626in;cropleft "0";croptop "1";cropright
"1";cropbottom "0";tempfilename 'PQXXQYSS.wmf';tempfile-properties "XPR";}}

\section{Vibrometry}

Just like each atom has a specific spectrum, and each instrument, each
engine, machine, or anything that vibrates has a spectrum. We can use this
to monitor the health of machinery, or even to identify a piece of
equipment. Laser or acoustic vibrometers are available commercially.\FRAME{%
dhFU}{3.7572in}{1.64in}{0pt}{\Qcb{{\protect\small Laser Vibrometer Schematic
(Public Domain Image from Laderaranch:
http://commons.wikimedia.org/wiki/File:LDV\_Schematic.png)}}}{}{Figure}{%
\special{language "Scientific Word";type "GRAPHIC";maintain-aspect-ratio
TRUE;display "USEDEF";valid_file "T";width 3.7572in;height 1.64in;depth
0pt;original-width 9.7293in;original-height 4.2309in;cropleft "0";croptop
"1";cropright "1";cropbottom "0";tempfilename
'PQXXQYST.bmp';tempfile-properties "XPR";}}They provide a way to monitor
equipment in places where it would be dangerous or even impossible to send a
person. The equipment also does not need to be shut down, a great benefit
for factories that are never shut down, or for a satellite system that
cannot be reached by anyone.

Fourier Series: Mathematics of Non-Sinusoidal Waves

We should take a quick look at the mathematics of non-sinusoidal waves.

Let' start with a superposition of many sinusoidal waves. The math looks
like this%
\[
y\left( t\right) =\dsum\limits_{n}\left( A_{n}\sin \left( 2\pi f_{n}t\right)
+B_{n}\cos \left( 2\pi f_{n}t\right) \right) 
\]%
where $A_{n}$ and $B_{n}$ are a series of coefficients and $f_{n}$ are the
harmonic series of frequencies.

Example: Fourier representation of a square wave.

For example, we could represent a function $f\left( x\right) $ with the
following series%
\begin{eqnarray}
f\left( x\right) &=&C_{o}+C_{1}\cos \left( \frac{2\pi }{\lambda }%
x+\varepsilon _{1}\right) \\
&&+C_{2}\cos \left( \frac{2\pi }{\frac{\lambda }{2}}x+\varepsilon _{2}\right)
\\
&&+C_{3}\cos \left( \frac{2\pi }{\frac{\lambda }{3}}x+\varepsilon _{3}\right)
\\
&&+\ldots \\
&&+C_{n}\cos \left( \frac{2\pi }{\frac{\lambda }{n}}x+\varepsilon _{n}\right)
\\
&&+\ldots
\end{eqnarray}

where we will let $\varepsilon _{i}=\omega _{i}t+\phi _{i}$

The $C^{\prime }s$ are just coefficients that tell us the amplitude of the
individual cosine waves. The more terms in the series we take, the better
the approximation we will have, with the series exactly matching $f\left(
x\right) $ when the number of terms, $N\rightarrow \infty .$

Usually we rewrite the terms of the series as 
\begin{equation}
C_{m}\cos \left( mkx+\varepsilon _{m}\right) =A_{m}\cos \left( mkx\right)
+B_{m}\sin \left( mkx\right)
\end{equation}%
where $k$ is the wavenumber%
\begin{equation}
k=\frac{2\pi }{\lambda }
\end{equation}%
and $\lambda $ is the wavelength of the complicated but still periodic
function $f\left( x\right) .$ Then we identify 
\begin{eqnarray}
A_{m} &=&C_{m}\cos \left( \varepsilon _{m}\right) \\
B_{m} &=&-C_{m}\sin \left( \varepsilon _{m}\right)
\end{eqnarray}%
then%
\begin{equation}
f\left( x\right) =\frac{A_{o}}{2}+\dsum\limits_{m-1}^{\infty }A_{m}\cos
\left( mkx\right) +\dsum\limits_{m-1}^{\infty }B_{m}\sin \left( mkx\right)
\label{Fourier Series}
\end{equation}%
where we separated out the $A_{o}/2$ term because it mikes things nicer
later.

Fourier Analysis

The process of finding the coefficients of the series is called \emph{%
Fourier analysis}. We start by integrating equation (\ref{Fourier Series})%
\begin{equation}
\int_{0}^{\lambda }f\left( x\right) dx=\int_{0}^{\lambda }\frac{A_{o}}{2}%
dx+\int_{0}^{\lambda }\dsum\limits_{m-1}^{\infty }A_{m}\cos \left(
mkx\right) dx+\int_{0}^{\lambda }\dsum\limits_{m-1}^{\infty }B_{m}\sin
\left( mkx\right) dx
\end{equation}%
We can see immediately that all the sine and cosine terms integrate to zero
(we integrated over a wavelength) so%
\begin{equation}
\int_{0}^{\lambda }f\left( x\right) dx=\int_{0}^{\lambda }\frac{A_{o}}{2}dx=%
\frac{A_{o}}{2}\lambda
\end{equation}%
We solve this for $A_{o}$%
\begin{equation}
A_{o}=\frac{2}{\lambda }\int_{0}^{\lambda }f\left( x\right) dx
\end{equation}

To find the rest of the coefficients we need to remind ourselves of the
orthogonality of sinusoidal functions%
\begin{eqnarray}
\int_{0}^{\lambda }\sin \left( akx\right) \cos \left( bkx\right) dx &=&0 \\
\int_{0}^{\lambda }\cos \left( akx\right) \cos \left( bkx\right) dx &=&\frac{%
\lambda }{2}\delta _{ab} \\
\int_{0}^{\lambda }\sin \left( akx\right) \sin \left( bkx\right) dx &=&\frac{%
\lambda }{2}\delta _{ab}
\end{eqnarray}%
where $\delta _{ab}$ is the Kronecker delta.

To find the coefficients, then, we multiply both sides of equation (\ref%
{Fourier Series}) by $\cos \left( lkx\right) $ where $l$ is a positive
integer. Then we integrate over one wavelength.%
\begin{eqnarray}
\int_{0}^{\lambda }f\left( x\right) \cos \left( lkx\right) dx
&=&\int_{0}^{\lambda }\frac{A_{o}}{2}\cos \left( lkx\right) dx \\
&&+\int_{0}^{\lambda }\dsum\limits_{m-1}^{\infty }A_{m}\cos \left(
mkx\right) \cos \left( lkx\right) dx \\
&&+\int_{0}^{\lambda }\dsum\limits_{m-1}^{\infty }B_{m}\sin \left(
mkx\right) \cos \left( lkx\right) dx
\end{eqnarray}%
which gives 
\begin{equation}
\int_{0}^{\lambda }f\left( x\right) \cos \left( mkx\right)
dx=\int_{0}^{\lambda }A_{m}\cos \left( mkx\right) \cos \left( mkx\right) dx
\end{equation}%
that is, only the term with two cosine functions where $l=m$ will be non
zero. So%
\begin{equation}
\int_{0}^{\lambda }f\left( x\right) \cos \left( mkx\right) dx=\frac{\lambda 
}{2}A_{m}
\end{equation}%
solving for $A_{m}$ we have%
\begin{equation}
A_{m}=\frac{2}{\lambda }\int_{0}^{\lambda }f\left( x\right) \cos \left(
mkx\right) dx
\end{equation}

We can perform the same steps to find $B_{m}$ only we use $\sin \left(
lkx\right) $ as the multiplier. Then we find%
\begin{equation}
B_{m}=\frac{2}{\lambda }\int_{0}^{\lambda }f\left( x\right) \sin \left(
mkx\right) dx
\end{equation}

\subsection{Square wave}

Let's find the series for a square wave using our Fourier analysis technique.

Let's take 
\begin{equation}
\lambda =2
\end{equation}%
\begin{equation}
f(x)=\left\{ 
\begin{array}{ccl}
1 & \text{if} & 0<x<\frac{\lambda }{2} \\ 
-1 & \text{if} & \frac{\lambda }{2}<x<\lambda%
\end{array}%
\right.
\end{equation}

\FRAME{dtbpFX}{2.3644in}{1.5774in}{0pt}{}{}{Plot}{\special{language
"Scientific Word";type "MAPLEPLOT";width 2.3644in;height 1.5774in;depth
0pt;display "USEDEF";plot_snapshots TRUE;mustRecompute FALSE;lastEngine
"MuPAD";xmin "-5.0020002";xmax "5.0020002";xviewmin "-5.0020002";xviewmax
"5.0020002";yviewmin "-2.00080009";yviewmax
"2.00090009";viewset"XY";rangeset"X";plottype 4;axesFont "Times New
Roman,12,0000000000,useDefault,normal";numpoints 100;plotstyle
"patch";axesstyle "normal";axestips FALSE;xis \TEXUX{x};var1name
\TEXUX{$x$};function \TEXUX{$\left\{
\MATRIX{3,2}{c}\VR{,,c,,,}{,,c,,,}{,,l,,,}{,,,,,}\HR{,,}\CELL{1}\CELL{%
\text{if}}\CELL{0<x<\frac{\left( 2\right)
}{2}}\CELL{-1}\CELL{\text{if}}\CELL{\frac{\left( 2\right) }{2}<x<\left(
2\right) }\right. $};linecolor "green";linestyle 1;pointstyle
"point";linethickness 3;lineAttributes "Solid";var1range
"-5.0020002,5.0020002";num-x-gridlines 100;curveColor
"[flat::RGB:0x00008000]";curveStyle "Line";function \TEXUX{$\left\{
\MATRIX{3,2}{c}\VR{,,c,,,}{,,c,,,}{,,l,,,}{,,,,,}\HR{,,}\CELL{1}\CELL{%
\text{if}}\CELL{\left( 2\right) <x<\frac{3\left( 2\right)
}{2}}\CELL{-1}\CELL{\text{if}}\CELL{\frac{3\left( 2\right) }{2}<x<2\left(
2\right) }\right. $};linecolor "green";linestyle 1;pointstyle
"point";linethickness 3;lineAttributes "Solid";var1range
"-5.0020002,5.0020002";num-x-gridlines 100;curveColor
"[flat::RGB:0x00008000]";curveStyle "Line";function \TEXUX{$\left\{
\MATRIX{3,2}{c}\VR{,,c,,,}{,,c,,,}{,,l,,,}{,,,,,}\HR{,,}\CELL{1}\CELL{%
\text{if}}\CELL{2\left( 2\right) <x<\frac{5\left( 2\right)
}{2}}\CELL{-1}\CELL{\text{if}}\CELL{\frac{5\left( 2\right) }{2}<x<3\left(
2\right) }\right. $};linecolor "green";linestyle 1;pointstyle
"point";linethickness 3;lineAttributes "Solid";var1range
"-5.0020002,5.0020002";num-x-gridlines 100;curveColor
"[flat::RGB:0x00008000]";curveStyle "Line";function \TEXUX{$\left\{
\MATRIX{3,2}{c}\VR{,,c,,,}{,,c,,,}{,,l,,,}{,,,,,}\HR{,,}\CELL{-1}\CELL{%
\text{if}}\CELL{0>x>-1\frac{\left( 2\right)
}{2}}\CELL{1}\CELL{\text{if}}\CELL{-\frac{\left( 2\right) }{2}>x>-\left(
2\right) }\right. $};linecolor "green";linestyle 1;pointstyle
"point";linethickness 3;lineAttributes "Solid";var1range
"-5.0020002,5.0020002";num-x-gridlines 100;curveColor
"[flat::RGB:0x00008000]";curveStyle "Line";function \TEXUX{$\left\{
\MATRIX{3,2}{c}\VR{,,c,,,}{,,c,,,}{,,l,,,}{,,,,,}\HR{,,}\CELL{-1}\CELL{%
\text{if}}\CELL{-\left( 2\right) >x>-\frac{3\left( 2\right)
}{2}}\CELL{1}\CELL{\text{if}}\CELL{-\frac{3\left( 2\right) }{2}>x>-2\left(
2\right) }\right. $};linecolor "green";linestyle 1;pointstyle
"point";linethickness 3;lineAttributes "Solid";var1range
"-5.0020002,5.0020002";num-x-gridlines 100;curveColor
"[flat::RGB:0x00008000]";curveStyle "Line";function \TEXUX{$\left\{
\MATRIX{3,2}{c}\VR{,,c,,,}{,,c,,,}{,,l,,,}{,,,,,}\HR{,,}\CELL{-1}\CELL{%
\text{if}}\CELL{-2\left( 2\right) >x>-\frac{5\left( 2\right)
}{2}}\CELL{1}\CELL{\text{if}}\CELL{-\frac{5\left( 2\right) }{2}>x>-3\left(
2\right) }\right. $};linecolor "green";linestyle 1;pointstyle
"point";linethickness 3;lineAttributes "Solid";var1range
"-5.0020002,5.0020002";num-x-gridlines 100;curveColor
"[flat::RGB:0x00008000]";curveStyle "Line";VCamFile
'PQXXR22K.xvz';valid_file "T";tempfilename
'PQXXQYSU.wmf';tempfile-properties "XPR";}}

since $f\left( x\right) is$ odd, $A_{m}=0$ for all $m$. We have%
\begin{equation}
B_{m}=\frac{2}{\lambda }\int_{0}^{\frac{\lambda }{2}}\left( 1\right) \sin
\left( mkx\right) dx+\frac{2}{\lambda }\int_{\frac{\lambda }{2}}^{\lambda
}\left( -1\right) \sin \left( mkx\right) dx
\end{equation}%
so%
\begin{equation}
B_{m}=\frac{1}{m\pi }\left( -\cos \left( mkx\right) \mathstrut \right\vert
_{0}^{\frac{\lambda }{2}}+\frac{1}{m\pi }\left( \cos \left( mkx\right)
\mathstrut \right\vert _{\frac{\lambda }{2}}^{\lambda }
\end{equation}%
Which is%
\begin{equation}
B_{m}=\frac{1}{m\pi }\left( 1\cos \left( m\frac{2\pi }{\lambda }x\right)
\mathstrut \right\vert _{0}^{\frac{\lambda }{2}}+\frac{1}{m\pi }\left( \cos
\left( m\frac{2\pi }{\lambda }x\right) \right\vert _{\frac{\lambda }{2}%
}^{\lambda }
\end{equation}%
so%
\begin{eqnarray}
B_{m} &=&\frac{1}{m\pi }\left( \left( -\cos \left( m\frac{2\pi }{\lambda }%
\frac{\lambda }{2}\right) \right) +\cos \left( m\frac{2\pi }{\lambda }\left(
0\right) \right) \right) \\
&&+\frac{1}{m\pi }\left( \left( \cos \left( m\frac{2\pi }{\lambda }\lambda
\right) -\cos \left( m\frac{2\pi }{\lambda }\frac{\lambda }{2}\right)
\right) \right)
\end{eqnarray}%
which is%
\begin{equation}
B_{m}=\frac{2}{m\pi }\left( 1-\cos \left( m\pi \right) \right)
\end{equation}

Our series is then just 
\begin{equation}
f\left( x\right) =\dsum\limits_{m-1}^{\infty }\frac{2}{m\pi }\left( 1-\cos
\left( m\pi \right) \right) \sin \left( mkx\right)
\end{equation}%
and we can write a few terms%
\begin{equation}
\begin{tabular}{|l|l|}
\hline
$Term$ &  \\ \hline
1 & $\frac{4}{\pi }\sin \left( kx\right) $ \\ \hline
2 & $0$ \\ \hline
3 & $\frac{4}{3\pi }\sin \left( 3kx\right) $ \\ \hline
4 & $0$ \\ \hline
5 & $\frac{4}{5\pi }\sin \left( 5kx\right) $ \\ \hline
\end{tabular}%
\end{equation}%
then the partial sum up to $m=5$ looks like%
\begin{equation}
f\left( x\right) =\frac{4}{\pi }\sin \left( kx\right) +\frac{4}{3\pi }\sin
\left( 3kx\right) +\frac{4}{5\pi }\sin \left( 5kx\right)
\end{equation}%
\FRAME{dtbpFX}{2.3644in}{1.5774in}{0pt}{}{}{Plot}{\special{language
"Scientific Word";type "MAPLEPLOT";width 2.3644in;height 1.5774in;depth
0pt;display "USEDEF";plot_snapshots TRUE;mustRecompute FALSE;lastEngine
"MuPAD";xmin "-5.1";xmax "5";xviewmin "-5.1";xviewmax "5";yviewmin
"-2";yviewmax "2.0001";viewset"XY";rangeset"X";plottype 4;axesFont "Times
New Roman,12,0000000000,useDefault,normal";numpoints 100;plotstyle
"patch";axesstyle "normal";axestips FALSE;xis \TEXUX{x};var1name
\TEXUX{$x$};function \TEXUX{$\left\{
\MATRIX{3,2}{c}\VR{,,c,,,}{,,c,,,}{,,l,,,}{,,,,,}\HR{,,}\CELL{1}\CELL{%
\text{if}}\CELL{0<x<\frac{\left( 2\right)
}{2}}\CELL{-1}\CELL{\text{if}}\CELL{\frac{\left( 2\right) }{2}<x<\left(
2\right) }\right. $};linecolor "green";linestyle 1;pointstyle
"point";linethickness 3;lineAttributes "Solid";var1range
"-5.1,5";num-x-gridlines 100;curveColor "[flat::RGB:0x00008000]";curveStyle
"Line";function \TEXUX{$\left\{
\MATRIX{3,2}{c}\VR{,,c,,,}{,,c,,,}{,,l,,,}{,,,,,}\HR{,,}\CELL{1}\CELL{%
\text{if}}\CELL{\left( 2\right) <x<\frac{3\left( 2\right)
}{2}}\CELL{-1}\CELL{\text{if}}\CELL{\frac{3\left( 2\right) }{2}<x<2\left(
2\right) }\right. $};linecolor "green";linestyle 1;pointstyle
"point";linethickness 3;lineAttributes "Solid";var1range
"-5.1,5";num-x-gridlines 100;curveColor "[flat::RGB:0x00008000]";curveStyle
"Line";function \TEXUX{$\left\{
\MATRIX{3,2}{c}\VR{,,c,,,}{,,c,,,}{,,l,,,}{,,,,,}\HR{,,}\CELL{1}\CELL{%
\text{if}}\CELL{2\left( 2\right) <x<\frac{5\left( 2\right)
}{2}}\CELL{-1}\CELL{\text{if}}\CELL{\frac{5\left( 2\right) }{2}<x<3\left(
2\right) }\right. $};linecolor "green";linestyle 1;pointstyle
"point";linethickness 3;lineAttributes "Solid";var1range
"-5.1,5";num-x-gridlines 100;curveColor "[flat::RGB:0x00008000]";curveStyle
"Line";function \TEXUX{$\left\{
\MATRIX{3,2}{c}\VR{,,c,,,}{,,c,,,}{,,l,,,}{,,,,,}\HR{,,}\CELL{-1}\CELL{%
\text{if}}\CELL{0>x>-1\frac{\left( 2\right)
}{2}}\CELL{1}\CELL{\text{if}}\CELL{-\frac{\left( 2\right) }{2}>x>-\left(
2\right) }\right. $};linecolor "green";linestyle 1;pointstyle
"point";linethickness 3;lineAttributes "Solid";var1range
"-5.1,5";num-x-gridlines 100;curveColor "[flat::RGB:0x00008000]";curveStyle
"Line";function \TEXUX{$\left\{
\MATRIX{3,2}{c}\VR{,,c,,,}{,,c,,,}{,,l,,,}{,,,,,}\HR{,,}\CELL{-1}\CELL{%
\text{if}}\CELL{-\left( 2\right) >x>-\frac{3\left( 2\right)
}{2}}\CELL{1}\CELL{\text{if}}\CELL{-\frac{3\left( 2\right) }{2}>x>-2\left(
2\right) }\right. $};linecolor "green";linestyle 1;pointstyle
"point";linethickness 3;lineAttributes "Solid";var1range
"-5.1,5";num-x-gridlines 100;curveColor "[flat::RGB:0x00008000]";curveStyle
"Line";function \TEXUX{$\left\{
\MATRIX{3,2}{c}\VR{,,c,,,}{,,c,,,}{,,l,,,}{,,,,,}\HR{,,}\CELL{-1}\CELL{%
\text{if}}\CELL{-2\left( 2\right) >x>-\frac{5\left( 2\right)
}{2}}\CELL{1}\CELL{\text{if}}\CELL{-\frac{5\left( 2\right) }{2}>x>-3\left(
2\right) }\right. $};linecolor "green";linestyle 1;pointstyle
"point";linethickness 3;lineAttributes "Solid";var1range
"-5.1,5";num-x-gridlines 100;curveColor "[flat::RGB:0x00008000]";curveStyle
"Line";function \TEXUX{$0.424\,41\sin 9.\,\allowbreak 424\,8x+\allowbreak
1.\,\allowbreak 273\,2\sin 3.\,\allowbreak 141\,6x+\allowbreak 0.254\,65\sin
15.\,\allowbreak 708x\allowbreak $};linecolor "blue";linestyle 1;pointstyle
"point";linethickness 1;lineAttributes "Solid";var1range
"-5.1,5";num-x-gridlines 100;curveColor "[flat::RGB:0x000000ff]";curveStyle
"Line";VCamFile 'PQXXR22J.xvz';valid_file "T";tempfilename
'PQXXQYSV.wmf';tempfile-properties "XPR";}}If we take many terms,%
\begin{eqnarray}
f\left( x\right) &=&\frac{4}{\pi }\sin \left( kx\right) +\frac{4}{3\pi }\sin
\left( 3kx\right) +\frac{4}{5\pi }\sin \left( 5kx\right) +\frac{4}{7\pi }%
\sin \left( 7kx\right) +\frac{4}{9\pi }\sin \left( 95kx\right) \\
&&+\frac{4}{11\pi }\sin \left( 11kx\right) +\frac{4}{13\pi }\sin \left(
13kx\right) +\frac{4}{15\pi }\sin \left( 15kx\right) +\frac{4}{17\pi }\sin
\left( 17kx\right) +\frac{4}{19\pi }\sin \left( 19kx\right)  \nonumber
\end{eqnarray}%
We see the function get closer and closer to a square wave.\FRAME{dtbpFX}{%
5.1975in}{1.1026in}{0pt}{}{}{Plot}{\special{language "Scientific Word";type
"MAPLEPLOT";width 5.1975in;height 1.1026in;depth 0pt;display
"USEDEF";plot_snapshots TRUE;mustRecompute FALSE;lastEngine "MuPAD";xmin
"-2.5";xmax "2.5";xviewmin "-2.51";xviewmax "2.5";yviewmin "-2";yviewmax
"2.0001";viewset"XY";rangeset"X";plottype 4;axesFont "Times New
Roman,12,0000000000,useDefault,normal";numpoints 100;plotstyle
"patch";axesstyle "normal";axestips FALSE;xis \TEXUX{x};var1name
\TEXUX{$x$};function \TEXUX{$\left\{
\MATRIX{3,2}{c}\VR{,,c,,,}{,,c,,,}{,,l,,,}{,,,,,}\HR{,,}\CELL{1}\CELL{%
\text{if}}\CELL{0<x<\frac{\left( 2\right)
}{2}}\CELL{-1}\CELL{\text{if}}\CELL{\frac{\left( 2\right) }{2}<x<\left(
2\right) }\right. $};linecolor "green";linestyle 1;pointstyle
"point";linethickness 3;lineAttributes "Solid";var1range
"-2.5,2.5";num-x-gridlines 100;curveColor
"[flat::RGB:0x00008000]";curveStyle "Line";rangeset"X";function
\TEXUX{$\left\{
\MATRIX{3,2}{c}\VR{,,c,,,}{,,c,,,}{,,l,,,}{,,,,,}\HR{,,}\CELL{1}\CELL{%
\text{if}}\CELL{\left( 2\right) <x<\frac{3\left( 2\right)
}{2}}\CELL{-1}\CELL{\text{if}}\CELL{\frac{3\left( 2\right) }{2}<x<2\left(
2\right) }\right. $};linecolor "green";linestyle 1;pointstyle
"point";linethickness 3;lineAttributes "Solid";var1range
"-2.51,2.5";num-x-gridlines 100;curveColor
"[flat::RGB:0x00008000]";curveStyle "Line";function \TEXUX{$\left\{
\MATRIX{3,2}{c}\VR{,,c,,,}{,,c,,,}{,,l,,,}{,,,,,}\HR{,,}\CELL{1}\CELL{%
\text{if}}\CELL{2\left( 2\right) <x<\frac{5\left( 2\right)
}{2}}\CELL{-1}\CELL{\text{if}}\CELL{\frac{5\left( 2\right) }{2}<x<3\left(
2\right) }\right. $};linecolor "green";linestyle 1;pointstyle
"point";linethickness 3;lineAttributes "Solid";var1range
"-2.51,2.5";num-x-gridlines 100;curveColor
"[flat::RGB:0x00008000]";curveStyle "Line";function \TEXUX{$\left\{
\MATRIX{3,2}{c}\VR{,,c,,,}{,,c,,,}{,,l,,,}{,,,,,}\HR{,,}\CELL{-1}\CELL{%
\text{if}}\CELL{0>x>-1\frac{\left( 2\right)
}{2}}\CELL{1}\CELL{\text{if}}\CELL{-\frac{\left( 2\right) }{2}>x>-\left(
2\right) }\right. $};linecolor "green";linestyle 1;pointstyle
"point";linethickness 3;lineAttributes "Solid";var1range
"-2.51,2.5";num-x-gridlines 100;curveColor
"[flat::RGB:0x00008000]";curveStyle "Line";function \TEXUX{$\left\{
\MATRIX{3,2}{c}\VR{,,c,,,}{,,c,,,}{,,l,,,}{,,,,,}\HR{,,}\CELL{-1}\CELL{%
\text{if}}\CELL{-\left( 2\right) >x>-\frac{3\left( 2\right)
}{2}}\CELL{1}\CELL{\text{if}}\CELL{-\frac{3\left( 2\right) }{2}>x>-2\left(
2\right) }\right. $};linecolor "green";linestyle 1;pointstyle
"point";linethickness 3;lineAttributes "Solid";var1range
"-2.51,2.5";num-x-gridlines 100;curveColor
"[flat::RGB:0x00008000]";curveStyle "Line";function \TEXUX{$\left\{
\MATRIX{3,2}{c}\VR{,,c,,,}{,,c,,,}{,,l,,,}{,,,,,}\HR{,,}\CELL{-1}\CELL{%
\text{if}}\CELL{-2\left( 2\right) >x>-\frac{5\left( 2\right)
}{2}}\CELL{1}\CELL{\text{if}}\CELL{-\frac{5\left( 2\right) }{2}>x>-3\left(
2\right) }\right. $};linecolor "green";linestyle 1;pointstyle
"point";linethickness 3;lineAttributes "Solid";var1range
"-2.51,2.5";num-x-gridlines 100;curveColor
"[flat::RGB:0x00008000]";curveStyle "Line";function \TEXUX{$\frac{4}{\pi
}\sin \pi x+\frac{4}{3\pi }\sin 3\pi x+\frac{4}{5\pi }\sin 5\pi
x+\frac{4}{7\pi }\sin 7\pi x+\allowbreak \frac{4}{11\pi }\sin 11\pi
x+\frac{4}{13\pi }\sin 13\pi x+\frac{4}{15\pi }\sin 15\pi x+\allowbreak
\frac{4}{17\pi }\sin 17\pi x+\frac{4}{19\pi }\sin 19\pi x+\frac{4}{9\pi
}\sin 95\pi x\allowbreak $};linecolor "blue";linestyle 1;pointstyle
"point";linethickness 1;lineAttributes "Solid";var1range
"-2.5,2.5";num-x-gridlines 411;curveColor
"[flat::RGB:0x000000ff]";curveStyle "Line";rangeset"X";VCamFile
'PQXXR22I.xvz';valid_file "T";tempfilename
'PQXXQYSW.wmf';tempfile-properties "XPR";}}

In the limit of infinitely many waves, the match would be perfect. But we
don't usually need an infinite number of terms. we can pick the part of the
spectrum that best represents the phenomena we desire to observe. For
example, oil based compounds all have specific spectral signatures in the
wavelength range between $3-5$ micrometers. If you wish to tell the
difference between gasoline and crude oil, you can restrict your study to
these wavelengths alone.

\chapter{Interference of Light Waves}

%TCIMACRO{%
%\TeXButton{Fundamental Concepts}{\hspace{-1.3in}{\Large Fundamental Concepts\vspace{0.25in}}}}%
%BeginExpansion
\hspace{-1.3in}{\Large Fundamental Concepts\vspace{0.25in}}%
%EndExpansion

\section{What is Light?}

Before the 19th century (1800's) light was assumed to be a stream of
particles. Newton was one of the chief proponents of this theory. The theory
was able to explain reflection of light from mirrors and other objects and
therefore explain vision. In 1678 Huygens showed that wave theory could also
explain reflection and vision.

In 1801 Thomas Young demonstrated that light had attributes that were best
explained by wave theory. We will study Young's experiment later today. The
crux of his experiment was to show that light displayed constructive and
destructive interference--clearly a wave phenomena! The theory of the nature
of light took a dramatic shift

In 1805 Joseph Smith was born in Sharon, Vermont.

In September of 1832 Joseph Smith received a revelation that said in part :

\begin{quotation}
For the word of the Lord is truth, and whatsoever is truth is light, and
whatsoever is light is Spirit, even the Spirit of Jesus Christ. And the
Spirit giveth light to every man that cometh into the world; and the Spirit
enlighteneth every man through the world, that hearkeneth to the voice of
the Spirit. (D\&C 84:45-46)
\end{quotation}

In December of 1832 Joseph Smith received another revelation that says in
part:

\begin{quotation}
This Comforter is the promise which I give unto you of eternal life, even
the glory of the celestial kingdom; which glory is that of the church of the
Firstborn, even of God, the holiest of all, through Jesus Christ his
Son---He that ascended up on high, as also he descended below all things, in
that he comprehended all things, that he might be in all and through all
things, the light of truth; which truth shineth. This is the light of
Christ. As also he is in the sun, and the light of the sun, and the power
thereof by which it was made. As also he is in the moon, and is the light of
the moon, and the power thereof by which it was made; as also the light of
the stars, and the power thereof by which they were made; and the earth
also, and the power thereof, even the earth upon which you stand. And the
light which shineth, which giveth you light, is through him who enlighteneth
your eyes, which is the same light that quickeneth your understandings;
which light proceedeth forth from the presence of God to fill the immensity
of space---the light which is in all things, which giveth life to all
things, which is the law by which all things are governed, even the power of
God who sitteth upon his throne, who is in the bosom of eternity, who is in
the midst of all things. (D\&C 88:5-12)
\end{quotation}

Light, even real, physical light, seems to be of interest to members of the
Restored Church.

In 1847 the saints entered the Salt Lake Valley.

In 1873 Maxwell published his findings that light is an electromagnetic wave
(something we will try to show before this course is over!).

Planck's work in quantization theory (1900) was used by Einstein In 1905 to
give an explantation of the photoelectric effect that again made light look
like a particle.

Current theory allows light to exhibit the characteristics of a wave in some
situations and like a particle in others. We will study both before the end
of the semester.

The results of Einstein's work give us the concept of a \emph{photon} or a
quantized unit of radiant energy. Each \textquotedblleft piece of
light\textquotedblright\ or photon has energy 
\begin{equation}
E=hf
\end{equation}%
where $f$ is the frequency of the light and $h$ is a constant 
\begin{equation}
h=6.63\times 10^{-34}\unit{J}\unit{s}
\end{equation}

The nature of light is fascinating and useful both in physical and religious
areas of thought.

\section{Measurements of the Speed of Light}

One of the great foundations of modern physical theory is that the speed of
light is constant in a vacuum. Galileo first tried to measure the speed of
light. He used two towers in town and placed a lantern and an assistant on
each tower. The lanterns had shades. The plan was for one assistant to
remove his shade, and then for the assistant on the other tower to remove
his shade as soon as he saw the light from the first lantern. Back at the
first tower, the first assistant would use a clock to determine the time
difference between when the first lantern was un-shaded, and when they saw
the light from the second tower. The light would have traveled twice the
inter-tower distance. Dividing that distance by the time would give the
speed of light. You can probably guess that this did not work. Light travels
very quickly. The clocks of Galileo's day could not measure such a small
time difference. Ole R\o mer was the first to succeed in measuring the speed
of light.

\subsection{R\o mer's Measurement of the speed of light}

\FRAME{dhFU}{1.1761in}{2.0626in}{0pt}{\Qcb{A diagram illustrating R\o mer's
determination of the speed of light. Point A is the Sun, piont B is Jupiter.
Point C is the immersion of Io into Jupiter's shadow at the start of an
eclipse}}{}{Figure}{\special{language "Scientific Word";type
"GRAPHIC";maintain-aspect-ratio TRUE;display "USEDEF";valid_file "T";width
1.1761in;height 2.0626in;depth 0pt;original-width 1.7252in;original-height
3.0468in;cropleft "0";croptop "1";cropright "1";cropbottom "0";tempfilename
'PQXXQYSX.wmf';tempfile-properties "XPR";}}R\o mer performed his measurement
in 1675, $\allowbreak 269$ years before digital devices existed!. He used
the period of revolution of Io, a moon of Jupiter, as Jupiter revolved
around the sun. He first measured the period of Io's rotation about Jupiter,
then he predicted an eclipse of Io three months later. But he found his
calculation was off by $600\unit{s}.$ After careful thought, he realized
that the Earth had moved in its orbit, and that the light had to travel the
extra distance due to the Earth's new position. Given R\o mer's best
estimate for the orbital radius of the earth and his time difference, R\o %
mer arrived at a estimate of $c=2.3\times 10^{8}\frac{\unit{m}}{\unit{s}}.$
Amazing for 1675!

\subsection{Fizeau's Measurement of the speed of light}

\FRAME{dtbpF}{2.5659in}{1.6881in}{0pt}{}{}{Figure}{\special{language
"Scientific Word";type "GRAPHIC";maintain-aspect-ratio TRUE;display
"USEDEF";valid_file "T";width 2.5659in;height 1.6881in;depth
0pt;original-width 2.5253in;original-height 1.6518in;cropleft "0";croptop
"1";cropright "1";cropbottom "0";tempfilename
'PQXXQYSY.wmf';tempfile-properties "XPR";}}Hippolyte Fizeau measured the
speed of light in 1849 using the apparatus indicated in the figure above. He
used a toothed wheel and a mirror and a beam of light. The light passed
through the open space in the wheel's teeth as the wheel rotated. Then was
reflected by the mirror. The speed would be%
\[
v=\frac{\Delta x}{\Delta t} 
\]%
We just need $\Delta x$ and $\Delta t.$

It is easy to see that 
\[
\Delta x=2d 
\]%
because the light travels twice the distance to the mirror ($d$) and back.
If the wheel turned just at the right angular speed, then the reflected
light would hit the next tooth and be blocked. Think of angular speed%
\[
\omega =\frac{\Delta \theta }{\Delta t} 
\]%
so the time difference would be 
\[
\Delta t=\frac{\Delta \theta }{\omega } 
\]%
we find $\Delta \theta $ by taking $2\pi $ and dividing by the number of
teeth on the wheel.%
\[
\Delta \theta =\frac{2\pi }{N_{teeth}} 
\]

Then the speed of light must be 
\begin{eqnarray*}
c &=&v=\frac{2d}{\frac{\Delta \theta }{\omega }} \\
&=&\frac{2d\omega }{\Delta \theta } \\
&=&\frac{2d\omega N_{teeth}}{2\pi } \\
&=&\frac{d\omega N_{teeth}}{\pi }
\end{eqnarray*}

then if we have $720$ teeth and $\omega $ is measured to be $d=7500\unit{m}$%
\begin{eqnarray*}
c &=&\frac{\left( 7500\unit{m}\right) \left( 172.\,\allowbreak 79\unit{Hz}%
\right) \left( 720\right) }{\pi } \\
&=&2.\,\allowbreak 97\times 10^{8}\frac{\unit{m}}{\unit{s}}
\end{eqnarray*}%
which is Fizeau's number and it is pretty good!

Modern measurements are performed in very much the same way that Fizeau did
his calculation. The current value is 
\begin{equation}
c=2.9979\times 10^{8}\frac{\unit{m}}{\unit{s}}
\end{equation}

\subsection{Faster than light}

The speed of light in a vacuum is constant, but in matter the speed of light
changes.

%TCIMACRO{%
%\TeXButton{Pass the photon (ball) demo}{\marginpar {
%\hspace{-0.5in}
%\begin{minipage}[t]{1in}
%\small{Pass the photon (ball) demo}
%\end{minipage}
%}}}%
%BeginExpansion
\marginpar {
\hspace{-0.5in}
\begin{minipage}[t]{1in}
\small{Pass the photon (ball) demo}
\end{minipage}
}%
%EndExpansion
We will study this in detail when we look at refraction. But for now, a
dramatic example is Cherenkov radiation. It is an eerie blue glow around the
core of nuclear reactors. It occurs when electrons are accelerated past the
speed of light in the water surrounding the core. The electrons emit light
and the light waves form a Doppler cone or a light-sonic boom! The result is
the blue glow.\FRAME{dtbpFUw}{2.9457in}{2.2024in}{0pt}{\Qcb{{\protect\small %
\ Cherenkov radiation (United States Department of Energy, image in the
public domain)}}}{}{Figure}{\special{language "Scientific Word";type
"GRAPHIC";maintain-aspect-ratio TRUE;display "USEDEF";valid_file "T";width
2.9457in;height 2.2024in;depth 0pt;original-width 6.2595in;original-height
4.6717in;cropleft "0";croptop "1";cropright "1";cropbottom "0";tempfilename
'PQXXQYSZ.wmf';tempfile-properties "XPR";}}This does bring up a problem in
terminology. What does the word \textquotedblleft medium\textquotedblright\
mean? We have used it to mean the substance through which a wave travels.
This substance must have the property of transferring energy between it's
parts, like the coils of a spring can transfer energy to each other, or like
air molecules can transfer energy by collision. For light the wave medium is
the electromagnetic field. This field can store and transfer energy (PH220).
But many books on physics call materials like glass a \textquotedblleft
medium\textquotedblright\ through which light travels. The water in our last
example is a \textquotedblleft medium\textquotedblright\ in this sense. Are
glass and water wave mediums for light? The answer is an emphatic NO! Light
does not need any matter to form it's wave. The wave medium is the
electromagnetic field. So we will have to keep this in mind as we allow
light to travel through matter. I will try to say that the light enters a
new \textquotedblleft material\textquotedblright\ to describe something like
light entering a piece of glass. But some books may call the matter a
\textquotedblleft medium.\textquotedblright\ You must remember that by this
they don't mean the wave medium.

\section{Interference and Young's Experiment}

Waves do some funny things when they encounter barriers. Think of a water
wave. If we pass the wave through a small opening in a barrier, the wave
can't all get through the small hole, but it can cause a disturbance right
at the opening. We know that a small disturbance will cause a wave. But this
wave will be due to a very small--almost point--source. Point sources make
spherical waves. So the waves on the other side of the small opening will be
nearly spherical. The smaller the opening the more pronounced the curving of
the wave, because the source (the hole) is more like a point source.\FRAME{%
dhF}{2.1378in}{2.0375in}{0pt}{}{}{Figure}{\special{language "Scientific
Word";type "GRAPHIC";maintain-aspect-ratio TRUE;display "USEDEF";valid_file
"T";width 2.1378in;height 2.0375in;depth 0pt;original-width
2.0989in;original-height 1.9986in;cropleft "0";croptop "1";cropright
"1";cropbottom "0";tempfilename 'PQXXQYT0.wmf';tempfile-properties "XPR";}}%
Now suppose we have two of these openings. We expect the two sources to make
curved waves and those waves can interfere. \FRAME{dhF}{2.2139in}{2.5901in}{%
0pt}{}{}{Figure}{\special{language "Scientific Word";type
"GRAPHIC";maintain-aspect-ratio TRUE;display "USEDEF";valid_file "T";width
2.2139in;height 2.5901in;depth 0pt;original-width 2.962in;original-height
3.4696in;cropleft "0";croptop "1";cropright "1";cropbottom "0";tempfilename
'PQXXQYT1.wmf';tempfile-properties "XPR";}}In the figure, we can already see
that there will be constructive and destructive interference were the waves
from the two holes meet. Thomas young predicted that we should see
constructive and destructive interference in light (he drew figures very
like the ones we have used to explain his idea). \FRAME{dhF}{2.3618in}{%
2.6654in}{0pt}{}{}{Figure}{\special{language "Scientific Word";type
"GRAPHIC";maintain-aspect-ratio TRUE;display "USEDEF";valid_file "T";width
2.3618in;height 2.6654in;depth 0pt;original-width 2.322in;original-height
2.623in;cropleft "0";croptop "1";cropright "1";cropbottom "0";tempfilename
'PQXXQYT2.wmf';tempfile-properties "XPR";}}

Young set up a coherent source of light and placed it in front of this
source a barrier with two very thin slits cut in it to test his idea.. He
set up a screen beyond the barrier and observed the pattern on the screen
formed by the light. This (in part) is what he saw%
%TCIMACRO{%
%\TeXButton{Young's Experiment demo}{\marginpar {
%\hspace{-0.5in}
%\begin{minipage}[t]{1in}
%\small{Young's Experiment demo}
%\end{minipage}
%}}}%
%BeginExpansion
\marginpar {
\hspace{-0.5in}
\begin{minipage}[t]{1in}
\small{Young's Experiment demo}
\end{minipage}
}%
%EndExpansion

\bigskip

\FRAME{dhF}{2.3618in}{0.7671in}{0pt}{}{}{Figure}{\special{language
"Scientific Word";type "GRAPHIC";maintain-aspect-ratio TRUE;display
"USEDEF";valid_file "T";width 2.3618in;height 0.7671in;depth
0pt;original-width 2.322in;original-height 0.7351in;cropleft "0";croptop
"1";cropright "1";cropbottom "0";tempfilename
'PQXXQYT3.wmf';tempfile-properties "XPR";}}%
%TCIMACRO{%
%\TeXButton{Question 123.26.5}{\marginpar {
%\hspace{-0.5in}
%\begin{minipage}[t]{1in}
%\small{Question 123.26.5}
%\end{minipage}
%}}}%
%BeginExpansion
\marginpar {
\hspace{-0.5in}
\begin{minipage}[t]{1in}
\small{Question 123.26.5}
\end{minipage}
}%
%EndExpansion
We see bright spots (constructive interference) and dark spots (destructive
interference). Only wave phenomena can interfere, so this is fairly good
evidence that light is a wave.

\subsection{Constructive Interference}

We can find the condition for getting a bright or a dark band if we think
about it a bit. The two waves coming from the two slits are just two
different waves. We can use our two wave mixing analysis with our
constructive and destructive interference criteria. For constructive
interference, the difference in phase must be a multiple of $2\pi .$ Let's
review this, but instead of $y_{\max }$ being the displacement of the waves,
let's write them as $E_{\max }$ because it is the electric field that is
carrying the wave and it doesn't really go up or down, it just gets a higher
value or lower value. Our waves will be 
\begin{eqnarray*}
E_{1} &=&E_{\max }\sin \left( k_{2}r_{2}-\omega _{2}t_{2}+\phi _{2}\right) \\
E_{2} &=&E_{\max }\sin \left( k_{1}r_{1}-\omega _{1}t_{1}+\phi _{1}\right)
\end{eqnarray*}%
and the resulting wave will be 
\begin{eqnarray*}
E_{r} &=&E_{\max }\sin \left( k_{2}r_{2}-\omega _{2}t_{2}+\phi _{2}\right)
+A\sin \left( k_{1}r_{1}-\omega _{1}t_{1}+\phi _{1}\right) \\
&=&2E_{\max }\cos \left( \frac{\left( k_{2}r_{2}-\omega _{2}t_{2}+\phi
_{2}\right) -\left( k_{1}r_{1}-\omega _{1}t_{1}+\phi _{1}\right) }{2}\right)
\\
&&\times \sin \left( \frac{\left( k_{2}r_{2}-\omega _{2}t_{2}+\phi
_{2}\right) +\left( k_{1}r_{1}-\omega _{1}t_{1}+\phi _{1}\right) }{2}\right)
\\
&=&2E_{\max }\cos \left( \frac{1}{2}\left[ \left( k_{2}r_{2}-\omega
_{2}t_{2}+\phi _{2}\right) -\left( k_{1}r_{1}-\omega _{1}t_{1}+\phi
_{1}\right) \right] \right) \\
&&\times \sin \left( \frac{\left( k_{2}r_{2}-\omega _{2}t_{2}+\phi
_{2}\right) +\left( k_{1}r_{1}-\omega _{1}t_{1}+\phi _{1}\right) }{2}\right)
\end{eqnarray*}%
As we are now well aware, the sine part is a combined wave, and the cosine
part is part of the amplitude. The amplitude can be written as 
\[
A=2E_{\max }\cos \left( \frac{1}{2}\left[ \left( k_{2}r_{2}-\omega
_{2}t_{2}+\phi _{2}\right) -\left( k_{1}r_{1}-\omega _{1}t_{1}+\phi
_{1}\right) \right] \right) 
\]%
We should pause and think about which of our values will change. The
frequency won't change, $\omega _{2}=\omega _{1}=\omega .$ There is no
slower material, so the wavelength won't change, $k_{2}=k_{1}=k.$ We want
the waves to mix at the same time so $t_{2}=t_{1}=t.$ And the new waves are
created from the old wave hitting the slits. As long as the original wave
hits both slits at once, then $\phi _{2}=\phi _{1}=\phi _{o}.$ We are left
with 
\begin{eqnarray*}
A &=&2E_{\max }\cos \left( \frac{1}{2}\left[ \left( kr_{2}-\omega t+\phi
_{o}\right) -\left( kr_{1}-\omega t+\phi _{0}\right) \right] \right) \\
&=&2E_{\max }\cos \left( \frac{1}{2}\left[ \left( kr_{2}\right) -\left(
kr_{1}\right) \right] \right) \\
&=&2E_{\max }\cos \left( \frac{1}{2}k\left( r_{2}-r_{1}\right) \right) \\
&=&2E_{\max }\cos \left( \frac{1}{2}k\Delta r\right)
\end{eqnarray*}

That means the path difference between the two slit-sources must be an even
number of wavelengths. We have been calling the path difference in the total
phase $\Delta x,$ or for spherical waves $\Delta r,$ but in optics it is
customary to call this path difference $\delta .$ So%
\[
\delta =\Delta r 
\]%
Then we can write the amplitude function as 
\[
A=2E_{\max }\cos \left( \frac{1}{2}k\delta \right) 
\]%
\FRAME{dhF}{5.3186in}{2.5849in}{0pt}{}{}{Figure}{\special{language
"Scientific Word";type "GRAPHIC";maintain-aspect-ratio TRUE;display
"USEDEF";valid_file "T";width 5.3186in;height 2.5849in;depth
0pt;original-width 3.6564in;original-height 1.7634in;cropleft "0";croptop
"1";cropright "1";cropbottom "0";tempfilename
'PQXXQYT4.wmf';tempfile-properties "XPR";}}so our total phase equation
becomes%
\[
\Delta \phi =k\delta 
\]%
We want to use this in our criteria for constructive or destructive
interference. But first, let's express $\delta =\Delta r$ in terms of
geometry that is easy to measure in our experiment. In this set up, the
screen is much farther away than $d,$ the slit distance, we can say that the
blue triangle is almost a right triangle, and then $\delta $ is 
\[
\delta =r_{2}-r_{1}\approx d\sin \theta 
\]%
Our wave repeats every $2\pi $ radians or every wavelength, $\lambda ,$ then
we have constructive interference (a bright spot) when%
\[
\Delta \phi =k\delta \approx kd\sin \theta =2\pi m\quad \left( m=0,\pm 1,\pm
2\ldots \right) 
\]%
We can solve this for $\delta $ 
\[
\delta =d\sin \theta =m\lambda \quad \left( m=0,\pm 1,\pm 2\ldots \right) 
\]%
and we can give $m$ a name. It is called the \emph{order number}. 
%TCIMACRO{%
%\TeXButton{Question 123.26.6}{\marginpar {
%\hspace{-0.5in}
%\begin{minipage}[t]{1in}
%\small{Question 123.26.6}
%\end{minipage}
%}} }%
%BeginExpansion
\marginpar {
\hspace{-0.5in}
\begin{minipage}[t]{1in}
\small{Question 123.26.6}
\end{minipage}
}
%EndExpansion
That is, if we are off by any number of whole wavelengths then our total
phase due to path difference will be a multiple of $2\pi $.

If we assume that $\lambda \ll d$ we can find the distance from the axis for
each fringe more easily. This guarantees that $\theta $ will be small. Using
the yellow triangle we see%
\[
\tan \theta =\frac{y}{L} 
\]%
but if $\theta $ is small this is just about the same as 
\[
\sin \theta =\frac{y}{L} 
\]%
because for small angles $\tan \theta \approx \sin \theta \approx \theta .$
So if $\theta $ is small then 
\begin{eqnarray*}
\delta &=&d\sin \theta \\
&=&d\frac{y}{L}
\end{eqnarray*}%
and for a bright spot or \emph{\textquotedblleft fringe\textquotedblright }
we find 
\[
d\frac{y}{L}=m\lambda 
\]%
Solving for the position of the bright spots gives%
\begin{equation}
y_{bright}\approx \frac{\lambda L}{d}m\quad \left( m=0,\pm 1,\pm 2\ldots
\right)
\end{equation}%
We can measure up from the central spot and predict where each successive
bright spot will be.

\subsection{Destructive Interference}

We can also find a condition for destructive interference. We know that a
path difference of an odd multiple of a half wavelength will give
distractive interference. so%
\[
\Delta \phi =k\delta \approx kd\sin \theta =\left( m+\frac{1}{2}\right) 2\pi
\qquad m=0,\pm 1,\pm 2,\pm 3,\cdots 
\]%
or using $k=2\pi /\lambda $ 
\[
\delta =d\sin \theta =\left( m+\frac{1}{2}\right) \lambda \quad \left(
m=0,\pm 1,\pm 2\ldots \right) 
\]%
will give a dark fringe. The location of the dark fringes will be 
\begin{equation}
y_{dark}\approx \frac{\lambda L}{d}\left( m+\frac{1}{2}\right) \quad \left(
m=0,\pm 1,\pm 2\ldots \right)
\end{equation}

\section{Double Slit Intensity Pattern}

The fringes we have seen are not just points, but are patterns that fade
from a maximum intensity. We can calculate the intensity pattern. We need to
know a little bit about electric fields to do this.

We can represent an electromagnetic wave using just the electric field (the
magnetic field pattern is very similar and can be derived from the electric
field pattern) as we did at the beginning of this lecture%
\begin{eqnarray*}
E_{1} &=&E_{\max }\sin \left( k_{2}r_{2}-\omega _{2}t_{2}+\phi _{2}\right) \\
E_{2} &=&E_{\max }\sin \left( k_{1}r_{1}-\omega _{1}t_{1}+\phi _{1}\right)
\end{eqnarray*}%
and the resulting wave will be 
\begin{eqnarray*}
E_{r} &=&E_{\max }\sin \left( k_{2}r_{2}-\omega _{2}t_{2}+\phi _{2}\right)
+A\sin \left( k_{1}r_{1}-\omega _{1}t_{1}+\phi _{1}\right) \\
&=&2E_{\max }\cos \left( \frac{\left( k_{2}r_{2}-\omega _{2}t_{2}+\phi
_{2}\right) -\left( k_{1}r_{1}-\omega _{1}t_{1}+\phi _{1}\right) }{2}\right)
\\
&&\times \sin \left( \frac{\left( k_{2}r_{2}-\omega _{2}t_{2}+\phi
_{2}\right) +\left( k_{1}r_{1}-\omega _{1}t_{1}+\phi _{1}\right) }{2}\right)
\\
&=&2E_{\max }\cos \left( \frac{1}{2}\left[ \left( k_{2}r_{2}-\omega
_{2}t_{2}+\phi _{2}\right) -\left( k_{1}r_{1}-\omega _{1}t_{1}+\phi
_{1}\right) \right] \right) \\
&&\times \sin \left( \frac{\left( k_{2}r_{2}-\omega _{2}t_{2}+\phi
_{2}\right) +\left( k_{1}r_{1}-\omega _{1}t_{1}+\phi _{1}\right) }{2}\right)
\end{eqnarray*}%
but now we know that we can simplify this because $\omega _{2}=\omega
_{1}=\omega ,$ $k_{2}=k_{1}=k,$ $t_{2}=t_{1}=t,$ and $\phi _{2}=\phi
_{1}=\phi _{o}.$%
\begin{eqnarray*}
E_{r} &=&2E_{\max }\cos \left( \frac{1}{2}k\delta \right) \sin \left( \frac{%
kr_{2}+kr_{1}}{2}-\omega t+\phi _{o}\right) \\
&=&2E_{\max }\cos \left( \frac{1}{2}\frac{2\pi }{\lambda }d\sin \theta
\right) \sin \left( k\frac{r_{2}+r_{1}}{2}-\omega t+\phi _{o}\right)
\end{eqnarray*}

We have a combined wave at point $P$ that is a traveling wave $\left( \sin
\left( k\frac{\left( r_{2}+r_{1}\right) }{2}-\omega t+\phi _{o}\right)
\right) $ but with amplitude $\left( 2E_{o}\cos \left( \frac{1}{2}\left( 
\frac{2\pi }{\lambda }d\sin \theta \right) \right) \right) $ that depends on
our total phase $\Delta \phi =\frac{2\pi }{\lambda }d\sin \theta .$

But the situation is more complicated because of how we detect light. Our
eyes, film, and most detectors measure the intensity of the light. We know
that 
\[
I=\frac{\mathcal{P}}{A} 
\]%
In PH 220 you will learn that the power is proportional to the square of the
electric field wave amplitude. 
\begin{equation}
\mathcal{P}\varpropto E^{2}
\end{equation}%
Then the intensity must also be proportional to the amplitude of the
electric field squared.%
\begin{eqnarray*}
I &=&\frac{P}{A}\varpropto E^{2} \\
&\varpropto &4E_{o}^{2}\cos ^{2}\left( \frac{1}{2}\left( \frac{2\pi }{%
\lambda }d\sin \theta \right) \right) \sin ^{2}\left( \frac{k\left(
r_{2}+r_{1}\right) }{2}-\omega t+\phi _{o}\right)
\end{eqnarray*}

Light detectors collect power for a set amount of time. So most light
detection will be a value averaged over a set \emph{integration time}. This
means that the detector sums up (or integrates) the amount of power received
over the detector time. Usually the integration time is much longer than a
period, so we need to time-average our intensity.%
\begin{eqnarray*}
\int_{\text{many periods}}Idt &\varpropto &=\int_{\text{many periods}%
}4E_{o}^{2}\cos ^{2}\left( \frac{1}{2}\left( \frac{2\pi }{\lambda }d\sin
\theta \right) \right) \sin ^{2}\left( \frac{k\left( r_{2}+r_{1}\right) }{2}%
-\omega t+\phi _{o}\right) dt \\
&=&4E_{o}^{2}\cos ^{2}\left( \frac{1}{2}\left( \frac{2\pi }{\lambda }d\sin
\theta \right) \right) \int_{\text{many periods}}\sin ^{2}\left( \frac{%
k\left( rx_{2}+r_{1}\right) }{2}-\omega t+\phi _{o}\right) dt
\end{eqnarray*}%
but the term%
\begin{equation}
\int_{\text{many periods}}\sin ^{2}\left( \frac{k\left( r_{2}+r_{1}\right) }{%
2}-\omega t+\phi o\right) dt=\frac{1}{2}
\end{equation}

To convince yourself of this, think that $\sin ^{2}\left( x\right) $ has a
maximum value of $1$ and a minimum of $0.$ Looking at the graph\FRAME{dhF}{%
1.9986in}{1.4927in}{0pt}{}{}{Figure}{\special{language "Scientific
Word";type "GRAPHIC";maintain-aspect-ratio TRUE;display "USEDEF";valid_file
"T";width 1.9986in;height 1.4927in;depth 0pt;original-width
1.9597in;original-height 1.4572in;cropleft "0";croptop "1";cropright
"1";cropbottom "0";tempfilename 'PQXXQYT5.wmf';tempfile-properties "XPR";}}%
should be believable that the average value over a period is $1/2.$ The
average over many periods will still be $1/2.$

So we have 
\begin{equation}
\bar{I}=\int_{\text{many periods}}Idt\varpropto 2E_{o}^{2}\cos ^{2}\left( 
\frac{1}{2}\left( \frac{2\pi }{\lambda }d\sin \theta \right) \right)
\end{equation}%
where $\bar{I}$ is the time average intensity. The important part is that
the time varying part has averaged out.

So, usually in optics, we ignore the fast fluctuating parts of such
calculations because we can't see them and so we write%
\[
I=I_{\max }\cos ^{2}\left( \frac{1}{2}\left( \frac{2\pi }{\lambda }d\sin
\theta \right) \right) 
\]%
where we have dropped the bar from the $I,$ but it is understood that the
intensity we report is a time average over many periods.

We should remind our selves, our intensity pattern 
\[
I=I_{\max }\cos ^{2}\left( \frac{1}{2}\frac{2\pi }{\lambda }d\sin \theta
\right) 
\]%
is really 
\[
I=I_{\max }\cos ^{2}\left( \frac{\Delta \phi }{2}\right) 
\]%
Which is just our amplitude squared for the mixing of two waves. All we have
done to find the intensity pattern is to find and expression for the phase
difference $\Delta \phi .$

Our intensity pattern should give the same location for the center of the
bright spots as we got before. Let's check that it works. We used the small
angle approximation before. It is still valid, so let's use it again now.
For for small angles 
\begin{eqnarray*}
I &=&I_{\max }\cos ^{2}\left( \frac{\pi d}{\lambda }\theta \right) \\
&=&I_{\max }\cos ^{2}\left( \frac{\pi d}{\lambda }\frac{y}{L}\right)
\end{eqnarray*}%
Then we have constructive interference when 
\[
\frac{\pi d}{\lambda }\frac{y}{L}=m\pi 
\]%
or%
\[
y=m\frac{L\lambda }{d} 
\]%
which is what we found before.

The plot of normalized intensity 
\[
\frac{I}{I_{\max }}=\cos ^{2}\left( \frac{\Delta \phi }{2}\right) 
\]%
verses $\Delta \phi /2$ is given next, \FRAME{dtbpFX}{2.2917in}{1.5281in}{0pt%
}{}{}{Plot}{\special{language "Scientific Word";type "MAPLEPLOT";width
2.2917in;height 1.5281in;depth 0pt;display "USEDEF";plot_snapshots
TRUE;mustRecompute FALSE;lastEngine "MuPAD";xmin "-15";xmax "15";xviewmin
"-15";xviewmax "15";yviewmin "0";yviewmax
"1";viewset"XY";rangeset"X";plottype 4;labeloverrides 3;x-label
"delta_phi/2";y-label "I /I_max";axesFont "Times New
Roman,12,0000000000,useDefault,normal";numpoints 100;plotstyle
"patch";axesstyle "normal";axestips FALSE;xis \TEXUX{x};var1name
\TEXUX{$x$};function \TEXUX{$\cos ^{2}\left( \frac{x}{2}\right) $};linecolor
"blue";linestyle 1;pointstyle "point";linethickness 1;lineAttributes
"Solid";var1range "-15,15";num-x-gridlines 100;curveColor
"[flat::RGB:0x000000ff]";curveStyle "Line";VCamFile
'PR953C06.xvz';valid_file "T";tempfilename
'PQXXQYT6.wmf';tempfile-properties "XPR";}}but we will find that we are not
quite through with this analysis. Next time we will find that there is
another compounding factor that reduces the intensity as we move away from
the midpoint.\FRAME{dtbpFX}{2.1871in}{1.4581in}{0pt}{}{}{Plot}{\special%
{language "Scientific Word";type "MAPLEPLOT";width 2.1871in;height
1.4581in;depth 0pt;display "USEDEF";plot_snapshots TRUE;mustRecompute
FALSE;lastEngine "MuPAD";xmin "-15";xmax "15";xviewmin "-15";xviewmax
"15";yviewmin "0";yviewmax "1";viewset"XY";rangeset"X";plottype
4;labeloverrides 3;x-label "delta_phi/2";y-label "I /I_max";axesFont "Times
New Roman,12,0000000000,useDefault,normal";numpoints 100;plotstyle
"patch";axesstyle "normal";axestips FALSE;xis \TEXUX{x};var1name
\TEXUX{$x$};function \TEXUX{$\cos ^{2}\left( \frac{x}{2}\right) \frac{4\sin
^{2}\left( \frac{0.2x}{2}\right) }{\left( 0.2x\right) ^{2}}$};linecolor
"blue";linestyle 1;pointstyle "point";linethickness 1;lineAttributes
"Solid";var1range "-15,15";num-x-gridlines 100;curveColor
"[flat::RGB:0x000000ff]";curveStyle "Line";VCamFile
'PR953F07.xvz';valid_file "T";tempfilename
'PQXXQYT7.wmf';tempfile-properties "XPR";}}Let's pause to remember what this
pattern means. This is the intensity of light due to interference. It is
instructive to match our intensity pattern to the pattern Young saw with our
graph.\FRAME{dhF}{4.4719in}{2.2606in}{0pt}{}{}{Figure}{\special{language
"Scientific Word";type "GRAPHIC";maintain-aspect-ratio TRUE;display
"USEDEF";valid_file "T";width 4.4719in;height 2.2606in;depth
0pt;original-width 4.4209in;original-height 2.2208in;cropleft "0";croptop
"1";cropright "1";cropbottom "0";tempfilename
'PQXXQYT8.wmf';tempfile-properties "XPR";}}The high intensity peaks are the
bright fringes and the low intensity troughs are the dark fringes. The
pattern moves smoothly and continuously from bright to dark.

\chapter{Many Slits, and Single Slits}

%TCIMACRO{%
%\TeXButton{Fundamental Concepts}{\hspace{-1.3in}{\Large Fundamental Concepts\vspace{0.25in}}}}%
%BeginExpansion
\hspace{-1.3in}{\Large Fundamental Concepts\vspace{0.25in}}%
%EndExpansion

\section{Diffraction Gratings}

%TCIMACRO{%
%\TeXButton{Rainbow Glasses}{\marginpar {
%\hspace{-0.5in}
%\begin{minipage}[t]{1in}
%\small{Rainbow Glasses}
%\end{minipage}
%}} }%
%BeginExpansion
\marginpar {
\hspace{-0.5in}
\begin{minipage}[t]{1in}
\small{Rainbow Glasses}
\end{minipage}
}
%EndExpansion
A diffraction grating is an optical element with many many parallel slits
spaced very close together. Here is a typical diffraction grating created by
etching lines in a piece of glass. The etchings scatter the light, but the
un-etched part allows the light to pass through. The un-etched parts are
essentially a series of slits.\FRAME{dhFU}{1.9984in}{1.625in}{0pt}{\Qcb{%
Surface of a diffraction grating (600 lines/mm). Image taken with optical
transmission microscope. (Image in the public domain courtesy Scapha)}}{}{%
Figure}{\special{language "Scientific Word";type
"GRAPHIC";maintain-aspect-ratio TRUE;display "USEDEF";valid_file "T";width
1.9984in;height 1.625in;depth 0pt;original-width 2.4474in;original-height
1.9847in;cropleft "0";croptop "1";cropright "1";cropbottom "0";tempfilename
'PQXXQYT9.wmf';tempfile-properties "XPR";}}A typical grating might have $%
5000 $ slits per unit centimeter. You have probably used a diffraction
grating to see rainbow colors in a beginning science class. Gratings are
usually made by cutting parallel grooves in a flat surface.

If we use $5000\frac{grooves}{\unit{cm}}$ for an example, we see that the
slit spacing is 
\begin{eqnarray}
d &=&\frac{1}{5000}\unit{cm} \\
&=&\allowbreak 2.0\times 10^{-6}\unit{m}
\end{eqnarray}%
Take a section of diffraction grating as shown below\FRAME{dhF}{1.9951in}{%
3.3788in}{0in}{}{}{Figure}{\special{language "Scientific Word";type
"GRAPHIC";maintain-aspect-ratio TRUE;display "USEDEF";valid_file "T";width
1.9951in;height 3.3788in;depth 0in;original-width 5.5893in;original-height
9.5207in;cropleft "0";croptop "1";cropright "1";cropbottom "0";tempfilename
'PQXXQYTA.wmf';tempfile-properties "XPR";}}

At some point, two of the slits will have a path difference that is a whole
wavelength, and we would expect a bright spot. But what about the other
slits? If we have a slit spacing such that each of the succeeding slits has
a path difference that is just an additional wavelength, then each of the
slits will contribute to the constructive interference at our point, and the
point will become a bright spot.

\FRAME{dhF}{0.9755in}{2.3679in}{0pt}{}{}{Figure}{\special{language
"Scientific Word";type "GRAPHIC";maintain-aspect-ratio TRUE;display
"USEDEF";valid_file "T";width 0.9755in;height 2.3679in;depth
0pt;original-width 1.6821in;original-height 4.1217in;cropleft "0";croptop
"1";cropright "1";cropbottom "0";tempfilename
'PQXXQYTB.wmf';tempfile-properties "XPR";}}Let's look at just two slits. The
light leaves each slit in phase with the light from the rest of the slits,
but at some distance $L$ away and at some angle $\theta $ we will have a
path difference%
\begin{equation}
\delta =d\sin \left( \theta _{bright}\right) =m\lambda \quad m=0,\pm 1,\pm
2,\ldots
\end{equation}%
because the path lengths are not all the same.

This equation tells us that each wavelength, $\lambda ,$ will experience
constructive interference at a slightly different angle $\theta _{bright}.$
Different frequencies will create bright spots at different angles. We have
found a way to create a spectrum with light waves. We often call a spectrum
made with visible light frequencies a rainbow. Knowing $d$ and $\theta $
allows an accurate calculation of $\lambda .$ This may seem a silly thing to
do, but suppose we add into our system a sample of a chemical to identify 
\FRAME{dhF}{3.5958in}{1.7465in}{0pt}{}{}{Figure}{\special{language
"Scientific Word";type "GRAPHIC";maintain-aspect-ratio TRUE;display
"USEDEF";valid_file "T";width 3.5958in;height 1.7465in;depth
0pt;original-width 3.6357in;original-height 1.7509in;cropleft "0";croptop
"1";cropright "1";cropbottom "0";tempfilename
'PQXXQYTC.wmf';tempfile-properties "XPR";}}

We could then record the intensity of the transmitted light as a function of
angle, which is equivalent to $\lambda .$ We can again generate a spectrum.
This is a traditional way to build a spectrometer and many such devices are
available today.%
%TCIMACRO{%
%\TeXButton{Demo a student spectrometer with a gas tube}{\marginpar {
%\hspace{-0.5in}
%\begin{minipage}[t]{1in}
%\small{Demo a student spectrometer with a gas tube}
%\end{minipage}
%}}}%
%BeginExpansion
\marginpar {
\hspace{-0.5in}
\begin{minipage}[t]{1in}
\small{Demo a student spectrometer with a gas tube}
\end{minipage}
}%
%EndExpansion

\subsection{Resolving power of diffraction gratings}

We noticed that with two slits, we got a bright spot for a particular
wavelength, but we didn't just get one bright spot. We got several. The same
is true for diffraction gratings. So we expect to get a rainbow, but we
really expect to get a series of rainbows. The integer $m$ tells us which
rainbow we have in the series. The integer $m$ is called the order number.

For two nearly equal wavelengths $\lambda _{1}$ and $\lambda _{2},$ we say
that the diffraction grating can resolve the wavelengths if we can
distinguish the two using the grating. The \emph{resolving power} of the
grating is defined as 
\begin{equation}
R=\frac{\left( \lambda _{1}+\lambda _{2}\right) }{2\left( \lambda
_{1}-\lambda _{2}\right) }=\frac{\bar{\lambda}}{\Delta \lambda }
\end{equation}%
We can show that for the $m$-th order diffraction, the resolving power is%
\begin{equation}
R=Nm
\end{equation}%
where $N$ is the number of slits. So our ability to distinguish wavelengths
increases with the number of slits and with the order (which is related to
how far off-axis we look).

Note that for $m=0$ we have no ability to resolve wavelengths. The central
peak is a mix of all wavelengths and usually looks white for normal
illumination.

That the resolution depends on the number of slits, $N,$ means that we can
improve our spectrometer by using more lines. Here is a representation of
what happens as we increase $N$\FRAME{dhF}{3.103in}{2.6654in}{0in}{}{}{Figure%
}{\special{language "Scientific Word";type "GRAPHIC";maintain-aspect-ratio
TRUE;display "USEDEF";valid_file "T";width 3.103in;height 2.6654in;depth
0in;original-width 3.0588in;original-height 2.623in;cropleft "0";croptop
"1";cropright "1";cropbottom "0";tempfilename
'PQXXQYTD.wmf';tempfile-properties "XPR";}}we can see that the peaks get
narrower as $N$ increases. These graphs are for a particular $\lambda .$ If
the peaks for a particular $\lambda $ get narrower, then there will be less
overlap with adjacent $\lambda ^{\prime }s$ which means that each wavelength
can more easily be resolved.

Spectrometers are used in many places. On that has some public interest
today is monitoring the atmosphere. Instruments like the one shown below
detect the amount of special gasses in the atmosphere using IR spectrometers.

\FRAME{dhFU}{4.6544in}{3.0554in}{0pt}{\Qcb{AIRS\ sensor, spectrometer
design, and global CO$_{2}$ data. (Images in the Public Domain courtesy NASA)%
}}{}{Figure}{\special{language "Scientific Word";type
"GRAPHIC";maintain-aspect-ratio TRUE;display "USEDEF";valid_file "T";width
4.6544in;height 3.0554in;depth 0pt;original-width 4.6017in;original-height
3.0113in;cropleft "0";croptop "1";cropright "1";cropbottom "0";tempfilename
'PQXXQYTE.wmf';tempfile-properties "XPR";}}The instrument shown is the AIRS
spectrometer. You can see in the diagram that it uses a grating
spectrometer. The picture of the Earth is a composite of AIRS data showing
the northern and southern bands of CO$_{2}.$

\section{Single Slits}

We have looked at interference from two slits, and for many slits. The two
slits acted like two coherent sources. We might expect that a single slit
will give only a single bright spot. But let's consider a single slit very
closely. To do this, let's return to the work of Huygens.\footnote{%
Huygens method is technically not a correct representation of what happens.
The actual wave leaving the single opening is a superposition of the
original wave, and the wave scattered from the sides of the opening. You can
see this scattering by tearing a small hole in a piece of paper and looking
through the hole at a light source. You will see the bright ring around the
hole where the edges of the paper are scattering the light. But the
mathematical result we will get using Huygens method gives a mathematically
identical result for the resulting wave leaving the slit with much less high
power math. So we will stick with Huygens in this class.} His idea for the
nature of light was simple. He suggested that every point on the wave front
of a light wave was the source (the disturbance) for a new set of small
spherical waves. In optics, the crests of the waves are often called
wavefronts. The next wavefront would be formed by the superposition of the
little \textquotedblleft wavelets.\textquotedblright\ Here is an example for
a plane wave and a spherical wave.\FRAME{dhF}{2.2208in}{1.8697in}{0pt}{}{}{%
Figure}{\special{language "Scientific Word";type
"GRAPHIC";maintain-aspect-ratio TRUE;display "USEDEF";valid_file "T";width
2.2208in;height 1.8697in;depth 0pt;original-width 4.0594in;original-height
3.4143in;cropleft "0";croptop "1";cropright "1";cropbottom "0";tempfilename
'PQXXQYTF.wmf';tempfile-properties "XPR";}}In each case we have drawn spots
on the wave front and drawn spherical waves around those spots. where the
wavefronts of the little wavelets combine, we have new wave front of our
wave. This is sort of what happens in bulk matter. Remember that light is
absorbed and re-emitted by the atoms of the material. This is why light
slows down in a material. Because of the time it is absorbed, it effectively
goes slower. But the light is not necessarily re-emitted in the same
direction. Sometimes it is, but sometimes it is not. This creates a small,
spherical wave (called a wavelet) that is emitted by that atom. So Huygens
idea is not too bad.

We can use this idea for a single slit and look at what happens as the light
goes through. Here is such a slit.

\FRAME{fhF}{2.0686in}{2.693in}{0pt}{}{\Qlb{SingleSlitGeometry}}{Figure}{%
\special{language "Scientific Word";type "GRAPHIC";maintain-aspect-ratio
TRUE;display "USEDEF";valid_file "T";width 2.0686in;height 2.693in;depth
0pt;original-width 2.0297in;original-height 2.6507in;cropleft "0";croptop
"1";cropright "1";cropbottom "0";tempfilename
'PQXXQYTG.wmf';tempfile-properties "XPR";}}

In the figure above, we have divided a single slit of width $a$ into two
parts, each of size $a/2.$ According to Huygens' principle, each position of
the slit acts as a source of light rays. So we can treat half a slit as two
coherent sources. These two sources should interfere. So what do we see when
we perform such an experiment?

\FRAME{dhF}{2.3065in}{0.8233in}{0in}{}{}{Figure}{\special{language
"Scientific Word";type "GRAPHIC";maintain-aspect-ratio TRUE;display
"USEDEF";valid_file "T";width 2.3065in;height 0.8233in;depth
0in;original-width 2.2667in;original-height 0.7904in;cropleft "0";croptop
"1";cropright "1";cropbottom "0";tempfilename
'PQXXQYTH.wmf';tempfile-properties "XPR";}}

The figure shows a diffraction pattern for a thin slit. There are several
terms that are in common use to describe the pattern

\begin{enumerate}
\item Central Maximum: The broad intense central band.

\item Secondary Maxima:The fainter bright bands to both sides of the central
maxima

\item Minima: The dark bands between the maxima
\end{enumerate}

\section{Narrow Slit Intensity Pattern}

Let's use figure \ref{SingleSlitGeometry} to find the dark minima of the
single slit pattern. First we should notice that figure \ref%
{SingleSlitGeometry} could have another set of rays that contribute to the
bright spot because they will also have a path difference of $\left(
a/2\right) \sin \theta .$ Let's fill these in. They are rays $2$ and $4$ of
the next figure. \FRAME{dhF}{2.1525in}{2.6507in}{0pt}{}{}{Figure}{\special%
{language "Scientific Word";type "GRAPHIC";maintain-aspect-ratio
TRUE;display "USEDEF";valid_file "T";width 2.1525in;height 2.6507in;depth
0pt;original-width 2.1136in;original-height 2.6091in;cropleft "0";croptop
"1";cropright "1";cropbottom "0";tempfilename
'PQXXQYTI.wmf';tempfile-properties "XPR";}}

Before we started with what we are now calling rays $1$ and $3.$ Ray $1$
travels a distance 
\begin{equation}
\delta =\frac{a}{2}\sin \left( \theta \right)
\end{equation}%
farther than ray $3.$ As we just argued, rays $2$ and $4$ also have the same
path difference, and so do rays $3$ and $5.$ If this path difference is $%
\lambda /2$ then we will have destructive interference. The condition for a
minima is then%
\begin{equation}
\frac{a}{2}\sin \left( \theta \right) =\pm \frac{\lambda }{2}
\end{equation}%
or%
\begin{equation}
\sin \left( \theta \right) =\pm \frac{\lambda }{a}
\end{equation}

Now we could also divide the slit into four equal parts. Then we have a path
difference of 
\begin{equation}
\delta =\frac{a}{4}\sin \left( \theta \right)
\end{equation}%
and to have destructive interference we need this path difference to be $%
\lambda /2$%
\begin{equation}
\frac{a}{4}\sin \left( \theta \right) =\pm \frac{\lambda }{2}
\end{equation}%
or%
\begin{equation}
\sin \left( \theta \right) =\pm \frac{2\lambda }{a}
\end{equation}%
We can keep going to find a minima at\FRAME{dhF}{1.9285in}{2.5114in}{0pt}{}{%
}{Figure}{\special{language "Scientific Word";type
"GRAPHIC";maintain-aspect-ratio TRUE;display "USEDEF";valid_file "T";width
1.9285in;height 2.5114in;depth 0pt;original-width 1.8905in;original-height
2.4708in;cropleft "0";croptop "1";cropright "1";cropbottom "0";tempfilename
'PQXXQYTJ.wmf';tempfile-properties "XPR";}}%
\begin{equation}
\sin \left( \theta \right) =\pm \frac{3\lambda }{a}
\end{equation}%
and in general at%
\begin{equation}
\sin \left( \theta \right) =m\frac{\lambda }{a}\quad m=\pm 1,\pm 2,\pm
3\ldots  \label{Single Slit minima}
\end{equation}

You might object that we did not find the bright spots, only the dark spots.
That's fine because the bright spots have to be in between the dark spots.
But we can do better. Let's look at the full intensity pattern for a single
slit.

\section{Intensity of the single-slit pattern}

I will not derive the intensity pattern for the single slit (though it is
not really too hard to do) but I\ will give it here%
\begin{equation}
I=I_{\max }\left( \frac{\sin \left( \frac{\pi }{\lambda }a\sin \theta
\right) }{\frac{\pi }{\lambda }a\sin \theta }\right) ^{2}
\end{equation}

Notice this has the form%
\[
\frac{\sin x}{x} 
\]%
which has a distinctive shape.\FRAME{dtbpFX}{4.4996in}{1.3958in}{0pt}{}{}{%
Plot}{\special{language "Scientific Word";type "MAPLEPLOT";width
4.4996in;height 1.3958in;depth 0pt;display "USEDEF";plot_snapshots
TRUE;mustRecompute FALSE;lastEngine "MuPAD";xmin "-30";xmax "30";xviewmin
"-30";xviewmax "30";yviewmin "-0.217351";yviewmax
"1.000122";viewset"XY";rangeset"X";plottype 4;labeloverrides 3;x-label
"x";y-label "sinc(x)";axesFont "Times New
Roman,12,0000000000,useDefault,normal";numpoints 100;plotstyle
"patch";axesstyle "normal";axestips FALSE;xis \TEXUX{x};var1name
\TEXUX{$x$};function \TEXUX{$\frac{\sin x}{x}$};linecolor "blue";linestyle
1;pointstyle "point";linethickness 3;lineAttributes "Solid";var1range
"-30,30";num-x-gridlines 100;curveColor "[flat::RGB:0x000000ff]";curveStyle
"Line";VCamFile 'PR953I08.xvz';valid_file "T";tempfilename
'PQXXQYTK.wmf';tempfile-properties "XPR";}}this is known as a sinc function
(pronounced like \textquotedblleft sink\textquotedblright ). It has a
central maximum as we would expect. Of course our pattern has a sinc squared 
\FRAME{dtbpFX}{4.4996in}{2.2502in}{0pt}{}{}{Plot}{\special{language
"Scientific Word";type "MAPLEPLOT";width 4.4996in;height 2.2502in;depth
0pt;display "USEDEF";plot_snapshots TRUE;mustRecompute FALSE;lastEngine
"MuPAD";xmin "-15";xmax "15";xviewmin "-15";xviewmax "15";yviewmin
"-1";yviewmax "1.002079";viewset"XY";rangeset"X";plottype 4;labeloverrides
3;x-label "pi*a*sin(theta)/lambda";y-label "I";axesFont "Times New
Roman,12,0000000000,useDefault,normal";numpoints 100;plotstyle
"patch";axesstyle "normal";axestips FALSE;xis \TEXUX{x};var1name
\TEXUX{$x$};function \TEXUX{$\left( \frac{\sin x}{x}\right) ^{2}$};linecolor
"blue";linestyle 1;pointstyle "point";linethickness 3;lineAttributes
"Solid";var1range "-15,15";num-x-gridlines 100;curveColor
"[flat::RGB:0x000000ff]";curveStyle "Line";VCamFile
'PR953I09.xvz';valid_file "T";tempfilename
'PQXXQYTL.wmf';tempfile-properties "XPR";}}You can see the central maximum
and the much weaker minima produced by this function. Indeed, it seems to
match what we saw very well. Putting it all together, our pattern looks like
this.\FRAME{dhF}{3.0329in}{1.535in}{0pt}{}{}{Figure}{\special{language
"Scientific Word";type "GRAPHIC";maintain-aspect-ratio TRUE;display
"USEDEF";valid_file "T";width 3.0329in;height 1.535in;depth
0pt;original-width 2.9888in;original-height 1.4987in;cropleft "0";croptop
"1";cropright "1";cropbottom "0";tempfilename
'PQXXQYTM.wmf';tempfile-properties "XPR";}}

This is really an interesting result. You might wonder why, when we found
the two slit interference pattern, there was no evidence of the single slit
fringing that we discovered in this chapter. After all, a double slit system
is made from single slits. Shouldn't there be some effect due to the fact
that the slits are individually single slits? The answer is that we did see
some hint of the single slit pattern. Remember the figure below. \FRAME{dhF}{%
4.4719in}{2.2606in}{0pt}{}{}{Figure}{\special{language "Scientific
Word";type "GRAPHIC";maintain-aspect-ratio TRUE;display "USEDEF";valid_file
"T";width 4.4719in;height 2.2606in;depth 0pt;original-width
4.4209in;original-height 2.2208in;cropleft "0";croptop "1";cropright
"1";cropbottom "0";tempfilename 'PQXXQYTN.wmf';tempfile-properties "XPR";}}%
The intensity of the peaks seems to fall off with distance from the center.
We dealt with only the center-most part of the pattern. If we draw the
pattern for larger angles, we see the following.\FRAME{dtbpF}{4.1883in}{%
2.3497in}{0pt}{}{}{Figure}{\special{language "Scientific Word";type
"GRAPHIC";maintain-aspect-ratio TRUE;display "USEDEF";valid_file "T";width
4.1883in;height 2.3497in;depth 0pt;original-width 4.1381in;original-height
2.3099in;cropleft "0";croptop "1";cropright "1";cropbottom "0";tempfilename
'PQXXQYTO.wmf';tempfile-properties "XPR";}}

It takes a bright laser or dark room to see the secondary groups of fringes
easily, but we can do it. We can also graph the intensity pattern. It is the
combination of the two slit and single slit pattern with the single slit
pattern acting and an envelope.

\begin{equation}
I=I_{\max }\cos ^{2}\left( \frac{\pi d\sin \left( \theta \right) }{\lambda }%
\right) \left( \frac{\sin \left( \frac{\pi a\sin \left( \theta \right) }{%
\lambda }\right) }{\frac{\pi a\sin \left( \theta \right) }{\lambda }}\right)
^{2}
\end{equation}%
Note that one of the double slit maxima is clobbered by a minimum in the
single slit pattern. We can find out the order of the missing maximum.
Recall that 
\[
d\sin \left( \theta \right) =m\lambda 
\]%
describes the maxima from the double slit. But%
\[
a\sin \left( \theta \right) =\lambda 
\]%
describes the minimum from the single slit. Dividing these yields%
\begin{eqnarray*}
\frac{d\sin \left( \theta \right) }{a\sin \left( \theta \right) } &=&\frac{%
m\lambda }{\lambda } \\
\frac{d}{a} &=&m
\end{eqnarray*}%
so the 
\begin{equation}
m=\frac{d}{a}
\end{equation}%
double slit maximum will be missing.

\section{Circular Apertures}

%TCIMACRO{%
%\TeXButton{Question 123.27.4}{\marginpar {
%\hspace{-0.5in}
%\begin{minipage}[t]{1in}
%\small{Question 123.27.4}
%\end{minipage}
%}}}%
%BeginExpansion
\marginpar {
\hspace{-0.5in}
\begin{minipage}[t]{1in}
\small{Question 123.27.4}
\end{minipage}
}%
%EndExpansion
%TCIMACRO{%
%\TeXButton{Question 123.27.5}{\marginpar {
%\hspace{-0.5in}
%\begin{minipage}[t]{1in}
%\small{Question 123.27.5}
%\end{minipage}
%}}}%
%BeginExpansion
\marginpar {
\hspace{-0.5in}
\begin{minipage}[t]{1in}
\small{Question 123.27.5}
\end{minipage}
}%
%EndExpansion
%TCIMACRO{%
%\TeXButton{Question 123.27.6}{\marginpar {
%\hspace{-0.5in}
%\begin{minipage}[t]{1in}
%\small{Question 123.27.6}
%\end{minipage}
%}}}%
%BeginExpansion
\marginpar {
\hspace{-0.5in}
\begin{minipage}[t]{1in}
\small{Question 123.27.6}
\end{minipage}
}%
%EndExpansion
Our analysis of light going through holes has been somewhat limited by
squarish holes or slits. But most optical systems, including our eyes don't
have square holes. So what happens when the hole is round? The situation is
as shown in the next figure.\FRAME{dhF}{2.341in}{1.2842in}{0pt}{}{}{Figure}{%
\special{language "Scientific Word";type "GRAPHIC";maintain-aspect-ratio
TRUE;display "USEDEF";valid_file "T";width 2.341in;height 1.2842in;depth
0pt;original-width 4.4763in;original-height 2.4431in;cropleft "0";croptop
"1";cropright "1";cropbottom "0";tempfilename
'PQXXQYTP.wmf';tempfile-properties "XPR";}}Before we discuss this situation,
let's think about the width of a single slit pattern. We remember that 
\[
\sin \left( \theta \right) =\left( 1\right) \frac{\lambda }{a} 
\]%
for the first minima,or that 
\[
\theta \approx \frac{\lambda }{a} 
\]%
and from the figure we can see that 
\[
\theta \approx \frac{y}{L} 
\]%
so long as $\theta $ is small, then we find the position of the first
minimum to be%
\[
y\approx \frac{\lambda }{a}L 
\]%
This is the distance from the center bright spot to the first dark spot. The
width of the bright spot is twice this distance%
\[
w\approx 2\frac{\lambda }{a}L 
\]%
We expect something like this for our circular aperture. The derivation is
not to hard, but it involves Bessel functions, which are beyond the math
requirement for this course. So I will give you the answer%
\[
\theta \approx 1.22\frac{\lambda }{D} 
\]%
That's right, the circular aperture (hole) only adds a factor of 1.22. And
as with the slit%
\[
\theta \approx \frac{y}{L} 
\]%
so%
\begin{equation}
y\approx 1.22\frac{\lambda }{D}L
\end{equation}%
and 
\begin{equation}
w\approx 2.44\frac{\lambda }{D}L
\end{equation}%
%TCIMACRO{%
%\TeXButton{Airy Pattern Demo}{\marginpar {
%\hspace{-0.5in}
%\begin{minipage}[t]{1in}
%\small{Airy Pattern Demo}
%\end{minipage}
%}}}%
%BeginExpansion
\marginpar {
\hspace{-0.5in}
\begin{minipage}[t]{1in}
\small{Airy Pattern Demo}
\end{minipage}
}%
%EndExpansion
The picture in most books is a little bit deceptive. The pattern looks a
little like the slit pattern. But the secondary maxima are actually very
small for the circular aperture case. Much smaller than the secondary maxima
in the slit case. Here is a larger version. \FRAME{dtbpFUX}{4.9061in}{%
3.3667in}{0pt}{\Qcb{{}}}{}{Plot}{\special{language "Scientific Word";type
"MAPLEPLOT";width 4.9061in;height 3.3667in;depth 0pt;display
"USEDEF";plot_snapshots TRUE;mustRecompute FALSE;lastEngine "MuPAD";xmin
"-20";xmax "20.0100";xviewmin "-20";xviewmax "20.0100";yviewmin
"-0.000100";yviewmax "1.000100";viewset"XY";rangeset"X";plottype
4;labeloverrides 3;x-label "k*a*sin(theta)";y-label "I/I_o";axesFont "Times
New Roman,12,0000000000,useDefault,normal";numpoints 100;plotstyle
"patch";axesstyle "normal";axestips FALSE;xis \TEXUX{x};var1name
\TEXUX{$x$};function \TEXUX{$\frac{1}{\allowbreak 0.25}\left( \text{
}\frac{J_{1}\left( x\right) }{x}\right) ^{2}$};linecolor "blue";linestyle
1;pointstyle "point";linethickness 3;lineAttributes "Solid";var1range
"-20,20.0100";num-x-gridlines 100;curveColor
"[flat::RGB:0x000000ff]";curveStyle "Line";VCamFile
'PRGS6V19.xvz';valid_file "T";tempfilename
'PRGRKJ05.wmf';tempfile-properties "XPR";}}

A three dimensional version of the intensity pattern from the circular
aperture.\FRAME{dtbpFUX}{4.7573in}{3.1713in}{0pt}{\Qcb{{}}}{}{Plot}{\special%
{language "Scientific Word";type "MAPLEPLOT";width 4.7573in;height
3.1713in;depth 0pt;display "USEDEF";plot_snapshots TRUE;mustRecompute
FALSE;lastEngine "MuPAD";xmin "-0.02";xmax "0.02";ymin "-0.02";ymax
"0.02";xviewmin "-0.02";xviewmax "0.02";yviewmin "-0.02";yviewmax
"0.02";zviewmin "-0.1";zviewmax "1";viewset"XYZ";rangeset"XYZ";phi 74;theta
-118;cameraDistance "1.19837";cameraOrientation
"[0,0,4.0674]";cameraOrientationFixed TRUE;plottype 5;axesFont "Times New
Roman,12,0000000000,useDefault,normal";num-x-gridlines 25;num-y-gridlines
25;plotstyle "hidden";axesstyle "normal";axestips FALSE;plotshading
"Z";lighting 0;xis \TEXUX{x};yis \TEXUX{y};var1name \TEXUX{$x$};var2name
\TEXUX{$y$};function \TEXUX{$\frac{1}{\allowbreak 0.25}\left( \text{
}\frac{J_{1}\left( \frac{2\pi \left( \sqrt{x^{2}+y^{2}}\right) \left(
0.05\right) }{\left( 500\times 10^{-6}\right) \left( 1\right) }\right)
}{\frac{2\pi \left( \sqrt{x^{2}+y^{2}}\right) \left( 0.05\right) }{\left(
500\times 10^{-6}\right) \left( 1\right) }}\right) ^{2}$};linestyle
1;pointstyle "point";linethickness 1;lineAttributes "Solid";var1range
"-0.02,0.02";var2range "-0.02,0.02";surfaceColor
"[linear:Z:RGB:0000000000:0000000000]";surfaceStyle "Hidden
Line";num-x-gridlines 75;num-y-gridlines 75;surfaceMesh
"Mesh";rangeset"XY";VCamFile 'PRGS1O15.xvz';valid_file "T";tempfilename
'PRGS1E06.wmf';tempfile-properties "XPR";}}With a bright enough laser, they
pattern becomes visible.\FRAME{dhFU}{2.7415in}{2.6757in}{0pt}{\Qcb{{}}}{}{%
Figure}{\special{language "Scientific Word";type
"GRAPHIC";maintain-aspect-ratio TRUE;display "USEDEF";valid_file "T";width
2.7415in;height 2.6757in;depth 0pt;original-width 1.0404in;original-height
1.0136in;cropleft "0";croptop "1";cropright "1";cropbottom "0";tempfilename
'PQXXQYTS.wmf';tempfile-properties "XPR";}}

\chapter{Interferometers and Rays}

%TCIMACRO{%
%\TeXButton{Fundamental Concepts}{\hspace{-1.3in}{\Large Fundamental Concepts\vspace{0.25in}}}}%
%BeginExpansion
\hspace{-1.3in}{\Large Fundamental Concepts\vspace{0.25in}}%
%EndExpansion

%TCIMACRO{%
%\TeXButton{Interferomenter Demo}{\marginpar {
%\hspace{-0.5in}
%\begin{minipage}[t]{1in}
%\small{Interferomenter Demo}
%\end{minipage}
%}}}%
%BeginExpansion
\marginpar {
\hspace{-0.5in}
\begin{minipage}[t]{1in}
\small{Interferomenter Demo}
\end{minipage}
}%
%EndExpansion
Before we leave wave properties of physics and go to the ray approximation,
we should study some devices that use interference.

\section{The Michelson Interferometer}

The Michelson interferometer is another device that uses path differences to
create interference fringes. \FRAME{dhF}{3.2007in}{3.1263in}{0pt}{}{}{Figure%
}{\special{language "Scientific Word";type "GRAPHIC";maintain-aspect-ratio
TRUE;display "USEDEF";valid_file "T";width 3.2007in;height 3.1263in;depth
0pt;original-width 3.1566in;original-height 3.0805in;cropleft "0";croptop
"1";cropright "1";cropbottom "0";tempfilename
'PQXXQYTT.wmf';tempfile-properties "XPR";}}The device is shown in the
figure. A coherent light source is used. The light beam is split into two
beams that are usually at $90\unit{%
%TCIMACRO{\U{b0}}%
%BeginExpansion
{{}^\circ}%
%EndExpansion
}$ apart. The beams are reflected off of two mirrors back along the same
path and are mixed at the telescope. The result (with perfect alignment) is
a target fringe pattern like the first two shown below. \FRAME{dhF}{3.5362in%
}{1.6466in}{0pt}{}{}{Figure}{\special{language "Scientific Word";type
"GRAPHIC";maintain-aspect-ratio TRUE;display "USEDEF";valid_file "T";width
3.5362in;height 1.6466in;depth 0pt;original-width 3.4895in;original-height
1.6103in;cropleft "0";croptop "1";cropright "1";cropbottom "0";tempfilename
'PQXXQYTU.wmf';tempfile-properties "XPR";}}If the alignment is off, you get
smaller fringes, but the system can still work. This is shown in the last
image in the previous figure.

In the figure, we have constructive interference in the center, but if we
move one of the mirrors half a wavelength, we would have destructive
interference and would see a dark spot in the center. This device gives us
the ability to measure distances on the order of the wavelength of the
light. When the distance is continuously changed, the pattern seems to grow
from the center (or collapse into the center).

Notice that if the mirror is moved $\frac{\lambda }{2},$ the path distance
changes by $\lambda $ because the light travels the distance to the mirror
and then back from the mirror (it travels the path twice!).

\section{Holography}

%TCIMACRO{%
%\TeXButton{Hologram demo-picture of woman}{\marginpar {
%\hspace{-0.5in}
%\begin{minipage}[t]{1in}
%\small{Hologram demo-picture of woman}
%\end{minipage}
%}}}%
%BeginExpansion
\marginpar {
\hspace{-0.5in}
\begin{minipage}[t]{1in}
\small{Hologram demo-picture of woman}
\end{minipage}
}%
%EndExpansion
%TCIMACRO{%
%\TeXButton{Hologram demo-chess pieces}{\marginpar {
%\hspace{-0.5in}
%\begin{minipage}[t]{1in}
%\small{Hologram demo-chess pieces}
%\end{minipage}
%}}}%
%BeginExpansion
\marginpar {
\hspace{-0.5in}
\begin{minipage}[t]{1in}
\small{Hologram demo-chess pieces}
\end{minipage}
}%
%EndExpansion
You may have seen holograms in the past. We have enough understanding of
light to understand how they are generated now.

\FRAME{dtbpF}{2.853in}{1.7772in}{0pt}{}{}{Figure}{\special{language
"Scientific Word";type "GRAPHIC";maintain-aspect-ratio TRUE;display
"USEDEF";valid_file "T";width 2.853in;height 1.7772in;depth
0pt;original-width 8.0834in;original-height 5.0246in;cropleft "0";croptop
"1";cropright "1";cropbottom "0";tempfilename
'PQXXQYTV.wmf';tempfile-properties "XPR";}}

A device for generating a hologram is shown in the figure above. Light from
a laser or other coherent source is expanded and split into two beams. One
travels to a photographic plate, the other is directed to an object. At the
object, light is scattered and the scattered light also reaches the
photographic plate. The combination of the direct and scattered beams
generates a complicated interference pattern.

The pattern can be developed (like you develop photographic film). Once
developed, it can be re- illuminated with a direct beam. The emulsion on the
plate creates complicated patterns of light transmission, which combine to
create interference. It is like a very complicated slit pattern or grating
pattern. The result is a three-dimensional image generated by the
interference. The interference pattern generates an image that looks like
the original object.\FRAME{dtbpF}{2.687in}{1.689in}{0pt}{}{}{Figure}{\special%
{language "Scientific Word";type "GRAPHIC";display "USEDEF";valid_file
"T";width 2.687in;height 1.689in;depth 0pt;original-width
8.0834in;original-height 5.0246in;cropleft "0";croptop "1";cropright
"1";cropbottom "0";tempfilename 'PQXXQYTW.wmf';tempfile-properties "XPR";}}

\section{Diffraction of X-rays by Crystals}

If we make the wavelength of light very small, then we can deal with very
small diffraction gratings. This concept is used to investigate the
structure of crystals with x-rays. The crystal latus of molecules or atoms
creates the regular pattern we need for a grating. The pattern is three
dimensional, so the patterns are complex.

Let's start with a simple crystal with a square regular latus. $NaCl$ has
such a structure.

\FRAME{dhF}{3.2038in}{1.4387in}{0in}{}{}{Figure}{\special{language
"Scientific Word";type "GRAPHIC";maintain-aspect-ratio TRUE;display
"USEDEF";valid_file "T";width 3.2038in;height 1.4387in;depth
0in;original-width 3.2375in;original-height 1.4378in;cropleft "0";croptop
"1";cropright "1";cropbottom "0";tempfilename
'PQXXQYTX.wmf';tempfile-properties "XPR";}}If we illuminate the crystal with
x-rays, the x-rays can reflect off the top layer of atoms, or off the second
layer of atoms (or off any other layer, but for now let's just consider two
layers). If the spacing between the layers is $d,$ then the path difference
will be 
\begin{equation}
\delta =2\left( d\sin \left( \theta \right) \right)
\end{equation}%
then for constructive interference%
\begin{equation}
2d\sin \left( \theta \right) =m\lambda \qquad m=1,2,3,\ldots
\end{equation}%
This is known as \emph{Bragg's law.} This relationship can be used to
measure the distance between the crystal planes.

A resulting pattern is given in the following figure.\FRAME{dhFU}{1.6489in}{%
1.6604in}{0pt}{\Qcb{{\protect\small Diffraction image of protein crystal.
Hen egg lysozyme, X-ray souce Bruker I}$\protect\mu ${\protect\small S, }$%
\protect\lambda =0.154188\unit{nm}${\protect\small , }$45\unit{kV}$%
{\protect\small , Exposure }$10\unit{s}.${\protect\small \ (image in the
public domain)}}}{}{Figure}{\special{language "Scientific Word";type
"GRAPHIC";maintain-aspect-ratio TRUE;display "USEDEF";valid_file "T";width
1.6489in;height 1.6604in;depth 0pt;original-width 1.6129in;original-height
1.6233in;cropleft "0";croptop "1";cropright "1";cropbottom "0";tempfilename
'PQXXQYTY.wmf';tempfile-properties "XPR";}}DNA\ makes in interesting
diffraction pattern.\FRAME{dhFU}{1.5809in}{1.8273in}{0pt}{\Qcb{%
{\protect\small X-ray diffraction pattern of DNA (image courtesy of the
National Institute of Health, image in the public domain)}}}{}{Figure}{%
\special{language "Scientific Word";type "GRAPHIC";maintain-aspect-ratio
TRUE;display "USEDEF";valid_file "T";width 1.5809in;height 1.8273in;depth
0pt;original-width 1.5437in;original-height 1.7902in;cropleft "0";croptop
"1";cropright "1";cropbottom "0";tempfilename
'PQXXQYTZ.wmf';tempfile-properties "XPR";}}

\section{Transition to the ray model}

%TCIMACRO{%
%\TeXButton{Question 223.28.1}{\marginpar {
%\hspace{-0.5in}
%\begin{minipage}[t]{1in}
%\small{Question 223.28.1}
%\end{minipage}
%}}}%
%BeginExpansion
\marginpar {
\hspace{-0.5in}
\begin{minipage}[t]{1in}
\small{Question 223.28.1}
\end{minipage}
}%
%EndExpansion
In the next figure, two waves of different wavelengths go through a single
opening. The wave representing the central maxima is shown in each case, but
not the secondary maxima.\FRAME{dhF}{2.7951in}{2.776in}{0pt}{}{}{Figure}{%
\special{language "Scientific Word";type "GRAPHIC";maintain-aspect-ratio
TRUE;display "USEDEF";valid_file "T";width 2.7951in;height 2.776in;depth
0pt;original-width 2.7527in;original-height 2.7337in;cropleft "0";croptop
"1";cropright "1";cropbottom "0";tempfilename
'PQXXQYU0.wmf';tempfile-properties "XPR";}}Notice that the smaller
wavelength has a narrower central maxima as we would expect from%
\[
\sin \left( \theta _{dark}\right) =1.22\frac{\lambda }{D} 
\]%
or 
\[
\theta _{dark}\approx 1.22\frac{\lambda }{D} 
\]%
we see that the ratio of the wavelength to the hole size determines the
angular extent of the central maxima. The smaller the ratio, the smaller the
central region. We can use this to explain why the wave nature of light was
so hard to find.

The patch of light on a screen that is created by light passing through the
aperture is created by the central maximum. \FRAME{dhF}{3.0191in}{2.2468in}{%
0pt}{}{}{Figure}{\special{language "Scientific Word";type
"GRAPHIC";maintain-aspect-ratio TRUE;display "USEDEF";valid_file "T";width
3.0191in;height 2.2468in;depth 0pt;original-width 2.975in;original-height
2.207in;cropleft "0";croptop "1";cropright "1";cropbottom "0";tempfilename
'PQXXQYU1.wmf';tempfile-properties "XPR";}}For the long wavelength (red) the
central maximum is larger than the screen. The short wavelength spot will be
wholly on the screen as shown. The geometric spot is what we would see if
the light traveled straight through the opening. Notice that the short
wavelength spot is closer to the size of the geometric spot. In the limit
that 
\[
\lambda \ll a 
\]%
or for circular openings 
\[
\lambda \ll D 
\]%
then 
\[
\theta \approx \frac{\lambda }{a}\approx 0 
\]%
or%
\[
\theta \approx \frac{\lambda }{D}\approx 0 
\]%
and the spot size would be very nearly equal to the geometric spot size.

This is the limit we will call the \emph{ray approximation}.

For most of mankind's time on the Earth, it was very hard to build holes
that were comparable to the size of a wavelength of visible light. So it is
no wonder that the waviness of light was missed for so many years.

But this ray limit is very useful for apertures the size of camera lenses.
So we will begin to use this small $\lambda ,$ large aperture approximation.

\section{The Ray Approximation in Geometric Optics}

In the last section we said that when the geometric spot size was larger
than the spot due to diffraction, we could ignore diffraction and use the
simpler ray model. This is usually true in our personal experiences. But
this may not be true in experiments or devices we design. We should see
where the crossover point is.

Intuitively, if the aperture and the spot are the same size, that ought to
be some sort of critical point. That is when the aperture size is equal to
the spot size. We found that for a circular aperture the spot width is 
\[
w\approx 2.44\frac{\lambda }{D_{aperture}}L 
\]%
We want the case where $w=D_{aperture}$

\[
D_{aperture}=2.44\frac{\lambda }{D_{aperture}}L 
\]%
This gives 
\[
D_{aperture}=\sqrt{2.44\lambda L} 
\]

Of course this is for round apertures, but for square apertures we know we
remove the $2.44.$ This gives about a millimeter for visible wavelengths.%
\begin{eqnarray*}
D_{aperture} &=&\sqrt{2.44\left( 500\unit{nm}\right) \left( 1\unit{m}\right) 
} \\
&=&1.\,\allowbreak 104\,5\times 10^{-3}\allowbreak \unit{m}
\end{eqnarray*}

for apertures much larger than a millimeter, we expect interference effects
due to diffraction through the aperture to be much harder to see. We expect
them to be easy to see if the aperture is smaller than a millimeter. But
what about when the aperture is about a millimeter in size? That is a
subject for PH375, and so we will avoid this case in this class. But this is
not too restrictive. Most good optical systems have apertures larger than $1%
\unit{mm}.$ Cell phone cameras may be an exception (but I don't consider
cell phone cameras to be good optical systems). Even our eyes have an
aperture that varies from about $2\unit{mm}$ to about $7\unit{mm},$ so most
common experiences in visible wavelengths will work fine with what we learn.
Note that for microwave or radio wave systems this may really not be true!

How about the other extreme? Suppose $\lambda \gg D.$ This is really beyond
our class (requires partial differential equations), but in the extreme
case, we can use reason to find out what happens. If the opening is much
smaller than the wavelength, then the wave does not see the opening, and no
wave is produced on the other side. This is the case of a microwave oven
door. If the wavelength is much larger than the spacing of the little dots
or lines that span the door, then the waves will not leave the interior of
the microwave oven. Of course as the wavelength becomes closer to $D$ this
is less true, and this case is more challenging to calculate, and we will
save it for a 300 level electrodynamics course.

To summarize%
\[
\begin{tabular}{ll}
$\lambda \ll D$ & Wave nature of light is not visible \\ 
$\lambda \approx D$ & Wave nature of light is apparent \\ 
$\lambda \gg D$ & Little to no penetration of aperture by the wave%
\end{tabular}%
\]

\section{The ray model and phase}

%TCIMACRO{%
%\TeXButton{Question 123.28.2}{\marginpar {
%\hspace{-0.5in}
%\begin{minipage}[t]{1in}
%\small{Question 123.28.2}
%\end{minipage}
%}}}%
%BeginExpansion
\marginpar {
\hspace{-0.5in}
\begin{minipage}[t]{1in}
\small{Question 123.28.2}
\end{minipage}
}%
%EndExpansion
There is a further complication that helps to explain why the wave nature of
light was not immediately apparent. Let's consider a light source.\FRAME{dhF%
}{0.6339in}{1.0957in}{0pt}{}{}{Figure}{\special{language "Scientific
Word";type "GRAPHIC";maintain-aspect-ratio TRUE;display "USEDEF";valid_file
"T";width 0.6339in;height 1.0957in;depth 0pt;original-width
1.3214in;original-height 2.3039in;cropleft "0";croptop "1";cropright
"1";cropbottom "0";tempfilename 'PQXXQYU2.wmf';tempfile-properties "XPR";}}%
For a typical light source, the filament is larger than about a millimeter,
so we should expect that diffraction should be hard to see. But the filament
is made of hot metal. The atoms of the hot metal emit light because of the
extra energy they have. The method of producing this light is that the
atom's excited electrons are in upper shells because of the extra thermal
energy provided by the electricity flowing through the filament. But the
electrons eventually fall to their proper shell, and in doing so they give
off the extra energy as light. It is not too hard to believe that this
process of exciting electrons and having them fall back down is a random
process. Each electron that moves starts a wave. The atoms have different
positions, so there will be a path difference $\Delta r$ between each atom's
waves. There will also be a time difference $\Delta t$ between when the
waves start. We can model this with a $\Delta \phi _{o}.$

%TCIMACRO{%
%\TeXButton{Question 123.28.3}{\marginpar {
%\hspace{-0.5in}
%\begin{minipage}[t]{1in}
%\small{Question 123.28.3}
%\end{minipage}
%}}}%
%BeginExpansion
\marginpar {
\hspace{-0.5in}
\begin{minipage}[t]{1in}
\small{Question 123.28.3}
\end{minipage}
}%
%EndExpansion
It is also true that not all of the electrons fall from the same shell. This
gives us different frequencies, so we expect beating between different waves
from different atoms. It is also true that we have millions of atoms, so we
have millions of waves.

Let's look at just two of these waves

\[
\begin{tabular}{l}
$\lambda =2$ \\ 
$k=\frac{2\pi }{\lambda }$ \\ 
$\omega =1$ \\ 
$\phi _{o}=\frac{\pi }{6}$ \\ 
$t=0$ \\ 
$E_{o}=1\frac{\unit{N}}{\unit{C}}$%
\end{tabular}%
\]

\[
E_{1}=E_{o}\sin \left( kx-\omega t\right) 
\]

\FRAME{dtbpFX}{2.156in}{0.6875in}{0pt}{}{}{Plot}{\special{language
"Scientific Word";type "MAPLEPLOT";width 2.156in;height 0.6875in;depth
0pt;display "USEDEF";plot_snapshots TRUE;mustRecompute FALSE;lastEngine
"MuPAD";xmin "0";xmax "5.001000";xviewmin "0";xviewmax "5.001000";yviewmin
"-2";yviewmax "2";viewset"XY";rangeset"X";plottype 4;labeloverrides
2;y-label "E";axesFont "Times New
Roman,12,0000000000,useDefault,normal";numpoints 100;plotstyle
"patch";axesstyle "normal";axestips FALSE;xis \TEXUX{x};var1name
\TEXUX{$x$};function \TEXUX{$\allowbreak \sin \pi x$};linecolor
"blue";linestyle 1;pointstyle "point";linethickness 3;lineAttributes
"Solid";var1range "0,5.001000";num-x-gridlines 100;curveColor
"[flat::RGB:0x000000ff]";curveStyle "Line";VCamFile
'PRGS7O1A.xvz';valid_file "T";tempfilename
'PQXXQYU3.wmf';tempfile-properties "XPR";}}

\[
E_{2}=E_{o}\sin \left( kx-\omega t+\phi _{o}\right) 
\]%
\FRAME{dtbpFX}{2.156in}{0.6875in}{0pt}{}{}{Plot}{\special{language
"Scientific Word";type "MAPLEPLOT";width 2.156in;height 0.6875in;depth
0pt;display "USEDEF";plot_snapshots TRUE;mustRecompute FALSE;lastEngine
"MuPAD";xmin "0";xmax "5.001000";xviewmin "0";xviewmax "5.001000";yviewmin
"-2";yviewmax "2";viewset"XY";rangeset"X";plottype 4;labeloverrides
2;y-label "E";axesFont "Times New
Roman,12,0000000000,useDefault,normal";numpoints 100;plotstyle
"patch";axesstyle "normal";axestips FALSE;xis \TEXUX{x};var1name
\TEXUX{$x$};function \TEXUX{$\sin \left( \frac{1}{6}\pi +\pi x\right)
$};linecolor "maroon";linestyle 1;pointstyle "point";linethickness
3;lineAttributes "Solid";var1range "0,5.001000";num-x-gridlines
100;curveColor "[flat::RGB:0x00800000]";curveStyle "Line";VCamFile
'PRGS7O1B.xvz';valid_file "T";tempfilename
'PQXXQYU4.wmf';tempfile-properties "XPR";}}then%
\[
E_{r}=E_{o}\sin \left( kx-\omega t\right) +E_{o}\sin \left( kx-\omega t+\phi
_{o}\right) 
\]

We found for these two waves%
\[
E_{r}=2E_{o}\cos \left( \frac{\phi _{o}}{2}\right) \sin \left( kx-\omega t+%
\frac{\phi _{o}}{2}\right) 
\]%
\FRAME{dtbpFX}{2.354in}{0.5846in}{0pt}{}{}{Plot}{\special{language
"Scientific Word";type "MAPLEPLOT";width 2.354in;height 0.5846in;depth
0pt;display "USEDEF";plot_snapshots TRUE;mustRecompute FALSE;lastEngine
"MuPAD";xmin "0";xmax "5.001000";xviewmin "0";xviewmax "5.001000";yviewmin
"-2";yviewmax "2";viewset"XY";rangeset"X";plottype 4;labeloverrides
2;y-label "E";axesFont "Times New
Roman,12,0000000000,useDefault,normal";numpoints 100;plotstyle
"patch";axesstyle "normal";axestips FALSE;xis \TEXUX{x};var1name
\TEXUX{$x$};function \TEXUX{$\sin \left( \frac{1}{6}\pi +\pi x\right) +\sin
\pi x$};linecolor "green";linestyle 1;pointstyle "point";linethickness
3;lineAttributes "Solid";var1range "0,5.001000";num-x-gridlines
100;curveColor "[flat::RGB:0x00008000]";curveStyle "Line";VCamFile
'PRGS7T1E.xvz';valid_file "T";tempfilename
'PQXXQYU5.wmf';tempfile-properties "XPR";}}

But suppose we complicate the situation by sending lots of waves at random
times, each with different amplitudes and wavelengths, down the rope. If we
look at a single point for a specific time, we might be experiencing
interference, but it would be hard to tell. Lets try this mathematically. I
will combine many waves with random phases, some coming from the right and
some coming from the left.

\begin{eqnarray*}
E_{1} &=&E_{o}\sin \left( 5x-\omega t+\frac{\pi }{4}\right) +0.5E_{o}\sin
\left( 0.2x-\omega t-\frac{\pi }{6}\right) \\
&&+3.6E_{o}\sin \left( .4x-\omega t+\frac{\pi }{10}\right) +4E_{o}\sin
\left( 20x-\omega t-\frac{\pi }{7}\right) \\
&&+.2E_{o}\sin \left( 15x-\omega t+1\right) +0.7E_{o}\sin \left( .7x-\omega
t-.25\right)
\end{eqnarray*}%
\FRAME{dtbpFX}{2.354in}{1.1943in}{0pt}{}{}{Plot}{\special{language
"Scientific Word";type "MAPLEPLOT";width 2.354in;height 1.1943in;depth
0pt;display "USEDEF";plot_snapshots TRUE;mustRecompute FALSE;lastEngine
"MuPAD";xmin "0";xmax "5";xviewmin "0";xviewmax "5";yviewmin "-20";yviewmax
"20";viewset"XY";rangeset"X";plottype 4;labeloverrides 2;y-label
"E";axesFont "Times New Roman,12,0000000000,useDefault,normal";numpoints
100;plotstyle "patch";axesstyle "normal";axestips FALSE;xis
\TEXUX{x};var1name \TEXUX{$x$};function \TEXUX{$0.2\sin \left( 15x+1\right)
+0.7\sin \left( 0.7x-0.25\right) +\allowbreak 0.5\sin \left(
0.2x-\frac{1}{6}\pi \right) +3.\,\allowbreak 6\sin \left( \frac{1}{10}\pi
+0.4x\right) +\allowbreak \sin \left( \frac{1}{4}\pi +5x\right) +4\sin
\left( 20x-\frac{1}{7}\pi \right) $};linecolor "blue";linestyle 1;pointstyle
"point";linethickness 1;lineAttributes "Solid";var1range
"0,5";num-x-gridlines 100;curveColor "[flat::RGB:0x000000ff]";curveStyle
"Line";VCamFile 'PRGS7T1F.xvz';valid_file "T";tempfilename
'PQXXQYU6.wmf';tempfile-properties "XPR";}}%
\begin{eqnarray*}
E_{2} &=&E_{o}\sin \left( 0.2x+\omega t+\pi \right) +2E_{o}\sin \left(
5x+\omega t+\frac{\pi }{6}\right) \\
&&+6E_{o}\sin \left( 0.4x+\omega t+\frac{\pi }{3.5}\right) +0.4E_{o}\sin
\left( 20x+\omega t-0\right) \\
&&+E_{o}\sin \left( 15x+\omega t+1\right) +0.7E_{o}\sin \left( .7x+\omega
t-4\right)
\end{eqnarray*}%
\FRAME{dtbpFX}{2.271in}{1.0343in}{0pt}{}{}{Plot}{\special{language
"Scientific Word";type "MAPLEPLOT";width 2.271in;height 1.0343in;depth
0pt;display "USEDEF";plot_snapshots TRUE;mustRecompute FALSE;lastEngine
"MuPAD";xmin "0";xmax "5.001000";xviewmin "0";xviewmax "5.001000";yviewmin
"-20";yviewmax "20";viewset"XY";rangeset"X";plottype 4;labeloverrides
2;y-label "E";axesFont "Times New
Roman,12,0000000000,useDefault,normal";numpoints 100;plotstyle
"patch";axesstyle "normal";axestips FALSE;xis \TEXUX{x};var1name
\TEXUX{$x$};function \TEXUX{$0.7\sin \left( 0.7x-4\right) +6\sin \left(
0.285\,71\pi +0.4x\right) +\allowbreak \sin \left( 15x+1\right) -\sin
0.2x+2\sin \left( \frac{1}{6}\pi +5x\right) +0.4\allowbreak \sin
20x$};linecolor "red";linestyle 1;pointstyle "point";linethickness
2;lineAttributes "Solid";var1range "0,5.001000";num-x-gridlines
100;curveColor "[flat::RGB:0x00ff0000]";curveStyle "Line";VCamFile
'PRGS7T1G.xvz';valid_file "T";tempfilename
'PQXXQYU7.wmf';tempfile-properties "XPR";}}

Then $E_{1}+E_{2}$ looks like\FRAME{dtbpFX}{2.3333in}{1.1182in}{0pt}{}{}{Plot%
}{\special{language "Scientific Word";type "MAPLEPLOT";width 2.3333in;height
1.1182in;depth 0pt;display "USEDEF";plot_snapshots TRUE;mustRecompute
FALSE;lastEngine "MuPAD";xmin "0";xmax "5.001000";xviewmin "0";xviewmax
"5.001000";yviewmin "-20";yviewmax "20";viewset"XY";rangeset"X";plottype
4;labeloverrides 2;y-label "E";axesFont "Times New
Roman,12,0000000000,useDefault,normal";numpoints 100;plotstyle
"patch";axesstyle "normal";axestips FALSE;xis \TEXUX{x};var1name
\TEXUX{$x$};function \TEXUX{$\allowbreak 0.7\sin \left( 0.7x-4\right) +6\sin
\left( 0.285\,71\pi +0.4x\right) +\allowbreak 1.\,\allowbreak 2\sin \left(
15x+1\right) +0.7\sin \left( 0.7x-0.25\right) +\allowbreak 0.5\sin \left(
0.2x-\frac{1}{6}\pi \right) +3.\,\allowbreak 6\sin \left( \frac{1}{10}\pi
+0.4x\right) -\allowbreak \sin 0.2x+\sin \left( \frac{1}{4}\pi +5x\right)
+2\sin \left( \frac{1}{6}\pi +5x\right) +4\sin \left( 20x-\frac{1}{7}\pi
\right) +\allowbreak 0.4\sin 20x$};linecolor "green";linestyle 1;pointstyle
"point";linethickness 1;lineAttributes "Solid";var1range
"0,5.001000";num-x-gridlines 100;curveColor
"[flat::RGB:0x00008000]";curveStyle "Line";VCamFile
'PRGS7V1H.xvz';valid_file "T";tempfilename
'PQXXQYU8.wmf';tempfile-properties "XPR";}}

In this example, you could think about the superposition of $E_{1}$ and $%
E_{2}$ and predict the outcome, but if there were millions of waves, each
with it's own wavelength, phase, and amplitude, the situation would be
hopeless. Note that the fluctuations in these waves are much more frequent
than our original waves. With all the added waves, we get a rapid change in
amplitude.

Now if these waves are light waves, our eyes and most detectors are not able
to react fast enough to detect the rapid fluctuations. So if there is
constructive or destructive interference that might be simple enough to
distinguish, we will miss it due to our detection systems' integration
times. To describe this rapidly fluctuating interference pattern that we
can't track with our detectors, we just say that light bulbs emit \emph{%
incoherent light}. The ray approximation assumes incoherent light.

But then light bulbs and hot ovens and most things must emit incoherent
light. Does any thing emit coherent light? Sure, today the easiest source of
coherent light is a laser. That is why I have used lasers in the class
demonstrations so far. Really though, even a laser is not perfectly
coherent. One property of the laser is that it produces light with a long 
\emph{coherence length}, or it produces light that can be treated under most
circumstances as begin monochromatic and having a single phase across the
wave for much of the beam length. Radar and microwave transmitters emit
coherent light (but at frequencies we can't see) and so do radio stations.

In the past, one could carefully create a monochromatic beam with filters.
Then split the beam into two beams and remix the two beams. This would
generate two mostly coherent sources if the distances traveled were not too
large. This is what Young did.

\section{Coherency}

%TCIMACRO{%
%\TeXButton{Question 123.28.4}{\marginpar {
%\hspace{-0.5in}
%\begin{minipage}[t]{1in}
%\small{Question 123.28.4}
%\end{minipage}
%}}}%
%BeginExpansion
\marginpar {
\hspace{-0.5in}
\begin{minipage}[t]{1in}
\small{Question 123.28.4}
\end{minipage}
}%
%EndExpansion
To be coherent,

\begin{enumerate}
\item A given part of the wave must maintain a constant phase with respect
to the rest of the wave.

\item The wave must be monochromatic
\end{enumerate}

These are very hard criteria to achieve. Most light, like that from our
light bulb, is not coherent.

In the next lecture, we will leave behind the wave nature of light and
consider in-coherent beams of light. Our goal will be to understand how
light moves around us to allow us to see things.

\chapter{Reflection and Refraction}

%TCIMACRO{%
%\TeXButton{Question 223.13.4}{\marginpar {
%\hspace{-0.5in}
%\begin{minipage}[t]{1in}
%\small{Question 223.13.4}
%\end{minipage}
%}}}%
%BeginExpansion
\marginpar {
\hspace{-0.5in}
\begin{minipage}[t]{1in}
\small{Question 223.13.4}
\end{minipage}
}%
%EndExpansion
In the movie \emph{Star Wars} inter-galactic star ships blast each other
with laser cannons. The laser beams streak across the screen. This is
dramatic, but not realistic. For us to see the light, some of the light must
get to our eyes. The light must either travel directly to our eyes from the
source, or it must bounce off of something. We can make a laser beam visible
by providing dust for the light to bounce off of so it travels to our eyes.
But unless that galaxy far far away is really dusty, Hollywood doesnt
understand light waves very well. Let's start our study of geometric optics
with reflection, the bouncing of light off of an object (like dust, or a
mirror) and then consider what happens when light enters a material like
glass or water.

%TCIMACRO{%
%\TeXButton{Fundamental Concepts}{\hspace{-1.3in}{\Large Fundamental Concepts\vspace{0.25in}}}}%
%BeginExpansion
\hspace{-1.3in}{\Large Fundamental Concepts\vspace{0.25in}}%
%EndExpansion

\begin{itemize}
\item Reflection

\item Specular and Diffuse reflection

\item Refraction

\item Total internal reflection
\end{itemize}

%TCIMACRO{%
%\TeXButton{Question 223.13.5}{\marginpar {
%\hspace{-0.5in}
%\begin{minipage}[t]{1in}
%\small{Question 223.13.5}
%\end{minipage}
%}}}%
%BeginExpansion
\marginpar {
\hspace{-0.5in}
\begin{minipage}[t]{1in}
\small{Question 223.13.5}
\end{minipage}
}%
%EndExpansion
Using the ray approximation we wish to find what happens when a bundle of
rays reaches a boundary between materials. If the material boundary is very
smooth, then the rays are reflected (bounce off) in a uniform way. This is
called \emph{specular }reflection%
%TCIMACRO{%
%\TeXButton{Specular and Diffuse Reflector Demo}{\marginpar {
%\hspace{-0.5in}
%\begin{minipage}[t]{1in}
%\small{Specular and Diffuse Reflector Demo}
%\end{minipage}
%}}}%
%BeginExpansion
\marginpar {
\hspace{-0.5in}
\begin{minipage}[t]{1in}
\small{Specular and Diffuse Reflector Demo}
\end{minipage}
}%
%EndExpansion
\FRAME{dhF}{2.3895in}{2.1344in}{0pt}{}{}{Figure}{\special{language
"Scientific Word";type "GRAPHIC";maintain-aspect-ratio TRUE;display
"USEDEF";valid_file "T";width 2.3895in;height 2.1344in;depth
0pt;original-width 2.3497in;original-height 2.0954in;cropleft "0";croptop
"1";cropright "1";cropbottom "0";tempfilename
'PQXXQYU9.wmf';tempfile-properties "XPR";}}If it is not smooth, then
something different happens. The rays are reflected, but they are reflected
randomly

\FRAME{dhF}{2.2131in}{1.7193in}{0pt}{}{}{Figure}{\special{language
"Scientific Word";type "GRAPHIC";maintain-aspect-ratio TRUE;display
"USEDEF";valid_file "T";width 2.2131in;height 1.7193in;depth
0pt;original-width 1.7936in;original-height 1.388in;cropleft "0";croptop
"1";cropright "1";cropbottom "0";tempfilename
'PQXXQYUA.wmf';tempfile-properties "XPR";}}This is called \emph{diffuse}
reflection%
%TCIMACRO{%
%\TeXButton{Question 223.13.6}{\marginpar {
%\hspace{-0.5in}
%\begin{minipage}[t]{1in}
%\small{Question 223.13.6}
%\end{minipage}
%}}}%
%BeginExpansion
\marginpar {
\hspace{-0.5in}
\begin{minipage}[t]{1in}
\small{Question 223.13.6}
\end{minipage}
}%
%EndExpansion

This difference can be seen in real life. In the next figure, the surface on
the left is a specular reflector and you can see the reflected light beam.
But the surface on the right photo is diffuse, no reflected beam is seen.%
\FRAME{dhF}{3.8441in}{1.1718in}{0pt}{}{}{Figure}{\special{language
"Scientific Word";type "GRAPHIC";maintain-aspect-ratio TRUE;display
"USEDEF";valid_file "T";width 3.8441in;height 1.1718in;depth
0pt;original-width 3.7948in;original-height 1.1381in;cropleft "0";croptop
"1";cropright "1";cropbottom "0";tempfilename
'PQXXQYUB.wmf';tempfile-properties "XPR";}}

We said the surface must be smooth for there to be specular reflection. What
does smooth mean? Generally the size of the rough spots must be much smaller
than a wavelength to be considered smooth. So suppose we have a red laser.
How small do the surface variations have to be for the surface to be
considered smooth? The wavelength of a $HeNe$ laser is 
\[
\lambda _{HeNe}=633\unit{nm} 
\]%
This is very small. Modern optics for remote sensing are often manufactured
to $1/10$ of a wavelength, which would be $63\unit{nm}.$ Mirrors are smooth
for visible light.

How about a microwave beam of light like your cell phone uses?

\begin{eqnarray*}
c &=&\lambda f \\
\lambda &=&\frac{c}{f}=\frac{3\times 10^{8}\frac{\unit{m}}{\unit{s}}}{1\unit{%
GHz}} \\
&=&\allowbreak 0.3\unit{m}
\end{eqnarray*}%
Rough brick walls are smooth for cell phone light!

We can see that we must be careful in our definition of \textquotedblleft
smooth.\textquotedblright

\section{Law of reflection}

%TCIMACRO{%
%\TeXButton{Ball Bounce Demo}{\marginpar {
%\hspace{-0.5in}
%\begin{minipage}[t]{1in}
%\small{Ball Bounce Demo}
%\end{minipage}
%}}}%
%BeginExpansion
\marginpar {
\hspace{-0.5in}
\begin{minipage}[t]{1in}
\small{Ball Bounce Demo}
\end{minipage}
}%
%EndExpansion

Experience shows that if we do have a smooth surface, that light bounces
much like a ball. This is why Newton though light was a particle. Suppose we
take a flat surface and we shine a light on it. We have a ray that
approaches at an angle $\theta _{i}$ measured from the normal. Then the
reflected ray will leave the surface with an angle $\theta _{r}$ measured
from the normal such that 
\[
\theta _{r}=\theta _{i} 
\]

\FRAME{dhF}{1.8585in}{1.1857in}{0in}{}{}{Figure}{\special{language
"Scientific Word";type "GRAPHIC";maintain-aspect-ratio TRUE;display
"USEDEF";valid_file "T";width 1.8585in;height 1.1857in;depth
0in;original-width 1.8213in;original-height 1.1519in;cropleft "0";croptop
"1";cropright "1";cropbottom "0";tempfilename
'PQXXQYUC.wmf';tempfile-properties "XPR";}}This is called the \emph{law of
reflection}.

%TCIMACRO{%
%\TeXButton{Question 223.13.7}{\marginpar {
%\hspace{-0.5in}
%\begin{minipage}[t]{1in}
%\small{Question 223.13.7}
%\end{minipage}
%}}}%
%BeginExpansion
\marginpar {
\hspace{-0.5in}
\begin{minipage}[t]{1in}
\small{Question 223.13.7}
\end{minipage}
}%
%EndExpansion
%TCIMACRO{%
%\TeXButton{Question 223.13.8}{\marginpar {
%\hspace{-0.5in}
%\begin{minipage}[t]{1in}
%\small{Question 223.13.8}
%\end{minipage}
%}}}%
%BeginExpansion
\marginpar {
\hspace{-0.5in}
\begin{minipage}[t]{1in}
\small{Question 223.13.8}
\end{minipage}
}%
%EndExpansion

\subsection{Retroreflection}

Let's take and example\FRAME{dtbpF}{1.5523in}{1.452in}{0pt}{}{}{Figure}{%
\special{language "Scientific Word";type "GRAPHIC";maintain-aspect-ratio
TRUE;display "USEDEF";valid_file "T";width 1.5523in;height 1.452in;depth
0pt;original-width 3.8579in;original-height 3.608in;cropleft "0";croptop
"1";cropright "1";cropbottom "0";tempfilename
'PQXXQYUD.wmf';tempfile-properties "XPR";}}

Let's take our system to be two mirrors set at a right angle. We have a beam
of light incident at angle $\theta _{1}$. By the law of reflection, it must
leave the mirror at $\theta _{2}=\theta _{1}.$ We can see that $\alpha $
must be $90\unit{%
%TCIMACRO{\U{b0}}%
%BeginExpansion
{{}^\circ}%
%EndExpansion
}-\theta _{2}$ and it is clear that $\theta _{3}=\alpha .$ By the law of
reflection, $\theta _{3}=\theta _{4}.$ Then, since 
\begin{eqnarray*}
90\unit{%
%TCIMACRO{\U{b0}}%
%BeginExpansion
{{}^\circ}%
%EndExpansion
} &=&\theta _{2}+\alpha \\
&=&\theta _{2}+\theta _{3}
\end{eqnarray*}%
and 
\[
90\unit{%
%TCIMACRO{\U{b0}}%
%BeginExpansion
{{}^\circ}%
%EndExpansion
}=\theta _{1}+\theta _{4} 
\]%
then the total angular change is 
\[
90\unit{%
%TCIMACRO{\U{b0}}%
%BeginExpansion
{{}^\circ}%
%EndExpansion
}+90\unit{%
%TCIMACRO{\U{b0}}%
%BeginExpansion
{{}^\circ}%
%EndExpansion
}=180\unit{%
%TCIMACRO{\U{b0}}%
%BeginExpansion
{{}^\circ}%
%EndExpansion
} 
\]%
or the outgoing ray is sent back toward the source! If we do this in three
dimensions we have a corner cube. Here is a picture of a radar corner cube
array.

\FRAME{dhFU}{2.1926in}{2.7913in}{0pt}{\Qcb{Radar retroreflector tower
located in the center of Yucca Flat dry lake bed. Used as a radar target by
maneuvering aircraft during "inert" contact fusing bomb drops at Yucca Flat.
Sandia National Laboratories conducted the tests on the lake bed from 1954
to 1956. (Image in the Public Domain in the United States)}}{}{Figure}{%
\special{language "Scientific Word";type "GRAPHIC";maintain-aspect-ratio
TRUE;display "USEDEF";valid_file "T";width 2.1926in;height 2.7913in;depth
0pt;original-width 2.1543in;original-height 2.7484in;cropleft "0";croptop
"1";cropright "1";cropbottom "0";tempfilename
'PQXXQYUE.wmf';tempfile-properties "XPR";}}And we left visible corner cube
arrays on the Moon so we can bounce laser beems off of them and monigor the
Earth-Moon distance.

\FRAME{dhFU}{2.2219in}{2.7638in}{0pt}{\Qcb{Apollo Retroreflector (Images in
the Public Domain courtesy NASA)}}{}{Figure}{\special{language "Scientific
Word";type "GRAPHIC";maintain-aspect-ratio TRUE;display "USEDEF";valid_file
"T";width 2.2219in;height 2.7638in;depth 0pt;original-width
2.1819in;original-height 2.7207in;cropleft "0";croptop "1";cropright
"1";cropbottom "0";tempfilename 'PQXXQYUF.wmf';tempfile-properties "XPR";}}%
%TCIMACRO{%
%\TeXButton{Question 223.13.9}{\marginpar {
%\hspace{-0.5in}
%\begin{minipage}[t]{1in}
%\small{Question 223.13.9}
%\end{minipage}
%}}}%
%BeginExpansion
\marginpar {
\hspace{-0.5in}
\begin{minipage}[t]{1in}
\small{Question 223.13.9}
\end{minipage}
}%
%EndExpansion

\section{Reflections, Objects, and seeing}

Armed with the law of reflection, we can start to understand how we see
things. Using the ray concept, we can say that a ray of light must leave the
light source. That ray then reflects from something. Suppose you look at the
person sitting next to you in class. We should wonder, how is it that we can
see them? We can only detect (see) light that gets to our eyes. Let's trace
the light from it's source (the light fixture in our room) to our eyes to
see how we see our neighbors.

Light comes from the ceiling lights and goes in all directions. Some of that
light hits our neighbor. And some of that light that hits our neighbor will
reflect. But is the person a specular or diffuse reflector? Once again, we
can only give an answer relative to the wavelength of light. For visible
light, your neighbors do not look like mirrors. The are diffuse reflectors.
Light bounces off of them in every direction. Some of that light that
reflects from your neighbor reflects in the direction of your eyes. That
light can be detected. Your eye is designed to take this diverging set of
rays and condense it into a picture of the person that your brain can
interpret.\FRAME{dhF}{3.173in}{1.3534in}{0pt}{}{}{Figure}{\special{language
"Scientific Word";type "GRAPHIC";maintain-aspect-ratio TRUE;display
"USEDEF";valid_file "T";width 3.173in;height 1.3534in;depth
0pt;original-width 3.128in;original-height 1.3188in;cropleft "0";croptop
"1";cropright "1";cropbottom "0";tempfilename
'PQXXQYUG.wmf';tempfile-properties "XPR";}}We tend to not draw the rays that
bounce off the diffuse reflector but that don't get to our eyes, because we
don't see them. So a ray diagram is usually much simpler.\FRAME{dhF}{2.2926in%
}{1.9951in}{0pt}{}{}{Figure}{\special{language "Scientific Word";type
"GRAPHIC";maintain-aspect-ratio TRUE;display "USEDEF";valid_file "T";width
2.2926in;height 1.9951in;depth 0pt;original-width 2.2528in;original-height
1.9562in;cropleft "0";croptop "1";cropright "1";cropbottom "0";tempfilename
'PQXXQYUH.wmf';tempfile-properties "XPR";}}This is easy to understand, but
we must keep in mind the wildly fluctuating waviness that is masked by our
macroscopic view.%
%TCIMACRO{%
%\TeXButton{Question 223.13.10}{\marginpar {
%\hspace{-0.5in}
%\begin{minipage}[t]{1in}
%\small{Question 223.13.10}
%\end{minipage}
%}}}%
%BeginExpansion
\marginpar {
\hspace{-0.5in}
\begin{minipage}[t]{1in}
\small{Question 223.13.10}
\end{minipage}
}%
%EndExpansion

We can use the idea of a ray diagram to solve problems. Suppose you hold a
mirror half a meter in front of you and look at your reflection. Where would
the reflection appear to be?\FRAME{dhF}{1.9847in}{1.9527in}{0pt}{}{}{Figure}{%
\special{language "Scientific Word";type "GRAPHIC";maintain-aspect-ratio
TRUE;display "USEDEF";valid_file "T";width 1.9847in;height 1.9527in;depth
0pt;original-width 1.9458in;original-height 1.9147in;cropleft "0";croptop
"1";cropright "1";cropbottom "0";tempfilename
'PQXXQYUI.wmf';tempfile-properties "XPR";}}Knowing that rays travel in
straight lines and that our mind interprets rays as going in straight lines,
then we can use rays to see where the light appears to be from. The image is
half a meter behind the mirror. Now suppose we look at an image of that
image in a mirror behind us. \FRAME{dhF}{2.6273in}{1.004in}{0pt}{}{}{Figure}{%
\special{language "Scientific Word";type "GRAPHIC";maintain-aspect-ratio
TRUE;display "USEDEF";valid_file "T";width 2.6273in;height 1.004in;depth
0pt;original-width 2.5858in;original-height 0.9712in;cropleft "0";croptop
"1";cropright "1";cropbottom "0";tempfilename
'PQXXQYUJ.wmf';tempfile-properties "XPR";}}The ray diagram makes it easy to
see that the image will appear to be $2\unit{m}$ behind the big mirror. You
might not feel that this was so easy, but you might find it is not so bad in
a problem in your near future.

\FRAME{dhF}{3.8303in}{2.8876in}{0pt}{}{}{Figure}{\special{language
"Scientific Word";type "GRAPHIC";maintain-aspect-ratio TRUE;display
"USEDEF";valid_file "T";width 3.8303in;height 2.8876in;depth
0pt;original-width 3.781in;original-height 2.8444in;cropleft "0";croptop
"1";cropright "1";cropbottom "0";tempfilename
'PQXXQYUK.wmf';tempfile-properties "XPR";}}

\section{Refraction}

Not all surfaces reflect all the light. Some, like the lenses shown below,
reflect some light at visible wavelengths, but are transparent so most of
the light travels through them. \FRAME{dhF}{3.2846in}{1.6042in}{0pt}{}{}{%
Figure}{\special{language "Scientific Word";type
"GRAPHIC";maintain-aspect-ratio TRUE;display "USEDEF";valid_file "T";width
3.2846in;height 1.6042in;depth 0pt;original-width 3.2396in;original-height
1.5679in;cropleft "0";croptop "1";cropright "1";cropbottom "0";tempfilename
'PQXXQYUL.wmf';tempfile-properties "XPR";}}We need a way to deal with
transparent materials. This is tricky, because different wavelengths of
light penetrate different materials in different ways. As an example, this
is also a lens\FRAME{dhFU}{1.0732in}{1.6285in}{0pt}{\Qcb{{\protect\small %
Infrared lens, but visible mirror (Image courtesy US Navy, image in the
public domain)}}}{}{Figure}{\special{language "Scientific Word";type
"GRAPHIC";maintain-aspect-ratio TRUE;display "USEDEF";valid_file "T";width
1.0732in;height 1.6285in;depth 0pt;original-width 5.4587in;original-height
8.3195in;cropleft "0";croptop "1";cropright "1";cropbottom "0";tempfilename
'PQXXQYUM.bmp';tempfile-properties "XPR";}}but it clearly is not transparent
at visible wavelengths. It is transparent in the infrared. So what might be
transparent at one wavelength might not be at another.

When light travels into a material, we say it is transmitted. The situation
is shown schematically below.

\FRAME{dhF}{2.348in}{1.9666in}{0in}{}{}{Figure}{\special{language
"Scientific Word";type "GRAPHIC";maintain-aspect-ratio TRUE;display
"USEDEF";valid_file "T";width 2.348in;height 1.9666in;depth
0in;original-width 2.3082in;original-height 1.9285in;cropleft "0";croptop
"1";cropright "1";cropbottom "0";tempfilename
'PQXXQYUN.wmf';tempfile-properties "XPR";}}

In the figure we see a ray incident on an air-glass boundary. Some of the
light is reflected just as we saw before. But some passes into the glass.
Notice that the angle between the normal and the new transmitted ray is 
\emph{not} equal to the incident ray. We say the ray has been bent or \emph{%
refracted} by the change in material. Many experiments were performed to
find a relationship between the incident and the refracted angles. It was
found that 
\begin{equation}
\frac{\sin \left( \theta _{2}\right) }{\sin \left( \theta _{1}\right) }=%
\frac{v_{2}}{v_{1}}=\text{constant}
\end{equation}%
Many optics books write this as 
\begin{equation}
\frac{\sin \left( \theta _{t}\right) }{\sin \left( \theta _{i}\right) }=%
\frac{v_{2}}{v_{1}}=\text{constant}
\end{equation}%
where the subscript $i$ stands for \textquotedblleft
incident\textquotedblright\ and the subscript $t$ stands for
\textquotedblleft transmitted.\textquotedblright\ Note that we are using the
fact that the speed of light changes in a material. We should probably
recall why this should occur

\section{Speed of light in a material}

In a vacuum, light travels as a disturbance in the electromagnetic field
with nothing to encounter. In a material (like glass) the light waves
continually hit atoms. We have not studied antennas, but I\ think many of
you know that an antenna works because the electrons in the metal act like
driven harmonic oscillators. The incoming radio waves drive the electron
motion. Here each atom has electrons, and the atoms act like little
antennas, their electrons moving and absorbing the light. But the atom
cannot keep the extra energy (PH433), so it is readmitted. It travels to the
next atom and the process repeats. Quantum mechanics tells us that there is
a time delay in the re-emission of the light. This causes the propagation of
the light to slow down. Thus the speed of light is slower in a material.%
\FRAME{dhF}{4.6267in}{1.8135in}{0pt}{}{}{Figure}{\special{language
"Scientific Word";type "GRAPHIC";maintain-aspect-ratio TRUE;display
"USEDEF";valid_file "T";width 4.6267in;height 1.8135in;depth
0pt;original-width 4.574in;original-height 1.7763in;cropleft "0";croptop
"1";cropright "1";cropbottom "0";tempfilename
'PQXXQYUO.wmf';tempfile-properties "XPR";}}As a mechanical analog, consider
a rolling barrel.\FRAME{dhF}{2.4604in}{1.8005in}{0pt}{}{}{Figure}{\special%
{language "Scientific Word";type "GRAPHIC";maintain-aspect-ratio
TRUE;display "USEDEF";valid_file "T";width 2.4604in;height 1.8005in;depth
0pt;original-width 2.4189in;original-height 1.7634in;cropleft "0";croptop
"1";cropright "1";cropbottom "0";tempfilename
'PQXXQYUP.wmf';tempfile-properties "XPR";}}As the barrel rolls from a flat
low-friction concrete to a higher-friction grass lawn, the friction slows
the barrel. If the barrel hits the lawn parallel to the boundary (so it's
velocity vector is perpendicular to the boundary), then the barrel continues
in the same direction at the slower speed. But if it hits at an angle, the
leading edge is slowed first. \FRAME{dhF}{2.3575in}{1.7236in}{0pt}{}{}{Figure%
}{\special{language "Scientific Word";type "GRAPHIC";maintain-aspect-ratio
TRUE;display "USEDEF";valid_file "T";width 2.3575in;height 1.7236in;depth
0pt;original-width 2.725in;original-height 1.9847in;cropleft "0";croptop
"1";cropright "1";cropbottom "0";tempfilename
'PQXXQYUQ.wmf';tempfile-properties "XPR";}}This makes the trailing edge
travel faster than the leading edge, and the barrel turns slightly.\FRAME{dhF%
}{2.4267in}{1.7711in}{0pt}{}{}{Figure}{\special{language "Scientific
Word";type "GRAPHIC";maintain-aspect-ratio TRUE;display "USEDEF";valid_file
"T";width 2.4267in;height 1.7711in;depth 0pt;original-width
3.7948in;original-height 2.7622in;cropleft "0";croptop "1";cropright
"1";cropbottom "0";tempfilename 'PQXXQYUR.wmf';tempfile-properties "XPR";}}

We expect the same behavior from light.\FRAME{dhF}{2.8236in}{2.0237in}{0pt}{%
}{}{Figure}{\special{language "Scientific Word";type
"GRAPHIC";maintain-aspect-ratio TRUE;display "USEDEF";valid_file "T";width
2.8236in;height 2.0237in;depth 0pt;original-width 2.7804in;original-height
1.9847in;cropleft "0";croptop "1";cropright "1";cropbottom "0";tempfilename
'PQXXQYUS.wmf';tempfile-properties "XPR";}}We can see that the left hand
side of the wave hits the slower (green) material first and slows down. The
rest of the wave front moves quicker. The result is the turning of the wave.%
\footnote{%
Once again this is a bit of a simplification, but it will due for now. If
you are lucky enough to take a junior level optics class, you will revisit
this.}

%TCIMACRO{%
%\TeXButton{Question 223.14.1}{\marginpar {
%\hspace{-0.5in}
%\begin{minipage}[t]{1in}
%\small{Question 223.14.1}
%\end{minipage}
%}}}%
%BeginExpansion
\marginpar {
\hspace{-0.5in}
\begin{minipage}[t]{1in}
\small{Question 223.14.1}
\end{minipage}
}%
%EndExpansion

\section{Change of wavelength}

We have found that when a wave enters a material, its speed may change. But
we remember from wave theory%
\begin{equation}
v=\lambda f
\end{equation}

But it is time to review: does $\lambda $ change, or does $f$ change? If you
will recall, we found that the change in speed at the boundary changes the
wavelength. Recall that if we go from a fast material to a slow material,
the forward part of the wave slows and the rest of the wave catches up to
it. \FRAME{dhF}{2.2727in}{1.2644in}{0pt}{}{}{Figure}{\special{language
"Scientific Word";type "GRAPHIC";maintain-aspect-ratio TRUE;display
"USEDEF";valid_file "T";width 2.2727in;height 1.2644in;depth
0pt;original-width 3.4618in;original-height 1.9147in;cropleft "0";croptop
"1";cropright "1";cropbottom "0";tempfilename
'PQXXQYUT.wmf';tempfile-properties "XPR";}}This will comperes pulses, and
lower the wavelength. Now that we know more about light we can also argue
that $f$ cannot change because 
\[
E=hf 
\]%
If $f$ changed, then we would either require an input of energy or we would
store energy at the boundary because%
\[
\Delta f=\frac{\Delta E}{h} 
\]%
This can't be true. If the wavelength changes, there is no such change in
energy.

Since 
\[
v_{1}=\lambda _{1}f 
\]%
and 
\[
v_{2}=\lambda _{2}f 
\]%
then the ratio%
\[
\frac{v_{1}}{v_{2}}=\frac{\lambda _{1}}{\lambda _{2}} 
\]%
and we again have our solution for the wavelength in the material%
\[
\lambda _{2}=\lambda _{1}\frac{v_{2}}{v_{1}} 
\]%
which agrees with our previous analysis.

\section{Index of refraction and Snell's Law}

%TCIMACRO{%
%\TeXButton{Question 223.14.2}{\marginpar {
%\hspace{-0.5in}
%\begin{minipage}[t]{1in}
%\small{Question 223.14.2}
%\end{minipage}
%}}}%
%BeginExpansion
\marginpar {
\hspace{-0.5in}
\begin{minipage}[t]{1in}
\small{Question 223.14.2}
\end{minipage}
}%
%EndExpansion
%TCIMACRO{%
%\TeXButton{Question 223.14.3}{\marginpar {
%\hspace{-0.5in}
%\begin{minipage}[t]{1in}
%\small{Question 223.14.3}
%\end{minipage}
%}}}%
%BeginExpansion
\marginpar {
\hspace{-0.5in}
\begin{minipage}[t]{1in}
\small{Question 223.14.3}
\end{minipage}
}%
%EndExpansion

Because the equation%
\begin{equation}
\frac{\sin \left( \theta _{1}\right) }{\sin \left( \theta _{2}\right) }=%
\frac{v_{2}}{v_{1}}=\text{constant}
\end{equation}%
has a constant ratio of velocities, it is convenient to define a term that
represents that ratio. We already have a concept that can help. The \emph{%
index of refraction} is just such a term. It assumes that one speed is the
speed of light in vacuum, $c.$%
\begin{equation}
n\equiv \frac{c}{v}
\end{equation}%
Then for our example 
\begin{equation}
\frac{\sin \left( \theta _{1}\right) }{\sin \left( \theta _{2}\right) }=%
\frac{1}{n}
\end{equation}

Suppose we don't have a vacuum (or air that is close to a vacuum). We can
write our formula as 
\begin{equation}
n_{1}\sin \left( \theta _{1}\right) =n_{2}\sin \left( \theta _{2}\right)
\end{equation}%
where we have determined 
\begin{equation}
n_{1}\equiv \frac{c}{v_{1}}
\end{equation}%
and 
\begin{equation}
n_{2}\equiv \frac{c}{v_{2}}
\end{equation}

This is called \emph{Snell's law of refraction}.

Using the index of refraction we can write our equation relating the ratio
of velocities and wavelengths as%
\begin{equation}
\frac{v_{1}}{v_{2}}=\frac{\lambda _{1}}{\lambda _{2}}=\frac{\frac{2}{n_{1}}}{%
\frac{c}{n_{2}}}=\frac{n_{2}}{n_{1}}
\end{equation}%
which gives 
\begin{equation}
\lambda _{1}n_{1}=\lambda _{2}n_{2}
\end{equation}%
and if we have vacuum and a single material we can find the index of
refraction from 
\begin{equation}
n=\frac{\lambda }{\lambda _{in}}
\end{equation}%
where $\lambda _{in}$ is the wavelength in the material.%
%TCIMACRO{%
%\TeXButton{Question 223.14.4}{\marginpar {
%\hspace{-0.5in}
%\begin{minipage}[t]{1in}
%\small{Question 223.14.4}
%\end{minipage}
%}}}%
%BeginExpansion
\marginpar {
\hspace{-0.5in}
\begin{minipage}[t]{1in}
\small{Question 223.14.4}
\end{minipage}
}%
%EndExpansion
%TCIMACRO{%
%\TeXButton{Question 223.14.5}{\marginpar {
%\hspace{-0.5in}
%\begin{minipage}[t]{1in}
%\small{Question 223.14.5}
%\end{minipage}
%}}}%
%BeginExpansion
\marginpar {
\hspace{-0.5in}
\begin{minipage}[t]{1in}
\small{Question 223.14.5}
\end{minipage}
}%
%EndExpansion

\section{Total Internal Reflection}

%TCIMACRO{%
%\TeXButton{Question 223.14.6}{\marginpar {
%\hspace{-0.5in}
%\begin{minipage}[t]{1in}
%\small{Question 223.14.6}
%\end{minipage}
%}}}%
%BeginExpansion
\marginpar {
\hspace{-0.5in}
\begin{minipage}[t]{1in}
\small{Question 223.14.6}
\end{minipage}
}%
%EndExpansion
Up to now we have assumed that light was coming from a region of low index
of refraction into a region of high index of refraction. We should pause to
look at what can happen if we go the other way.\FRAME{dhF}{3.0744in}{2.5815in%
}{0pt}{}{}{Figure}{\special{language "Scientific Word";type
"GRAPHIC";maintain-aspect-ratio TRUE;display "USEDEF";valid_file "T";width
3.0744in;height 2.5815in;depth 0pt;original-width 3.0303in;original-height
2.5399in;cropleft "0";croptop "1";cropright "1";cropbottom "0";tempfilename
'PQXXQYUU.wmf';tempfile-properties "XPR";}}

We start with Snell's law%
\begin{equation}
n_{1}\sin \theta _{1}=n_{2}\sin \theta _{2}
\end{equation}%
but this time $n=n_{1}$ and $n_{2}=1$ so 
\begin{equation}
n\sin \theta _{1}=\sin \theta _{2}
\end{equation}%
which gives 
\begin{equation}
\theta _{2}=\sin ^{-1}\left( n\sin \theta _{1}\right)
\end{equation}%
If we take $n=1.33$ (water) we can plot this expression as a function of $%
\theta _{1}$

\FRAME{dtbpF}{3.0312in}{2.0211in}{0pt}{}{}{Plot}{\special{language
"Scientific Word";type "MAPLEPLOT";width 3.0312in;height 2.0211in;depth
0pt;display "USEDEF";plot_snapshots TRUE;mustRecompute FALSE;lastEngine
"MuPAD";xmin "0";xmax "1.2";xviewmin "0";xviewmax "1.2";yviewmin
"0";yviewmax "2";viewset"XY";rangeset"X";plottype 4;labeloverrides 3;x-label
"Theta 2";y-label "Theta 1";axesFont "Times New
Roman,12,0000000000,useDefault,normal";numpoints 100;plotstyle
"patch";axesstyle "normal";axestips FALSE;xis \TEXUX{x};var1name
\TEXUX{$x$};function \TEXUX{$\sin ^{-1}\left( 1.33\sin x\right) $};linecolor
"blue";linestyle 1;pointstyle "point";linethickness 1;lineAttributes
"Solid";var1range "0,1.2";num-x-gridlines 100;curveColor
"[flat::RGB:0x000000ff]";curveStyle "Line";VCamFile
'PRQ7NS06.xvz';valid_file "T";tempfilename
'PQXXQYUV.wmf';tempfile-properties "XPR";}}we see that at $\theta
_{1}=\allowbreak 0.850\,91\unit{rad}$ $(48.\,\allowbreak 754\unit{%
%TCIMACRO{\U{b0}}%
%BeginExpansion
{{}^\circ}%
%EndExpansion
})$ the curve becomes infinitely steep. If we use this value in our equation
this gives%
\begin{eqnarray}
\theta _{2} &=&\sin ^{-1}\left( n\sin \left( \allowbreak 0.850\,91\right)
\right) \\
&=&1.\,\allowbreak 570\,8\unit{rad} \\
&=&90\unit{%
%TCIMACRO{\U{b0}}%
%BeginExpansion
{{}^\circ}%
%EndExpansion
}
\end{eqnarray}%
The light skims along the edge of the water! \FRAME{dhF}{2.3756in}{1.9251in}{%
0pt}{}{}{Figure}{\special{language "Scientific Word";type
"GRAPHIC";maintain-aspect-ratio TRUE;display "USEDEF";valid_file "T";width
2.3756in;height 1.9251in;depth 0pt;original-width 2.3359in;original-height
1.887in;cropleft "0";croptop "1";cropright "1";cropbottom "0";tempfilename
'PQXXQYUW.wmf';tempfile-properties "XPR";}}We can find the value of $\theta
_{1}$ that makes this happen without graphing. Set $\theta _{2}=90\unit{%
%TCIMACRO{\U{b0}}%
%BeginExpansion
{{}^\circ}%
%EndExpansion
}$ then%
\begin{equation}
\theta _{1}=\theta _{c}\equiv \sin ^{-1}\left( \frac{1}{n}\right)
\end{equation}%
We give this value of $\theta _{1}$ a special name. It is the \emph{critical
angle} for internal reflection. But what happens if we go farther than this (%
$\theta _{1}>\theta _{c}$). We will no longer have a transmitted ray. The
ray will be reflected. This is why when you dive into a pool and look up,
you see a region of the roof of the pool area (or sky) but off to the side
of the pool the surface looks mirrored. It is also why you sometimes see the
sides of a fish tank appear to be mirrored when you look through the front. 
\FRAME{dhF}{1.7616in}{1.3258in}{0pt}{}{}{Figure}{\special{language
"Scientific Word";type "GRAPHIC";maintain-aspect-ratio TRUE;display
"USEDEF";valid_file "T";width 1.7616in;height 1.3258in;depth
0pt;original-width 1.7236in;original-height 1.2912in;cropleft "0";croptop
"1";cropright "1";cropbottom "0";tempfilename
'PQXXQYUX.wmf';tempfile-properties "XPR";}}More importantly, it is why cut
gems (like diamonds) sparkle. They capture the light with facets that are
cut at angles that create total internal reflection. The light that enters
the gem comes back out the front (We will study how to make the pretty
colored sparkles next time).%
%TCIMACRO{%
%\TeXButton{Question 223.14.7}{\marginpar {
%\hspace{-0.5in}
%\begin{minipage}[t]{1in}
%\small{Question 223.14.7}
%\end{minipage}
%}}}%
%BeginExpansion
\marginpar {
\hspace{-0.5in}
\begin{minipage}[t]{1in}
\small{Question 223.14.7}
\end{minipage}
}%
%EndExpansion
%TCIMACRO{%
%\TeXButton{Question 223.14.8}{\marginpar {
%\hspace{-0.5in}
%\begin{minipage}[t]{1in}
%\small{Question 223.14.8}
%\end{minipage}
%}}}%
%BeginExpansion
\marginpar {
\hspace{-0.5in}
\begin{minipage}[t]{1in}
\small{Question 223.14.8}
\end{minipage}
}%
%EndExpansion
%TCIMACRO{%
%\TeXButton{Question 223.14.9}{\marginpar {
%\hspace{-0.5in}
%\begin{minipage}[t]{1in}
%\small{Question 223.14.9}
%\end{minipage}
%}}}%
%BeginExpansion
\marginpar {
\hspace{-0.5in}
\begin{minipage}[t]{1in}
\small{Question 223.14.9}
\end{minipage}
}%
%EndExpansion

\section{Fiber Optics}

Beyond pretty pebbles, this effect is very useful! It is the heart and soul
of fiber optics. \FRAME{dtbpF}{3.1993in}{1.7882in}{0in}{}{}{Figure}{\special%
{language "Scientific Word";type "GRAPHIC";maintain-aspect-ratio
TRUE;display "USEDEF";valid_file "T";width 3.1993in;height 1.7882in;depth
0in;original-width 3.233in;original-height 1.7935in;cropleft "0";croptop
"1";cropright "1";cropbottom "0";tempfilename
'PQXXQYUY.wmf';tempfile-properties "XPR";}}

An interior material with a lower index of refraction is inclosed in a
cladding with a higher index. This creates a light pipe that traps the light
in the fiber. \FRAME{dhF}{3.4662in}{1.535in}{0pt}{}{}{Figure}{\special%
{language "Scientific Word";type "GRAPHIC";maintain-aspect-ratio
TRUE;display "USEDEF";valid_file "T";width 3.4662in;height 1.535in;depth
0pt;original-width 3.4203in;original-height 1.4987in;cropleft "0";croptop
"1";cropright "1";cropbottom "0";tempfilename
'PQXXQYUZ.wmf';tempfile-properties "XPR";}}

Modern fibers don't always have a hard boundary. The fibers have a gradual
change in index of refraction that changes the direction of the light
gradually. This keeps the light in the fiber but tends to direct along the
fiber so the beam is not crisscrossing as it goes.

The cutting edge of fiber design today uses hollow fibers or fibers filled
with different index material.

\FRAME{dtbpFU}{1.6319in}{1.5973in}{0pt}{\Qcb{{\protect\small Hollow-Core
Fiber (Courtesy Defense Advanced Research Projects Agency (DARPA), image in
the public domain)}}}{}{Figure}{\special{language "Scientific Word";type
"GRAPHIC";maintain-aspect-ratio TRUE;display "USEDEF";valid_file "T";width
1.6319in;height 1.5973in;depth 0pt;original-width 1.6338in;original-height
1.5992in;cropleft "0";croptop "1";cropright "1";cropbottom "0";tempfilename
'PQXXQYV0.wmf';tempfile-properties "XPR";}}

\chapter{Images and Color}

%TCIMACRO{%
%\TeXButton{Fundamental Concepts}{\hspace{-1.3in}{\Large Fundamental Concepts\vspace{0.25in}}}}%
%BeginExpansion
\hspace{-1.3in}{\Large Fundamental Concepts\vspace{0.25in}}%
%EndExpansion

\begin{itemize}
\item An image is a diverging set of rays in a recognizable pattern

\item Images can be formed by refraction

\item The index of refraction is wavelength dependent

\item That different wavelengths bend different amounts when refracted is
called \emph{dispersion}

\item White light is a superposition of many other frequencies
\end{itemize}

Let's think about what an image is.%
%TCIMACRO{%
%\TeXButton{Make Images with Lens Demo}{\marginpar {
%\hspace{-0.5in}
%\begin{minipage}[t]{1in}
%\small{Make Images with Lens Demo}
%\end{minipage}
%}} }%
%BeginExpansion
\marginpar {
\hspace{-0.5in}
\begin{minipage}[t]{1in}
\small{Make Images with Lens Demo}
\end{minipage}
}
%EndExpansion
Take a piece of paper and a lens, and hold up the lens in a darkened room
that has some bright object in it. Move the lens or the paper back and
forth, and at just the right distance, a miniature picture of the bright
object will appear. We should think about what the word \textquotedblleft
picture\textquotedblright\ means in this sense.

\FRAME{dhF}{2.6835in}{2.0237in}{0pt}{}{}{Figure}{\special{language
"Scientific Word";type "GRAPHIC";maintain-aspect-ratio TRUE;display
"USEDEF";valid_file "T";width 2.6835in;height 2.0237in;depth
0pt;original-width 2.6411in;original-height 1.9847in;cropleft "0";croptop
"1";cropright "1";cropbottom "0";tempfilename
'PQXXQYV1.wmf';tempfile-properties "XPR";}}We have talked about how we see
objects. Remember the BYU-I guys from last time.\FRAME{dhF}{3.173in}{1.3534in%
}{0pt}{}{}{Figure}{\special{language "Scientific Word";type
"GRAPHIC";maintain-aspect-ratio TRUE;display "USEDEF";valid_file "T";width
3.173in;height 1.3534in;depth 0pt;original-width 3.128in;original-height
1.3188in;cropleft "0";croptop "1";cropright "1";cropbottom "0";tempfilename
'PQXXQYV2.wmf';tempfile-properties "XPR";}}Our eyes gather rays that are
diverging from the object because light has bounced off of the object. Our
eyes intersect a diverging set of rays that form a definite pattern. That
diverging set of rays forming a pattern is the picture of the object.

So when we say that the lens has formed a miniature picture of our object,
we mean that the lens has somehow formed a diverging set of rays that form a
pattern that looks like the pattern formed by the diverging set of rays
coming from the object, itself. In other words, the object forms a diverging
set of rays. And our lens forms a duplicate set of rays in the same pattern,
so we see the same thing. The lens' version is smaller, upside down, and
backwards, but it is still essentially the same pattern.

As a first step to see how this works, consider our fish tank again. It
would be bad on the fish, but think about looking at a fish in air. The room
light would bounce off of the fish, and we would have a diverging set of
rays from every point on the fish (see next figure). \FRAME{dhF}{4.4235in}{%
2.2139in}{0pt}{}{}{Figure}{\special{language "Scientific Word";type
"GRAPHIC";maintain-aspect-ratio TRUE;display "USEDEF";valid_file "T";width
4.4235in;height 2.2139in;depth 0pt;original-width 5.4916in;original-height
2.7337in;cropleft "0";croptop "1";cropright "1";cropbottom "0";tempfilename
'PQXXQYV3.wmf';tempfile-properties "XPR";}}We can see that the picture is
made from every point on the fish being \textquotedblleft
imaged\textquotedblright\ to a point on our retina. We collect the rays
leaving every point on the fish, and bring them to corresponding points on
the retina to make the picture.

It will take us a few lectures to see exactly how this is done, but as a
first step, let's consider the fish tank, itself. Put the fish back in the
tank and look at it (next figure).

\FRAME{dhF}{3.6763in}{2.3445in}{0pt}{}{}{Figure}{\special{language
"Scientific Word";type "GRAPHIC";maintain-aspect-ratio TRUE;display
"USEDEF";valid_file "T";width 3.6763in;height 2.3445in;depth
0pt;original-width 3.6288in;original-height 2.3039in;cropleft "0";croptop
"1";cropright "1";cropbottom "0";tempfilename
'PQXXQYV4.wmf';tempfile-properties "XPR";}}Rays still come from the fish.
But we now know that the change from a slow material to a fast material will
bend the light. These bent rays are collected by our eyes, and the picture
of the fish is formed on the retina just as before. But our eyes interpret
the light as though it went in straight lines with no bends (dotted lines in
the last figure). our mind is designed to believe light travels in straight
lines, so our mind tells us there is a fish, but that the fish head (and
every other part of the fish) is closer than it really is. We call this
apparent fish at the closer location an image of the fish, because this is
where we think the diverging set of rays come from that form the fish
pattern.

The next figure shows the details of the rays leaving a dot on the fish head
(with the angles exagerated so it's easier to see them). \FRAME{dhF}{3.5639in%
}{3.3486in}{0pt}{}{}{Figure}{\special{language "Scientific Word";type
"GRAPHIC";maintain-aspect-ratio TRUE;display "USEDEF";valid_file "T";width
3.5639in;height 3.3486in;depth 0pt;original-width 3.5172in;original-height
3.3027in;cropleft "0";croptop "1";cropright "1";cropbottom "0";tempfilename
'PQXXQYV5.wmf';tempfile-properties "XPR";}}A spot on the fish head is our
object for this set of rays. The distance from the fish-head dot and the
edge of the water/air boundary is called the \emph{object distance} and is
given the symbol $s.$

The distance from the image of the fish-head dot to the edge of the
water/air boundary is called the \emph{image distance} and is given the
symbol $s^{\prime }.$ Note that this is not a derivative, it is just a
distance like $s,$ because it appears to be where the rays come from, but it
is a different distance because of the refraction of the rays, so to make it
look difference we put a prime mark on it.

We can find where the image is $\left( s^{\prime }\right) $ knowing $s.$ We
can see from the figure that%
\begin{eqnarray*}
\ell &=&s\tan \theta _{1} \\
&=&s^{\prime }\tan \theta _{2}
\end{eqnarray*}%
so%
\[
s\tan \theta _{1}=s^{\prime }\tan \theta _{2} 
\]%
or%
\[
s\frac{\tan \theta _{1}}{\tan \theta _{2}}=s^{\prime } 
\]%
from Snell's law, we know that 
\[
\frac{\sin \theta _{1}}{\sin \theta _{2}}=\frac{n_{2}}{n_{1}} 
\]

Usually we can take the small angle approximation. This would limit our
analysis to rays that are near the central axes. Let's call this central
axis the \emph{optics axis} and the rays that are not too far away from this
axis \emph{paraxial rays.} Then for our small angles we can write 
\[
\tan \theta _{i}\approx \sin \theta _{i} 
\]%
so%
\[
s\frac{\tan \theta _{1}}{\tan \theta _{2}}\approx s\frac{\sin \theta _{1}}{%
\sin \theta _{2}}=s\frac{n_{2}}{n_{1}}=s^{\prime } 
\]%
and we have the image distance

\[
s^{\prime }=s\frac{n_{2}}{n_{1}} 
\]

This is not so useful unless you have some burning need to know where your
fish are in a tank. But we now have the vocabulary to discuss the larger
problem of how a lens works, which we will take up next time.

\section{Dispersion}

\FRAME{dhF}{4.3613in}{2.2745in}{0pt}{}{}{Figure}{\special{language
"Scientific Word";type "GRAPHIC";maintain-aspect-ratio TRUE;display
"USEDEF";valid_file "T";width 4.3613in;height 2.2745in;depth
0pt;original-width 4.3102in;original-height 2.2347in;cropleft "0";croptop
"1";cropright "1";cropbottom "0";tempfilename
'PQXXQYV6.wmf';tempfile-properties "XPR";}}%
%TCIMACRO{%
%\TeXButton{Question 123.15.1}{\marginpar {
%\hspace{-0.5in}
%\begin{minipage}[t]{1in}
%\small{Question 123.15.1}
%\end{minipage}
%}}}%
%BeginExpansion
\marginpar {
\hspace{-0.5in}
\begin{minipage}[t]{1in}
\small{Question 123.15.1}
\end{minipage}
}%
%EndExpansion

Who hasn't played with a prism? We immediately recognize a rainbow. But why
does the prism make a rainbow? The secret lies in the nature of the
refractive index.\FRAME{fhFU}{4.3533in}{2.6201in}{0pt}{\Qcb{Index of
refraction as a function of wavelength ( Ohara optical glass
http://www.oharacorp.com/fused-silica-quartz.html data and Schott optical
glass data http://www.uqgoptics.com/materials\_glasses\_schott.aspx)}}{\Qlb{%
Dispersion of Glass}}{Figure}{\special{language "Scientific Word";type
"GRAPHIC";maintain-aspect-ratio TRUE;display "USEDEF";valid_file "T";width
4.3533in;height 2.6201in;depth 0pt;original-width 5.1436in;original-height
3.084in;cropleft "0";croptop "1";cropright "1";cropbottom "0";tempfilename
'dispersion0.wmf';tempfile-properties "XNPR";}}

Notice that in the figure, the index of refraction depends on wavelength.
This means that as light enters a material, different wavelengths will be
refracted at different angles. White light is a superposition of many
wavelengths of light. Thus white light is pulled apart by refraction into a
rainbow. This process is called \emph{dispersion}.

The graph tells us that blue light bends more than red light. \FRAME{dhF}{%
2.4379in}{1.2816in}{0pt}{}{}{Figure}{\special{language "Scientific
Word";type "GRAPHIC";maintain-aspect-ratio TRUE;display "USEDEF";valid_file
"T";width 2.4379in;height 1.2816in;depth 0pt;original-width
6.5207in;original-height 3.4143in;cropleft "0";croptop "1";cropright
"1";cropbottom "0";tempfilename 'PQXXQYV7.wmf';tempfile-properties "XPR";}}

We call the change in direction measured from the original direction of
travel, $\delta ,$ the \emph{angle of deviation}. The colors we can see are
called the visible spectrum. Note that this is a second way to make a
visible light spectrum. The first way used a diffraction grating and the
wave nature of light. Although this dispersive element (prism) method of
making a spectrum uses the ides of geometric optics, it is still the wave
nature of light that makes the waves of different wavelengths refract
differently.

%TCIMACRO{%
%\TeXButton{Question 123.15.1}{\marginpar {
%\hspace{-0.5in}
%\begin{minipage}[t]{1in}
%\small{Question 123.15.1}
%\end{minipage}
%}}}%
%BeginExpansion
\marginpar {
\hspace{-0.5in}
\begin{minipage}[t]{1in}
\small{Question 123.15.1}
\end{minipage}
}%
%EndExpansion
%TCIMACRO{%
%\TeXButton{Question 123.15.2}{\marginpar {
%\hspace{-0.5in}
%\begin{minipage}[t]{1in}
%\small{Question 123.15.2}
%\end{minipage}
%}}}%
%BeginExpansion
\marginpar {
\hspace{-0.5in}
\begin{minipage}[t]{1in}
\small{Question 123.15.2}
\end{minipage}
}%
%EndExpansion

Let's look at a natural rainbow. The dispersion is caused buy small droplets
of water. The white sunlight enters the drop and is dispersed. It bounces
off the back of the drop and then leaves the drop, again being dispersed.
Red light leaves the drop at about $42\unit{%
%TCIMACRO{\U{b0}}%
%BeginExpansion
{{}^\circ}%
%EndExpansion
}$ from its input direction, and blue light leaves at about $40\unit{%
%TCIMACRO{\U{b0}}%
%BeginExpansion
{{}^\circ}%
%EndExpansion
}.$\FRAME{dhF}{3.1964in}{2.3039in}{0pt}{}{}{Figure}{\special{language
"Scientific Word";type "GRAPHIC";maintain-aspect-ratio TRUE;display
"USEDEF";valid_file "T";width 3.1964in;height 2.3039in;depth
0pt;original-width 9.1488in;original-height 6.5778in;cropleft "0";croptop
"1";cropright "1";cropbottom "0";tempfilename
'PQXXQYV8.wmf';tempfile-properties "XPR";}}This effectively separates all
the different wavelengths and we see a rainbow at angles between $40\unit{%
%TCIMACRO{\U{b0}}%
%BeginExpansion
{{}^\circ}%
%EndExpansion
}$ and $42\unit{%
%TCIMACRO{\U{b0}}%
%BeginExpansion
{{}^\circ}%
%EndExpansion
}$ from the incoming light direction.

%TCIMACRO{%
%\TeXButton{Question 123.15.3}{\marginpar {
%\hspace{-0.5in}
%\begin{minipage}[t]{1in}
%\small{Question 123.15.3}
%\end{minipage}
%}}}%
%BeginExpansion
\marginpar {
\hspace{-0.5in}
\begin{minipage}[t]{1in}
\small{Question 123.15.3}
\end{minipage}
}%
%EndExpansion

\subsection{Calculation of n using a prism}

Let's do a problem using the idea of refraction in a prism. Let's find the
index of refraction of a the prism material. Suppose we make a prism as
shown. We know the angle $\Phi $ and can measure the exit angle $\delta .$
In terms of these two variables, what is $n?$

\FRAME{dhF}{2.0539in}{1.433in}{0pt}{}{}{Figure}{\special{language
"Scientific Word";type "GRAPHIC";maintain-aspect-ratio TRUE;display
"USEDEF";valid_file "T";width 2.0539in;height 1.433in;depth
0pt;original-width 5.1024in;original-height 3.5526in;cropleft "0";croptop
"1";cropright "1";cropbottom "0";tempfilename
'PQXXQYV9.wmf';tempfile-properties "XPR";}}

Using the notation indicated in the figure, we choose $\theta _{1}$ such
that the interior ray is horizontal.\footnote{%
WARNING! in the upcoming homework problem you can't make the same
assumptions!} This is a refraction problem, so we will want to use Snell's
law. 
\[
n_{1}\sin \theta _{1}=n_{2}\sin \theta _{2} 
\]%
Thus, we need to find $\theta _{1}$ and $\theta _{2}.$ Knowing $\Phi $ and $%
\delta ,$ and realizing $n_{1}=1,$ we can find $\theta _{2}$ and $\theta
_{1}.$ Then use 
\[
n_{1}\sin \theta _{1}=n_{2}\sin \theta _{2} 
\]%
to find $n_{2}.$ Let's see one way to do this. In geometry it is fair game
to extend lines and even make some new lines of our own. By doing this we
can realize that

\begin{equation}
\theta _{1}=\theta _{2}+\alpha
\end{equation}%
and that 
\begin{equation}
\delta =180-\beta
\end{equation}%
and it is also true that

\begin{equation}
180=\beta +2\alpha
\end{equation}

Then

\begin{equation}
\delta =2\alpha
\end{equation}%
and%
\begin{equation}
\alpha =\frac{\delta }{2}
\end{equation}

Now also realize that

\begin{equation}
90=\alpha +\theta _{2}+\phi
\end{equation}

and

\begin{equation}
180=\Phi +2\alpha +2\phi
\end{equation}%
or%
\begin{equation}
90=\frac{\Phi }{2}+\alpha +\phi
\end{equation}%
Then%
\begin{eqnarray}
\alpha +\theta _{2}+\phi &=&\frac{\Phi }{2}+\alpha +\phi \\
\theta _{2} &=&\frac{\Phi }{2}
\end{eqnarray}%
We can put these in our equation for $\theta _{1}$%
\begin{eqnarray}
\theta _{1} &=&\theta _{2}+\alpha \\
&=&\frac{\Phi }{2}+\frac{\delta }{2} \\
&=&\frac{\Phi +\delta }{2}
\end{eqnarray}%
Now we can use Snell's Law%
\begin{eqnarray}
\sin \left( \theta _{1}\right) &=&n\sin \left( \theta _{2}\right) \\
\sin \left( \frac{\Phi +\delta }{2}\right) &=&n\sin \left( \frac{\Phi }{2}%
\right)
\end{eqnarray}%
then%
\begin{equation}
n=\frac{\sin \left( \frac{\Phi }{2}\right) }{\sin \left( \frac{\Phi +\delta 
}{2}\right) }
\end{equation}%
and since we know $\Phi $ and $\delta $, we can find $n.$

\section{Filters and other color phenomena}

Of course, we have assumed without statement, that white light is made up of
all the colors of the rainbow. We should ask why a red shirt is red, or why
passing light through a green film makes the light look green as it leaves.

Both of these phenomena are examples of removing wavelengths from white
light.

In the case of the red shirt, the red dye in the cloth absorbs all of the
visible colors except red. The red is reflected, so the shirt looks red. The
filter is much the same. The green pigment in the film causes nearly all
visible colors to be absorbed except green. So only green light is
transmitted. This is why leaves are green. Chlorophyll absorbs red and blue
wavelengths, so the green is reflected or transmitted.

\FRAME{dhFU}{2.4738in}{2.19in}{0pt}{\Qcb{{\protect\small Chlorophyll
Spectrum (Public Domain image courtesy Kurzon)}}}{}{Figure}{\special%
{language "Scientific Word";type "GRAPHIC";maintain-aspect-ratio
TRUE;display "USEDEF";valid_file "T";width 2.4738in;height 2.19in;depth
0pt;original-width 2.4336in;original-height 2.1508in;cropleft "0";croptop
"1";cropright "1";cropbottom "0";tempfilename
'PQXXQZVA.wmf';tempfile-properties "XPR";}}

\chapter{Ray Diagrams and Thin Lenses}

Back in Newton's day, there were no electronic computers, or calculators, or
mechanical adding machines. Early optics researchers did math with a pen and
paper. This is one reason they liked small angle approximations. The
approximation allowed them to do harder problems using easy math. And so
long as the things they built worked, the approximations were good enough.
We are going to use another approximation in this lecture. It is called the
thin lens approximation. It will make the math that describes lenses for,
say, telescopes or microscopes, much easier. We will also introduce a way to
draw lenses and light that will tell you how the lens works. We will use
this drawing approach to do homework and examination problems. Professional
optical designers use computer codes that do this drawing very accurately to
help the optical engineer understand the system they are designing and to
look for mistakes. So this drawing scheme is very useful. Because it makes
optical systems so much easier to understand, let's start with the drawings
and then work toward a mathematical description of thin lenses.

%TCIMACRO{%
%\TeXButton{Fundamental Concepts}{\hspace{-1.3in}{\Large Fundamental Concepts\vspace{0.25in}}}}%
%BeginExpansion
\hspace{-1.3in}{\Large Fundamental Concepts\vspace{0.25in}}%
%EndExpansion

\begin{enumerate}
\item We can describe how a lens operates with just three easy-to-draw rays.

\item The magnification is given by $m\equiv -\frac{h^{\prime }}{h}=-\frac{%
s^{\prime }}{s}$

\item A semi-infinite bump can be described by the equation $\frac{n_{1}}{s}+%
\frac{n_{2}}{s^{\prime }}=\frac{\left( n_{2}-n_{1}\right) }{R}$

\item For a thin lens, we can describe where the light goes using $\frac{1}{s%
}+\frac{1}{s^{\prime }}=\frac{1}{f}$where $\frac{1}{f}=\left( n-1\right)
\left( \frac{1}{R_{1}}-\frac{1}{R_{2}}\right) $

\item There is a sign convention for all of these equations.
\end{enumerate}

Before we do a lot of math to describe how lenses work, lets think about our
early childhood experiences. You may have burned things with a magnifying
glass\footnote{%
If you didn't do this as a child, consider trying it now. Be responsible and
safe, but it is valuable to see how this works.}. Using the idea of a ray
diagram, here is what happens.\FRAME{dhF}{3.8017in}{2.2329in}{0pt}{}{}{Figure%
}{\special{language "Scientific Word";type "GRAPHIC";maintain-aspect-ratio
TRUE;display "USEDEF";valid_file "T";width 3.8017in;height 2.2329in;depth
0pt;original-width 3.7533in;original-height 2.1923in;cropleft "0";croptop
"1";cropright "1";cropbottom "0";tempfilename
'PQXXQZVB.wmf';tempfile-properties "XPR";}}The rays from the Sun come from
so far away that they are essentially parallel. We know that these rays come
together to a fine point that can start a fire. The point where these rays
converge is important to us. We will call this the \emph{focal point}.

Knowing that the light will follow the same paths either direction, we would
expect that if we put a light source at the focal distance, the rays should
come out parallel. \FRAME{dhF}{4.1926in}{2.4561in}{0pt}{}{}{Figure}{\special%
{language "Scientific Word";type "GRAPHIC";maintain-aspect-ratio
TRUE;display "USEDEF";valid_file "T";width 4.1926in;height 2.4561in;depth
0pt;original-width 4.1433in;original-height 2.4146in;cropleft "0";croptop
"1";cropright "1";cropbottom "0";tempfilename
'PQXXQZVC.wmf';tempfile-properties "XPR";}}This is one way manufacturers
make LED flashlights.

We need one other bit of information that we already have seen from basic
refraction.\FRAME{dhF}{1.996in}{1.6933in}{0pt}{}{}{Figure}{\special{language
"Scientific Word";type "GRAPHIC";maintain-aspect-ratio TRUE;display
"USEDEF";valid_file "T";width 1.996in;height 1.6933in;depth
0pt;original-width 3.6841in;original-height 3.122in;cropleft "0";croptop
"1";cropright "1";cropbottom "0";tempfilename
'PQXXQZVD.wmf';tempfile-properties "XPR";}}A flat block does refract the
light, but when the light leaves the block it is only displaced, it retains
the original direction. With these three ray scenarios, we can describe how
a lens works.

We know that for every point on the object, we get millions of reflected
rays that diverge. In the next figure, rays of light are reflecting off of
the large, black arrow. The lens must collect these rays together to form
the corresponding point on the image.

\FRAME{dhF}{4.1373in}{2.4275in}{0pt}{}{}{Figure}{\special{language
"Scientific Word";type "GRAPHIC";maintain-aspect-ratio TRUE;display
"USEDEF";valid_file "T";width 4.1373in;height 2.4275in;depth
0pt;original-width 4.088in;original-height 2.3869in;cropleft "0";croptop
"1";cropright "1";cropbottom "0";tempfilename
'PQXXQZVE.wmf';tempfile-properties "XPR";}}Of course, this happens for every
point on the image. \FRAME{dhF}{4.1926in}{2.4561in}{0pt}{}{}{Figure}{\special%
{language "Scientific Word";type "GRAPHIC";maintain-aspect-ratio
TRUE;display "USEDEF";valid_file "T";width 4.1926in;height 2.4561in;depth
0pt;original-width 4.1433in;original-height 2.4146in;cropleft "0";croptop
"1";cropright "1";cropbottom "0";tempfilename
'PQXXQZVF.wmf';tempfile-properties "XPR";}}But we usually pick the top of
the object. If we place the bottom of the object on the \emph{optic axis}%
\footnote{%
The line drawn on the figure thorugh the middle of the lens.}, the bottom of
the image will also be on the optic axis. So knowing where the bottom of the
image is, and finding the top of the image gives a pretty good idea of where
the rest of the image must be. So we will draw diagrams for the top of the
object to find the top of the image.

But suppose this is not true? For example, when we use a camera, we do not
align the optical system on an axis before we shoot.\FRAME{dhF}{3.2292in}{%
1.8421in}{0pt}{}{}{Figure}{\special{language "Scientific Word";type
"GRAPHIC";maintain-aspect-ratio TRUE;display "USEDEF";valid_file "T";width
3.2292in;height 1.8421in;depth 0pt;original-width 3.1842in;original-height
1.8049in;cropleft "0";croptop "1";cropright "1";cropbottom "0";tempfilename
'PQXXQZVG.wmf';tempfile-properties "XPR";}}We can, of course, trace two rays
for the bottom of the image in this case and find the location of the bottom
of the image.

Drawing millions of rays is impractical, and fortunately, not needed. We can
choose three simple rays that leave the top of the object , and see where
these rays converge to form the top of the image. Let's start with a ray
that travels from the top of the object and travels parallel to the optic
axis. We recognize this ray as being like one of the rays from the Sun. It
comes in parallel, so it will leave the lens and travel through the focal
point.\FRAME{dhF}{4.3889in}{2.5815in}{0pt}{}{}{Figure}{\special{language
"Scientific Word";type "GRAPHIC";maintain-aspect-ratio TRUE;display
"USEDEF";valid_file "T";width 4.3889in;height 2.5815in;depth
0pt;original-width 4.3379in;original-height 2.5399in;cropleft "0";croptop
"1";cropright "1";cropbottom "0";tempfilename
'PQXXQZVH.wmf';tempfile-properties "XPR";}}For a second easy ray, lets take
the case that is like our flat block. Near the center of the lens, the sides
are nearly flat. So we expect that the ray will leave in about the same
direction as it was going before it struck the lens.\FRAME{dhF}{3.1522in}{%
1.8542in}{0pt}{}{}{Figure}{\special{language "Scientific Word";type
"GRAPHIC";maintain-aspect-ratio TRUE;display "USEDEF";valid_file "T";width
3.1522in;height 1.8542in;depth 0pt;original-width 8.9949in;original-height
5.2736in;cropleft "0";croptop "1";cropright "1";cropbottom "0";tempfilename
'PQXXQZVI.wmf';tempfile-properties "XPR";}}Technically this ray should be
shifted. This is where our thin lens approximation comes in. If the lens is
thin, then the ray going through the center of the lens won't be shifted
much, and we can ignore the shift. We just dray ray 2 as a straight line.

Two rays are really enough to determine where the top of the image will be,
but there is a third ray that is easy to draw, so let's draw it to give us
more confidence in our answer. That ray is one that leaves the top of the
object and passes through the focal point on the object side of the lens.
This situation we also recognize. Since the ray goes through the focal
point, it is as though the light came from that point. This is like our LED
flashlight case. This ray will leave the lens parallel to the optic axis.

\FRAME{dhF}{4.2627in}{2.4976in}{0pt}{}{}{Figure}{\special{language
"Scientific Word";type "GRAPHIC";maintain-aspect-ratio TRUE;display
"USEDEF";valid_file "T";width 4.2627in;height 2.4976in;depth
0pt;original-width 4.2125in;original-height 2.4569in;cropleft "0";croptop
"1";cropright "1";cropbottom "0";tempfilename
'PQXXQZVJ.wmf';tempfile-properties "XPR";}}Where all three rays intersect,
we will have the top of the image. \FRAME{dhF}{4.9753in}{2.1067in}{0pt}{}{}{%
Figure}{\special{language "Scientific Word";type
"GRAPHIC";maintain-aspect-ratio TRUE;display "USEDEF";valid_file "T";width
4.9753in;height 2.1067in;depth 0pt;original-width 4.9216in;original-height
2.0678in;cropleft "0";croptop "1";cropright "1";cropbottom "0";tempfilename
'PQXXQZVK.wmf';tempfile-properties "XPR";}}Notice that in this case, the
image is upside down. That is normal. Also notice that it is smaller than
the object. We say that the image is magnified, which may seem a little bit
strange. But in optics, a magnification of greater than one means that the
image is bigger than the object. This is like a movie projector that makes a
large image of a small film segment. The magnification can be equal to one,
meaning the object and image are the same size. And finally the
magnification can be less than one. This means that the image is smaller
than the object. This is a convenient definition, because then we can use
the same equation to describe all three situations.%
\[
m\equiv \frac{\text{Image height}}{\text{Object height}}=-\frac{h^{\prime }}{%
h} 
\]%
where $h$ is the object height, and $h^{\prime }$ is the image height.
Notice the negative sign. By convention (meaning physicists got together and
voted on this) we say that an upside down image has a negative
magnification. You just have to memorize this, there is no obvious reason
for this except it is mathematically convenient.

We will find that 
\[
m=-\frac{s^{\prime }}{s} 
\]

%TCIMACRO{%
%\TeXButton{Question 123.15.4}{\marginpar {
%\hspace{-0.5in}
%\begin{minipage}[t]{1in}
%\small{Question 123.15.4}
%\end{minipage}
%}}}%
%BeginExpansion
\marginpar {
\hspace{-0.5in}
\begin{minipage}[t]{1in}
\small{Question 123.15.4}
\end{minipage}
}%
%EndExpansion

\section{Virtual Images with Lenses}

Lets take another case and draw a ray diagram. This time let's place the
object closer than the focal distance. This is the case when we use a lens
as a magnifying glass. The rays will look like this.\FRAME{dhF}{3.6201in}{%
2.3713in}{0pt}{}{}{Figure}{\special{language "Scientific Word";type
"GRAPHIC";maintain-aspect-ratio TRUE;display "USEDEF";valid_file "T";width
3.6201in;height 2.3713in;depth 0pt;original-width 3.5725in;original-height
2.3315in;cropleft "0";croptop "1";cropright "1";cropbottom "0";tempfilename
'PQXXQZVL.wmf';tempfile-properties "XPR";}}Notice that these rays never
converge! We won't get an image that could project on a paper. But we know
that there is an image, we can look through the lens and see it! And that is
the key. The image does not really exist. There is no place where light
gathers and then diverges into a pattern that we recognize. But we look
through the lens, and our mind interprets the diverging rays coming from the
lens as though they had only traveled in straight lines. If we extend these
rays backwards along straight lines, they appear to come from a common
point. This is the point they would have had to have come from if there were
no lens. We believe we see an image at this location. But no light really
goes there! Because this image is not really made from light diverging from
this position, we call it a \emph{virtual image}. The image we formed before
that could be projected on a screen is called a \emph{real image}.

By convention, we say the distance from the lens to the virtual image is a
negative value.

\subsection{Diverging Lenses}

So far our lenses have only been the sort that work as magnifying glasses.
We call these \emph{converging lenses.} These lenses are fatter in the
middle and thinner on the edges. Because of this they are sometimes called 
\emph{convex lenses}. By convention, we say the focal distance for this type
of lens is positive. For this reason, they are often called \emph{positive
lenses.}

But what if we make a lens that is thinner in the middle and thicker on the
edges. We can call this sort of lens a \emph{concave lens}, and we will give
it a negative focal length by convention, so we can also call it a \emph{%
negative lens}. But what would this lens do? If we think about our three
rays, ray $1$ won't be bent toward the optic axis for this type of lens. In
fact, if we observe an object through this lens, ray number 1 will appear to
come from the focal point. Ray number 2 will still go through the middle of
the lens, and if the lens is thin enough, ray 2 will pass through undeviated.

\FRAME{dhF}{3.2984in}{1.7582in}{0pt}{}{}{Figure}{\special{language
"Scientific Word";type "GRAPHIC";maintain-aspect-ratio TRUE;display
"USEDEF";valid_file "T";width 3.2984in;height 1.7582in;depth
0pt;original-width 3.2534in;original-height 1.7201in;cropleft "0";croptop
"1";cropright "1";cropbottom "0";tempfilename
'PQXXQZVM.wmf';tempfile-properties "XPR";}}finally ray three leaves the
object in the direction of the far focal point. It will hit the lens and
leave parallel to the optic axis. From the figure we see that these three
rays will never converge. We expect they will form a virtual image. If we
extend the rays backward as shown, we see that the extensions all meet at a
point. The rays leaving the lens appear to come from this point. This is the
location of the top of the virtual image of the object.

You might wonder what good such a lens could do, but we will find that this
type of lens is used to correct vision for nearsighted people. It's likely
that many people in our class have this type of lens with them either on
their eye (contacts) or on their nose (eye glasses).

\section{Semi-Infinite Lenses and the Semi-Infinite Lens Image Equation}

%TCIMACRO{%
%\TeXButton{Question 123.16.1}{\marginpar {
%\hspace{-0.5in}
%\begin{minipage}[t]{1in}
%\small{Question 123.165.1}
%\end{minipage}
%}}}%
%BeginExpansion
\marginpar {
\hspace{-0.5in}
\begin{minipage}[t]{1in}
\small{Question 123.165.1}
\end{minipage}
}%
%EndExpansion
We learned how to find an image location graphically, now let's do it
algebraically. Let's start by thinking of a special case for refraction. A
curved surface on a very large piece of glass. We will assume that the piece
of glass is semi-infinate, but all it has to be is very large.

We can call this a semi-infinate bump of glass.\FRAME{dhF}{3.2491in}{1.1865in%
}{0pt}{}{}{Figure}{\special{language "Scientific Word";type
"GRAPHIC";maintain-aspect-ratio TRUE;display "USEDEF";valid_file "T";width
3.2491in;height 1.1865in;depth 0pt;original-width 5.2555in;original-height
1.9009in;cropleft "0";croptop "1";cropright "1";cropbottom "0";tempfilename
'PQXXQZVN.wmf';tempfile-properties "XPR";}}

Take a point object that either glows, or has rays of light reflecting from
it. The rays leave the object and reach the surface of the glass. The rays
will refract at the surface. Each bends toward the normal, but because of
the curvature of the glass, the rays all converge toward the center. We can
identify this convergence point as the image of the point object. At the
surface we can find the refracted angles using Snell's law%
\[
n_{1}\sin \theta _{1}=n_{2}\sin \theta _{2} 
\]

We will again use the small angle approximation. then $\theta _{1}$ and $%
\theta _{2}$ are small and none of the rays are very far away from the axis.
This is called the \emph{paraxial}\footnote{%
If you speak Spanish (or Latin) you will recognize this as meaning
\textquotedblleft by the axis.\textquotedblright} approximation. Snell's law
becomes 
\[
n_{1}\theta _{1}=n_{2}\theta _{2} 
\]

\FRAME{dtbpF}{5.3065in}{2.0306in}{0in}{}{}{Figure}{\special{language
"Scientific Word";type "GRAPHIC";maintain-aspect-ratio TRUE;display
"USEDEF";valid_file "T";width 5.3065in;height 2.0306in;depth
0in;original-width 5.2503in;original-height 1.9917in;cropleft "0";croptop
"1";cropright "1";cropbottom "0";tempfilename
'PS8C4K03.wmf';tempfile-properties "XPR";}}Using the more detailed figure
above, we observe triangles $SAC$ and $PAC.$ We can see that for triangle $%
SAC$ the top angle labeled $\eta $, plus $\theta _{1}$ must be $180.$ 
\[
\eta +\theta _{1}=180\unit{%
%TCIMACRO{\U{b0}}%
%BeginExpansion
{{}^\circ}%
%EndExpansion
} 
\]%
We also know that the sum of interior angles of a triangle must equal $180.$
So%
\[
180\unit{%
%TCIMACRO{\U{b0}}%
%BeginExpansion
{{}^\circ}%
%EndExpansion
}=\alpha +\beta +\eta 
\]%
then 
\begin{eqnarray*}
\eta +\theta _{1} &=&\alpha +\beta +\eta \\
\theta _{1} &=&\alpha +\beta
\end{eqnarray*}%
Likewise, from triangle $PAC$,%
\[
\beta +\psi =180\unit{%
%TCIMACRO{\U{b0}}%
%BeginExpansion
{{}^\circ}%
%EndExpansion
} 
\]

\[
180\unit{%
%TCIMACRO{\U{b0}}%
%BeginExpansion
{{}^\circ}%
%EndExpansion
}=\psi +\gamma +\theta _{2} 
\]%
so then 
\[
\beta =\theta _{2}+\gamma 
\]%
then, 
\[
\theta _{2}=\beta -\gamma 
\]

and we can write our paraxial Snell's law as%
\begin{eqnarray*}
n_{1}\theta _{1} &=&n_{2}\theta _{2} \\
n_{1}\left( \alpha +\beta \right) &=&n_{2}\left( \beta -\gamma \right) \\
n_{1}\alpha +n_{1}\beta &=&n_{2}\beta -n_{2}\gamma \\
n_{1}\alpha +n_{2}\gamma &=&n_{2}\beta -n_{1}\beta \\
n_{1}\alpha +n_{2}\gamma &=&\beta \left( n_{2}-n_{1}\right)
\end{eqnarray*}

Looking at the figure. We see that $d$ is a leg of three different right
triangles ($SAV,$ $ACV$, and $PAV$). The ray in the figure is clearly not a
paraxial ray. If we use a paraxial ray, then the point $V$ will approach the
air-glass boundary. When this happens, then $SV=s,$ $VC=R$, and $%
VP=s^{\prime }.$ So we can write

\begin{eqnarray*}
\tan \alpha &\approx &\alpha \approx \frac{d}{s} \\
\tan \beta &\approx &\beta \approx \frac{d}{R} \\
\tan \gamma &\approx &\gamma \approx \frac{d}{s^{\prime }}
\end{eqnarray*}%
so our Snell's law becomes%
\begin{eqnarray*}
n_{1}\alpha +n_{2}\gamma &=&\beta \left( n_{2}-n_{1}\right) \\
n_{1}\frac{d}{s}+n_{2}\frac{d}{s^{\prime }} &=&\frac{d}{R}\left(
n_{2}-n_{1}\right)
\end{eqnarray*}%
We can divide out the $d^{\prime }s$%
\begin{equation}
\frac{n_{1}}{s}+\frac{n_{2}}{s^{\prime }}=\frac{\left( n_{2}-n_{1}\right) }{R%
}  \label{Bump_Equation}
\end{equation}

We can use this formula to convince ourselves that no matter what the angle
is (providing it is small), the rays will form an image at $P.$

Real images will be in the glass for this special case (we will fix this
with non-infinite lenses soon) so we need to change sign conventions.

\[
\begin{tabular}{|l|l|l|}
\hline
\textbf{Quantity} & \textbf{Positive if} & \textbf{Negative if} \\ \hline
{\small Object location }$\left( s\right) $ & 
\begin{tabular}{l}
{\small Object is in front of interface} \\ 
{\small \ surface}%
\end{tabular}
& 
\begin{tabular}{l}
{\small Object is in back of interface} \\ 
{\small \ surface (virtual object)}%
\end{tabular}%
{\small \ } \\ \hline
{\small Image location }$\left( s^{\prime }\right) $ & 
\begin{tabular}{l}
{\small Image is in back of interface} \\ 
{\small \ surface (real image)}%
\end{tabular}
& 
\begin{tabular}{l}
{\small Image is in front of interface} \\ 
{\small \ surface (virtual image)}%
\end{tabular}
\\ \hline
{\small Image height }$\left( h^{\prime }\right) $ & 
\begin{tabular}{l}
{\small Image is upright}%
\end{tabular}
& 
\begin{tabular}{l}
{\small Image is inverted}%
\end{tabular}
\\ \hline
{\small Radius }$\left( R\right) $ & 
\begin{tabular}{l}
{\small Center of curvature is in} \\ 
{\small back of interface surface}%
\end{tabular}%
{\small \ } & 
\begin{tabular}{l}
{\small Center of curvature is in} \\ 
{\small front of interface surface}%
\end{tabular}%
{\small \ } \\ \hline
\end{tabular}%
\]

We could go through the entire derivation and switch the indices of
refraction. It turns out we get the same equation. The results will be
different, but the equation is the same.

\subsection{Flat Refracting surfaces}

\FRAME{dhF}{1.983in}{1.855in}{0pt}{}{}{Figure}{\special{language "Scientific
Word";type "GRAPHIC";maintain-aspect-ratio TRUE;display "USEDEF";valid_file
"T";width 1.983in;height 1.855in;depth 0pt;original-width
3.7395in;original-height 3.4973in;cropleft "0";croptop "1";cropright
"1";cropbottom "0";tempfilename 'PQXXQZVP.wmf';tempfile-properties "XPR";}}

Let's return to our fish tank. The fish tank has an interface, but it is
flat. Can we use our equation (\ref{Bump_Equation}) to describe this?

The answer is yes, if we let $R=\infty .$ This makes sense for a flat
surface. If we have an infinitely large sphere, then our small part of that
spherical surface that makes up the fish tank wall will be very flat.

Then%
\[
\frac{n_{1}}{s}+\frac{n_{2}}{s^{\prime }}=\frac{\left( n_{2}-n_{1}\right) }{%
\infty } 
\]%
or%
\[
\frac{n_{1}}{s}+\frac{n_{2}}{s^{\prime }}=0 
\]%
we see that 
\[
s^{\prime }=-s\frac{n_{2}}{n_{1}} 
\]

This is what we got before for this case, except before we just got the
distance, and now we have included the effects of our sign convention. The
negative sign means that the image is in front of the surface. By
\textquotedblleft in front\textquotedblright\ we always mean to follow the
light from the source (fish) to the optical boundary. This boundary is the
water/air boundary of the tank, so the fact that our image is in the water
means that our image is in front of the optical boundary. As we know, this
means the image is virtual.

\section{Thin Lenses}

Lets' find an equation for a spherical surface once more. But this time,
let's let it be more practical and not make the \textquotedblleft
lens\textquotedblright\ semi-infinate. We will need to deal with two sides
of the lens because (usually) both will be curved.

We found that for refraction%
\[
\frac{n_{1}}{s}+\frac{n_{2}}{s^{\prime }}=\frac{\left( n_{2}-n_{1}\right) }{R%
} 
\]

but we did this for a spherical bump on a semi-infinite piece of glass. For
this new problem let's make a few assumptions:

\begin{enumerate}
\item We have two spherical surfaces, with $R_{1}$ and $R_{2}$ as the radii
of curvature

\item We have only paraxial rays

\item The image formed by one refractive surface serves as the object for
the second surface

\item The lens is not very thick (the thickness is much smaller than both $%
R_{1}$ and $R_{2})$
\end{enumerate}

The answer we will get is quite simple

\begin{equation}
\frac{1}{s}+\frac{1}{s^{\prime }}=\frac{1}{f}  \label{Thin_lens}
\end{equation}%
where%
\begin{equation}
\frac{1}{f}=\left( n-1\right) \left( \frac{1}{R_{1}}-\frac{1}{R_{2}}\right)
\label{Lens_maker}
\end{equation}%
but to appreciate what it means, lets find out where it comes from.

\subsection{Derivation of the lens equation}

Consider the optical element in the figure below. Notice that our object is
a dot, so our image will also be a dot. This is not as boring as it sounds
if we consider any object can be considered as a collection of dots.\FRAME{%
dhF}{4.0542in}{2.4561in}{0pt}{}{}{Figure}{\special{language "Scientific
Word";type "GRAPHIC";maintain-aspect-ratio TRUE;display "USEDEF";valid_file
"T";width 4.0542in;height 2.4561in;depth 0pt;original-width
4.0041in;original-height 2.4146in;cropleft "0";croptop "1";cropright
"1";cropbottom "0";tempfilename 'PQXXQZVQ.wmf';tempfile-properties "XPR";}}%
Light enters at a spherical surface on the left hand side. We use a point
object located at $S$ on the principal axis. We could put out dot object
anywhere, but let's put it on the axis and trace two rays. The ray along the
principal axis crosses each spherical surface at right angles, and therefore
traves straight through the optic (this makes on ray very easy to trace!).
The second ray hits the first spherical surface at point $A.$ It is
refracted and travels to point $B$. It is again refracted and travels toward
the principal axis, crossing at $P.$ The image location is the intersection
of these rays, so we have an image at $P.$

Lets study the surfaces separately

Surface 1:

\FRAME{dhF}{3.7879in}{2.5676in}{0pt}{}{}{Figure}{\special{language
"Scientific Word";type "GRAPHIC";maintain-aspect-ratio TRUE;display
"USEDEF";valid_file "T";width 3.7879in;height 2.5676in;depth
0pt;original-width 3.7395in;original-height 2.5261in;cropleft "0";croptop
"1";cropright "1";cropbottom "0";tempfilename
'PQXXQZVR.wmf';tempfile-properties "XPR";}}

Let's treat surface $1$ as though surface two did not exist. The light would
bend at point $A$ and head off into the lens material. This is just our
semi-infinite bump problem so we know that 
\[
\frac{n_{1}}{s}+\frac{n_{2}}{s^{\prime }}=\frac{\left( n_{2}-n_{1}\right) }{R%
} 
\]

We can consider $n_{1}=1$ and $n_{2}=n$ for a air-glass interface and noting
that $s_{1}^{\prime }$ is negative. Then%
\begin{equation}
\frac{1}{s_{1}}-\frac{n}{s_{1}^{\prime }}=\frac{\left( n-1\right) }{R_{1}}
\label{Surface1}
\end{equation}

Note that our rays are \emph{not} converging in the glass. We can find the
image formed by this side of our lens by tracing the diverging rays backward
as we did for the fish tank. The image formed from the first side of the
lens is virtual.

Surface 2

Now consider the second surface.

\FRAME{dhF}{4.2065in}{2.079in}{0pt}{}{}{Figure}{\special{language
"Scientific Word";type "GRAPHIC";maintain-aspect-ratio TRUE;display
"USEDEF";valid_file "T";width 4.2065in;height 2.079in;depth
0pt;original-width 4.1572in;original-height 2.0401in;cropleft "0";croptop
"1";cropright "1";cropbottom "0";tempfilename
'PQXXQZVS.wmf';tempfile-properties "XPR";}}The second surface sees light
diverging as though it came from a semi-infinite piece of glass with the
object at $P_{1}.$ It's not true, the light came from $s.$ But the second
surface can't tell the difference. It only knows light is incident in a
particular pattern, and that pattern appears to come from point $P_{1}.$ So
we can treat the situation at surface $2$ as though the object is the
virtual image formed by surface $1.$ So 
\[
s_{2}=s_{1}^{\prime }+t 
\]

We again use our refractive equation%
\[
\frac{n_{1}}{s}+\frac{n_{2}}{s^{\prime }}=\frac{\left( n_{2}-n_{1}\right) }{R%
} 
\]%
but we identify $n_{1}=n$ and $n_{2}=1$. We have for surface $2$%
\begin{equation}
\frac{n}{s_{2}}+\frac{1}{s_{2}^{\prime }}=\frac{\left( 1-n\right) }{R_{2}}
\end{equation}%
or%
\begin{equation}
\frac{n}{s_{1}^{\prime }+t}+\frac{1}{s_{2}^{\prime }}=\frac{\left(
1-n\right) }{R_{2}}  \label{Surface2}
\end{equation}

Now we take our thin lens approximation. Let $t\rightarrow 0.$ Then
equations (\ref{Surface1}) and (\ref{Surface2}) become%
\[
\frac{1}{s_{1}}-\frac{n}{s_{1}^{\prime }}=\frac{\left( n-1\right) }{R_{1}} 
\]

\[
\frac{n}{s_{1}}+\frac{1}{s_{2}^{\prime }}=\frac{\left( 1-n\right) }{R_{2}} 
\]

Adding these two equations yields%
\[
\frac{1}{s_{1}}-\frac{n}{s_{1}^{\prime }}+\frac{n}{s_{1}^{\prime }}+\frac{1}{%
s_{2}^{\prime }}=\frac{\left( n-1\right) }{R_{1}}+\frac{\left( 1-n\right) }{%
R_{2}} 
\]%
or%
\[
\frac{1}{s_{1}}+\frac{1}{s_{2}^{\prime }}=\left( n-1\right) \left( \frac{1}{%
R_{1}}-\frac{1}{R_{2}}\right) 
\]

This equation is very useful. It reminds us of the mirror equation (well, a
little) . If we again let $s_{1}=\infty $ (put the object at $\infty $ so
the rays enter surface $1$ parallel) we find%
\[
\frac{1}{s_{2}^{\prime }}=\left( n-1\right) \left( \frac{1}{R_{1}}-\frac{1}{%
R_{2}}\right) 
\]%
The spot where the rays gather if the object is infinitely far away is the
focal point, $f.$ so for parallel rays we can identify $s_{2}^{\prime }=f$
as the focal length of the optic%
\[
\frac{1}{f}=\left( n-1\right) \left( \frac{1}{R_{1}}-\frac{1}{R_{2}}\right) 
\]

which is known as the \emph{lens makers' equation}.

Then we have a relationship between the object distance in front of the
lens, and the final image in back of the lens: 
\begin{eqnarray*}
\frac{1}{s_{1}}+\frac{1}{s_{2}^{\prime }} &=&\left( n-1\right) \left( \frac{1%
}{R_{1}}-\frac{1}{R_{2}}\right) \\
&=&\frac{1}{f}
\end{eqnarray*}%
We can drop the subscripts (which we can do now that we let $t=0$ since the
internal distances for the inside points are not important).%
\[
\frac{1}{s}+\frac{1}{s^{\prime }}=\frac{1}{f} 
\]%
This is called the \emph{thin lens equation}. The resulting approximate
geometry is shown below.

\FRAME{dhF}{4.4581in}{1.3396in}{0pt}{}{}{Figure}{\special{language
"Scientific Word";type "GRAPHIC";maintain-aspect-ratio TRUE;display
"USEDEF";valid_file "T";width 4.4581in;height 1.3396in;depth
0pt;original-width 4.4071in;original-height 1.305in;cropleft "0";croptop
"1";cropright "1";cropbottom "0";tempfilename
'PQXXQZVT.wmf';tempfile-properties "XPR";}}

Of course any real object is made of lots of points, and not all of the
points are on an axis. But each point will be imaged to a corresponding
point on the image. Here is an example with three points in the object (the $%
s_{i}$ points) and where their images are (the $p_{i}$ points).\FRAME{dhF}{%
2.3255in}{1.8792in}{0pt}{}{}{Figure}{\special{language "Scientific
Word";type "GRAPHIC";maintain-aspect-ratio TRUE;display "USEDEF";valid_file
"T";width 2.3255in;height 1.8792in;depth 0pt;original-width
4.2817in;original-height 3.4558in;cropleft "0";croptop "1";cropright
"1";cropbottom "0";tempfilename 'PQXXQZVU.wmf';tempfile-properties "XPR";}}%
but it would work for millions of points. Our simple analysis explains the
formation of actual images and not just point images.%
%TCIMACRO{%
%\TeXButton{Question 123.16.2}{\marginpar {
%\hspace{-0.5in}
%\begin{minipage}[t]{1in}
%\small{Question 123.16.2}
%\end{minipage}
%}}}%
%BeginExpansion
\marginpar {
\hspace{-0.5in}
\begin{minipage}[t]{1in}
\small{Question 123.16.2}
\end{minipage}
}%
%EndExpansion

\section{Sign Convention}

We need to add to our sign convention table a second radius, and the focal
length.

\[
\begin{tabular}{|l|l|l|}
\hline
\textbf{Quantity} & \textbf{Positive if} & \textbf{Negative if} \\ \hline
{\small Object location }$\left( s\right) $ & 
\begin{tabular}{l}
{\small Object is in front of surface} \\ 
\end{tabular}
& 
\begin{tabular}{l}
{\small Object is in back of surface} \\ 
{\small (virtual object)}%
\end{tabular}%
{\small \ } \\ \hline
{\small Image location }$\left( s^{\prime }\right) $ & 
\begin{tabular}{l}
{\small Image is in back of surface} \\ 
{\small (real image)}%
\end{tabular}
& 
\begin{tabular}{l}
{\small Image is in front of surface } \\ 
{\small (virtual image)}%
\end{tabular}
\\ \hline
{\small Image height }$\left( h^{\prime }\right) $ & 
\begin{tabular}{l}
{\small Image is upright}%
\end{tabular}
& 
\begin{tabular}{l}
{\small Image is inverted}%
\end{tabular}
\\ \hline
{\small Radius }$\left( R_{1}\text{ and }R_{2}\right) $ & 
\begin{tabular}{l}
{\small Center of curvature is in} \\ 
{\small back of surface}%
\end{tabular}%
{\small \ } & 
\begin{tabular}{l}
{\small Center of curvature is in} \\ 
{\small front of surface}%
\end{tabular}%
{\small \ } \\ \hline
Focal length $\left( f\right) $ & Converging lens & Diverging lens \\ \hline
\end{tabular}%
\]

\section{Magnification}

We defined the magnification earlier as a comparison of the image height to
the object height. 
\begin{equation}
m=\frac{h^{\prime }}{h}
\end{equation}%
But we can see that we could write this another way using simple
trigonometry. \FRAME{dtbpF}{2.834in}{1.5108in}{0in}{}{}{Figure}{\special%
{language "Scientific Word";type "GRAPHIC";maintain-aspect-ratio
TRUE;display "USEDEF";valid_file "T";width 2.834in;height 1.5108in;depth
0in;original-width 2.7916in;original-height 1.4745in;cropleft "0";croptop
"1";cropright "1";cropbottom "0";tempfilename
'PS8BF600.wmf';tempfile-properties "XPR";}}Note that 
\begin{eqnarray*}
\tan \theta &=&\frac{h}{s} \\
\tan \theta &=&\frac{h^{\prime }}{s^{\prime }}
\end{eqnarray*}%
then 
\[
\frac{h}{s}=\frac{h^{\prime }}{s^{\prime }} 
\]%
or 
\[
\frac{h^{\prime }}{h}=\frac{s^{\prime }}{s} 
\]%
so we can write the magnification for a converging lens as minus the ratio
of the image distance over the object distance. 
\[
m=-\frac{s^{\prime }}{s} 
\]%
The minus sign comes from our sign convention because the image will be
upside down.

We found the thin lens formula using converging lenses, but it works for
diverging lenses as well, so long as the thin lens approximation is valid.

\chapter{Images Formed by Mirrors and Combinations of Lenses}

%TCIMACRO{%
%\TeXButton{Fundamental Concepts}{\hspace{-1.3in}{\Large Fundamental Concepts\vspace{0.25in}}}}%
%BeginExpansion
\hspace{-1.3in}{\Large Fundamental Concepts\vspace{0.25in}}%
%EndExpansion

\begin{itemize}
\item Curved Mirrors can form images

\item We can write the magnification in terms of the object and image
distances 
\[
m=-\frac{s^{\prime }}{s} 
\]

\item The mirror formula for imaging is almost the same as the thin lens
formula. The focal length part is quite different.%
\begin{eqnarray*}
\frac{1}{s}+\frac{1}{s^{\prime }} &=&\frac{1}{f} \\
f &=&\frac{R}{2}
\end{eqnarray*}

\item The magnification of a two-lens system is just the product of the
magnifications of the individual lenses $m_{\text{combined}}=m_{1}m_{2}$

\item If the two lenses in a two-lens system are placed so the distance
between them is essentially zero, then the focal length of the two-lens
system is given by 
\[
\frac{1}{f}=\frac{1}{f_{1}}+\frac{1}{f_{2}} 
\]%
where $f_{1}$ and $f_{2}$ are the focal lengths of the individual lenses

\item If the two lenses in a two-lens system are not places so the distance
between them is essentially zero, the previous equation is not valid!
\end{itemize}

\section{Imaging with Mirrors}

%TCIMACRO{%
%\TeXButton{Question 123.32.1}{\marginpar {
%\hspace{-0.5in}
%\begin{minipage}[t]{1in}
%\small{Question 123.32.1}
%\end{minipage}
%}}}%
%BeginExpansion
\marginpar {
\hspace{-0.5in}
\begin{minipage}[t]{1in}
\small{Question 123.32.1}
\end{minipage}
}%
%EndExpansion
All of us have looked in a mirror at some time. We know what to expect. We
see an image of ourselves. To study mirrors mathematically, we need to
establish a sign convention and some standard notation.

\FRAME{dtbpF}{1.452in}{1.5281in}{0in}{}{}{Figure}{\special{language
"Scientific Word";type "GRAPHIC";maintain-aspect-ratio TRUE;display
"USEDEF";valid_file "T";width 1.452in;height 1.5281in;depth
0in;original-width 1.4166in;original-height 1.4909in;cropleft "0";croptop
"1";cropright "1";cropbottom "0";tempfilename
'PQXXQZVV.wmf';tempfile-properties "XPR";}}In the figure above, we have a
person observing an object $O$ in a mirror. The object is located at a
distance $s$ from the mirror. Just like for lenses, we will call this the 
\emph{object distance}. The image appears to be located at a point $I$
beyond the mirror a distance $s^{\prime }.$ Like before, we will call this
the \emph{image distance}.

\section{How Images are Formed}

This isn't our first look at mirrors, so let's review a bit. Images are
located at a point from which rays of light diverge or at a point from which
rays of light appear to diverge. This makes sense if you recall how our
brain processes the light signals that we see. Our eyes intercept rays of
light diverging from an object. Our brain processes those rays as though
they traveled only in straight lines. So if we can create a situation that
makes rays diverge in the same way the object did, we will have an image of
the object.

\subsection{Virtual Images}

We already know that mirrors often create what we call \emph{virtual} images
because the image appears to be created from diverging rays from behind the
mirror. If we look behind the mirror, no rays exist there (if they did
exist, they would not make it through the mirror!).

\FRAME{dtbpF}{2.6757in}{1.5195in}{0in}{}{}{Figure}{\special{language
"Scientific Word";type "GRAPHIC";maintain-aspect-ratio TRUE;display
"USEDEF";valid_file "T";width 2.6757in;height 1.5195in;depth
0in;original-width 2.6333in;original-height 1.4832in;cropleft "0";croptop
"1";cropright "1";cropbottom "0";tempfilename
'PQXXQZVW.wmf';tempfile-properties "XPR";}}

Let's look at a simple image as shown in the figure above. The object is an
arrow. We could trace all the rays that diverge from this object and build a
very nice representation of the arrow (like ray tracing-based computer
graphics--the way movies like \emph{Toy Story} are made) but that would take
time and computation power. For lenses, we only really needed to use two
rays, and to remember what the object looked like. The same is true for
mirrors.

We pick one ray from the top of the arrow that travels straight to the
mirror. This ray will travel a distance $s$ and bounce back. We pick a
second ray from point $P$ that travels the path $PR.$ This ray bounces off
the mirror at an angle $\theta .$ So it appears that the tip of the arrow is
at position $P^{\prime }$ and the rays from the tip appear to travel the
paths $P^{\prime }P$ and $P^{\prime }R.$ All this was review from our study
of the law of reflection. Now let's go a little bit deeper into image
formation by mirrors.

\subsection{Mirror reversal}

Look into a mirror. Raise your right hand. \FRAME{dtbpF}{3.9998in}{2.2399in}{%
0pt}{}{}{Figure}{\special{language "Scientific Word";type
"GRAPHIC";maintain-aspect-ratio TRUE;display "USEDEF";valid_file "T";width
3.9998in;height 2.2399in;depth 0pt;original-width 3.9505in;original-height
2.2001in;cropleft "0";croptop "1";cropright "1";cropbottom "0";tempfilename
'PSBQTW00.wmf';tempfile-properties "XPR";}}Your image raises what appears to
be a left hand. It looks like a mirror switches the left and right sides of
the image. But it might be better to note that the image raised hand is on
the same side of the room as your real raised hand. Think of lying sideways
on the ground in front of the mirror and raise a hand. Note that your image
feet are on the same side of the room as your real feet. And note that as
you put one hand up, the image hand also goes up. This is not exactly
left-right reversal. What is happening? \FRAME{dtbpF}{4.4936in}{2.5166in}{0in%
}{}{}{Figure}{\special{language "Scientific Word";type
"GRAPHIC";maintain-aspect-ratio TRUE;display "USEDEF";valid_file "T";width
4.4936in;height 2.5166in;depth 0in;original-width 4.4417in;original-height
2.4751in;cropleft "0";croptop "1";cropright "1";cropbottom "0";tempfilename
'PSBXLS02.wmf';tempfile-properties "XPR";}}The light from your raised hand
would hit the mirror, reflect, and then be absorbed by your eye. Your brain
tells you the light traveled in a straight line. So it looks like you have a
raised hand in the mirror. But what has happened is that the light has
traveled from your hand, to the mirror and back to your eye. This is not
really left-right reversal. We need a name to describe what is really
happening. The name is odd. To see why we use this name, notice that from
your perspective (moving left to right across the figure) you first have the
back part of your hand, then the front of your hand and then the front of
the image hand, then the back of the image hand. We call this back-front
reversal. It comes from the reversal in the order of parts of the object,
like the front and back of your hand. So we say that a flat mirror performs
a front-back reversal, not a left-right reversal.

\subsection{Concave Mirrors}

Concave mirrors can form images. I'm sure you know that many telescopes are
made with mirrors. We should see how this works. We recall the law of
reflection\FRAME{dhF}{1.8585in}{1.1857in}{0pt}{}{}{Figure}{\special{language
"Scientific Word";type "GRAPHIC";maintain-aspect-ratio TRUE;display
"USEDEF";valid_file "T";width 1.8585in;height 1.1857in;depth
0pt;original-width 1.8213in;original-height 1.1519in;cropleft "0";croptop
"1";cropright "1";cropbottom "0";tempfilename
'PQXXQZVY.wmf';tempfile-properties "XPR";}}

\[
\theta _{i}=\theta _{r} 
\]

Armed with this, we can see what would happen if we curved the mirror
surface. Each ray has a different normal due to the curvature of the mirror.
The result is that parallel rays all meet at a spot on the axis.

\FRAME{dhF}{3.243in}{2.079in}{0pt}{}{}{Figure}{\special{language "Scientific
Word";type "GRAPHIC";maintain-aspect-ratio TRUE;display "USEDEF";valid_file
"T";width 3.243in;height 2.079in;depth 0pt;original-width
3.1981in;original-height 2.0401in;cropleft "0";croptop "1";cropright
"1";cropbottom "0";tempfilename 'PQXXQZVZ.wmf';tempfile-properties "XPR";}}%
If we place something at this location, we could start a fire! We have just
re-invented the solar cooker! Look at the point where all the rays meet.
This is just like the lens situation where parallel rays met at a point
after passing an optical element. This point must be a focal point.

\subsection{Paraxial Approximation for Mirrors}

The correct shape of a mirror is more like a parabola, but parabolas are
hard to machine or build. Spherical shapes are relatively easy. So we often
see spherical mirrors just like we often see spherical lenses. This will
work so long as we allow only rays that make small angles with respect to
the principal axis. We can see why this works if we plot a sphere and a
parabola (and a hyperbola). For small deviations from the center, the shape
of the functions all look alike. \FRAME{dhF}{4.3613in}{1.3949in}{0pt}{}{}{%
Figure}{\special{language "Scientific Word";type
"GRAPHIC";maintain-aspect-ratio TRUE;display "USEDEF";valid_file "T";width
4.3613in;height 1.3949in;depth 0pt;original-width 4.3102in;original-height
1.3604in;cropleft "0";croptop "1";cropright "1";cropbottom "0";tempfilename
'PQXXQZW0.wmf';tempfile-properties "XPR";}}We would expect the reflections
to be similar under these circumstances, so, if we meet the criteria for the
paraxial approximation, our spherical mirrors should work. Note that when
you need the entire mirror, say, in a communications antenna, you must do
better than a spherical approximation to the correct shape for your mirror.
Your satellite dish is likely not a spherical section. \FRAME{dtbpF}{3.1004in%
}{1.9in}{0pt}{}{}{Figure}{\special{language "Scientific Word";type
"GRAPHIC";maintain-aspect-ratio TRUE;display "USEDEF";valid_file "T";width
3.1004in;height 1.9in;depth 0pt;original-width 4.6414in;original-height
2.8331in;cropleft "0";croptop "1";cropright "1";cropbottom "0";tempfilename
'PQXXQZW1.wmf';tempfile-properties "XPR";}}

Like with our flat mirror, we will measure distances from the mirror surface
(from point $V$ in the figure). We can find the image location, $s^{\prime
}, $ by again taking two rays. We could use any of billions of rays. But
just like for lenses, let's try to pick rays that are easy to draw. One
convenient ray is the ray that passes through the center of curvature, $C.$
This ray will strike the mirror surface at right angles and bounce back
along the same path. The incident angle will be zero, so the reflected angle
must be zero by the law of reflection. That is the yellow ray in the figure.

Another convenient ray is the ray from the tip to point $V.$ This ray will
bounce back with angle $\theta $. Right at point $V$ the mirror surface is
perpendicular to the optic axis. This makes the bounce of the blue ray just
like the a bounce from a flat mirror! That is easy to draw. Where these two
reflected rays cross, we will find the image of the tip of our arrow.
Knowing the shape of the arrow and that the bottom is on the axis, we can
fill in the rest of the image.

We can calculate the magnification for this case. We use the gold triangle
to determine that 
\[
\tan \theta =\frac{h}{s} 
\]%
and the blue triangle to determine that 
\[
\tan \theta =\frac{-h^{\prime }}{s^{\prime }} 
\]%
We want to indicate that the image is inverted by making it's sign negative.
So we have arbitrarily added the negative sign to make the equation fit our
sign convention. This gives us $\tan \theta $ instead of $-\tan \theta $
when $h^{\prime }$ is inverted (when $h^{\prime },$ itself, is negative).
This is just a convention. But we will use it. So our magnification would be%
\[
m=\frac{h^{\prime }}{h} 
\]%
But we can write 
\begin{eqnarray*}
h &=&s\tan \theta \\
h^{\prime } &=&-s^{\prime }\tan \theta
\end{eqnarray*}%
so the magnification could be written as 
\[
m=\frac{h^{\prime }}{h}=\frac{-s^{\prime }\tan \theta }{s\tan \theta }=-%
\frac{s^{\prime }}{s} 
\]%
This is the same definition for magnification that we found for lenses.

\section{Mirror Equation}

We can further exploit this geometry to get a relationship between $s,$ $%
s^{\prime },$ and $R.$ Notice that 
\[
\tan \alpha =\frac{h}{s-R} 
\]%
and that 
\[
\tan \alpha =\frac{-h^{\prime }}{R-s^{\prime }} 
\]

Then%
\[
\frac{h}{s-R}=\frac{-h^{\prime }}{R-s^{\prime }} 
\]%
or%
\[
\frac{R-s^{\prime }}{s-R}=-\frac{h^{\prime }}{h} 
\]

We can use our magnification definition to replace $h^{\prime }/h$%
\[
\frac{R-s^{\prime }}{s-R}=\frac{s^{\prime }}{s} 
\]%
we perform some algebra%
\begin{eqnarray*}
\left( R-s^{\prime }\right) s &=&s^{\prime }\left( s-R\right) \\
-s^{\prime }s+Rs &=&ps^{\prime }-Rs^{\prime } \\
Rs+Rs^{\prime } &=&ss^{\prime }+s^{\prime }s \\
\frac{Rs^{\prime }}{Rss^{\prime }}+\frac{Rs}{Rss^{\prime }} &=&\frac{%
2ss^{\prime }}{Rss^{\prime }} \\
\frac{1}{s}+\frac{1}{s^{\prime }} &=&\frac{2}{R}
\end{eqnarray*}%
This is called the \emph{mirror equation}. Notice how similar this is to the
thin lens equation! In fact, if 
\[
\frac{1}{f}=\frac{2}{R} 
\]
would be exactly the same equation.

\subsection{Focal Point}

Now that we know the mirror equation, let's let $s$ be very large (for
example, let $s$ be the distance to the Sun). Then 
\begin{eqnarray*}
\frac{1}{\infty }+\frac{1}{s^{\prime }} &\approx &\frac{2}{R} \\
0+\frac{1}{s^{\prime }} &\approx &\frac{2}{R} \\
\frac{1}{s^{\prime }} &\approx &\frac{2}{R}
\end{eqnarray*}%
or 
\[
s^{\prime }\approx \frac{R}{2} 
\]

This is a special image point. But we really already know what to call this
special place where parallel rays come together. We call it the \emph{focal
point} and the distance from the mirror to the point $f\ $is called the 
\emph{focal length}. We see that, indeed 
\begin{equation}
f=\frac{R}{2}
\end{equation}%
so we can write the mirror equation as%
\begin{equation}
\frac{1}{s}+\frac{1}{s^{\prime }}=\frac{1}{f}
\end{equation}

For a mirror, the value of $f$ does not depend on the mirror material
because in optical mirrors there is no glass in front of the metallic
surface (this is not true for lenses optics).

\section{Ray Diagrams for Mirrors}

We have been drawing diagrams to find where images are formed for lenses, we
should do the same for mirrors. We use a similar set of three rays. These
rays are defined as follows:

\subsubsection{Principal rays for a concave mirror:}

\begin{enumerate}
\item Ray 1 is drawn from the top of the object such that its reflected ray
must pass through $f$.

\item Ray 2 is drawn from the top of the object through the focal point to
reflect parallel to the principal axis.

\item Ray 3 is drawn from the top of the object through the center of
curvature. This ray will be incident on the mirror surface at a right angle
and will be reflected back on itself.
\end{enumerate}

\FRAME{dhF}{2.8228in}{1.804in}{0pt}{}{}{Figure}{\special{language
"Scientific Word";type "GRAPHIC";maintain-aspect-ratio TRUE;display
"USEDEF";valid_file "T";width 2.8228in;height 1.804in;depth
0pt;original-width 3.7948in;original-height 2.4146in;cropleft "0";croptop
"1";cropright "1";cropbottom "0";tempfilename
'PQXXQZW2.wmf';tempfile-properties "XPR";}}

We can do the same for an object closer than a focal length\FRAME{dhF}{%
2.9482in}{1.8498in}{0pt}{}{}{Figure}{\special{language "Scientific
Word";type "GRAPHIC";maintain-aspect-ratio TRUE;display "USEDEF";valid_file
"T";width 2.9482in;height 1.8498in;depth 0pt;original-width
3.7533in;original-height 2.3454in;cropleft "0";croptop "1";cropright
"1";cropbottom "0";tempfilename 'PQXXQZW3.wmf';tempfile-properties "XPR";}}

We also may have a mirror that curves, but curves the other way.

\subsection{Principal rays for a convex mirror:}

\begin{enumerate}
\item Ray 1 is drawn from the top of the object such that its reflected ray
appears to have come from $F$.

\item Ray 2 is drawn from the top of the object to reflect parallel to the
principal axis.

\item Ray 3 is drawn from the top of the object so that it appears to have
come from the center of curvature. This ray will be incident on the mirror
surface at a right angle and will be reflected back on itself.
\end{enumerate}

\FRAME{dhF}{2.7397in}{2.1767in}{0pt}{}{}{Figure}{\special{language
"Scientific Word";type "GRAPHIC";maintain-aspect-ratio TRUE;display
"USEDEF";valid_file "T";width 2.7397in;height 2.1767in;depth
0pt;original-width 2.6974in;original-height 2.1369in;cropleft "0";croptop
"1";cropright "1";cropbottom "0";tempfilename
'PQXXQZW4.wmf';tempfile-properties "XPR";}}We should tabulate our sign
convention for mirrors like we did for lenses%
\[
\begin{tabular}{|l|l|l|}
\hline
\textbf{Quantity} & \textbf{Positive if} & \textbf{Negative if} \\ \hline
{\small Object location }$\left( s\right) $ & 
\begin{tabular}{l}
{\small Object is in front of surface} \\ 
{\small (real object)}%
\end{tabular}
& 
\begin{tabular}{l}
{\small Object is in back of surface} \\ 
{\small (virtual object)}%
\end{tabular}%
{\small \ } \\ \hline
{\small Image location }$\left( s^{\prime }\right) $ & 
\begin{tabular}{l}
{\small Image is in front of surface } \\ 
{\small (real image)}%
\end{tabular}
& 
\begin{tabular}{l}
{\small Image is in back of surface} \\ 
{\small (virtual image)}%
\end{tabular}
\\ \hline
{\small Image height }$\left( h^{\prime }\right) $ & 
\begin{tabular}{l}
{\small Image is upright}%
\end{tabular}
& 
\begin{tabular}{l}
{\small Image is inverted}%
\end{tabular}
\\ \hline
{\small Radius }$\left( R_{1}\text{ and }R_{2}\right) $ & 
\begin{tabular}{l}
{\small Center of curvature is in} \\ 
{\small front of surface}%
\end{tabular}%
{\small \ } & 
\begin{tabular}{l}
{\small Center of curvature is in} \\ 
{\small back of surface}%
\end{tabular}%
{\small \ } \\ \hline
Focal length $\left( f\right) $ & In {\small front of surface} & In {\small %
back of surface} \\ \hline
\end{tabular}%
\]

\section{Aberrations}

It's time to ask what happens when we don't use paraxial rays (when the
small angle approximation is not valid) but our equations are based on using
only small angles or paraxial rays. You might guess that we will have
problems in our imagery. and we know that imagery problems are given a
special name, \textquotedblleft aberrations.\textquotedblright

Our first aberration came from lenses refracting different colors of light
to form different images at different image locations. We called this
\textquotedblleft chromatic aberration\textquotedblright\ because it
involved a problem with colors. But now we have a problem because we made
the mirror the wring shape. It should be a parabola, but instead it is a
section of a sphere. We will call the problem this creates, \emph{spherical
aberration.}

You might wonder why we would use the wrong shape if it causes problems in
our imagery. The reason is that spherical shapes are easier to make than
parabolas or hyperbole, or other shapes. So optics manufacturers have been
using spherical optics for centuries. But we saw that our thin lens and thin
mirror equations only work if we have paraxial rays. So, we should ask, if
we used a different shape, would we be good without the paraxial
restriction? The answer is yes, but for lenses we need hyperbolic shapes,
and for mirrors we need parabolic shapes. A hyperbolic shaped lens is more
than ten times the cost of a spherical shaped lens. So often we continue to
use the spherical shapes even though they are wrong.

Spherical aberration has a similar effect to chromatic aberration. We end up
with different images of our object at different image locations. But this
time the color of the light does not matter. What matters is how the light
enters the mirror. If we let rays converge from any direction from our
spherical mirror we find that the rays do not form a single image. \FRAME{%
dtbpF}{3.5976in}{2.1975in}{0in}{}{}{Figure}{\special{language "Scientific
Word";type "GRAPHIC";maintain-aspect-ratio TRUE;display "USEDEF";valid_file
"T";width 3.5976in;height 2.1975in;depth 0in;original-width
3.55in;original-height 2.1577in;cropleft "0";croptop "1";cropright
"1";cropbottom "0";tempfilename 'PQXXQZW5.wmf';tempfile-properties "XPR";}}%
In the figure above, notice that some rays (given red lines but remember
color doesn't matter here, just the angle) converge to one focal point (the
red one) and other rays (given blue lines) converge to a different Instead,
they converge on a volume near where the image should be. Rays from larger
angles converge at different distances than rays from small angles. This
problem is known as \emph{spherical aberration}. Most of the time, we will
point our optics so the object is near the principal axis, so we can make
the paraxial approximation that fixes this problem.%
%TCIMACRO{%
%\TeXButton{Question 223.16.5}{\marginpar {
%\hspace{-0.5in}
%\begin{minipage}[t]{1in}
%\small{Question 223.16.5}
%\end{minipage}
%}}}%
%BeginExpansion
\marginpar {
\hspace{-0.5in}
\begin{minipage}[t]{1in}
\small{Question 223.16.5}
\end{minipage}
}%
%EndExpansion

The same problem happens with lenses\FRAME{dhF}{1.8178in}{1.5973in}{0pt}{}{}{%
Figure}{\special{language "Scientific Word";type
"GRAPHIC";maintain-aspect-ratio TRUE;display "USEDEF";valid_file "T";width
1.8178in;height 1.5973in;depth 0pt;original-width 2.9888in;original-height
2.623in;cropleft "0";croptop "1";cropright "1";cropbottom "0";tempfilename
'PQXXQZW6.wmf';tempfile-properties "XPR";}}Again non-paraxial and paraxial
rays focus at different spots because we have the wrong shape for our lens.
Spherical aberration was made famous as the main problem with the Hubble
Telescope.

There are many aberrations that come from making lenses that are easy to
manufacture, but that are not the perfect shape. We won't study these in
this class. If you are curious, we cover these in PH375.

\section{Combinations of Lenses}

We found that to correct chromatic aberration we used two lenses. Together
they are called an \textquotedblleft achromat\textquotedblright\ and all
good cameras use achromats to fix chromatic aberration. We can do something
similar to correct for spherical aberration. But in each case, we are
combining two lenses (or even more!). How do we predict what the lens system
will do? It turns out that we have all we need to know to combine lenses
already.

To combine lenses, we do the same thing we did for the two surfaces of a
thin lens. We form the image from the first lens as though the second lens
is not there. Then we use the image from the first lens as the object for
the second lens. Suppose we take two lenses of focal lengths $f_{1}$ and $%
f_{2}$ and place them a distance $d$ apart. \FRAME{dhF}{3.1912in}{1.2834in}{%
0pt}{}{}{Figure}{\special{language "Scientific Word";type
"GRAPHIC";maintain-aspect-ratio TRUE;display "USEDEF";valid_file "T";width
3.1912in;height 1.2834in;depth 0pt;original-width 5.9093in;original-height
2.3592in;cropleft "0";croptop "1";cropright "1";cropbottom "0";tempfilename
'PQXXQZW7.wmf';tempfile-properties "XPR";}}

Because this system would use a magnified image as the object for lens $2,$
the final magnification is the product of the two lens magnifications

\begin{equation}
m_{\text{combined}}=m_{1}m_{2}
\end{equation}

For the first lens we have%
\begin{equation}
\frac{1}{s_{1}}+\frac{1}{s_{1}^{\prime }}=\frac{1}{f_{1}}
\end{equation}%
where $s_{1}^{\prime }$ is our first lens image distance. We can solve for $%
s_{1}^{\prime }$%
\begin{equation}
s_{1}^{\prime }=\frac{s_{1}f_{1}}{s_{1}-f_{1}}  \label{q}
\end{equation}

We then take as the second object distance 
\[
s_{2}=d-s_{1}^{\prime } 
\]

we use the lens formula again.%
\[
\frac{1}{s_{2}}+\frac{1}{s_{2}^{\prime }}=\frac{1}{f_{2}} 
\]%
and again find the image distance%
\[
s_{2}^{\prime }=\frac{s_{2}f_{2}}{s_{2}-f_{2}} 
\]%
but we can use our value of $s_{2}$ to find%
\begin{eqnarray*}
s_{2}^{\prime } &=&\frac{\left( d-s_{1}^{\prime }\right) f_{2}}{\left(
d-s_{1}^{\prime }\right) -f_{2}} \\
&=&\frac{\left( d-s_{1}^{\prime }\right) f_{2}}{d-s_{1}^{\prime }-f_{2}}
\end{eqnarray*}%
We have and expression relating the image distances, $d$ and $f_{2}.$ But we
would really like to have an expression that relates $s_{1}$ and $%
s_{2}^{\prime }.$ Lets use 
\[
s_{1}^{\prime }=\frac{s_{1}f_{1}}{s_{1}-f_{1}} 
\]

and substitute it into our expression for $s_{2}^{\prime }$%
\[
s_{2}^{\prime }=\frac{\left( d-\frac{s_{1}f_{1}}{s_{1}-f_{1}}\right) f_{2}}{%
d-\frac{s_{1}f_{1}}{s_{1}-f_{1}}-f_{2}} 
\]%
This looks messy, but we can do some simplification%
\begin{equation}
s_{2}^{\prime }=\frac{df_{2}-\frac{s_{1}f_{1}f_{2}}{s_{1}-f_{1}}}{d-f_{2}-%
\frac{s_{1}f_{1}}{s_{1}-f_{1}}}  \label{TwoLensesSeparated}
\end{equation}%
Well, it is still a little messy, but we have achieved our goal. We have $%
s_{2}^{\prime }$ in therms of the focal lengths, $d,$ and $s_{1}.$

Suppose we let $d\rightarrow 0.$ Then%
\begin{eqnarray*}
s_{2}^{\prime } &=&\frac{-\frac{s_{1}f_{1}f_{2}}{s_{1}-f_{1}}}{-f_{2}-\frac{%
s_{1}f_{1}}{s_{1}-f_{1}}} \\
&=&\frac{\frac{s_{1}f_{1}f_{2}}{s_{1}-f_{1}}}{\frac{f_{2}\left(
s_{1}-f_{1}\right) }{s_{1}-f_{1}}+\frac{s_{1}f_{1}}{s_{1}-f_{1}}} \\
&=&\frac{s_{1}f_{1}f_{2}}{f_{2}s_{1}-f_{2}f_{1}+s_{1}f_{1}} \\
&=&\frac{s_{1}f_{1}f_{2}}{s_{1}\left( f_{2}+f_{1}\right) -f_{2}f_{1}}
\end{eqnarray*}%
So%
\[
s_{2}^{\prime }=\frac{s_{1}f_{1}f_{2}}{s_{1}\left( f_{2}+f_{1}\right)
-f_{2}f_{1}} 
\]%
Lets undo the math that brought us $s_{2}^{\prime }$ in the first place 
\begin{eqnarray*}
\frac{1}{s_{2}^{\prime }} &=&\frac{s_{1}\left( f_{2}+f_{1}\right) -f_{2}f_{1}%
}{s_{1}f_{1}f_{2}} \\
&=&\frac{s_{1}\left( f_{2}+f_{1}\right) }{s_{1}f_{1}f_{2}}-\frac{f_{2}f_{1}}{%
s_{1}f_{1}f_{2}} \\
&=&\frac{\left( f_{2}+f_{1}\right) }{f_{1}f_{2}}-\frac{1}{s_{1}}
\end{eqnarray*}%
or%
\[
\frac{1}{s_{2}^{\prime }}+\frac{1}{s_{1}}=\frac{\left( f_{2}+f_{1}\right) }{%
f_{1}f_{2}} 
\]%
Which looks very like the lens formula with 
\[
\frac{1}{f}=\frac{\left( f_{2}+f_{1}\right) }{f_{1}f_{2}} 
\]%
If we unwind this expression, we find%
\begin{equation}
\frac{1}{f}=\frac{f_{2}}{f_{1}f_{2}}+\frac{f_{1}}{f_{1}f_{2}}  \nonumber
\end{equation}%
\begin{equation}
\frac{1}{f}=\frac{1}{f_{1}}+\frac{1}{f_{2}}
\end{equation}

This is how we combine thin lenses. We see that the two lenses are
equivalent to a single lens with focal length $f$ as long as they are close
together.

Of course, we had to place our lenses right next to each other for this to
work. This is not the case for a telescope or microscope. We should look at
such a case. There is no need for more math. We can go back to equation (\ref%
{TwoLensesSeparated}). 
\[
s_{2}^{\prime }=\frac{df_{2}-\frac{s_{1}f_{1}f_{2}}{s_{1}-f_{1}}}{d-f_{2}-%
\frac{s_{1}f_{1}}{s_{1}-f_{1}}} 
\]%
But let's look at a case using ray diagrams. For this case, let's take two
lenses, and let's have the first lens make a real image. Once again, let's
have that image be the object for the second lens. But this time, let's move
the second lens so that the image from the first lens (object for the second
lens) is closer to the second lens than $f_{2}.$ If that is the case, the
second lens works like a magnifier. The final image is enlarged.

\FRAME{dhF}{2.6074in}{1.6622in}{0pt}{}{}{Figure}{\special{language
"Scientific Word";type "GRAPHIC";maintain-aspect-ratio TRUE;display
"USEDEF";valid_file "T";width 2.6074in;height 1.6622in;depth
0pt;original-width 4.9355in;original-height 3.1367in;cropleft "0";croptop
"1";cropright "1";cropbottom "0";tempfilename
'PQXXQZW8.wmf';tempfile-properties "XPR";}}In our next lecture, we will take
on to common optical systems that have more than one lens and see how they
work.

\chapter{The Camera and Eyes}

%TCIMACRO{%
%\TeXButton{Fundamental Concepts}{\hspace{-1.3in}{\Large Fundamental Concepts\vspace{0.25in}}}}%
%BeginExpansion
\hspace{-1.3in}{\Large Fundamental Concepts\vspace{0.25in}}%
%EndExpansion

\begin{itemize}
\item Cameras and other imaging systems use a strange term called and
f-number to tell what the intensity of the image will be%
\[
I\propto \frac{1}{\left( f/\#\right) ^{2}} 
\]%
where $f/\#$ is the symbol for f-number and the f-number is given by%
\[
f/\#\equiv \frac{f}{D} 
\]

\item Medical people measure the power of a lens in diopters. A diopter is
one over the focal length with the focal length measured in meters.
\end{itemize}

\section{The Camera}

in 1900 George Eastman introduced the Brownie Camera. This event has changed
society dramatically. The idea behind a camera is very simple.\FRAME{dhF}{%
1.9908in}{1.1329in}{0pt}{}{}{Figure}{\special{language "Scientific
Word";type "GRAPHIC";maintain-aspect-ratio TRUE;display "USEDEF";valid_file
"T";width 1.9908in;height 1.1329in;depth 0pt;original-width
3.9487in;original-height 2.2347in;cropleft "0";croptop "1";cropright
"1";cropbottom "0";tempfilename 'PQXXQZW9.wmf';tempfile-properties "XPR";}}

The camera has a lens (often a compound lens like the ones we have just
discussed) and a screen for projecting a real image created by the lens.

Let's take an example camera. Say we wish to take a picture of Aunt Sally at
a family reunion. Aunt Sally is about $1.\,\allowbreak 5\unit{m}$ tall. She
is standing about $5\unit{m}$ away. Then to fit the image of Aunt Sally on
our $35\unit{mm}$ detector, we must have $h^{\prime }=0.035\unit{m}$ so our
camera description would be

\[
\begin{tabular}{l}
$h=1.5\unit{m}$ \\ 
$h^{\prime }=0.035\unit{m}$ \\ 
$s=5\unit{m}$ \\ 
$f=0.058\unit{m}$%
\end{tabular}%
\]

We wish to find $s^{\prime }$ and $m$ so we can design our camera. Let's do $%
m$ first. 
\begin{eqnarray*}
m &=&\frac{h^{\prime }}{h}=\frac{-0.035\unit{m}}{1.5\unit{m}} \\
&=&-2.\,\allowbreak 333\,3\times 10^{-2}
\end{eqnarray*}%
so our image is small and inverted. The small size we wanted. We want to be
able to include a small sensor array to capture the image created by our
lens. Bigger arrays cost more, so reasonably small is tood. Our experience
with thing lenses tells us that we should expecte that the image would be
inverted. A digital camera uses it's built-in computer to turn the image
right side up for us on the display on the back of the camera.

Now let's find $s^{\prime }.$ From the thin lens formula we know 
\[
\frac{1}{s}+\frac{1}{s^{\prime }}=\frac{1}{f_{eq}} 
\]%
where we are treating the complicated compound lens of the camera like one
equivalent lens with a focal length $f_{eq}.$ We can rearrange the thin lens
formula to solve for $s^{\prime }$ 
\begin{eqnarray*}
s^{\prime } &=&\frac{fs}{s-f} \\
&=&5.\,\allowbreak 868\,1\times 10^{-2}\unit{m} \\
&=&58.681\unit{mm}
\end{eqnarray*}%
so our detector must be $58.681\unit{mm}$ from the lens. We build a camera
such that the distance from the lens to the detector array is $58.681\unit{mm%
}.$

Now suppose we want to photograph a $1000\unit{m}$ tower from $2\unit{km}$
away. Then%
\begin{eqnarray*}
m &=&-\frac{0.035\unit{m}}{1000\unit{m}} \\
&=&-3.\,\allowbreak 5\times 10^{-5}
\end{eqnarray*}%
and 
\begin{eqnarray*}
s^{\prime } &=&\frac{\left( 0.058\unit{m}\right) \left( 2000\unit{m}\right) 
}{2000\unit{m}-\left( 0.058\unit{m}\right) } \\
&=&5.\,\allowbreak 800\,2\times 10^{-2}\unit{m} \\
&=&58.002\unit{mm}
\end{eqnarray*}%
It appears that our camera must allow the lens-sensor distance to change.
This is why you need a focus adjustment on the lens of a good camera.
Objects far away require a different $s^{\prime }$ value than objects that
are close. Usually you twist the lens housing to make this adjustment (the
lens housing has a threaded screw system that increases or decreases $%
s^{\prime }$ as you twist. Consumer cameras often have a motor that makes
this adjustment for you. Even some cell phone cameras do this. You may see
the lens move back and forth as someone takes a picture.

There are several things that govern whether a picture will be good. When
you buy a quality manual lens, it will be marked in $f\#s$. The
specification of an automatic lens will also be given in terms of $f/\#s$.
To help us buy such lenses, we should understand what the terminology means.

Most things we want to take a photograph of are much farther than $58\unit{mm%
}$ from the camera. For such objects we can revisit the magnification.%
\[
m=-\frac{s^{\prime }}{s} 
\]%
but from the thin lens formula%
\[
\frac{1}{s}+\frac{1}{s^{\prime }}=\frac{1}{f} 
\]%
If $s\gg f$ then we can say that $1/s\approx 0$ and so $s^{\prime }\approx
f. $ Then%
\[
m\approx -\frac{f}{s} 
\]%
and we see that the size of the image is directly proportional to the focal
distance. If we change the focal distance, we can change the size of the
image. This is how a zoom lens works. Usually a zoom lens is a compound
lens, and the focal length is changed by increasing the distance between the
component lenses. This is what your camera is doing when it zooms in and out
when you push the telephoto button. Some of the image goes off the edges of
the sensor, so what is actually detected is just the center of what the lens
saw, but what you do see on the sensor is magnified.

Remember we studied intensity

\[
I=\frac{P}{A} 
\]

Photographic film and digital focal plane arrays detect the intensity of
light falling on them. We can see that the area of our image depends on our
magnification, which depends on $s^{\prime }$ and for our distant objects it
is proportional to $f.$ The image area is proportional to $h^{\prime 2}$
which is proportional to $s^{\prime 2}\approx f^{2}.$ So we can say that the
area is proportional to $f^{2}.$ Then 
\[
I\propto \frac{P}{f^{2}} 
\]%
The power entering the camera is proportional to the size of the aperture
(hole the light goes through). If the camera has a circular opening, this
area is proportional to the square of the diameter of the opening, $D^{2}$ so%
\[
I\propto \frac{D^{2}}{f^{2}} 
\]%
This ratio is useful because it tells us how much intensity we get in terms
of things we can easily know. Good cameras have changeable aperture sizes,
and good lenses have changeable focal lengths. But by using the combination
of these two terms, we can ensure we will get enough light (but not too
much) when we take the picture.

It would be good to give this number a special name. But instead, we named
the ratio 
\begin{equation}
f/\#\equiv \frac{f}{D}
\end{equation}%
It is called the $f/\#$ (pronounced f-number) so 
\begin{equation}
I\propto \frac{1}{\left( f/\#\right) ^{2}}
\end{equation}%
So good cameras have adjustable lens systems marked in $f/\#^{\prime }s$.
Typical values are $f/2.8,$ $f/4,$ $f/5.6,$ $f/8,$ $f/11,$ and $f/16.$

This terminology is use for telescope design as well. The Hubble telescope
is an $f/24$ Ritchey-Chretien Cassegrainian system with a $2.4\unit{m}$
diameter aperture. The effective focal length is $57.6\unit{m}$.

\FRAME{dhF}{3.7593in}{2.3298in}{0pt}{}{}{Figure}{\special{language
"Scientific Word";type "GRAPHIC";maintain-aspect-ratio TRUE;display
"USEDEF";valid_file "T";width 3.7593in;height 2.3298in;depth
0pt;original-width 3.7118in;original-height 2.29in;cropleft "0";croptop
"1";cropright "1";cropbottom "0";tempfilename
'PQXXQZWA.wmf';tempfile-properties "XPR";}}

\section{The Eye}

\FRAME{dhF}{3.48in}{1.3673in}{0pt}{}{}{Figure}{\special{language "Scientific
Word";type "GRAPHIC";maintain-aspect-ratio TRUE;display "USEDEF";valid_file
"T";width 3.48in;height 1.3673in;depth 0pt;original-width
3.4342in;original-height 1.3327in;cropleft "0";croptop "1";cropright
"1";cropbottom "0";tempfilename 'PQXXQZWB.wmf';tempfile-properties "XPR";}}

The figure above shows the parts of the eye. The eye is like a camera in its
operation, but is much more complex. It is truly a marvel. The parts that
concern us are the cornea, crystalline lens, pupil, and the retina. \FRAME{%
dhF}{2.1586in}{1.5584in}{0pt}{}{}{Figure}{\special{language "Scientific
Word";type "GRAPHIC";maintain-aspect-ratio TRUE;display "USEDEF";valid_file
"T";width 2.1586in;height 1.5584in;depth 0pt;original-width
4.2402in;original-height 3.0528in;cropleft "0";croptop "1";cropright
"1";cropbottom "0";tempfilename 'PQXXQZWC.wmf';tempfile-properties "XPR";}}%
The Cornea-lens system refracts the light onto the retina, which detects the
light. The lens is focused by a set of mussels that flatten the lens to
change it's focal length. The focusing process is different from a standard
camera. The camera moves the lens to achieve a different image distance. Our
eye can't change the distance between the lens system and the retina. So our
eye changes the shape of the lens, changing it's focal length. \FRAME{dhFU}{%
3.2387in}{1.446in}{0pt}{\Qcb{The crystaline lens becomes thicker, and
therefore more curved when the cilliary musscle flexes. Austin Flint,
\textquotedblleft The Eye as an Optical Instrument,\textquotedblright\ \emph{%
Popular Science Monthly,} Vol. 45, p203, 1894 (Image in the public domain)}}{%
}{Figure}{\special{language "Scientific Word";type
"GRAPHIC";maintain-aspect-ratio TRUE;display "USEDEF";valid_file "T";width
3.2387in;height 1.446in;depth 0pt;original-width 4.1987in;original-height
1.8602in;cropleft "0";croptop "1";cropright "1";cropbottom "0";tempfilename
'PQXXQZWD.wmf';tempfile-properties "XPR";}}

The focusing system is called accommodation. This system becomes less
effective at about the time you reach an age of 40 years because the lens
becomes less flexible. The closest point that can be focused by
accommodation is called the near point. It is about $25\unit{cm}$ on
average. There is, of course, no such thing as an average person, all of us
are a little bit different. You young students probably have a much shorter
near point than $25\unit{cm}.$ For those of us that are a little older, $25%
\unit{cm}$ or more is more likely.

The farthest point that can be focused is a long way away. It is called the
far point. Both the near and far points degrade with years leading to
bifocal glasses (and much irritation because you can't see as well, but I'm
really not complaining because at least I can see).

The iris changes the area of the pupil (the aperture of the eye). The pupil
is, on average, about $7\unit{mm}$ in diameter. This acts like the aperture
adjustment of a camera.

\subsection{Nearsightedness}

%TCIMACRO{%
%\TeXButton{Question 223.18.3}{\marginpar {
%\hspace{-0.5in}
%\begin{minipage}[t]{1in}
%\small{Question 223.18.3}
%\end{minipage}
%}}}%
%BeginExpansion
\marginpar {
\hspace{-0.5in}
\begin{minipage}[t]{1in}
\small{Question 223.18.3}
\end{minipage}
}%
%EndExpansion
In some people the cornea-lens system focuses in front of the retina.
Usually this is because the shape of their eyes is not spherical but is
elongated along the optic axis of the eye. This is called nearsightedness or
myopia.

\FRAME{dhF}{3.5475in}{2.4189in}{0pt}{}{}{Figure}{\special{language
"Scientific Word";type "GRAPHIC";maintain-aspect-ratio TRUE;display
"USEDEF";valid_file "T";width 3.5475in;height 2.4189in;depth
0pt;original-width 4.7547in;original-height 3.2335in;cropleft "0";croptop
"1";cropright "1";cropbottom "0";tempfilename
'PQXXQZWE.wmf';tempfile-properties "XPR";}}So their perfectly good
cornea-lens system makes a great image in the vitreous humor (the jellylike
stuff that fills the eye) and not on the retina where it would be detected.
From your experience with lenses you know this would produce a blurry image.
And that is what nearsighted people see much of the time. We can help
nearsighted people by effectively changing the cornea-lens system of the eye
by adding another lens. We want a diverging lens that makes the light more
spread apart so that the lens system of the eye can make it focus where the
retina actually is. Alternately we could flatten the cornea, itself, with
laser ablation.

\subsection{Farsightedness}

Sometimes the cornea-lens system focuses in back of the retina. This is
usually because the eye grew too flat or \textquotedblleft
oblate.\textquotedblright\ This is called farsightedness or hyperopia. Once
again we can fix the problem by adding an additional lens. This time a
converging lens.

\FRAME{dhF}{3.7593in}{2.693in}{0pt}{}{}{Figure}{\special{language
"Scientific Word";type "GRAPHIC";maintain-aspect-ratio TRUE;display
"USEDEF";valid_file "T";width 3.7593in;height 2.693in;depth
0pt;original-width 3.7118in;original-height 2.6507in;cropleft "0";croptop
"1";cropright "1";cropbottom "0";tempfilename
'PQXXQZWF.wmf';tempfile-properties "XPR";}}The converging lens will make the
effective focal length of the system shorter, so that it can form an image
where the retina actually is.

It would be convenient and very understandable if medical professionals
prescribed eye glasses by telling us what focal length we needed to correct
our vision. But that would not be the medical way! Eye glasses use a
different unit of measure to describe how they bend light. The unit is the 
\emph{diopter} and it is equal one over the focal length, but with the focal
length measured in meters. 
\begin{equation}
\text{diopter}=\frac{1}{f\left( \unit{m}\right) }
\end{equation}%
This measurement is called the \emph{power }of the lens. It is just the same
as giving the focal length, but less clear for science students.

\subsection{Color Perception}

The eye detects different colors. The respecters called cones can detect
red, green, and blue light.\FRAME{dhF}{2.4587in}{1.0326in}{0pt}{}{}{Figure}{%
\special{language "Scientific Word";type "GRAPHIC";maintain-aspect-ratio
TRUE;display "USEDEF";valid_file "T";width 2.4587in;height 1.0326in;depth
0pt;original-width 4.3656in;original-height 1.8178in;cropleft "0";croptop
"1";cropright "1";cropbottom "0";tempfilename
'PQXXQZWG.wmf';tempfile-properties "XPR";}}The eye combines the red, green,
and blue response to allow us to perceive many different colors.

Most digital cameras also have red, green, and, blue pixels to provide color
to images. The detectors in digital cameras are often have much narrower
frequency bands than the eye. Likewise, television displays and monitors
have red, green, and blue pixels. By targeting the eye receptors, power need
not be wasted in creating light that is not detected well by the eye. The
difference in band-width can cause problems in color mixing. Yellow school
busses (perceived as different amounts of green and red light) may be
reddish or green if the bandwidths are chosen poorly.

The science of human visual perception of imagery is called \emph{image
science}. There are many applications for this field, from forensics to
intelligence gathering.

\chapter{Optical Systems that Magnify}

%TCIMACRO{%
%\TeXButton{Fundamental Concepts}{\hspace{-1.3in}{\Large Fundamental Concepts\vspace{0.25in}}}}%
%BeginExpansion
\hspace{-1.3in}{\Large Fundamental Concepts\vspace{0.25in}}%
%EndExpansion

\begin{itemize}
\item Angular magnification is a comparison of how big an image looks with
and without an optical system.

\item Telescopes and Microscopes are double lens systems

\item Resolution is a name we give to the fundamental blurriness in
geometrical optical systems due to the wave nature of light.
\end{itemize}

\section{Angular Magnification}

We already encountered the simple magnifier when we studied ray diagrams.
But by this point in our study of optics you are probably wondering about
our definition of magnification. If you are an Idahoan and are out hunting,
when you look through your binoculars or scope you don't want to know how
big the image is compared the actual dear, you want to know if you can see
the deer better than you could with just your eyes. The magnification on
your scope doesn't compare the image size to the object size, but it is
comparison between two optical systems, one is just your eyes, the other is
your eyes and the scope working together.

Let's use the simple magnifier that we know to define a new kind of
magnification that does this comparison between two optical systems. We can
use what we know about easy rays to draw to describe both optical systems.
We usually use three principle rays, but for this analysis, let's just use
one for each side of an object. Since the image is on the retina, we can see
where these singular rays strike the retina and understand the size of the
image on our eye light detection system. Let's choose to draw the rays that
go straight through the middle of the lens of the eye (because they are the
easiest ones!).

If we pick a ray from the top of our object that goes through the center of
the lens, that ray won't seem to change direction at all. It will hit the
retina to form the top of the image of the object. We can do the same for
the bottom of the object. Then we can see from the next figure \FRAME{dtbpF}{%
2.367in}{0.646in}{0pt}{}{}{Figure}{\special{language "Scientific Word";type
"GRAPHIC";maintain-aspect-ratio TRUE;display "USEDEF";valid_file "T";width
2.367in;height 0.646in;depth 0pt;original-width 2.3263in;original-height
0.6149in;cropleft "0";croptop "1";cropright "1";cropbottom "0";tempfilename
'PQXXQZWH.wmf';tempfile-properties "XPR";}}that the angle $\theta _{o}$
subtends\footnote{%
The angle that \textquotedblleft subtends\textquotedblright\ and object is
the one who's bounding lines just hit either side of the object.} both the
object and the image of the object. If you think about the previous figure,
you will see that if the angle were to increase, so would the size of the
image on the retina. We can increase this angle by, say, moving the object
closer to our eyes.\FRAME{dtbpF}{3.0606in}{1.695in}{0pt}{}{}{Figure}{\special%
{language "Scientific Word";type "GRAPHIC";maintain-aspect-ratio
TRUE;display "USEDEF";valid_file "T";width 3.0606in;height 1.695in;depth
0pt;original-width 3.0173in;original-height 1.6579in;cropleft "0";croptop
"1";cropright "1";cropbottom "0";tempfilename
'PQXXQZWI.wmf';tempfile-properties "XPR";}}

When we get to our $25\unit{cm}$ we reach the limit of the eye for focusing. 
\FRAME{dtbpF}{3.4048in}{1.1208in}{0pt}{}{}{Figure}{\special{language
"Scientific Word";type "GRAPHIC";maintain-aspect-ratio TRUE;display
"USEDEF";valid_file "T";width 3.4048in;height 1.1208in;depth
0pt;original-width 3.3589in;original-height 1.0871in;cropleft "0";croptop
"1";cropright "1";cropbottom "0";tempfilename
'PQXXQZWJ.wmf';tempfile-properties "XPR";}}If we move the object any closer,
it will appear blurry. We call this position, the closest point where we can
place an object and still bring it into focus with our eye, the \emph{near
point.} Thus the maximum value of $\theta $ will be at the near point for
unaided viewing. We will call this maximum unaided angle $\theta _{o}.$

But suppose we want to see this object in more detail. We can use a
magnifying glass. \FRAME{dtbpF}{4.0171in}{2.4933in}{0pt}{}{}{Figure}{\special%
{language "Scientific Word";type "GRAPHIC";maintain-aspect-ratio
TRUE;display "USEDEF";valid_file "T";width 4.0171in;height 2.4933in;depth
0pt;original-width 3.9678in;original-height 2.4517in;cropleft "0";croptop
"1";cropright "1";cropbottom "0";tempfilename
'PQXXQZWK.wmf';tempfile-properties "XPR";}} If we place the object closer to
the magnifying glass than the focal distance$(s<f)$, then we have a virtual
image with magnification 
\begin{equation}
m=\frac{-s^{\prime }}{s}
\end{equation}%
which is larger than one and positive (because $s^{\prime }$ is negative).

But what we really want to know is how much bigger the image looks with the
lens than it did without the lens. By looking at what happens to the rays
when they enter our eye, we can see why the image looks bigger. \FRAME{dtbpF%
}{3.0692in}{2.751in}{0in}{}{}{Figure}{\special{language "Scientific
Word";type "GRAPHIC";maintain-aspect-ratio TRUE;display "USEDEF";valid_file
"T";width 3.0692in;height 2.751in;depth 0in;original-width
3.0251in;original-height 2.7086in;cropleft "0";croptop "1";cropright
"1";cropbottom "0";tempfilename 'PQXXQZWL.wmf';tempfile-properties "XPR";}}

The magnifying glass has bent the light, and the bent rays make a bigger
angle, $\theta ,$ so the image on our retina is bigger. Since the image
fills more of our retina, we perceive the image as being bigger.

\FRAME{dtbpF}{3.0268in}{2.7095in}{0in}{}{}{Figure}{\special{language
"Scientific Word";type "GRAPHIC";maintain-aspect-ratio TRUE;display
"USEDEF";valid_file "T";width 3.0268in;height 2.7095in;depth
0in;original-width 2.9827in;original-height 2.6671in;cropleft "0";croptop
"1";cropright "1";cropbottom "0";tempfilename
'PQXXQZWM.wmf';tempfile-properties "XPR";}}

We need a mathematical formula that will tell us how much bigger the image
on our retina will be. Notice from the figure that 
\begin{eqnarray*}
\tan \theta _{o} &=&\frac{h_{eye}^{\prime }}{d} \\
\tan \theta &=&\frac{h_{lens-eye}^{\prime }}{d}
\end{eqnarray*}%
so 
\begin{eqnarray*}
h_{eye}^{\prime } &=&d\tan \theta _{o} \\
h_{lens-eye}^{\prime } &=&d\tan \theta
\end{eqnarray*}%
then if we compare the new, larger image on the retina formed with the
lens-eye system to the one formed with just the eye, we get 
\[
\frac{h_{lens-eye}^{\prime }}{h_{eye}^{\prime }}=\frac{d\tan \theta }{d\tan
\theta _{o}}=\frac{\tan \theta }{\tan \theta _{o}} 
\]%
and if we once again use the small angle approximation 
\[
\frac{h_{lens-eye}^{\prime }}{h_{eye}^{\prime }}\approx \frac{\theta }{%
\theta _{o}} 
\]%
this would tell us how much bigger our object looks when viewed with the
magnifying glass compared to how it looked without the magnifying glass.
This is just what we want! Let's give this a new symbol 
\begin{equation}
M=\frac{\theta }{\theta _{o}}
\end{equation}

Remember, this is really different than the magnification we have studied
before. The magnification we have been using compared the size of the image
with the size of the object. We call $M$ the \emph{angular magnification.}

So, the angular magnification compares how big the object seems to be with
and without a lens or lens system. It is really a comparison between the
size of the real image on the retina formed with just our eye, and the one
formed with the magnifier.

If the virtual image formed is farther than the near point of the eye, ($%
s^{\prime }>\symbol{126}25\unit{cm}$) the image on our retina will be
smaller than it would be at the near point because it is farther away. If
the virtual image is closer than the near point, it will be fuzzy because
the eye cannot focus closer than the near point. Thus, the value of $M$ will
be maximum when $s^{\prime }$ for the magnifying glass is at the near point
of the eye. We can find where to place the image so that we get maximum
magnification. Taking just the magnifier, and placing the image at about $-25%
\unit{cm},$%
\begin{eqnarray*}
\frac{1}{s}+\frac{1}{s^{\prime }} &=&\frac{1}{f} \\
\frac{1}{s}+\frac{1}{-25\unit{cm}} &=&\frac{1}{f}
\end{eqnarray*}%
and so%
\[
\frac{1}{s}=\frac{-25\unit{cm}-f}{-f\left( 25\unit{cm}\right) } 
\]%
or%
\begin{equation}
s=\frac{\left( 25\unit{cm}\right) f}{25\unit{cm}+f}
\end{equation}

Using small angle approximations, we can write%
\[
\tan \theta _{o}=\frac{h}{25\unit{cm}}\approx \theta _{o} 
\]%
and 
\[
\tan \theta =\frac{h}{s}\approx \theta 
\]%
then the maximum angular magnification is%
\begin{eqnarray*}
m_{\max } &=&\frac{\theta }{\theta _{o}}=\frac{\frac{h}{s}}{\frac{h}{25\unit{%
cm}}} \\
&=&\frac{25\unit{cm}}{\frac{25\unit{cm}f}{25\unit{cm}+f}} \\
&=&\frac{25\unit{cm}+f}{f} \\
&=&1+\frac{25\unit{cm}}{f}
\end{eqnarray*}

We can also find the minimum magnification by letting $s$ be at $f.$ This
gives 
\[
\theta =\frac{h}{f} 
\]%
\begin{eqnarray*}
m_{\min } &=&\frac{\theta }{\theta _{o}}=\frac{\frac{h}{f}}{\frac{h}{25\unit{%
cm}}} \\
&=&\frac{25\unit{cm}}{f}
\end{eqnarray*}

We use the idea of a simple magnifier in combination with other lenses to
make the magnification happen in telescopes, microscopes, and other
instruments that magnify. So all of these systems can be described using the
idea of an angular magnification.

\section{The Microscope}

To see things that are very small, we add another lens to our simple
magnifier. We will place this lens near the object and call the lens the 
\emph{objective} because it is next to the object. We will keep a simple
magnifier and place it near the eye. This lens will be called the \emph{%
eyepiece} because it is near your eye.

The objective will have a very short focal length. The eyepiece will have a
longer focal length (a few centimeters).\FRAME{dtbpF}{4.7498in}{3.4158in}{0pt%
}{}{}{Figure}{\special{language "Scientific Word";type
"GRAPHIC";maintain-aspect-ratio TRUE;display "USEDEF";valid_file "T";width
4.7498in;height 3.4158in;depth 0pt;original-width 5.2855in;original-height
3.7927in;cropleft "0";croptop "1";cropright "1";cropbottom "0";tempfilename
'PQXXQZWN.wmf';tempfile-properties "XPR";}}

We separate the lenses by a distance $L$ where 
\begin{eqnarray*}
L &>&f_{o} \\
L &>&f_{e}
\end{eqnarray*}

We place the object just outside the focal point of the objective. The image
formed by the objective lens is then real and inverted. We use this image as
the object for the eyepiece. The image formed is upright and virtual, but it
looks upside down because the object for the eyepiece (first image for the
objective) is upside down.

Of course, we want to know the magnification of the system. We know what the
eyepiece will do because it is being used as just a simple magnifier. Recall
that magnifications are a factor. Think from our basic equation 
\[
m=\frac{h^{\prime }}{h} 
\]%
tells us that 
\[
h^{\prime }=mh 
\]%
or in words, $m$ is the factor by which $h^{\prime }$ is bigger (or smaller)
than $h.$ So magnifications are factors. For example, $h\prime $ could be $%
10 $ time bigger. That means if we want to know the total magnification of
the system we start with the magnification of the eyepiece $M_{e},$ say, $20$
times bigger than it would look with just our eye, but we have to account
for eyepiece's object (the image from the first lens) being bigger than the
actual object. If you think about it, it makes sense that if the objective
makes the object look $10$ times bigger, and the eyepiece makes the image
look $20$ times bigger than if you looked at it with your eye, the system
makes it look $200$ times bigger. The combined magnification for the
two-lens system is 
\begin{equation}
m=m_{o}M_{e}
\end{equation}%
The minimum magnification of the eye piece will be roughly 
\[
M_{e_{\min }}\approx \frac{25\unit{cm}}{f_{e}} 
\]%
and the maximum will be 
\[
M_{e_{\max }}=1+\frac{25\unit{cm}}{f_{e}} 
\]%
But remember the object for the eyepiece is the image from the first lens.
And that image is larger than the object by an amount \ 
\[
m_{1}=m_{o}=\frac{-s_{o}^{\prime }}{s_{o}} 
\]%
and let's estimate how big the magnification due to the first lens will be.
Because $s_{o}\approx f_{o}.$ and $s_{o}^{\prime }\approx L$ (roughly) 
\[
m_{o}=\frac{-s_{0}^{\prime }}{s_{0}}\approx -\frac{L}{f_{o}} 
\]

The combined magnification for the two-lens system is about 
\begin{equation}
m_{system}=m_{o}M_{e}=-\frac{L}{f_{o}}\frac{25\unit{cm}}{f_{e}}
\end{equation}%
this is the minimum magnification (because we used the minimum magnification
formula for the eyepiece).

\section{Telescopes}

There are two main types of telescopes \emph{refracting} and \emph{reflecting%
}. We will study refracting telescopes first.

\subsection{Refracting Telescopes}

Like the microscope, we combine two lenses and call one the objective and
the other the eyepiece. The eyepiece again plays the role of a simple
magnifier, magnifying the image produced by the objective.\FRAME{dtbpF}{%
5.5158in}{2.6083in}{0pt}{}{}{Figure}{\special{language "Scientific
Word";type "GRAPHIC";maintain-aspect-ratio TRUE;display "USEDEF";valid_file
"T";width 5.5158in;height 2.6083in;depth 0pt;original-width
5.4587in;original-height 2.5668in;cropleft "0";croptop "1";cropright
"1";cropbottom "0";tempfilename 'PQXXQZWO.wmf';tempfile-properties "XPR";}}%
We again form a real, inverted image with the objective. We are now looking
at distant objects, so the image distance $s_{o}^{\prime }\approx f_{o}.$
Once again, we use the image from the objective as the object for the
eyepiece. The eye piece forms an upright virtual image (that looks inverted
because the object for the eyepiece is the image from the objective, and the
real image from the objective is inverted). The largest magnification is
when the rays exit the eyepiece parallel to the principal axis. Then the
image from the eyepiece is formed at near infinity (but it is very big, so
it is easy to see). This gives a lens separation of $f_{o}+f_{e}$ which will
be roughly the length of the telescope tube.

The angular magnification will be 
\begin{equation}
M=\frac{\theta }{\theta _{o}}
\end{equation}%
where $\theta _{o}$ is the angle subtended by the object at the objective
(see figure above) and $\theta $ is subtended by the final image at the
viewer's eye. Consider $s_{o}$ is very large. We see from the figure that 
\begin{equation}
\tan \theta _{o}=-\frac{h_{o}^{\prime }}{f_{o}}
\end{equation}%
and with $s_{o}$ large we can use small angles. 
\begin{equation}
\theta _{o}=-\frac{h_{o}^{\prime }}{f_{o}}
\end{equation}

The angle $\theta $ will be the angle formed by rays bent by the lens of the
eyepiece. This angle will be the same as the angle formed by a ray traveling
from the tip of the first image and traveling parallel to the principal
axis. This ray is bent by the objective to pass through $f_{e.}$ Then 
\begin{equation}
\tan \theta =\frac{h_{o}^{\prime }}{f_{e}}\approx \theta
\end{equation}%
so

The magnification is then%
\begin{equation}
M=\frac{\theta }{\theta _{o}}=\frac{\frac{h^{\prime }}{f_{e}}}{-\frac{%
h^{\prime }}{f_{o}}}=-\frac{f_{o}}{f_{e}}
\end{equation}

\subsection{Reflecting Telescopes}

Reflecting telescopes use a series of mirrors to replace the objective lens.
Usually, there is an eyepiece that is refractive (although there need not
be, radio frequency telescopes rarely have refractive pieces).\FRAME{dhF}{%
3.6348in}{2.7354in}{0pt}{}{}{Figure}{\special{language "Scientific
Word";type "GRAPHIC";maintain-aspect-ratio TRUE;display "USEDEF";valid_file
"T";width 3.6348in;height 2.7354in;depth 0pt;original-width
3.5864in;original-height 2.693in;cropleft "0";croptop "1";cropright
"1";cropbottom "0";tempfilename 'PQXXQZWP.wmf';tempfile-properties "XPR";}}

There are two reasons to build reflective telescopes. The first is that
reflective optics do not suffer from chromatic aberration. The second is
that large mirrors are much easier to make and mount than refractive optics.
The Keck Observatory in Hawaii has a $10\unit{m}$ reflective system. The
largest refractive system is a $1\unit{m}$ system. The Hubble telescope has
a $2.5\unit{m}$ aperture.

The telescope pictured in the figure is a Newtonian, named after Newton, who
designed this focus mechanism. Many other designs exist. Popular designs for
space applications include the cassigrain telescope.

The rough design of a reflective telescope can be worked out using
refractive pieces, then the rough details of the reflective optics can be
formed.

\section{Resolution}

We have emphasized that an extended object can be viewed as a collection of
point objects. Then the image is formed from the collection if images of
those point objects. We thought about the design of cameras and telescopes
and other optical systems using ray optics. It would be great if optical
systems could form images with infinite precision, but it turns out they
can't. And it is the fact that light acts as a wave prevents this from being
true! Our wave nature of light comes back to complicate our simple optical
designs!

Because light is really a wave, the images of the point objects won't be
points, themselves. They will be little circular central maxima with dim
concentric ring patterns. \FRAME{dtbpF}{1.8792in}{1.7374in}{0in}{}{}{Figure}{%
\special{language "Scientific Word";type "GRAPHIC";maintain-aspect-ratio
TRUE;display "USEDEF";valid_file "T";width 1.8792in;height 1.7374in;depth
0in;original-width 1.8421in;original-height 1.7002in;cropleft "0";croptop
"1";cropright "1";cropbottom "0";tempfilename
'PQXXQZWQ.wmf';tempfile-properties "XPR";}} If we have more than one point
of light, we will get two such patterns. \FRAME{dtbpF}{1.5281in}{1.2851in}{%
0in}{}{}{Figure}{\special{language "Scientific Word";type
"GRAPHIC";maintain-aspect-ratio TRUE;display "USEDEF";valid_file "T";width
1.5281in;height 1.2851in;depth 0in;original-width 1.4909in;original-height
1.2496in;cropleft "0";croptop "1";cropright "1";cropbottom "0";tempfilename
'PQXXQZWR.wmf';tempfile-properties "XPR";}}and our image of an extended
object (like Aunt Sally) is made up of many, many points all reflecting
light into our camera lens. We want each of those points of light to create
corresponding points on our image. But instead they are creating small
circles of light, and form the above figures we can see that those circles
will overlap. All this makes our images fuzzy.

The quality of our image depends on how poorly a point object is imaged. If
each point object makes a large circle of light on the screen or detector
array, we get a very confusing image (it will look blurry to us). Let's
review why this will happen so we can know how to minimize the effect.

We already know that if we take light and pass it through a single slit, we
get an intensity pattern that has a central bright region.\FRAME{dhF}{%
2.9914in}{2.2191in}{0pt}{}{}{Figure}{\special{language "Scientific
Word";type "GRAPHIC";maintain-aspect-ratio TRUE;display "USEDEF";valid_file
"T";width 2.9914in;height 2.2191in;depth 0pt;original-width
2.9473in;original-height 2.1785in;cropleft "0";croptop "1";cropright
"1";cropbottom "0";tempfilename 'PQXXQZWS.wmf';tempfile-properties "XPR";}}

Remember that normal objects will be made up of many small points of light
(either due to reflection or glowing) and each of these will form such an
intensity pattern on a screen. Here is a bright point source that is not on
the axis, an we see that it too makes a bright spot on the screen (and
smaller bright spots or rings, depending on the shape of the aperture) 
\FRAME{dhF}{2.7674in}{2.0513in}{0pt}{}{}{Figure}{\special{language
"Scientific Word";type "GRAPHIC";maintain-aspect-ratio TRUE;display
"USEDEF";valid_file "T";width 2.7674in;height 2.0513in;depth
0pt;original-width 2.725in;original-height 2.0124in;cropleft "0";croptop
"1";cropright "1";cropbottom "0";tempfilename
'PQXXQZWT.wmf';tempfile-properties "XPR";}}So our images will be made up of
many central bright spots, each of which represents a point of light from
the object. These central bright spots may overlap, (and their secondary
maxima certainly will overlap).

Let's take a simple case of two points of light, $S_{1}$ and $S_{2}.$ If we
take a single slit and pass light from two distant point sources through the
slit, we do not get two sharp images of the point sources. Instead, we get
two diffraction patterns.\FRAME{dhF}{3.3537in}{2.6507in}{0pt}{}{}{Figure}{%
\special{language "Scientific Word";type "GRAPHIC";maintain-aspect-ratio
TRUE;display "USEDEF";valid_file "T";width 3.3537in;height 2.6507in;depth
0pt;original-width 3.3088in;original-height 2.6091in;cropleft "0";croptop
"1";cropright "1";cropbottom "0";tempfilename
'PQXXQZWU.wmf';tempfile-properties "XPR";}}

If these patterns are formed sufficiently far from each other, it is easy to
tell they were formed from two distinct objects. Each point became a small
blur, but that is really not so bad. We can still tell that the two blurs
came from different sources. If our pixel size is about the same size of the
blur, we may not even notice the blurriness in the digital imagery.\FRAME{dhF%
}{3.6487in}{2.8738in}{0pt}{}{}{Figure}{\special{language "Scientific
Word";type "GRAPHIC";maintain-aspect-ratio TRUE;display "USEDEF";valid_file
"T";width 3.6487in;height 2.8738in;depth 0pt;original-width
3.6011in;original-height 2.8305in;cropleft "0";croptop "1";cropright
"1";cropbottom "0";tempfilename 'PQXXQZWV.wmf';tempfile-properties "XPR";}}
But if the patterns are formed close to each other, it gets hard to tell
whether they were formed from two objects or one bright object. We now have
a problem. Suppose you are trying to look at a star and see if it has a
planet. But all you can see is a blur. You can't tell if there is one source
of light or two.

Long ago an early researcher titled Lord Rayleigh developed a test to
determine if you can distinguish between two diffraction patterns. When the
central maximum of one point's image falls on the first minimum of anther
point's image, the images are said to be just resolved. This test is known
as \emph{Rayleigh's criterion}.

We can find the required separation for a slit. Remember that 
\begin{equation}
\sin \left( \theta \right) =m\frac{\lambda }{a}\quad m=\pm 1,\pm 2,\pm
3\ldots
\end{equation}%
gives the minima (dark spots) for a single slit. We want the first minimum,
so $m=1$%
\begin{equation}
\sin \left( \theta \right) =\frac{\lambda }{a}
\end{equation}%
Remember that 
\[
\tan \theta =\frac{y}{L} 
\]%
where $L$ is the distance from the slit to the viewing screen, and $y$ is
the vertical location of the dark spot. If we place the second image maximum
so it is just at this location, the two images will be just barely
resolvable. In the small angle approximation, $\sin \left( \theta \right)
\approx \theta $ so%
\begin{equation}
\theta _{\min }=\frac{\lambda }{a}
\end{equation}

Now you may be saying to yourself that you don't often take pictures through
single illuminated slits, so this is nice, but not really very interesting.
But suppose, instead, that we image a circular aperture. We won't go through
all the math (there are Bessel functions involved) but the criterion becomes%
\begin{equation}
\theta _{\min }=1.22\frac{\lambda }{D}
\end{equation}%
where $D$ is the aperture diameter.

Still, you might think, \textquotedblleft I don't like pictures taken
through small circles any better than through small
slits!\textquotedblright\ Yet, in fact, you do. Most cameras have circular
apertures. The light that passes into your phone camera must pass though the
circular lens. And, of course, your eyes have circular apertures.

The Rayleigh criteria tells you, based on your camera aperture size, how a
point source will be imaged on the film or sensor array. If we consider
extended sources (like your favorite car or Aunt Sally) to be collections of
many point sources, then we have a way to tell what features will be clearly
resolved on the image and what features will not (like you may not be able
to see the lettering on the car to tell what model it is, or you may not be
able to distinguish between the gem stones in Aunt Sally's necklace because
the image is too blurry to see these features clearly).

\FRAME{dtbpFU}{3.6893in}{1.4105in}{0pt}{\Qcb{Pattern from two resolved
circular slits.}}{}{Figure}{\special{language "Scientific Word";type
"GRAPHIC";maintain-aspect-ratio TRUE;display "USEDEF";valid_file "T";width
3.6893in;height 1.4105in;depth 0pt;original-width 3.6417in;original-height
1.375in;cropleft "0";croptop "1";cropright "1";cropbottom "0";tempfilename
'PQXXQZWW.wmf';tempfile-properties "XPR";}}\FRAME{dtbpFU}{3.6893in}{1.3595in%
}{0pt}{\Qcb{Rayleigh Criteria: Pattern from two circular soruces where the
sources are close enough that the maximum from one pattern is placed on the
minimum of the other. Lord Rayleigh gave this as the criteria for just
barily being resolved.}}{}{Figure}{\special{language "Scientific Word";type
"GRAPHIC";maintain-aspect-ratio TRUE;display "USEDEF";valid_file "T";width
3.6893in;height 1.3595in;depth 0pt;original-width 3.6417in;original-height
1.3249in;cropleft "0";croptop "1";cropright "1";cropbottom "0";tempfilename
'PQXXQZWX.wmf';tempfile-properties "XPR";}}Astronomers sometimes use
Sparrow's criteria for two sources being resolved. It is shown below.\FRAME{%
dtbpFU}{3.6893in}{1.3595in}{0pt}{\Qcb{Sparrow Criteria: This is a less
concervative resolution criteria than Rayleigh. When the intenisty pattern
is flat on the top, there must be two sources. This criterial is used in
astronomy.}}{}{Figure}{\special{language "Scientific Word";type
"GRAPHIC";maintain-aspect-ratio TRUE;display "USEDEF";valid_file "T";width
3.6893in;height 1.3595in;depth 0pt;original-width 3.6417in;original-height
1.3249in;cropleft "0";croptop "1";cropright "1";cropbottom "0";tempfilename
'PQXXQZWY.wmf';tempfile-properties "XPR";}}Since astronomers just want to
know if there are two stars or one, it is enough to see that the intensity
pattern went flat at the center. That must mean that there are two stars.
But if the stars are any closer, the flat center becomes a peak and the
stars are unresolved.

\FRAME{dtbpFU}{3.6893in}{1.3595in}{0pt}{\Qcb{Two circular sources unresolved}%
}{}{Figure}{\special{language "Scientific Word";type
"GRAPHIC";maintain-aspect-ratio TRUE;display "USEDEF";valid_file "T";width
3.6893in;height 1.3595in;depth 0pt;original-width 3.6417in;original-height
1.3249in;cropleft "0";croptop "1";cropright "1";cropbottom "0";tempfilename
'PQXXQZWZ.wmf';tempfile-properties "XPR";}}Here is what the easily resolved,
Rayleigh resolved, Sparrow resolved, and unresolved cases look like. \FRAME{%
dtbpF}{3.5613in}{1.0793in}{0pt}{}{}{Figure}{\special{language "Scientific
Word";type "GRAPHIC";maintain-aspect-ratio TRUE;display "USEDEF";valid_file
"T";width 3.5613in;height 1.0793in;depth 0pt;original-width
3.5146in;original-height 1.0456in;cropleft "0";croptop "1";cropright
"1";cropbottom "0";tempfilename 'PQXXQZX0.wmf';tempfile-properties "XPR";}}

This concludes our study of waves and optics!

You may be left feeling that we have really just gotten started, and you
would be right. If you are a physics major (or curious and have elective
credit) you will study waves more in PH295 and you may study more optics in
PH375. Both are fantastic experiences. Even if you don't take a further
class in these topics, you can study them on your own. There are many
wonderful books on Optics.

Our goal this semester was to study the motion of many things and wave
motion was a partial fulfilment of that goal. But not all motions of many
objects are as uniform as wave motion. What if, say, the air molecules in
our room instead of moving in simple harmonic motion like in a wave, they
had random velocities? This is type of motion that we hinted about when we
talked about thermal energy in PH121. We will take on this kind of motion
next in this course.

\chapter{Fluids and Pressure}

We are changing topics radically. PH121 taught us how individual objects
move. So far we have learned that wave motion is an organized motion of many
objects. But often many objects move in less organized ways. We will take on
less organized motion for the last part of this course. Let's review some
basic properties of matter. Matter is made of many atoms. So matter is an
example of many objects that might just move.

%TCIMACRO{%
%\TeXButton{Fundamental Concepts}{\hspace{-1.3in}{\Large Fundamental Concepts\vspace{0.25in}}}}%
%BeginExpansion
\hspace{-1.3in}{\Large Fundamental Concepts\vspace{0.25in}}%
%EndExpansion

\begin{itemize}
\item Compressibility of fluids

\item Density of fluids

\item Pressure is a force spread over an area

\item Pressure increases with depth in a fluid
\end{itemize}

\section{Fluids}

You are probably aware that there are four states of matter

\[
\begin{tabular}{|l|}
\hline
solid \\ \hline
liquid \\ \hline
gas \\ \hline
plasma \\ \hline
\end{tabular}%
\]

Believe it or not, plasma\footnote{%
This is not the kind of plasma that you donate!} is the most common, because
stars are made of plasma. Planets are sometimes made of solids, liquids, or
gases, but the great glowing stars are plasma. (I am ignoring the mysterious
form of \textquotedblleft dark matter\textquotedblright\ because so far we
don't know what it is). Plasma is a heated gas that is ionized. We will
mostly ignore this state, because unless you are dealing with neon signs,
fluorescent lights, or the like, you don't encounter plasmas in every day
experience.

\subsection{Solids}

We can view solids as having a set of forces that keep the molecules in
place much as though they were attached using springs. Solids can have
definite organization. If so, they are called crystals. You should observe
the crystals around the Romney building if you have not already.\FRAME{dhF}{%
2.2926in}{1.6328in}{0pt}{}{}{Figure}{\special{language "Scientific
Word";type "GRAPHIC";maintain-aspect-ratio TRUE;display "USEDEF";valid_file
"T";width 2.2926in;height 1.6328in;depth 0pt;original-width
2.2528in;original-height 1.5965in;cropleft "0";croptop "1";cropright
"1";cropbottom "0";tempfilename 'PQXXQZX1.wmf';tempfile-properties "XPR";}}%
If the solid lacks definite order in its organization, it is called
amorphous.\FRAME{dhF}{1.3258in}{1.5428in}{0pt}{}{}{Figure}{\special{language
"Scientific Word";type "GRAPHIC";maintain-aspect-ratio TRUE;display
"USEDEF";valid_file "T";width 1.3258in;height 1.5428in;depth
0pt;original-width 2.0721in;original-height 2.4146in;cropleft "0";croptop
"1";cropright "1";cropbottom "0";tempfilename
'PQXXQZX2.wmf';tempfile-properties "XPR";}}

\subsection{Liquids}

The molecules in a liquid are less tightly bound than those in a solid. That
is why they can flow. In the next figure, the atoms are bound by one
spring-like force. But the atoms are not tied together in a tight set of
bonds like a solid. \FRAME{dhF}{2.9352in}{2.2329in}{0pt}{}{}{Figure}{\special%
{language "Scientific Word";type "GRAPHIC";maintain-aspect-ratio
TRUE;display "USEDEF";valid_file "T";width 2.9352in;height 2.2329in;depth
0pt;original-width 2.8911in;original-height 2.1923in;cropleft "0";croptop
"1";cropright "1";cropbottom "0";tempfilename
'PQXXQZX3.wmf';tempfile-properties "XPR";}}

\subsection{Gasses}

The molecules in a gas are not bound to each other--not at all.

\FRAME{dhF}{2.3756in}{2.2883in}{0pt}{}{}{Figure}{\special{language
"Scientific Word";type "GRAPHIC";maintain-aspect-ratio TRUE;display
"USEDEF";valid_file "T";width 2.3756in;height 2.2883in;depth
0pt;original-width 2.3359in;original-height 2.2485in;cropleft "0";croptop
"1";cropright "1";cropbottom "0";tempfilename
'PQXXQZX4.wmf';tempfile-properties "XPR";}}We have an intuitive feel for
what a fluid is. But let's make a more formal definition.

\subsection{What is a fluid?}

For the next few lectures we will study fluids, but what is a fluid?

\begin{Note}
A \textbf{Fluid} is a collection of molecules that are randomly arranged and
are at most held together by weak cohesive forces and by forces exerted by
the walls of a container.
\end{Note}

Thus, liquids and gasses are definitely fluids. Solids are generally
considered not fluids. But how about Jello%
%TCIMACRO{\TeXButton{TM}{\textsuperscript{\texttrademark}}}%
%BeginExpansion
\textsuperscript{\texttrademark}%
%EndExpansion
? or a combination of corn starch and water (sometimes called
\textquotedblleft ooblick\textquotedblright )? These are non-Newtonian
fluids. That is, things that are sort of solid and sort of not. But we will
stick with things (at least at first) that are definitely fluids, and
generally fluids with negligible friction. This is a little like in PH121
when we studied frictionless surfaces. How many surfaces are truly
frictionless? Very few! you might guess that there are few fluids that have
no friction, and you would be right. But just like with PH121, the
assumption of frictionless fluids makes the math easier, and that is good
when we are starting a now topic.

\section{Pressure}

%TCIMACRO{%
%\TeXButton{Vollyball Demo}{\marginpar {
%\hspace{-0.5in}
%\begin{minipage}[t]{1in}
%\small{Vollyball Demo}
%\end{minipage}
%}}}%
%BeginExpansion
\marginpar {
\hspace{-0.5in}
\begin{minipage}[t]{1in}
\small{Vollyball Demo}
\end{minipage}
}%
%EndExpansion
We have already seen pressure in this course. But it's been a while, so
let's review. Consider a situation where we ask six of our class members to
come up and press on the ball from all directions. Suppose further that we
ask each person to exert a force on the ball. And suppose we ask the person
to use the area of their hand to exert the force. The motion of the ball,
and even it's shape depended on both the force (magnitude and direction) and
the area involved in each push.

Noticing that each person is exerting a force but that the force is not
acting on one point, but is spread out over an area, we recognize that each
person is exerting a pressure on the ball\ 
\begin{equation}
P\equiv \frac{F}{A}
\end{equation}%
\mathstrut

Now consider a ball sitting in a room surrounded by air. The air is a fluid,
so it's molecules are quite free to move around. Because there is some
thermal energy in the room (even in Rexburg) the molecules will have some
kinetic energy. So the air molecules will hit the ball. This will cause a
force on the surface of the ball. And that force will be spread over the
entire surface area of the ball. This is a force spread over an area. This
is a pressure. We call this \emph{air pressure}. This air force due to the
colliding molecules is like having the hands on the pushing on the ball.

This force due to individual molecules is small and only lasts during the
collision. But in the room we have many molecules, and many the molecules
impact the ball. The molecules also impact the walls of the room. Suppose
that every time a molecule bounces back from one wall it ends up headed back
to the opposite wall bounces back again toward the first wall. If the
molecules keep coming, there will be a force on the wall quite a bit of the
time. At least, on average there is a force, anyway. This is the force that
causes air pressure. The molecules impact the walls, and the ball, and us,
and everything in the room all over the surface area of each object. The
result is air pressure on every object in the room.

Likewise, the water pressure in a swimming pool is caused by moving water
molecules. You should convince yourself that the reason the water stays in
the pool is partly because the air molecules bounce against the water
surface exerting a pressure on the water!

\subsection{Working with the definition of pressure}

We can work with equation \ref{Pressure Definition} to define the force due
to pressure%
\[
F\equiv PA 
\]%
and can define a force due to the pressure at an element of area $dA$ 
\[
dF\equiv PdA 
\]%
Where there is a differential, expect that some time in the future we will
integrate!

But before we go on, let's see what the units of pressure would be. We have
a force divided by an area.

\[
1\allowbreak \frac{\unit{N}}{\unit{m}^{2}}\allowbreak =1\unit{Pa} 
\]%
The symbol is the $\unit{Pa},$ and the unit is called the \emph{pascal}.
This is the name of a famous scientist.

\subsection{Pressure Example:}

Let's do a pressure problem together,

Problem statement:

A $50.0\unit{kg}$ woman balances on one heel of a pair of high heeled shoes.
If the heel is circular and has a radius of $0.5000\unit{cm},$ what pressure
does she exert on the floor?

Drawing

\FRAME{dhF}{1.1044in}{0.4877in}{0pt}{}{}{Figure}{\special{language
"Scientific Word";type "GRAPHIC";maintain-aspect-ratio TRUE;display
"USEDEF";valid_file "T";width 1.1044in;height 0.4877in;depth
0pt;original-width 1.0706in;original-height 0.4575in;cropleft "0";croptop
"1";cropright "1";cropbottom "0";tempfilename
'PQXXQZX5.wmf';tempfile-properties "XPR";}}

Variables

\[
\begin{tabular}{lll}
& Known &  \\ 
$M$ & Mass of woman & $M=50\unit{kg}$ \\ 
$r$ & Radius of woman's heal & $r=0.500\unit{cm}$ \\ 
$g$ & acceleration do to gravity & $g=9.8\frac{\unit{m}}{\unit{s}^{2}}$ \\ 
& Unknown &  \\ 
$F$ & Force &  \\ 
$x$ & Coordinate Axis &  \\ 
$z$ & Coordinate Axis &  \\ 
$A$ & Area of woman's heal & 
\end{tabular}%
\]

Basic Equations

\begin{eqnarray*}
F &=&ma \\
A &=&\pi r^{2} \\
P &=&\frac{F}{A}
\end{eqnarray*}

Symbolic Solution

\[
A=\pi r^{2} 
\]

\[
F=ma=Mg 
\]%
\[
P=\frac{F}{A}=\frac{Mg}{\pi r^{2}} 
\]

Numerical Solution

\begin{eqnarray*}
P &=&\frac{Mg}{\pi r^{2}} \\
&=&\frac{50\unit{kg}\ast 9.8\frac{\unit{m}}{\unit{s}^{2}}}{\pi \left( 0.500%
\unit{cm}\right) ^{2}} \\
&=&\frac{490.0\frac{\unit{m}}{\unit{s}^{2}}\unit{kg}}{\allowbreak 0.785\,40%
\unit{cm}^{2}}\frac{\left( 100\unit{cm}\right) ^{2}}{\unit{m}^{2}} \\
&=&\frac{490.0\frac{\unit{m}}{\unit{s}^{2}}\unit{kg}}{\allowbreak 0.785\,40}%
\frac{100000}{\unit{m}^{2}} \\
&=&6.\,\allowbreak 238\,9\times 10^{6}\allowbreak \frac{\unit{kg}\unit{m}}{%
\unit{s}^{2}\unit{m}^{2}} \\
&=&6.\,\allowbreak 238\,9\times 10^{6}\unit{Pa}
\end{eqnarray*}%
\[
P=6.24\unit{MPa} 
\]

Units Check

\[
\frac{\unit{kg}\frac{\unit{m}}{\unit{s}^{2}}}{\unit{cm}^{2}}=\frac{\unit{kg}%
\frac{\unit{m}}{\unit{s}^{2}}}{\unit{cm}^{2}}\frac{\left( 100\unit{cm}%
\right) ^{2}}{\unit{m}^{2}}=10000\frac{\unit{kg}\frac{\unit{m}}{\unit{s}^{2}}%
}{\unit{m}^{2}}=10000\unit{Pa} 
\]

Units

Check

Reasonableness

This seems like a large number, but I have had a high heeled person step on
my toe, so I believe it!

%TCIMACRO{%
%\TeXButton{Bed of Nails}{\marginpar {
%\hspace{-0.5in}
%\begin{minipage}[t]{1in}
%\small{Bed of Nails}
%\end{minipage}
%}}}%
%BeginExpansion
\marginpar {
\hspace{-0.5in}
\begin{minipage}[t]{1in}
\small{Bed of Nails}
\end{minipage}
}%
%EndExpansion

\section{Variation of Pressure with Depth}

%TCIMACRO{%
%\TeXButton{Stacked Hymnbook}{\marginpar {
%\hspace{-0.5in}
%\begin{minipage}[t]{1in}
%\small{Stacked Hymnbook}
%\end{minipage}
%}}}%
%BeginExpansion
\marginpar {
\hspace{-0.5in}
\begin{minipage}[t]{1in}
\small{Stacked Hymnbook}
\end{minipage}
}%
%EndExpansion

If you have been in a swimming pool, you probably have noticed that the
pressure due to the water feels different at the surface than it does at the
bottom of the deep end. Intuitively, we can say that the pressure at the
bottom is larger than the pressure at the top. Let's see if we can show that
this is true.

Take a glass of water or some other fluid. Let's look at just a part of the
fluid (the darker section in figure \ref{Pressure Variation initial}). Let's
treat this \textquotedblleft parcel of fluid\textquotedblright\ as a
distinct body and look at the forces acting on it.

We need some way of labeling our forces. My way is kind of simple, but will
work for now.\FRAME{dtbpF}{2.405in}{0.8925in}{0pt}{}{}{Figure}{\special%
{language "Scientific Word";type "GRAPHIC";maintain-aspect-ratio
TRUE;display "USEDEF";valid_file "T";width 2.405in;height 0.8925in;depth
0pt;original-width 3.3702in;original-height 1.2324in;cropleft "0";croptop
"1";cropright "1";cropbottom "0";tempfilename
'PQXXQZX6.wmf';tempfile-properties "XPR";}}

Consider Newton's second law. The sum of the forces must be equal to zero
(Why?--think of the acceleration of the parcel of fluid)\FRAME{dtbpF}{%
2.8599in}{1.8628in}{0pt}{}{}{Figure}{\special{language "Scientific
Word";type "GRAPHIC";maintain-aspect-ratio TRUE;display "USEDEF";valid_file
"T";width 2.8599in;height 1.8628in;depth 0pt;original-width
2.8167in;original-height 1.8256in;cropleft "0";croptop "1";cropright
"1";cropbottom "0";tempfilename 'PRX4W400.wmf';tempfile-properties "XPR";}}%
\[
\overrightarrow{F}_{net}=0=\mathbf{F}_{pt}+\mathbf{F}_{pb}+\mathbf{F}_{pmr}+%
\mathbf{F}_{pml}+\mathbf{W} 
\]%
\begin{eqnarray*}
F_{net_{x}} &=&0=0+0-F_{pmr}+F_{pml}-0 \\
F_{net_{y}} &=&0=-F_{pt}+F_{pb}-0+0-W
\end{eqnarray*}

The $x$-part tells us that 
\[
F_{pmr}=F_{pml} 
\]%
which is not too much of a surprise. It seems reasonable that the side
forces must be equal if the parcel doesn't accelerate. The $y$-part gives%
\[
F_{pb}-F_{pt}=W 
\]

The definition of pressure gives%
\begin{eqnarray*}
P &=&\frac{F}{A} \\
F &=&PA
\end{eqnarray*}%
\FRAME{dtbpF}{2.751in}{1.8706in}{0pt}{}{}{Figure}{\special{language
"Scientific Word";type "GRAPHIC";maintain-aspect-ratio TRUE;display
"USEDEF";valid_file "T";width 2.751in;height 1.8706in;depth
0pt;original-width 2.7086in;original-height 1.8334in;cropleft "0";croptop
"1";cropright "1";cropbottom "0";tempfilename
'PSYENW00.wmf';tempfile-properties "XPR";}}

So, for the top of our parcel of fluid, the force on the top must be%
\[
F_{pt}=P_{t}A 
\]%
and for the bottom the force must be%
\[
F_{pb}=P_{b}A 
\]

Recalling that 
\begin{equation}
m=\rho 
%TCIMACRO{\TeXButton{V}{\ooalign{\hfil$V$\hfil\cr\kern0.1em--\hfil\cr}}}%
%BeginExpansion
\ooalign{\hfil$V$\hfil\cr\kern0.1em--\hfil\cr}%
%EndExpansion
\end{equation}%
where $\rho $ is the density and $%
%TCIMACRO{\TeXButton{V}{\ooalign{\hfil$V$\hfil\cr\kern0.1em--\hfil\cr}}}%
%BeginExpansion
\ooalign{\hfil$V$\hfil\cr\kern0.1em--\hfil\cr}%
%EndExpansion
$ is the volume, and substituting%
\begin{eqnarray*}
F_{pb}-F_{pt} &=&W \\
P_{b}A-P_{t}A &=&-mg \\
\left( P_{b}-P_{t}\right) A &=&-\rho Vg \\
\left( P_{b}-P_{t}\right) A &=&-\rho hAg
\end{eqnarray*}%
where we have used the fact that the volume must be the area of the top or
bottom, $A,$ multiplied by the length, $h,$ of our parcel. 
\begin{equation}
%TCIMACRO{\TeXButton{V}{\ooalign{\hfil$V$\hfil\cr\kern0.1em--\hfil\cr}}}%
%BeginExpansion
\ooalign{\hfil$V$\hfil\cr\kern0.1em--\hfil\cr}%
%EndExpansion
=Ah
\end{equation}%
\FRAME{dtbpF}{1.5947in}{1.8706in}{0pt}{}{}{Figure}{\special{language
"Scientific Word";type "GRAPHIC";maintain-aspect-ratio TRUE;display
"USEDEF";valid_file "T";width 1.5947in;height 1.8706in;depth
0pt;original-width 1.5584in;original-height 1.8334in;cropleft "0";croptop
"1";cropright "1";cropbottom "0";tempfilename
'PSYENW01.wmf';tempfile-properties "XPR";}}

We can solve this for the pressure at the bottom 
\[
P_{b}=P_{t}+\rho gh 
\]

Units Check

\[
\unit{Pa}\Bumpeq \unit{Pa}+\frac{\unit{kg}}{\unit{m}^{3}}\frac{\unit{m}}{%
\unit{s}^{2}}\unit{m} 
\]%
the last term can be written as%
\[
\left( \frac{\unit{kg}\unit{m}}{\unit{s}^{2}}\right) \frac{1}{\unit{m}^{2}}%
\Rightarrow \frac{\unit{N}}{\unit{m}^{2}}\Rightarrow \unit{Pa} 
\]%
so the units check.

This is a profound statement! (and a new basic equation for us). The
pressure is larger at the bottom of the pool than the top. And it is larger
by the amount $\rho gh$ where $\rho $ is the water density, $g$ is the
acceleration due to gravity, and $h$ is how far down we go to get to the
bottom.

Note that we used a box shaped volume, but the result is general. To see
this, view any arbitrary volume as consisting of little boxes. As we make
the box size small, we approximate the actual volume (seem familiar from
calculus?). Since it works for each box that makes up the volume
individually, it will work for the whole volume (it takes a minute to think
about this).

\section{Pascal's Law}

Let's step back and look at our new equation%
\[
P_{b}=P_{t}+\rho gh 
\]%
This gives the pressure at some depth, knowing the pressure at the top. On
what does $P_{b}$ depend? We can see that it only depends on $P_{t},$ $\rho
, $ $g,$ and $h.$ Suppose I\ take a complex system like a hydraulic jack

\FRAME{dtbpF}{2.258in}{1.3085in}{0pt}{}{}{Figure}{\special{language
"Scientific Word";type "GRAPHIC";maintain-aspect-ratio TRUE;display
"USEDEF";valid_file "T";width 2.258in;height 1.3085in;depth
0pt;original-width 2.2182in;original-height 1.2739in;cropleft "0";croptop
"1";cropright "1";cropbottom "0";tempfilename
'PQXXQZX7.wmf';tempfile-properties "XPR";}}Suppose the pressure on the left
hand side at the top is $P_{t\left( left\right) }.$ Then the pressure at the
left hand side bottom would be 
\[
P_{b(left)}=P_{t(left)}+\rho gh_{left} 
\]%
What is the pressure along the bottom of the jack system? Well we don't know
a value, but we know it is the same pressure all along the bottom. It can
only depend on $h$ (since $P_{t},$ $\rho ,$ $g$ are the same and $h$ is the
same simply because we are looking at a location along the bottom). So%
\[
P_{b(left)}=P_{b(right)} 
\]%
then the pressure on the top right must be 
\[
P_{t\left( right\right) }=P_{b\left( right\right) }-\rho gh_{right} 
\]

We could write this as 
\begin{eqnarray*}
P_{t\left( right\right) } &=&P_{b(left)}-\rho gh_{right} \\
&=&P_{t(left)}+\rho gh_{left}-\rho gh_{right}
\end{eqnarray*}

Now suppose I\ increase the pressure at the top at the left by pushing down
(increasing the force) on the piston on the left side. Then%
\begin{equation}
P_{t\left( left\right) }\rightarrow P_{t\left( left\right) }+\Delta P
\end{equation}%
The pressure at the bottom of the left side will change too%
\begin{eqnarray*}
P_{b\left( left\right) } &=&\left( P_{t\left( left\right) }+\Delta P\right)
+\rho gh \\
&=&\Delta P+P_{t\left( left\right) }+\rho gh
\end{eqnarray*}%
And since the pressure at the bottom left changes, and the pressure at the
bottom right must be the same as the pressure at the bottom left, the
pressure must change all along the whole bottom. 
\begin{eqnarray*}
P_{b\left( right\right) } &=&P_{b\left( left\right) } \\
&=&P_{t\left( left\right) }+\Delta P+\rho gh_{left}
\end{eqnarray*}%
and from what we did before we can see that the pressure at the top right
must be 
\begin{eqnarray*}
P_{t\left( right\right) } &=&P_{b\left( right\right) }-\rho gh_{right} \\
&=&(P_{t\left( left\right) }+\Delta P+\rho gh_{left})-\rho gh_{right} \\
&=&(P_{t\left( left\right) }+\Delta P+\rho gh_{left})-\rho gh_{right} \\
&=&(P_{t\left( left\right) }+\rho gh_{left}-\rho gh_{right})+\Delta P
\end{eqnarray*}%
which is the pressure we had on the right side before the change plus the
exact same change in pressure that we applied on the left side! We can see
that we have changed the pressure at the top right as well, and by the same
amount. This is an amazingly simple result! If I\ change the pressure at one
point in the fluid, I\ automatically change the pressure in the rest of the
fluid. This result is so useful it has a name.

\begin{Note}
Pascal's Law: \textbf{a change in the pressure applied to a fluid is
transmitted undiminished to every point of the fluid and to the walls of the
container.}
\end{Note}

There is an assumption that we have made in stating Pascal's law. We have
assumed that the fluid is incompressible. Pascal's law won't work for air,
because we can compress air. But it will work for oil or water because these
fluids are (mostly) incompressible.

So now there is a pressure change at piston $2.$ Since we changed the
pressure $P_{1}$ from $P_{t}$ to $P_{t}+\Delta P,$ and we know that this
pressure change will be transmitted to all parts of the fluid, so we can
tell that at piston $2$ we will change the pressure from it's original
pressure, $P_{2i}$ to $P_{2i}+\Delta P.$

The jack works because the areas of the two pistons are different. $A_{1}$
is small, and $A_{2}$ is large. Then an applied force 
\[
F_{1}=\Delta PA_{1} 
\]%
will cause a force 
\[
F_{2}=\Delta PA_{2} 
\]%
on the second piston Because $A_{2}>A_{1}$ then 
\[
F_{2}>F_{1} 
\]

In many cases we can sum up pressure over an area because the pressure will
be the same at every point within the area. Our hydraulic jack is one of
these cases. In our next lecture we will use Pascal's principle and our new
pressure as a function of depth equation in some important problems.

\chapter{Using and Measuring Pressure and Buoyancy}

%TCIMACRO{%
%\TeXButton{Fundamental Concepts}{\hspace{-1.3in}{\Large Fundamental Concepts\vspace{0.25in}}}}%
%BeginExpansion
\hspace{-1.3in}{\Large Fundamental Concepts\vspace{0.25in}}%
%EndExpansion

\begin{enumerate}
\item Details of a Hydraulic Press (Pascal's Law)

\item Barometers

\item Manometers

\item Buoyancy
\end{enumerate}

\section{Hydraulic Press}

\FRAME{dtbpF}{2.3981in}{2.0306in}{0in}{}{}{Figure}{\special{language
"Scientific Word";type "GRAPHIC";maintain-aspect-ratio TRUE;display
"USEDEF";valid_file "T";width 2.3981in;height 2.0306in;depth
0in;original-width 2.3583in;original-height 1.9917in;cropleft "0";croptop
"1";cropright "1";cropbottom "0";tempfilename
'PQXXQZX8.wmf';tempfile-properties "XPR";}}

Let's start were we left off last lecture, with a hydraulic jack. We usually
have a constant force acting on the jack area.

\begin{equation}
F=PA
\end{equation}%
To begin with we usually have atmospheric pressure pushing on the jack
piston, 
\begin{equation}
F_{atm}=P_{atm}A_{in}
\end{equation}%
but we wish to add a force $F_{in}$ to this. So on the input side, we have,
using Newton's second law 
\begin{equation}
\Sigma F=F_{atm}+F_{in}=P_{atm}A_{in}+\Delta P_{in}A_{in}
\end{equation}%
where 
\begin{equation}
F_{in}=\Delta P_{in}A_{in}
\end{equation}%
is due to our push on the jack input. We will ignore the atmospheric
pressure for now, since we only care about a change in pressure for this
problem. By changing the pressure on the input side we have changed the
pressure by $\Delta P_{in}.$ On the output side (the lifting side) we still
have%
\begin{equation}
F=PA
\end{equation}%
but we expect our $\Delta P_{in}$ to be transmitted throughout the entire
fluid. Then we can just call it $\Delta P$ (no $in$ ). So the output will
have an amount of force added to it. On this side%
\begin{equation}
F_{out}=\Delta P_{out}A_{out}=\Delta PA_{out}
\end{equation}%
since $\Delta P$ in both equations is the same when the two sides are at the
same elevation, then%
\begin{equation}
\frac{F_{out}}{A_{out}}=\frac{F_{in}}{A_{in}}
\end{equation}

So how does your car's hydraulic jack work?%
\begin{equation}
F_{out}=F_{in}\frac{A_{out}}{A_{in}}
\end{equation}%
If $A_{out}>A_{in}$ a much smaller $F_{in}$ can produce a large $F_{out}$
(you already knew that, didn't you!).

It is important to note that we have assumed our jack fluid is not
compressible. So the volume of fluid leaving the cylinder at the input side
must be the same as the volume of fluid entering the output side. Since $%
A_{out}>A_{in,}$ it is clear that the output piston will travel a much
smaller distance than the input piston. This is why you have to pump quite a
lot on the input side of your jack to move a car a relatively small distance.

\section{Pressure Measurements}

Now that we understand pressure and how it changes in fluids, we can study
several pressure measurement devices.

\subsection{The Barometer}

%TCIMACRO{%
%\TeXButton{Bring a Barometer}{\marginpar {
%\hspace{-0.5in}
%\begin{minipage}[t]{1in}
%\small{Bring a Barometer}
%\end{minipage}
%}}}%
%BeginExpansion
\marginpar {
\hspace{-0.5in}
\begin{minipage}[t]{1in}
\small{Bring a Barometer}
\end{minipage}
}%
%EndExpansion
%TCIMACRO{%
%\TeXButton{Test Tube Barometer}{\marginpar {
%\hspace{-0.5in}
%\begin{minipage}[t]{1in}
%\small{Test Tube Barometer}
%\end{minipage}
%}}}%
%BeginExpansion
\marginpar {
\hspace{-0.5in}
\begin{minipage}[t]{1in}
\small{Test Tube Barometer}
\end{minipage}
}%
%EndExpansion
The barometer is a common pressure measuring device. It consists of \FRAME{%
dtbpF}{1.2799in}{2.0578in}{0in}{}{}{Figure}{\special{language "Scientific
Word";type "GRAPHIC";maintain-aspect-ratio TRUE;display "USEDEF";valid_file
"T";width 1.2799in;height 2.0578in;depth 0in;original-width
1.2755in;original-height 2.0684in;cropleft "0";croptop "1";cropright
"1";cropbottom "0";tempfilename 'PQXXQZX9.wmf';tempfile-properties "XPR";}}

\begin{enumerate}
\item a long tube closed at one end

\item a dish

\item Mercury or another fluid
\end{enumerate}

The dish and the tube are filled with the fluid. The tube is inverted in the
dish. The pressure at the top of the tube is essentially zero. This is
because the weight of the fluid pulls the mass of fluid downward. If the
fluid is mercury, it is massive enough to leave a vacuum behind.

The pressure at point $A$ and point $B$ must be the same (or fluid would
flow until $P_{A}=P_{B}$).

It would be useful to find out how high up the mercury (or other fluid) will
stay 
\[
P_{b}=P_{t}+\rho gh 
\]%
At the top we have zero pressure, so $P_{t}=0.$ Note, this is not
atmospheric pressure, \textbf{it is vacuum}. Then for this special case 
\begin{equation}
P_{b}=0+\rho gh
\end{equation}

And we can solve for the height of the fluid%
\begin{equation}
h=\frac{P_{b}}{\rho g}
\end{equation}%
Let's consider what $P_{b}$ must be. The air outside the barometer pushes
down on the fluid. When we first construct a barometer, we fill the tall
tube with fluid and then turn it upside down in the dish. The fluid will
fall until the force due to air pressure matches the weight of the column of
fluid. Then the pressure at $B$ must be atmospheric pressure. Since the
pressure at point $A$ equals the pressure at point $B,$ we can use this as a
convenient $P_{b}.$ So 
\[
P_{b}=P_{atm} 
\]%
and 
\begin{equation}
h=\frac{P_{atm}}{\rho g}
\end{equation}%
Thus as the atmospheric air pressure changes, the height of the mercury
changes. We often hear of atmospheric pressure given in $\unit{mm}$ of $Hg$.
This is why.%
%TCIMACRO{%
%\TeXButton{Question 123.5.1}{\marginpar {
%\hspace{-0.5in}
%\begin{minipage}[t]{1in}
%\small{Question 123.5.1}
%\end{minipage}
%}}}%
%BeginExpansion
\marginpar {
\hspace{-0.5in}
\begin{minipage}[t]{1in}
\small{Question 123.5.1}
\end{minipage}
}%
%EndExpansion

\subsection{Manometer}

%TCIMACRO{%
%\TeXButton{Question 123.5.2}{\marginpar {
%\hspace{-0.5in}
%\begin{minipage}[t]{1in}
%\small{Question 123.5.2}
%\end{minipage}
%}}}%
%BeginExpansion
\marginpar {
\hspace{-0.5in}
\begin{minipage}[t]{1in}
\small{Question 123.5.2}
\end{minipage}
}%
%EndExpansion
The manometer finds an unknown pressure \FRAME{dtbpF}{1.5629in}{1.962in}{0in%
}{}{}{Figure}{\special{language "Scientific Word";type
"GRAPHIC";maintain-aspect-ratio TRUE;display "USEDEF";valid_file "T";width
1.5629in;height 1.962in;depth 0in;original-width 1.5646in;original-height
1.9709in;cropleft "0";croptop "1";cropright "1";cropbottom "0";tempfilename
'PQXXQZXA.wmf';tempfile-properties "XPR";}}The atmospheric pressure $P_{atm}$
is applied on one end of the tube. In our example it is on the right side,
where the tube is open at the top. The pressure to be tested is applied to
the other side, in our case, the left side. The pressure at point $A$ must
equal the pressure at point $B$. Why? (think of their heights and $\rho gh$%
). The pressure at point $A$ is the pressure to be measured. Again 
\begin{equation}
P_{A}=P_{t}+\rho gh
\end{equation}%
where we take $P_{A}$ to be the test pressure, and solve for the height $h.$

\section{Buoyant Forces, Archimedes' Principle}

%TCIMACRO{%
%\TeXButton{Question 123.5.3}{\marginpar {
%\hspace{-0.5in}
%\begin{minipage}[t]{1in}
%\small{Question 123.5.3}
%\end{minipage}
%}}}%
%BeginExpansion
\marginpar {
\hspace{-0.5in}
\begin{minipage}[t]{1in}
\small{Question 123.5.3}
\end{minipage}
}%
%EndExpansion
Let's investigate the net force on our block of water that we have used
before. We found the net force must be zero, or it would be accelerating. 
\begin{eqnarray*}
\Sigma F_{x} &=&ma_{x}=0=P_{m}A_{side}-P_{m}A_{side} \\
\Sigma F_{y} &=&ma_{y}=0=-P_{atm}A_{top}+P_{bottom}A_{bottom}-W
\end{eqnarray*}%
But suppose I\ ignore the weight of the parcel of water, $W=mg$. Then there
would be a net force due to all other forces in the $\mathbf{\hat{\jmath}}$
direction. That force must be matching the downward force due to gravity.%
\[
W=P_{bottom}A_{bottom}-P_{atm}A_{top} 
\]%
We could give this net force due to all the water pressure forces a symbol,
say \textquotedblleft $B$.\textquotedblright 
\[
W=P_{bottom}A_{bottom}-P_{atm}A_{top}=B 
\]%
\FRAME{dtbpF}{2.1698in}{1.4157in}{0pt}{}{}{Figure}{\special{language
"Scientific Word";type "GRAPHIC";maintain-aspect-ratio TRUE;display
"USEDEF";valid_file "T";width 2.1698in;height 1.4157in;depth
0pt;original-width 4.4201in;original-height 2.8746in;cropleft "0";croptop
"1";cropright "1";cropbottom "0";tempfilename
'PQXXQZXB.wmf';tempfile-properties "XPR";}}Then $B$ would be equal to the
weight of the parcel of water!

Now take a beach ball sized parcel (piece) of water\FRAME{dtbpF}{2.6844in}{%
2.0306in}{0in}{}{}{Figure}{\special{language "Scientific Word";type
"GRAPHIC";maintain-aspect-ratio TRUE;display "USEDEF";valid_file "T";width
2.6844in;height 2.0306in;depth 0in;original-width 2.6411in;original-height
1.9917in;cropleft "0";croptop "1";cropright "1";cropbottom "0";tempfilename
'PRX52703.wmf';tempfile-properties "XPR";}}The parcel is in equilibrium just
as before. We know there is a force on the parcel of water due to gravity.
It is downward. Just as before, the ball-shaped parcel of water is not
sinking so again there is no net force. Then if we add up all the forces due
to the water pressure, only, we must have a net pressure force that is
upward to balance the downward force due to gravity. Again this is what we
called \textquotedblleft $B$.\textquotedblright\ And again $B$ is equal to
the weight of the parcel of water.

The idea of a net force due to pressure due to a fluid is so useful we give
it its own name We will call the net force due to pressure from the fluid
the \emph{Buoyant force}.

Note that this is \textbf{not} the net force, it is just the sum of the
forces due to the water pressure. It does not include gravity or any tension
or spring or any other forces. It just contains the pressure forces due to
the fluid.

Note also that the buoyant force is not some new kind of force. It is a name
we give to the sum of the pressure forces due acting on an object in the
fluid. Like we call a group of soldiers a battalion or a group of cows a
heard, the sum of the group of pressure forces due to a fluid is given the
name \textquotedblleft buoyant force.\textquotedblright\ But herds and
battalions are not new things, they are groups of cows and soldiers. A
Buoyant force is not a new force, it is the vector sum of a group of
pressure forces.

Because the parcel is not sinking, we can determine that in this special
case 
\begin{equation}
B=W_{\text{fluid parcel}}\qquad \text{Floating}
\end{equation}%
for this situation that we call floating (these are magnitudes, are the
directions the same?). The term on the right is the weight of the water or
fluid.

Suppose we replace this amount of water with the beach ball. The weight of
the beach ball is different than the weight of the water parcel. But will
any of the pressure forces be different? The two volumes are exactly the
same!

It turns out that the buoyant force will be exactly the same for the beach
ball as it would for the beach ball-shaped parcel of water. We can calculate
the buoyant force by thinking of replacing the actual object we have with an
equal volume parcel of water. The buoyant force will be equal to the weight
of that equal volume of the fluid. We could say that to replace the water
with the beach ball we have \emph{displaced} a beach ball volume's worth of
water. The water that was at the location of the beach ball is now somewhere
else, so it is displaced. Then the buoyant force would be equal to the
weight of the water that was displaced.%
\begin{equation}
B=W_{\text{displaced fluid}}
\end{equation}%
Of course, the weight of the beach ball is far less than the weight of the
water displaced%
\[
W_{ball}<W_{\text{displaced fluid}} 
\]%
and this is why the beach ball accelerates upward.

It turns out that this idea of the buoyant force being equal to the weight
of an equal volume of fluid is general! This concept of the Buoyant force
being equal to the weight of the water that fits inside our volume is called 
\emph{Archimedes' principle}.

\begin{Note}
The magnitude of the buoyant force equals the weight of the fluid displaced
by the object.
\end{Note}

%TCIMACRO{%
%\TeXButton{Question 123.5.4}{\marginpar {
%\hspace{-0.5in}
%\begin{minipage}[t]{1in}
%\small{Question 123.5.4}
%\end{minipage}
%}}}%
%BeginExpansion
\marginpar {
\hspace{-0.5in}
\begin{minipage}[t]{1in}
\small{Question 123.5.4}
\end{minipage}
}%
%EndExpansion

\section{Buoyant Force: Two Cases}

We can gain insight into this concept of a buoyant force by considering more
floating objects. Let's consider an object that is floating submerged in the
fluid (like a fish or submarine) and one that is floating at the surface and
partially extends out of the fluid (like a floating log or boat).

\subsection{Totally submerged object}

\FRAME{dtbpF}{2.0557in}{1.7789in}{0in}{}{}{Figure}{\special{language
"Scientific Word";type "GRAPHIC";maintain-aspect-ratio TRUE;display
"USEDEF";valid_file "T";width 2.0557in;height 1.7789in;depth
0in;original-width 2.0176in;original-height 1.7417in;cropleft "0";croptop
"1";cropright "1";cropbottom "0";tempfilename
'PRX58W05.wmf';tempfile-properties "XPR";}}Consider a solid object, a block
of some material, placed in water. The weight of the solid object is $Mg.$

The magnitude of the buoyant force is equal to the weight of an equal volume
of water. If we imagine replacing the block with a block-sized volume of
water, then 
\begin{eqnarray*}
B &=&m_{_{fluid}}g \\
&=&\rho _{_{fluid}}%
%TCIMACRO{\TeXButton{V}{\ooalign{\hfil$V$\hfil\cr\kern0.1em--\hfil\cr}}}%
%BeginExpansion
\ooalign{\hfil$V$\hfil\cr\kern0.1em--\hfil\cr}%
%EndExpansion
_{object}g \\
&=&\rho _{_{fluid}}g%
%TCIMACRO{\TeXButton{V}{\ooalign{\hfil$V$\hfil\cr\kern0.1em--\hfil\cr}}}%
%BeginExpansion
\ooalign{\hfil$V$\hfil\cr\kern0.1em--\hfil\cr}%
%EndExpansion
_{block}
\end{eqnarray*}%
Think Archimedes: The weight of an equal volume of water was displaced by
the block. So we have the volume of the block as part of our equation for
the buoyant force.

If the object has mass $M$ then we can write the mass of the block as 
\begin{eqnarray*}
M &=&\rho _{object}%
%TCIMACRO{\TeXButton{V}{\ooalign{\hfil$V$\hfil\cr\kern0.1em--\hfil\cr}}}%
%BeginExpansion
\ooalign{\hfil$V$\hfil\cr\kern0.1em--\hfil\cr}%
%EndExpansion
_{object} \\
&=&\rho _{block}%
%TCIMACRO{\TeXButton{V}{\ooalign{\hfil$V$\hfil\cr\kern0.1em--\hfil\cr}}}%
%BeginExpansion
\ooalign{\hfil$V$\hfil\cr\kern0.1em--\hfil\cr}%
%EndExpansion
_{block}
\end{eqnarray*}%
and the weight of the block is then 
\begin{eqnarray*}
W &=&Mg \\
&=&\rho _{object}%
%TCIMACRO{\TeXButton{V}{\ooalign{\hfil$V$\hfil\cr\kern0.1em--\hfil\cr}}}%
%BeginExpansion
\ooalign{\hfil$V$\hfil\cr\kern0.1em--\hfil\cr}%
%EndExpansion
_{object}g \\
&=&\rho _{block}%
%TCIMACRO{\TeXButton{V}{\ooalign{\hfil$V$\hfil\cr\kern0.1em--\hfil\cr}}}%
%BeginExpansion
\ooalign{\hfil$V$\hfil\cr\kern0.1em--\hfil\cr}%
%EndExpansion
_{block}g
\end{eqnarray*}%
The net force is then 
\begin{eqnarray*}
F_{net} &=&B-W=\rho _{fluid}g%
%TCIMACRO{\TeXButton{V}{\ooalign{\hfil$V$\hfil\cr\kern0.1em--\hfil\cr}}}%
%BeginExpansion
\ooalign{\hfil$V$\hfil\cr\kern0.1em--\hfil\cr}%
%EndExpansion
_{object}-\rho _{object}%
%TCIMACRO{\TeXButton{V}{\ooalign{\hfil$V$\hfil\cr\kern0.1em--\hfil\cr}}}%
%BeginExpansion
\ooalign{\hfil$V$\hfil\cr\kern0.1em--\hfil\cr}%
%EndExpansion
_{object}g \\
&=&\rho _{fluid}g%
%TCIMACRO{\TeXButton{V}{\ooalign{\hfil$V$\hfil\cr\kern0.1em--\hfil\cr}}}%
%BeginExpansion
\ooalign{\hfil$V$\hfil\cr\kern0.1em--\hfil\cr}%
%EndExpansion
_{block}-\rho _{block}%
%TCIMACRO{\TeXButton{V}{\ooalign{\hfil$V$\hfil\cr\kern0.1em--\hfil\cr}}}%
%BeginExpansion
\ooalign{\hfil$V$\hfil\cr\kern0.1em--\hfil\cr}%
%EndExpansion
_{block}g \\
&=&g%
%TCIMACRO{\TeXButton{V}{\ooalign{\hfil$V$\hfil\cr\kern0.1em--\hfil\cr}}}%
%BeginExpansion
\ooalign{\hfil$V$\hfil\cr\kern0.1em--\hfil\cr}%
%EndExpansion
_{block}(\rho _{fluid}-\rho _{block})
\end{eqnarray*}%
Consider this last equation. If the density of the block is large, larger
than the density of the fluid, then $(\rho _{fluid}-\rho _{block})$ is
negative, and the force will be negative. The block will accelerate
downward. If the density of the block is smaller than the density of the
fluid, then $(\rho _{fluid}-\rho _{block})$ will be positive and the block
will accelerate upward. If the density of the block is just the same as the
density of the fluid, the block will float in place (or at least sink or
float upward at a constant rate. A submarine adjusts it's density to dive or
to surface.

We did this calculation for our block, but it could have been any object. So
in general%
\begin{equation}
F_{net}=g%
%TCIMACRO{\TeXButton{V}{\ooalign{\hfil$V$\hfil\cr\kern0.1em--\hfil\cr}}}%
%BeginExpansion
\ooalign{\hfil$V$\hfil\cr\kern0.1em--\hfil\cr}%
%EndExpansion
_{object}(\rho _{fluid}-\rho _{object})
\end{equation}

If the object has a density that is less than the fluid, the object will be
accelerated upward. And if the density of the object is greater than the
density of the fluid the object will be accelerated downward.%
%TCIMACRO{%
%\TeXButton{bb in a test tube demo}{\marginpar {
%\hspace{-0.5in}
%\begin{minipage}[t]{1in}
%\small{bb in a test tube demo}
%\end{minipage}
%}}}%
%BeginExpansion
\marginpar {
\hspace{-0.5in}
\begin{minipage}[t]{1in}
\small{bb in a test tube demo}
\end{minipage}
}%
%EndExpansion

Note that the motion of a totally submerged object is determined only by the
relative density! In a way, this could be confusing because we did not allow
our object to change volume as it moved. If we had an object that could
change size, then it changes density as it changes size. A balloon full of
air released from the bottom of a swimming pool does change volume as it
rises. So in that case we would need to know $\Delta 
%TCIMACRO{\TeXButton{V}{\ooalign{\hfil$V$\hfil\cr\kern0.1em--\hfil\cr}}}%
%BeginExpansion
\ooalign{\hfil$V$\hfil\cr\kern0.1em--\hfil\cr}%
%EndExpansion
_{object}$ and how the density, $\rho _{object},$ changes as the balloon
moves.

\subsection{Partially submerged object}

%TCIMACRO{%
%\TeXButton{Question 123.5.5}{\marginpar {
%\hspace{-0.5in}
%\begin{minipage}[t]{1in}
%\small{Question 123.5.5}
%\end{minipage}
%}}}%
%BeginExpansion
\marginpar {
\hspace{-0.5in}
\begin{minipage}[t]{1in}
\small{Question 123.5.5}
\end{minipage}
}%
%EndExpansion

Now let's consider a partially submerged object\FRAME{dtbpF}{1.9545in}{%
1.8204in}{0in}{}{}{Figure}{\special{language "Scientific Word";type
"GRAPHIC";maintain-aspect-ratio TRUE;display "USEDEF";valid_file "T";width
1.9545in;height 1.8204in;depth 0in;original-width 1.9164in;original-height
1.7832in;cropleft "0";croptop "1";cropright "1";cropbottom "0";tempfilename
'PRX58904.wmf';tempfile-properties "XPR";}}

Let's assume $\rho _{object}<\rho _{fluid}$ (why? think of $\rho
_{fluid}-\rho _{object}$). We assume static equilibrium, which means we are
observing this floating object when it is just sitting still relative to the
objects around it, not accelerating.

Since $a=0$ then 
\[
\Sigma F=ma=0 
\]%
and%
\[
\Sigma F=B-W 
\]%
so for an object floating on the surface 
\[
B=-W 
\]

Lets call the volume of the fluid displaced by the object $%
%TCIMACRO{\TeXButton{V}{\ooalign{\hfil$V$\hfil\cr\kern0.1em--\hfil\cr}}}%
%BeginExpansion
\ooalign{\hfil$V$\hfil\cr\kern0.1em--\hfil\cr}%
%EndExpansion
_{fluid}$. Now this is not the same as $%
%TCIMACRO{\TeXButton{V}{\ooalign{\hfil$V$\hfil\cr\kern0.1em--\hfil\cr}}}%
%BeginExpansion
\ooalign{\hfil$V$\hfil\cr\kern0.1em--\hfil\cr}%
%EndExpansion
_{object}$ for this case! Part of the object is sticking out of the fluid,
so $%
%TCIMACRO{\TeXButton{V}{\ooalign{\hfil$V$\hfil\cr\kern0.1em--\hfil\cr}}}%
%BeginExpansion
\ooalign{\hfil$V$\hfil\cr\kern0.1em--\hfil\cr}%
%EndExpansion
_{fluid}<%
%TCIMACRO{\TeXButton{V}{\ooalign{\hfil$V$\hfil\cr\kern0.1em--\hfil\cr}}}%
%BeginExpansion
\ooalign{\hfil$V$\hfil\cr\kern0.1em--\hfil\cr}%
%EndExpansion
_{object}.$ But, it is still true that the weight of the displaced volume of
fluid gives the buoyant force 
\[
B=\rho _{fluid}g%
%TCIMACRO{\TeXButton{V}{\ooalign{\hfil$V$\hfil\cr\kern0.1em--\hfil\cr}}}%
%BeginExpansion
\ooalign{\hfil$V$\hfil\cr\kern0.1em--\hfil\cr}%
%EndExpansion
_{fluid} 
\]

Then returning to our force equation%
\begin{eqnarray*}
B &=&-W \\
\rho _{fluid}g%
%TCIMACRO{\TeXButton{V}{\ooalign{\hfil$V$\hfil\cr\kern0.1em--\hfil\cr}}}%
%BeginExpansion
\ooalign{\hfil$V$\hfil\cr\kern0.1em--\hfil\cr}%
%EndExpansion
_{fluid} &=&-\rho _{object}%
%TCIMACRO{\TeXButton{V}{\ooalign{\hfil$V$\hfil\cr\kern0.1em--\hfil\cr}}}%
%BeginExpansion
\ooalign{\hfil$V$\hfil\cr\kern0.1em--\hfil\cr}%
%EndExpansion
_{object}g
\end{eqnarray*}%
or%
\begin{equation}
\frac{%
%TCIMACRO{\TeXButton{V}{\ooalign{\hfil$V$\hfil\cr\kern0.1em--\hfil\cr}}}%
%BeginExpansion
\ooalign{\hfil$V$\hfil\cr\kern0.1em--\hfil\cr}%
%EndExpansion
_{fluid}}{%
%TCIMACRO{\TeXButton{V}{\ooalign{\hfil$V$\hfil\cr\kern0.1em--\hfil\cr}}}%
%BeginExpansion
\ooalign{\hfil$V$\hfil\cr\kern0.1em--\hfil\cr}%
%EndExpansion
_{object}}=\frac{\rho _{object}}{\rho _{fluid}}
\end{equation}%
The fraction of the volume of a floating object that is below the fluid
surface is equal to the ratio of the density of the object to that of the
fluid.

%TCIMACRO{%
%\TeXButton{Question 123.5.5}{\marginpar {
%\hspace{-0.5in}
%\begin{minipage}[t]{1in}
%\small{Question 123.5.5}
%\end{minipage}
%}}}%
%BeginExpansion
\marginpar {
\hspace{-0.5in}
\begin{minipage}[t]{1in}
\small{Question 123.5.5}
\end{minipage}
}%
%EndExpansion
In the case of an iceberg, not all of the ice is visible above the surface.
The density of ice is about $920\frac{\unit{kg}}{\unit{m}^{3}}$ and that of
sea water is about $1000\frac{\unit{kg}}{\unit{m}^{3}}.$ So for ice%
\[
\frac{%
%TCIMACRO{\TeXButton{V}{\ooalign{\hfil$V$\hfil\cr\kern0.1em--\hfil\cr}}}%
%BeginExpansion
\ooalign{\hfil$V$\hfil\cr\kern0.1em--\hfil\cr}%
%EndExpansion
_{fluid}}{%
%TCIMACRO{\TeXButton{V}{\ooalign{\hfil$V$\hfil\cr\kern0.1em--\hfil\cr}}}%
%BeginExpansion
\ooalign{\hfil$V$\hfil\cr\kern0.1em--\hfil\cr}%
%EndExpansion
_{object}}=\frac{920\frac{\unit{kg}}{\unit{m}^{3}}}{1030\frac{\unit{kg}}{%
\unit{m}^{3}}}=0.893\,2 
\]%
\[
%TCIMACRO{\TeXButton{V}{\ooalign{\hfil$V$\hfil\cr\kern0.1em--\hfil\cr}}}%
%BeginExpansion
\ooalign{\hfil$V$\hfil\cr\kern0.1em--\hfil\cr}%
%EndExpansion
_{fluid}=0.89%
%TCIMACRO{\TeXButton{V}{\ooalign{\hfil$V$\hfil\cr\kern0.1em--\hfil\cr}}}%
%BeginExpansion
\ooalign{\hfil$V$\hfil\cr\kern0.1em--\hfil\cr}%
%EndExpansion
_{object} 
\]%
that is the fluid displaced is $89\%$ of the ice. Just about $10\%$ of the
ice sticks up out of the water. I \FRAME{dtbpFU}{3.5302in}{2.3931in}{0pt}{%
\Qcb{{\protect\small \ Iceberg with both beautiful blue-green submerged
portion and a reflection of the surface ice and snow. Approximately 90 per
cent of the iceberg is submerged. Antarctica, Palmer Peninsula, Northern
area. (Dr. Mike Goebel, NOAA NMFS SWFSC, NOAA NMFS SWFSC Antarctic Marine
Living Resources (AMLR) Program.)}}}{}{Figure}{\special{language "Scientific
Word";type "GRAPHIC";maintain-aspect-ratio TRUE;display "USEDEF";valid_file
"T";width 3.5302in;height 2.3931in;depth 0pt;original-width
3.5692in;original-height 2.4099in;cropleft "0";croptop "1";cropright
"1";cropbottom "0";tempfilename 'PQXXQZXC.wmf';tempfile-properties "XPR";}}

\chapter{Conservation of Energy for Fluid Flow}

So far, we have only dealt with fluids in equilibrium. The topic of fluid
dynamics is a complicated mathematical field that requires differential
equations to do with any exactness. But with some simplifying assumptions,
we can get a feel for how fluids flow.

%TCIMACRO{%
%\TeXButton{Fundamental Concepts}{\hspace{-1.3in}{\Large Fundamental Concepts\vspace{0.25in}}}}%
%BeginExpansion
\hspace{-1.3in}{\Large Fundamental Concepts\vspace{0.25in}}%
%EndExpansion

\begin{enumerate}
\item Continuity

\item Bernoulli's Equation
\end{enumerate}

\section{Moving Fluids}

To start with, let's make some definitions.

New Definitions: \FRAME{dtbpF}{2.7086in}{3.5803in}{0in}{}{}{Figure}{\special%
{language "Scientific Word";type "GRAPHIC";maintain-aspect-ratio
TRUE;display "USEDEF";valid_file "T";width 2.7086in;height 3.5803in;depth
0in;original-width 2.6671in;original-height 3.5336in;cropleft "0";croptop
"1";cropright "1";cropbottom "0";tempfilename
'PQXXQZXD.wmf';tempfile-properties "XPR";}}

%TCIMACRO{%
%\TeXButton{Question 123.6.1}{\marginpar {
%\hspace{-0.5in}
%\begin{minipage}[t]{1in}
%\small{Question 123.6.1}
%\end{minipage}
%}}}%
%BeginExpansion
\marginpar {
\hspace{-0.5in}
\begin{minipage}[t]{1in}
\small{Question 123.6.1}
\end{minipage}
}%
%EndExpansion
%TCIMACRO{%
%\TeXButton{Question 123.6.2}{\marginpar {
%\hspace{-0.5in}
%\begin{minipage}[t]{1in}
%\small{Question 123.6.2}
%\end{minipage}
%}}}%
%BeginExpansion
\marginpar {
\hspace{-0.5in}
\begin{minipage}[t]{1in}
\small{Question 123.6.2}
\end{minipage}
}%
%EndExpansion
%TCIMACRO{%
%\TeXButton{Question 123.6.3}{\marginpar {
%\hspace{-0.5in}
%\begin{minipage}[t]{1in}
%\small{Question 123.6.3}
%\end{minipage}
%}}}%
%BeginExpansion
\marginpar {
\hspace{-0.5in}
\begin{minipage}[t]{1in}
\small{Question 123.6.3}
\end{minipage}
}%
%EndExpansion

\begin{enumerate}
\item Laminar: Steady flow, if each particle of the fluid follows a smooth
path, such that the paths of the different particles never cross each other.

\item Turbulent: non-laminar flow characterized by whirlpool or eddy regions

\item Viscosity: term describing the internal friction of a fluid (Honey is
more viscus than water)

\item Irrotational: flow having no angular momentum

\item Streamline: a line indicating the path taken by a particle under
steady flow
\end{enumerate}

Sadly, we will limit our study to laminar, nonviscus, irrotational,
incompressible fluids. This is a little bit like saying we will only study
frictionless surfaces in PH121! But to go beyond non-viscus laminar flow we
need the math of partial differential equations, which is not a prereq for
this class. So we will have to content ourselves with this restriction. For
those of you who are interested (or are ME\ majors) you can take our junior
level fluid dynamic course taught by the Mechanical Engineering Department.

\section{Equation of continuity}

%TCIMACRO{%
%\TeXButton{Question 123.6.4}{\marginpar {
%\hspace{-0.5in}
%\begin{minipage}[t]{1in}
%\small{Question 123.6.4}
%\end{minipage}
%}}}%
%BeginExpansion
\marginpar {
\hspace{-0.5in}
\begin{minipage}[t]{1in}
\small{Question 123.6.4}
\end{minipage}
}%
%EndExpansion

\FRAME{dtbpF}{2.1638in}{2.6083in}{0in}{}{}{Figure}{\special{language
"Scientific Word";type "GRAPHIC";maintain-aspect-ratio TRUE;display
"USEDEF";valid_file "T";width 2.1638in;height 2.6083in;depth
0in;original-width 2.1248in;original-height 2.5668in;cropleft "0";croptop
"1";cropright "1";cropbottom "0";tempfilename
'PRX5DD07.wmf';tempfile-properties "XPR";}}

Let's consider a pipe that changes in size and position. We allow an ideal
(non-compressable) fluid to flow through the pipe in laminar flow. The
segment of fluid is the shaded part in the top part of figure \ref%
{Continuity}. In a time $\Delta t$ the fluid at the left hand side moves a
distance 
\[
\Delta x_{1}=v_{1}\Delta t 
\]

If $A_{1}$ is the area of the pipe at the left hand side (LHS) then the mass
contained in the shaded region (bottom left part of figure \ref{Continuity})
is%
\[
m_{1}=\rho A_{1}\Delta x_{1}=\rho A_{1}v_{1}\Delta t 
\]%
where $\rho $ is the density of the fluid. Likewise, for the RHS region in
the bottom of figure \ref{Continuity} 
\[
m_{2}=\rho A_{2}\Delta x_{2}=\rho A_{2}v_{2}\Delta t 
\]%
Unless we create, destroy, or pool mass, the mass that crosses $A_{2}$ must
equal the mass that crosses $A_{1}$%
\begin{eqnarray*}
m_{1} &=&m_{2} \\
\rho A_{1}v_{1}\Delta t &=&\rho A_{2}v_{2}\Delta t
\end{eqnarray*}%
or%
\begin{equation}
A_{1}v_{1}=A_{2}v_{2}
\end{equation}%
This is the \emph{equation of continuity for fluids} (remember that we
assumed ideal fluids).

Since we assumed that no mass was created, destroyed, or pooled as the fluid
flowed, this is what we might call \textquotedblleft conservation of
mass.\textquotedblright\ And as usual, when something is conserved in a
problem, we can use it to solve problems. With this equation, knowing the
diameter of a pipe and how it changes tells us about the speed of the fluid
in the pipe and how it changes. Of course, there could be situations where
mass is not conserved (say, a leaky pipe). So conservation of mass is an
assumption that we have to check. But for now let's assume that the equation
of continuity holds for our problems.

\section{Bernoulli's Equation}

It would be great if we could find the pressure as well as the speed of the
fluid inside our non-uniform pipe. We would need an equation that included
both pressure and speed. Let's try to find such and equation!

In PH121 we developed to ways of looking at motion problems. One was the
force picture where we had to deal with the vector nature of forces and
motion. The other was the energy picture where we could solve problems
without considering directions (but we gave up being able to find
directions). Usually the energy picture made solving problems easier. And
except for the flow direction, we don't really want to know the direction of
the motion of each molecule in our fluid, so giving up direction doesn't
seem too bad. So, the work energy theorem is something we know from PH121.
Let's start there, 
\[
W=\Delta K+\Delta U 
\]%
We still need to use forces to find the work done on the parcel of water,
but if we choose our axes carefully, we can keep the work equation in one
dimension. \FRAME{dtbpF}{2.1923in}{1.1857in}{0pt}{}{}{Figure}{\special%
{language "Scientific Word";type "GRAPHIC";maintain-aspect-ratio
TRUE;display "USEDEF";valid_file "T";width 2.1923in;height 1.1857in;depth
0pt;original-width 3.058in;original-height 1.6414in;cropleft "0";croptop
"1";cropright "1";cropbottom "0";tempfilename
'PRX5J40B.wmf';tempfile-properties "XPR";}}

\subsection{Work done in $\Delta t$}

Let's review a little of what we learned in PH121. Suppose we have two guys,
Normal Guy and Super Guy.

\FRAME{dtbpF}{1.6986in}{1.3216in}{0in}{}{}{Figure}{\special{language
"Scientific Word";type "GRAPHIC";maintain-aspect-ratio TRUE;display
"USEDEF";valid_file "T";width 1.6986in;height 1.3216in;depth
0in;original-width 1.7021in;original-height 1.3181in;cropleft "0";croptop
"1";cropright "1";cropbottom "0";tempfilename
'PQXXQZXE.wmf';tempfile-properties "XPR";}}And suppose they want to get to
the top of a tall building. Normal guy has to take the stairs, but Super guy
can jump right up to the top. \FRAME{dtbpF}{2.3549in}{2.5944in}{0in}{}{}{%
Figure}{\special{language "Scientific Word";type
"GRAPHIC";maintain-aspect-ratio TRUE;display "USEDEF";valid_file "T";width
2.3549in;height 2.5944in;depth 0in;original-width 2.3709in;original-height
2.6157in;cropleft "0";croptop "1";cropright "1";cropbottom "0";tempfilename
'PQXXQZXF.wmf';tempfile-properties "XPR";}}Who has done more work?

We know the change in potential energy is the same for both. And we know
that 
\[
W=-\Delta U 
\]%
so the amount of work is the same for both cases! Now let's return to our
fluid flow. \FRAME{dtbpF}{3.6729in}{1.6284in}{0pt}{}{}{Figure}{\special%
{language "Scientific Word";type "GRAPHIC";maintain-aspect-ratio
TRUE;display "USEDEF";valid_file "T";width 3.6729in;height 1.6284in;depth
0pt;original-width 3.6253in;original-height 1.5921in;cropleft "0";croptop
"1";cropright "1";cropbottom "0";tempfilename
'PRX6FP0G.wmf';tempfile-properties "XPR";}}

On the left hand side (LHS) of our pipe we have a force due to pressure from
the fluid in the rest of the LHS\ of the pipe. That fluid has molecules that
will bang in to our small marked parcel of fluid. The force from the LHS
fluid in the rest of the pipe will be 
\begin{equation}
F_{1}=P_{1}A_{1}\widehat{\mathbf{i}}
\end{equation}

On the right hand side (RHS) there is also more fluid in the rest of the RHS
of the pipe. Our fluid will bang into that fluid, and since the fluid is
non-compressible, the rest of the fluid in the RHS\ will push back. The
force will be%
\begin{equation}
F_{2}=-P_{2}A_{2}\widehat{\mathbf{i}}
\end{equation}

From PH121 we remember that works is 
\[
w=\int \mathbf{F}\cdot \mathbf{dx} 
\]%
and since our force is not changing this would be simply%
\begin{equation}
w=\mathbf{F}\cdot \mathbf{\Delta x}
\end{equation}

We need to find the work done in moving our particular parcel of fluid from
the left hand side to the right hand side. As the parcel moves in the $\hat{%
\imath}$ direction the amount of work will slowly change (like Normal Guy
going up the stairs) but really the amount of work only depends on the
beginning and ending conditions. So we could ignore all the details of
moving the parcel and envision the parcel jumping from the left up to the
right side of the pipe (like in the Super Guy case).

Then for the LHS 
\[
W_{1}=P_{1}A_{1}\Delta x_{1} 
\]

and for the RHS 
\[
W_{2}=-P_{2}A_{2}\Delta x_{2} 
\]%
The total work will be the sum of these two works. But before we add them
together, let's make some substitutions that will make our final formula for
the total work more meaningful.

Lets define the volume of our fluid segment

\begin{equation}
%TCIMACRO{\TeXButton{V}{\ooalign{\hfil$V$\hfil\cr\kern0.1em--\hfil\cr}}}%
%BeginExpansion
\ooalign{\hfil$V$\hfil\cr\kern0.1em--\hfil\cr}%
%EndExpansion
=A_{1}\Delta x_{1}
\end{equation}%
Note that this volume is NOT the volume of the entire segment. It is just
the part that is $\Delta x_{1}$ long. Then from the same arguments that lead
to the equation of continuity, we see that the volume of the marked part of
the fluid must not change as it flows%
%TCIMACRO{%
%\TeXButton{Question 123.6.5}{\marginpar {
%\hspace{-0.5in}
%\begin{minipage}[t]{1in}
%\small{Question 123.6.5}
%\end{minipage}
%}}}%
%BeginExpansion
\marginpar {
\hspace{-0.5in}
\begin{minipage}[t]{1in}
\small{Question 123.6.5}
\end{minipage}
}%
%EndExpansion
%TCIMACRO{%
%\TeXButton{Question 123.6.6}{\marginpar {
%\hspace{-0.5in}
%\begin{minipage}[t]{1in}
%\small{Question 123.6.6}
%\end{minipage}
%}}}%
%BeginExpansion
\marginpar {
\hspace{-0.5in}
\begin{minipage}[t]{1in}
\small{Question 123.6.6}
\end{minipage}
}%
%EndExpansion
\[
%TCIMACRO{\TeXButton{V}{\ooalign{\hfil$V$\hfil\cr\kern0.1em--\hfil\cr}}}%
%BeginExpansion
\ooalign{\hfil$V$\hfil\cr\kern0.1em--\hfil\cr}%
%EndExpansion
=A_{1}\Delta x_{1}=A_{2}\Delta x_{2} 
\]%
The we can write the work equations as%
\[
W_{1}=P_{1}V 
\]%
and 
\[
W_{2}=-P_{2}%
%TCIMACRO{\TeXButton{V}{\ooalign{\hfil$V$\hfil\cr\kern0.1em--\hfil\cr}}}%
%BeginExpansion
\ooalign{\hfil$V$\hfil\cr\kern0.1em--\hfil\cr}%
%EndExpansion
\]%
So finally, the total work done is 
\begin{eqnarray}
W &=&W_{1}+W_{2} \\
&=&P_{1}%
%TCIMACRO{\TeXButton{V}{\ooalign{\hfil$V$\hfil\cr\kern0.1em--\hfil\cr}}}%
%BeginExpansion
\ooalign{\hfil$V$\hfil\cr\kern0.1em--\hfil\cr}%
%EndExpansion
-P_{2}%
%TCIMACRO{\TeXButton{V}{\ooalign{\hfil$V$\hfil\cr\kern0.1em--\hfil\cr}} }%
%BeginExpansion
\ooalign{\hfil$V$\hfil\cr\kern0.1em--\hfil\cr}
%EndExpansion
\nonumber \\
&=&\left( P_{1}-P_{2}\right) 
%TCIMACRO{\TeXButton{V}{\ooalign{\hfil$V$\hfil\cr\kern0.1em--\hfil\cr}} }%
%BeginExpansion
\ooalign{\hfil$V$\hfil\cr\kern0.1em--\hfil\cr}
%EndExpansion
\nonumber
\end{eqnarray}%
so, from the work-energy theorem%
\begin{eqnarray*}
W &=&\Delta K+\Delta U \\
\left( P_{1}-P_{2}\right) 
%TCIMACRO{\TeXButton{V}{\ooalign{\hfil$V$\hfil\cr\kern0.1em--\hfil\cr}} }%
%BeginExpansion
\ooalign{\hfil$V$\hfil\cr\kern0.1em--\hfil\cr}
%EndExpansion
&=&\Delta K+\Delta U
\end{eqnarray*}%
We have an expression for the left hand side of our work-energy theorem! Now
for the right hand side $\Delta K+\Delta U$

\subsection{Kinetic Energy}

\FRAME{dtbpF}{2.4664in}{1.8049in}{0pt}{}{}{Figure}{\special{language
"Scientific Word";type "GRAPHIC";maintain-aspect-ratio TRUE;display
"USEDEF";valid_file "T";width 2.4664in;height 1.8049in;depth
0pt;original-width 2.4249in;original-height 1.7677in;cropleft "0";croptop
"1";cropright "1";cropbottom "0";tempfilename
'PRX6FP0H.wmf';tempfile-properties "XPR";}}If we think about the energy of
the segment, we will realize that after time $\Delta t,$ it is as though
most of the fluid did not move. We can't tell one part of the fluid from
another. The shaded region in figure \ref{fluids shaded region} is occupied
before and after $\Delta t.$ We can treat this problem as if we moved fluid
from region 1 to region 2 (Super Guy vs. Normal Guy again!) and left the
rest of the segment alone!

So, ignoring the rest of the fluid, we can see that if the velocity of our
parcel of water changed, then the kinetic energy of the parcel must change
as well

\begin{equation}
\Delta K=\frac{1}{2}mv_{2}^{2}-\frac{1}{2}mv_{1}^{2}
\end{equation}%
(the mass is the same as we found before).

\subsection{Potential Energy}

\FRAME{dtbpF}{2.911in}{1.2739in}{0pt}{}{}{Figure}{\special{language
"Scientific Word";type "GRAPHIC";maintain-aspect-ratio TRUE;display
"USEDEF";valid_file "T";width 2.911in;height 1.2739in;depth
0pt;original-width 3.9583in;original-height 1.7158in;cropleft "0";croptop
"1";cropright "1";cropbottom "0";tempfilename
'PRX5KQ0C.wmf';tempfile-properties "XPR";}}

Let's look at the change in potential energy for the parcel of fluid. Once
again we can consider just the beginning and ending case (think of Super Guy
and Normal Guy once again). We work as though we moved a mass $m=m_{1}=m_{2}$
from $y_{1}$ to $y_{2}.$

\begin{equation}
\Delta U=mgy_{2}-mgy_{1}
\end{equation}%
Since our pipe went upward as the parcel of water flows, we gained potential
energy. But if the pipe went downward we would loose potential energy.

\subsection{Total Work done on the system}

Let's now assemble our work-energy theorem. We had found the work so far, 
\[
\left( P_{1}-P_{2}\right) 
%TCIMACRO{\TeXButton{V}{\ooalign{\hfil$V$\hfil\cr\kern0.1em--\hfil\cr}}}%
%BeginExpansion
\ooalign{\hfil$V$\hfil\cr\kern0.1em--\hfil\cr}%
%EndExpansion
=\Delta K+\Delta U 
\]%
but now we know $\Delta K$ and $\Delta U$ We have calculated $W$, $\Delta K$%
, and $\Delta U$ for our parcel of water, so let's substitute in what we
have found. 
\begin{equation}
\left( \left( P_{1}-P_{2}\right) 
%TCIMACRO{\TeXButton{V}{\ooalign{\hfil$V$\hfil\cr\kern0.1em--\hfil\cr}}}%
%BeginExpansion
\ooalign{\hfil$V$\hfil\cr\kern0.1em--\hfil\cr}%
%EndExpansion
\right) =\left( \frac{1}{2}mv_{2}^{2}-\frac{1}{2}mv_{1}^{2}\right) +\left(
mgy_{2}-mgy_{1}\right)
\end{equation}%
This equation describes our fluid flow much like the equation of continuity,
but now it has pressure involved! We have achieved our goal. But we can make
this equation a little easier to understand if we play a trick by dividing
by $V$ 
\begin{equation}
\left( \left( P_{1}-P_{2}\right) \right) =\left( \frac{1}{2}\frac{m}{%
%TCIMACRO{\TeXButton{V}{\ooalign{\hfil$V$\hfil\cr\kern0.1em--\hfil\cr}}}%
%BeginExpansion
\ooalign{\hfil$V$\hfil\cr\kern0.1em--\hfil\cr}%
%EndExpansion
}v_{2}^{2}-\frac{1}{2}\frac{m}{%
%TCIMACRO{\TeXButton{V}{\ooalign{\hfil$V$\hfil\cr\kern0.1em--\hfil\cr}}}%
%BeginExpansion
\ooalign{\hfil$V$\hfil\cr\kern0.1em--\hfil\cr}%
%EndExpansion
}v_{1}^{2}\right) +\left( \frac{m}{%
%TCIMACRO{\TeXButton{V}{\ooalign{\hfil$V$\hfil\cr\kern0.1em--\hfil\cr}}}%
%BeginExpansion
\ooalign{\hfil$V$\hfil\cr\kern0.1em--\hfil\cr}%
%EndExpansion
}gy_{2}-\frac{m}{%
%TCIMACRO{\TeXButton{V}{\ooalign{\hfil$V$\hfil\cr\kern0.1em--\hfil\cr}}}%
%BeginExpansion
\ooalign{\hfil$V$\hfil\cr\kern0.1em--\hfil\cr}%
%EndExpansion
}gy_{1}\right)
\end{equation}%
and recalling that 
\[
\rho =\frac{m}{%
%TCIMACRO{\TeXButton{V}{\ooalign{\hfil$V$\hfil\cr\kern0.1em--\hfil\cr}}}%
%BeginExpansion
\ooalign{\hfil$V$\hfil\cr\kern0.1em--\hfil\cr}%
%EndExpansion
} 
\]

then%
\[
\left( P_{1}-P_{2}\right) =\left( \frac{1}{2}\rho v_{2}^{2}-\frac{1}{2}\rho
v_{1}^{2}\right) +\left( \rho gy_{2}-\rho gy_{1}\right) 
\]

Rearranging gives%
\begin{equation}
P_{1}+\frac{1}{2}\rho v_{1}^{2}+\rho gy_{1}=P_{2}+\frac{1}{2}\rho
v_{2}^{2}+\rho gy_{2}
\end{equation}%
which completes our goal. We have an equation that contains the velocity 
\emph{and} the pressure of the fluid.

Sometimes you see this written as 
\begin{equation}
P+\frac{1}{2}\rho v^{2}+\rho gy=\text{constant}
\end{equation}%
which really should excite us. This is another conservation equation. The
thing that is being conserved is $P+\frac{1}{2}\rho v^{2}+\rho gy.$ We have
not named this quantity, but whatever it is, it is not changing as the fluid
flows. Our experience with conservation equations tells us that this new
conservation law will help us solve problems. This equation is so useful we
give it a name. It is called \emph{Bernoulli's equation} (after the
scientist that came up with it).

%TCIMACRO{%
%\TeXButton{Question 123.6.7}{\marginpar {
%\hspace{-0.5in}
%\begin{minipage}[t]{1in}
%\small{Question 123.6.7}
%\end{minipage}
%}}}%
%BeginExpansion
\marginpar {
\hspace{-0.5in}
\begin{minipage}[t]{1in}
\small{Question 123.6.7}
\end{minipage}
}%
%EndExpansion
%TCIMACRO{%
%\TeXButton{Question 123.6.8}{\marginpar {
%\hspace{-0.5in}
%\begin{minipage}[t]{1in}
%\small{Question 123.6.8}
%\end{minipage}
%}}}%
%BeginExpansion
\marginpar {
\hspace{-0.5in}
\begin{minipage}[t]{1in}
\small{Question 123.6.8}
\end{minipage}
}%
%EndExpansion

\section{Applications of Fluid Dynamics}

%TCIMACRO{%
%\TeXButton{Ping Pong Ball Demo}{\marginpar {
%\hspace{-0.5in}
%\begin{minipage}[t]{1in}
%\small{Ping Pong Ball Demo}
%\end{minipage}
%}}}%
%BeginExpansion
\marginpar {
\hspace{-0.5in}
\begin{minipage}[t]{1in}
\small{Ping Pong Ball Demo}
\end{minipage}
}%
%EndExpansion
%TCIMACRO{%
%\TeXButton{Question 123.6.9}{\marginpar {
%\hspace{-0.5in}
%\begin{minipage}[t]{1in}
%\small{Question 123.6.9}
%\end{minipage}
%}}}%
%BeginExpansion
\marginpar {
\hspace{-0.5in}
\begin{minipage}[t]{1in}
\small{Question 123.6.9}
\end{minipage}
}%
%EndExpansion
%TCIMACRO{%
%\TeXButton{Question 123.6.10}{\marginpar {
%\hspace{-0.5in}
%\begin{minipage}[t]{1in}
%\small{Question 123.6.10}
%\end{minipage}
%}}}%
%BeginExpansion
\marginpar {
\hspace{-0.5in}
\begin{minipage}[t]{1in}
\small{Question 123.6.10}
\end{minipage}
}%
%EndExpansion
An object moving through a fluid experiences a force due to any effect that
causes the fluid to change its direction as it flows past the object.

Some influencing factors:

\begin{enumerate}
\item Shape

\item orientation

\item spinning motion

\item Texture

\item Airfoil\FRAME{dtbpFU}{3.3874in}{1.8503in}{0pt}{\Qcb{{\protect\small %
Pressure on an airfoil. (Michael Belisle, public domain)}}}{}{Figure}{%
\special{language "Scientific Word";type "GRAPHIC";maintain-aspect-ratio
TRUE;display "USEDEF";valid_file "T";width 3.3874in;height 1.8503in;depth
0pt;original-width 3.4238in;original-height 1.8564in;cropleft "0";croptop
"1";cropright "1";cropbottom "0";tempfilename
'PQXXQZXG.wmf';tempfile-properties "XPR";}}The air that travels over the top
part of the wing is deflected (changes direction) as it flows past the wing.
There is an upward force (lift) due to this effect.

\item @@@@@@@@@@@@@@@ Not done yet.
\end{enumerate}

\chapter{Elasticity-State Variables}

We dealt with liquids, let's briefly think about solids in our preparation
for studying thermodynamics.

%TCIMACRO{%
%\TeXButton{Fundamental Concepts}{\hspace{-1.3in}{\Large Fundamental Concepts\vspace{0.25in}}}}%
%BeginExpansion
\hspace{-1.3in}{\Large Fundamental Concepts\vspace{0.25in}}%
%EndExpansion

\begin{enumerate}
\item Elasticity

\item States of Matter

\item Density
\end{enumerate}

\section{Deformation of Solids}

We are working our way toward thermodynamics--the study of how many objects
move due to thermal energy. We will need to know a little bit more about
solids. Specifically we need to know how solids bend, stretch, and break
because all of these things can happen due to thermal energy.

Stretching, bending, and breaking all require forces to be applied to our
objects. But in our study of pressure, we know the area involved in applying
the force matters. We called a force spread out over an area a pressure.
This works for fluids. But it turns out that pressure is only one kind of
force spread out over an area. For solids the general form of a spread out
force is called a \emph{stress}. Let's start with some definitions.

\begin{enumerate}
\item Stress: Force per unit area causing deformation

\item Strain is a measure of the amount of deformation
\end{enumerate}

If we don't allow the stress to get too big, the strain is proportional to
the stress. The next figure is a stress vs. strain plot for bending pine
beams.\FRAME{dtbpFU}{3.1886in}{2.3471in}{0pt}{\Qcb{{\protect\small %
Stress-strain diagrams of two longleaf pine beams. E.L. = elastic limit. The
areas of the triangles 0(EL)A and 0(EL)B represent the elastic resilience of
the dry and green beams, respectively (Public Domain Image from Samuel J.
Record, The Mechanical Properties of Wood Including a Discussion of the
Factors Affecting the Mechanical Properties, and Methods of Timber Testing,
http://www.gutenberg.org/ebooks/12299)}}}{\Qlb{stress_stsrain_wood}}{Figure}{%
\special{language "Scientific Word";type "GRAPHIC";maintain-aspect-ratio
TRUE;display "USEDEF";valid_file "T";width 3.1886in;height 2.3471in;depth
0pt;original-width 2.4984in;original-height 1.8317in;cropleft "0";croptop
"1";cropright "1";cropbottom "0";tempfilename
'PQXXQZXH.wmf';tempfile-properties "XPR";}}Clearly at some point, the stress
is to great, and a solid pine beam will break. All of this is due to the
same type of forces that create normal and tension forces. That is, the
bonding of the atoms that make up the solid. A strain is present because the
stress (applied spread out force) works against the bonding forces. If the
bonds are weak, there is more deformation. At the breaking point the bonds
are ripped apart.

Liquids will not experience strains other than pressure, because they can
flow. Fluid bonds don't support a particular shape, so we can't reasonably
talk about a deformation of a fluid.\footnote{%
There are some fluids, the non-Newtonian fluids, for which this is not true.}

\subsection{Young's Modulus}

A modulus is a constant that describes how a solid can be deformed. Think of
it like the spring constant $k$ that tells us how hard it is to stretch or
compress a spring.

Suppose we pull on a rod with a force $\overrightarrow{\mathbf{F}}\mathbf{.}$%
\FRAME{dtbpF}{5.1463in}{2.7346in}{0in}{}{}{Figure}{\special{language
"Scientific Word";type "GRAPHIC";maintain-aspect-ratio TRUE;display
"USEDEF";valid_file "T";width 5.1463in;height 2.7346in;depth
0in;original-width 5.2155in;original-height 2.7576in;cropleft "0";croptop
"1";cropright "1";cropbottom "0";tempfilename
'PQXXQZXI.wmf';tempfile-properties "XPR";}}

The rod will be stressed. We call this \emph{tensile stress} which comes
from a pull . This pull will stretch the solid's molecular bonds (like a
tension force).

We write the stress as 
\begin{equation}
\frac{F}{A}
\end{equation}%
which looks like a pressure, but this time we mean that we are pulling on
the beam so that the pull is spread over the cross sectional area (marked $A$
in the figure above). That is not much like a pressure!

By pulling on the beam, the beam will stretch. That stretching is a strain.
We write the strain as the percent change in shape. In this case, the
percent change in length%
\begin{equation}
\frac{\Delta L}{L_{o}}
\end{equation}%
If the stress is not too great, then we can write a linear equation that
relates the stress to the strain. 
\begin{equation}
\frac{F}{A}=Y\frac{\Delta L}{L_{o}}  \label{Young's equation}
\end{equation}%
this linear equation only works in the elastic region of a stress vs. strain
curve for the particular object. For example, if we pull on a pine beam we
get the stress vs. strain curve in figure\ref{stress_stsrain_wood}. Our
equation \ref{Young's equation} only works for the red part marked
\textquotedblleft Elastic Region\textquotedblright\ on the curve. But this
is just the region that we want to use to build buildings and airplanes and
supercollider support structures, etc. We don't want to build buildings with
the stress on the boards causing the boards to bend and break! So this
equation is useful.

The constant of proportionality, $Y,$ is called \emph{Young's Modulus}. If
we write this as%
\begin{eqnarray}
F &=&\frac{YA}{L_{o}}\Delta L \\
F &=&k\Delta L
\end{eqnarray}%
it looks very much like Hook's law. Our restoring force or our
\textquotedblleft pull\textquotedblright\ is the rod pulling back with a
force like $F=-k\left( x-x_{o}\right) .$

Young's modulus depends on the microscopic properties of the solid (which we
will not study in detail, but these are our stretchy/squishy forces between
atoms that make normal and tension forces). We will look up Young's modulus
for each material in tables like the following:%
\[
\begin{tabular}{|l|l|}
\hline
Material\QQfnmark{%
*Ledbetter, H. M., Physical Properties Data Compilations Relevant to Energy
Storage, US National Bureaus of Standards, 1982. Different alloys have
different properties, so for any real work see the original tables in the
original publication.}$^{,}$\QQfnmark{%
\dag Average values from numbers given in various text books. These numbers
should be taken as example values and more exact numbers found for any real
work.} & Young's Modulus $\left( \unit{N}/\unit{m}^{2}\right) $ \\ \hline
Aluminum$^{\ast }$ & $7\times 10^{10}$ \\ \hline
Titanium$^{\ast }$ & $11.\times 10^{10}$ \\ \hline
Steel$^{\ast }$ & $20.\times 10^{10}$ \\ \hline
Carbon Steel$^{\ast }$ & $21.\times 10^{10}$ \\ \hline
Lead$^{\dag }$ & $1.6\times 10^{10}$ \\ \hline
Brass$^{\dag }$ & $10.\times 10^{10}$ \\ \hline
Concrete$^{\dag }$ & $2.0\times 10^{10}$ \\ \hline
Nylon$^{\dag }$ & $0.5\times 10^{10}$ \\ \hline
Bone$^{\dag }$ (arm or leg) & $1.5\times 10^{10}$ \\ \hline
\begin{tabular}{l}
Pine Wood$^{\dag }$ (parallel to grain) \\ 
(perpendicular to grain)%
\end{tabular}
& 
\begin{tabular}{l}
$1.0\times 10^{10}$ \\ 
$0.1\times 10^{10}$%
\end{tabular}
\\ \hline
\end{tabular}%
\QQfntext{-1}{
*Ledbetter, H. M., Physical Properties Data Compilations Relevant to Energy
Storage, US National Bureaus of Standards, 1982. Different alloys have
different properties, so for any real work see the original tables in the
original publication.}
\QQfntext{1}{
\dag Average values from numbers given in various text books. These numbers
should be taken as example values and more exact numbers found for any real
work.}
\]

Let's take an example. Suppose we have a steel piano wire. One end of the
piano wire is fixed in place, and the other is connected to a peg that can
be rotated so the piano wire is tightened. This is what you do when you tune
a piano, you turn the peg and tighten the wire. Suppose we tighten the wire
so that there will be $980\unit{N}$ of tension on the wire. The wire is $1.6%
\unit{m}$ long and has a diameter of $2\unit{mm}.$ How much will the wire
stretch as we tighten the peg? We will need to know that the Young's modulus
for steel is $20\times 10^{10}\unit{N}/\unit{m}^{3}.$

Here is a summary of what we know%
\begin{eqnarray*}
T &=&980\unit{N} \\
L_{o} &=&1.6\unit{m} \\
d &=&2\unit{mm} \\
Y &=&200\times 10^{9}\unit{N}/\unit{m}^{2}
\end{eqnarray*}

Let's start with our stress/strain basic equation%
\[
\frac{F}{A}=Y\frac{\Delta L}{L_{o}} 
\]%
and solve for $\Delta L$%
\[
\frac{F}{A}\frac{L_{o}}{Y}=\Delta L 
\]%
the force is our tension, and the cross sectional area will be 
\[
A=\pi \frac{d^{2}}{4} 
\]%
\[
\frac{T}{\pi \frac{d^{2}}{4}}\frac{L_{o}}{Y}=\Delta L 
\]%
\[
\frac{4T}{\pi d^{2}}\frac{L_{o}}{Y}=\Delta L 
\]%
then 
\begin{eqnarray*}
\Delta L &=&\frac{4\left( 980\unit{N}\right) }{\pi \left( 2\unit{mm}\right)
^{2}}\frac{\left( 1.6\unit{m}\right) }{\left( 200\times 10^{9}\unit{N}/\unit{%
m}^{2}\right) }= \\
&=&\allowbreak 2.\,\allowbreak 495\,5\times 10^{-3}\unit{m}
\end{eqnarray*}%
so the wire stretched about a quarter of a centimeter.

Of course, if we pull too hard the material will stretch, and finally break.
The stress that can break the bonds holding the material together is called
the \emph{ultimate strength}.

\[
\begin{tabular}{|l|l|}
\hline
Material\QQfnmark{%
*Ledbetter, H. M., Physical Properties Data Compilations Relevant to Energy
Storage, US National Bureaus of Standards, 1982. Different alloys have
different properties, so for any real work see the original tables in the
original publication.}$^{,}$\QQfnmark{%
\dag Average values from numbers given in various text books. These numbers
should be taken as example values and more exact numbers found for any real
work.} & Tensile Ultimate Strength $\left( \unit{N}/\unit{m}^{2}\right) $ \\ 
\hline
Aluminum$^{\ast }$ & $4.2\times 10^{8}$ \\ \hline
Titanium$^{\ast }$ & $9.9\times 10^{8}$ \\ \hline
Steel$^{\ast }$ & $17.0\times 10^{8}$ \\ \hline
Carbon Steel$^{\ast }$ & $6.5\times 10^{8}$ \\ \hline
Bone$^{\dag }$ (arm or leg) & $1.3\times 10^{6}$ \\ \hline
\end{tabular}%
\QQfntext{-1}{
*Ledbetter, H. M., Physical Properties Data Compilations Relevant to Energy
Storage, US National Bureaus of Standards, 1982. Different alloys have
different properties, so for any real work see the original tables in the
original publication.}
\QQfntext{1}{
\dag Average values from numbers given in various text books. These numbers
should be taken as example values and more exact numbers found for any real
work.}
\]

\subsection{Shear Modulus}

A shear stress is also a force acting on an area, but it is a force parallel
to the surface of the object. This is like pushing on the top of Jello%
%TCIMACRO{\TeXButton{TM}{\textsuperscript{\texttrademark}}}%
%BeginExpansion
\textsuperscript{\texttrademark}%
%EndExpansion
. The Jello will deform.\FRAME{dtbpF}{3.2011in}{2.1651in}{0pt}{}{}{Figure}{%
\special{language "Scientific Word";type "GRAPHIC";maintain-aspect-ratio
TRUE;display "USEDEF";valid_file "T";width 3.2011in;height 2.1651in;depth
0pt;original-width 3.2339in;original-height 2.1775in;cropleft "0";croptop
"1";cropright "1";cropbottom "0";tempfilename
'PQXXQZXJ.wmf';tempfile-properties "XPR";}}Again the stress is%
\begin{equation}
\frac{F}{A}
\end{equation}%
But the area is very different that the area we used in tensile stress. Now
we will use the area of the top of the Jello. The strain is the percentage
change in the $x$ position of the top surface, relative to the height of the
Jello 
\begin{equation}
\frac{\Delta x}{h}
\end{equation}%
Or, in other words, $\Delta x$ is how far the top surface moves, and $h$ is
the height of the object. The strain is the ratio of these two quantities.

The stress and strain are related by%
\begin{equation}
\frac{F}{A}=S\frac{\Delta x}{h}
\end{equation}%
where $S$ is the shear modulus. Of course, this equation is also only good
if we are in the linear region of the stress vs. strain curve. At some point
you will just sheer off the top of the jello\footnote{%
For example, if you are scooping off the layer of Jello that has shredded
carrots in it.}. But so long as we are in the linear region, our equation
works. Again, to find $S$ for a particular material, we should look it in a
table.

\[
\begin{tabular}{|l|l|}
\hline
Material\QQfnmark{\S Average values from numbers given in various text
books. These numbers should be taken as example values and more exact
numbers found for any real work. *Depends on how much water and what
temperatre the Jello and what agents you put in it for strength. See for
example A. Bigi \emph{et al}. Biomatrials 22 (2001) 763-768.} & Sheer
Modulus $\left( \unit{N}/\unit{m}^{2}\right) $ \\ \hline
Steel$^{\S }$ & $80\times 10^{9}$ \\ \hline
Cast Iron$^{\S }$ & $40\times 10^{9}$ \\ \hline
Brass$^{\S }$ & $135\times 10^{9}$ \\ \hline
Aluminum$^{\S }$ & $25\times 10^{9}$ \\ \hline
Bone$^{\S }$ (arm or leg) & $80\times 10^{9}$ \\ \hline
Jello$^{\ast }$ & $0$ to $8\times 10^{4}$ \\ \hline
\end{tabular}%
\QQfntext{0}{\S Average values from numbers given in various text books.
These numbers should be taken as example values and more exact numbers found
for any real work. *Depends on how much water and what temperatre the Jello
and what agents you put in it for strength. See for example A. Bigi \emph{et
al}. Biomatrials 22 (2001) 763-768.}
\]

\subsection{Bulk Modulus}

This one is harder. It tells us how much a solid can be squashed or,
compressed. Let's define 
\begin{equation}
\Delta P=\frac{\Delta F}{A}
\end{equation}%
where now $A$ is the outer surface area of the solid. If we were dealing
with a fluid, this would be pressure. Often fluids are very hard to compress
(we assumed they were not at all compressible). Many solids are easy to
compress. Think of Styrofoam or any foam rubber. For solids $\Delta F/A$
where the force is all around the solid and the area is the surface area is
a stress. The strain is the percentage change in volume (change amount over
the original volume)%
\begin{equation}
\frac{\Delta 
%TCIMACRO{\TeXButton{V}{\ooalign{\hfil$V$\hfil\cr\kern0.1em--\hfil\cr}}}%
%BeginExpansion
\ooalign{\hfil$V$\hfil\cr\kern0.1em--\hfil\cr}%
%EndExpansion
}{%
%TCIMACRO{\TeXButton{V}{\ooalign{\hfil$V$\hfil\cr\kern0.1em--\hfil\cr}}}%
%BeginExpansion
\ooalign{\hfil$V$\hfil\cr\kern0.1em--\hfil\cr}%
%EndExpansion
}
\end{equation}%
and the relationship is%
\begin{equation}
\Delta P=-B\frac{\Delta 
%TCIMACRO{\TeXButton{V}{\ooalign{\hfil$V$\hfil\cr\kern0.1em--\hfil\cr}}}%
%BeginExpansion
\ooalign{\hfil$V$\hfil\cr\kern0.1em--\hfil\cr}%
%EndExpansion
}{%
%TCIMACRO{\TeXButton{V}{\ooalign{\hfil$V$\hfil\cr\kern0.1em--\hfil\cr}}}%
%BeginExpansion
\ooalign{\hfil$V$\hfil\cr\kern0.1em--\hfil\cr}%
%EndExpansion
}
\end{equation}

$B$ is the \emph{bulk modulus}, which again, we look up in tables.%
\[
\begin{tabular}{|l|l|}
\hline
Material\QQfnmark{$\ddag $Average values from numbers given in various text
books. These numbers should be taken as example values and more exact
numbers found for any real work.} & Bulk Modulus $\left( \unit{N}/\unit{m}%
^{2}\right) $ \\ \hline
Steel$^{\ddag }$ & $140\times 10^{9}$ \\ \hline
Cast Iron$^{\ddag }$ & $90\times 10^{9}$ \\ \hline
Brass$^{\ddag }$ & $80\times 10^{9}$ \\ \hline
Aluminum$^{\ddag }$ & $70\times 10^{9}$ \\ \hline
Water$^{\ddag }$ & $2\times 10^{9}$ \\ \hline
Ethyl Alcohol$^{\ddag }$ & $1\times 10^{9}$ \\ \hline
Mercury$^{\ddag }$ & $2.5\times 10^{9}$ \\ \hline
Air$^{\ddag }$ & $1.01\times 10^{5}$ \\ \hline
\end{tabular}%
\QQfntext{0}{$\ddag $Average values from numbers given in various text
books. These numbers should be taken as example values and more exact
numbers found for any real work.}
\]%
The minus sign may be surprising. But in our definition of bulk modulus, we
are thinking of compression. So usually $\Delta 
%TCIMACRO{\TeXButton{V}{\ooalign{\hfil$V$\hfil\cr\kern0.1em--\hfil\cr}}}%
%BeginExpansion
\ooalign{\hfil$V$\hfil\cr\kern0.1em--\hfil\cr}%
%EndExpansion
$ is negative. The minus sigh means that for compression this formula gives
positive values. Also notice that there is a $\Delta P$ or $\Delta F$ in our
formula. Usually we start compressing a solid while it is experiencing air
pressure. So we don't start from zero stress. We change the force from the
force due to air pressure to some larger compressive force.

\section{Ultimate Strength of Materials}

%TCIMACRO{%
%\TeXButton{Arch Demo}{\marginpar {
%\hspace{-0.5in}
%\begin{minipage}[t]{1in}
%\small{Arch Demo}
%\end{minipage}
%}}}%
%BeginExpansion
\marginpar {
\hspace{-0.5in}
\begin{minipage}[t]{1in}
\small{Arch Demo}
\end{minipage}
}%
%EndExpansion

Many materials are stronger under compression than under tension. That is,
it is harder to squash them than to pull them apart. \FRAME{dtbpFU}{3.0969in%
}{2.4734in}{0pt}{\Qcb{Thomas Roger Smith John Slate, Architecture Classic
and Early Christian, London, William Clows and Sons, Limited, London, 1882)}%
}{}{Figure}{\special{language "Scientific Word";type
"GRAPHIC";maintain-aspect-ratio TRUE;display "USEDEF";valid_file "T";width
3.0969in;height 2.4734in;depth 0pt;original-width 3.0528in;original-height
2.4327in;cropleft "0";croptop "1";cropright "1";cropbottom "0";tempfilename
'PQXXQZXK.wmf';tempfile-properties "XPR";}}Arches are made based on this
principle. By placing bocks in an arch, the weight of the wall presses down
on the arch. The stones of the arch are compressed into each other. The
force is extended into the walls next to the archway.

The block at the top holds the arch together and distributes the weight from
above to the other blocks in the arch. \FRAME{dtbpF}{2.1855in}{1.8724in}{0pt%
}{}{}{Figure}{\special{language "Scientific Word";type
"GRAPHIC";maintain-aspect-ratio TRUE;display "USEDEF";valid_file "T";width
2.1855in;height 1.8724in;depth 0pt;original-width 2.1979in;original-height
1.8786in;cropleft "0";croptop "1";cropright "1";cropbottom "0";tempfilename
'PQXXQZXL.wmf';tempfile-properties "XPR";}}It is called the \emph{key stone}%
. This is what Joseph Smith was talking about when he said the \emph{Book of
Mormon} is the \textquotedblleft keystone of our
religion.\textquotedblright\ It holds the rest together! In the figure, the
vertical and horizontal components of the forces due to the building load
and due to the surrounding arch stones are shown. These must balance if the
arch is to be stable.

If we look at the blocks next to the key stone we see that the blocks to the
side and below must support this block. \FRAME{dtbpF}{2.0587in}{2.4232in}{0pt%
}{}{}{Figure}{\special{language "Scientific Word";type
"GRAPHIC";maintain-aspect-ratio TRUE;display "USEDEF";valid_file "T";width
2.0587in;height 2.4232in;depth 0pt;original-width 2.0693in;original-height
2.441in;cropleft "0";croptop "1";cropright "1";cropbottom "0";tempfilename
'PQXXQZXM.wmf';tempfile-properties "XPR";}} The key stone pushes with force
components downward and to the right. The next block must push up and to the
left to balance the force due to the keystone and the building load on this
block. \FRAME{dtbpF}{4.1564in}{2.7594in}{0pt}{}{}{Figure}{\special{language
"Scientific Word";type "GRAPHIC";maintain-aspect-ratio TRUE;display
"USEDEF";valid_file "T";width 4.1564in;height 2.7594in;depth
0pt;original-width 4.207in;original-height 2.7834in;cropleft "0";croptop
"1";cropright "1";cropbottom "0";tempfilename
'PQXXQZXN.wmf';tempfile-properties "XPR";}}This distribution of force along
the facing areas of the arch stones continues all the way through the arch.
In practice, it looks like this\FRAME{dtbpF}{1.3766in}{2.0569in}{0pt}{}{}{%
Figure}{\special{language "Scientific Word";type
"GRAPHIC";maintain-aspect-ratio TRUE;display "USEDEF";valid_file "T";width
1.3766in;height 2.0569in;depth 0pt;original-width 1.3739in;original-height
2.0667in;cropleft "0";croptop "1";cropright "1";cropbottom "0";tempfilename
'PQXXQZXO.wmf';tempfile-properties "XPR";}}But notice that we must provide a
buttressing force from the rest of the structure on the bottom most stone.
If you don't have wall to provide this force, you might be in trouble.
Suppose you want to build an arched ceiling for a cathedral. \FRAME{dtbpFU}{%
3.5612in}{2.3745in}{0pt}{\Qcb{Arched Ceiling of the Cathedral of Milan}}{}{%
Figure}{\special{language "Scientific Word";type
"GRAPHIC";maintain-aspect-ratio TRUE;display "USEDEF";valid_file "T";width
3.5612in;height 2.3745in;depth 0pt;original-width 3.6003in;original-height
2.3913in;cropleft "0";croptop "1";cropright "1";cropbottom "0";tempfilename
'PQXXQZXP.wmf';tempfile-properties "XPR";}}You would need to build a large
solid mass outside the building to provide the buttressing force. But this
might look bulky and ugly. So you could provide the buttressing force with a
series of \textquotedblleft flying buttresses\textquotedblright\ or
structures that can provide the buttressing force at points along the length
of the building. The buttress is there to provide an opposing force on the
outside of the wall where there would be no next arch. \FRAME{dtbpFU}{3.21in%
}{2.2237in}{0pt}{\Qcb{{\protect\small Flying Buttress (Dictionary of French
Architecture from 11th to 16th Century (1856) by Eug\`{e}ne Viollet-le-Duc
(1814-1879), Image public domain)}}}{}{Figure}{\special{language "Scientific
Word";type "GRAPHIC";maintain-aspect-ratio TRUE;display "USEDEF";valid_file
"T";width 3.21in;height 2.2237in;depth 0pt;original-width
3.2428in;original-height 2.2379in;cropleft "0";croptop "1";cropright
"1";cropbottom "0";tempfilename 'PQXXQZXQ.wmf';tempfile-properties "XPR";}}%
Here is an example of a buttress from the Duomo in Milan, Italy. \FRAME{%
dtbpFU}{4.3134in}{2.0729in}{0pt}{\Qcb{\textquotedblleft Flying
Buttress\textquotedblright\ Supporting the Arched Ceiling of the Cathedral
of Milan}}{}{Figure}{\special{language "Scientific Word";type
"GRAPHIC";maintain-aspect-ratio TRUE;display "USEDEF";valid_file "T";width
4.3134in;height 2.0729in;depth 0pt;original-width 4.3675in;original-height
2.0835in;cropleft "0";croptop "1";cropright "1";cropbottom "0";tempfilename
'PQXXQZXR.wmf';tempfile-properties "XPR";}}

At some point, though, if we put too much force on an object, it will break.
Our normal (atom squashing) or tension (bond stretching) forces can't push
back or stretch enough and the atoms are torn from each other. This is
described by the \emph{ultimate strength}\textbf{\ }parameters given in the
next table. Here is a table for ultimate strength for some materials. 
\[
\begin{tabular}{|l|l|c|c|}
\hline
{\small Material}\QQfnmark{%
\dag Average values from numbers given in various text books. These numbers
should be taken as example values and more exact numbers found for any real
work.} & 
\begin{tabular}{c}
{\small Ultimate } \\ 
{\small Tensile strength} \\ 
$\left( \unit{N}/\unit{m}^{2}\right) $%
\end{tabular}
& 
\begin{tabular}{c}
{\small Ultimate} \\ 
{\small \ Sheer Strength} \\ 
$\left( \unit{N}/\unit{m}^{2}\right) $%
\end{tabular}
& 
\begin{tabular}{c}
{\small Ultimate } \\ 
{\small Compressive Strength} \\ 
$\left( \unit{N}/\unit{m}^{2}\right) $%
\end{tabular}
\\ \hline
{\small Steel}$^{\dag }$ & ${\small 500\times 10}^{6}$ & 
\multicolumn{1}{|l|}{${\small 250\times 10}^{6}$} & \multicolumn{1}{|l|}{$%
{\small 500\times 10}^{6}$} \\ \hline
{\small Cast Iron}$^{\dag }$ & ${\small 170\times 10}^{6}$ & 
\multicolumn{1}{|l|}{${\small 170\times 10}^{6}$} & \multicolumn{1}{|l|}{$%
{\small 550\times 10}^{6}$} \\ \hline
{\small Brass}$^{\dag }$ & ${\small 250\times 10}^{6}$ & 
\multicolumn{1}{|l|}{${\small 200\times 10}^{6}$} & \multicolumn{1}{|l|}{$%
{\small 250\times 10}^{6}$} \\ \hline
{\small Aluminum}$^{\dag }$ & ${\small 200\times 10}^{6}$ & 
\multicolumn{1}{|l|}{${\small 200\times 10}^{6}$} & \multicolumn{1}{|l|}{$%
{\small 200\times 10}^{6}$} \\ \hline
{\small Nylon}$^{\dag }$ & ${\small 500\times 10}^{6}$ & 
\multicolumn{1}{|l|}{} & \multicolumn{1}{|l|}{} \\ \hline
{\small Bone}$^{\dag }${\small \ (arm or leg)} & ${\small 130\times 10}^{6}$
& \multicolumn{1}{|l|}{} & \multicolumn{1}{|l|}{${\small 170\times 10}^{6}$}
\\ \hline
\end{tabular}%
\QQfntext{0}{
\dag Average values from numbers given in various text books. These numbers
should be taken as example values and more exact numbers found for any real
work.}
\]

Suppose we return to our piano wire. If we keep turning the peg, the wire
will break (which is quite dangerous!). How much tension will break the
wire? We need the ultimate tensile strength for steel, $T_{u}=500\times
10^{6}\unit{N}/\unit{m}^{2}.$ The ultimate stress that will break the wire
is given by%
\[
\frac{F}{A}=T_{u} 
\]%
Think about what this means, we would expect that a wire made of steel would
break before a large beam made of steel would break. The force is not enough
to describe the motion, we also need to know how spread out that force is
over the cross sectional area of the wire or beam. So to find the tension
that will break the wire we need 
\[
F=T_{u}A 
\]%
\[
T=T_{u}\pi \frac{d^{2}}{4} 
\]%
\begin{eqnarray*}
T &=&\left( 500\times 10^{6}\unit{N}/\unit{m}^{2}\right) \pi \frac{\left( 2%
\unit{mm}\right) ^{2}}{4} \\
&=&1570.\,\allowbreak 8\unit{N} \\
&\approx &1600\unit{N}
\end{eqnarray*}

\section{State Variables}

%TCIMACRO{%
%\TeXButton{Question 123.7.2}{\marginpar {
%\hspace{-0.5in}
%\begin{minipage}[t]{1in}
%\small{Question 123.7.2}
%\end{minipage}
%}}}%
%BeginExpansion
\marginpar {
\hspace{-0.5in}
\begin{minipage}[t]{1in}
\small{Question 123.7.2}
\end{minipage}
}%
%EndExpansion
We learned before about solids, liquids, and gases. We learned that the
degree of bondedness made the difference between these states.

We are starting a new topic, and we will find that physics did not develop
as a cohesive whole. The new topic is called \emph{Thermodynamics} and it
developed independently of the theories of oscillation and waves and optics.
Thus the language we use will be different, and sometimes old words will
need new meanings.

The states of matter are often called \emph{phases} in thermodynamics. This
use of the word \textquotedblleft phase\textquotedblright\ has nothing to do
with the phase of an oscillator or wave. It is just the condition of being
solid, liquid or gas.

To change from solid to liquid, or from liquid to gas, is called a \emph{%
phase change}.

To describe that state of a sample of matter, we use a set of variables like
mass, volume, pressure, density, and energy. We have used these throughout
PH 121. Thermodynamics has it's own set of descriptive variables to add to
these: moles, number density, and Temperature. These variables, along with
our original set, are called \emph{state variables} and together, they
describe the conditions of a particular pice of matter.

\section{Temperature and the Zeroth Law of Thermodynamics}

%TCIMACRO{%
%\TeXButton{Question 123.7.3}{\marginpar {
%\hspace{-0.5in}
%\begin{minipage}[t]{1in}
%\small{Question 123.7.3}
%\end{minipage}
%}}}%
%BeginExpansion
\marginpar {
\hspace{-0.5in}
\begin{minipage}[t]{1in}
\small{Question 123.7.3}
\end{minipage}
}%
%EndExpansion
%TCIMACRO{%
%\TeXButton{Question 123.7.4}{\marginpar {
%\hspace{-0.5in}
%\begin{minipage}[t]{1in}
%\small{Question 123.7.4}
%\end{minipage}
%}}}%
%BeginExpansion
\marginpar {
\hspace{-0.5in}
\begin{minipage}[t]{1in}
\small{Question 123.7.4}
\end{minipage}
}%
%EndExpansion
We need a basis for talking about thermal systems. We sort of have an
intuition for temperature, so let's use our intuition for now.

\FRAME{dtbpF}{2.3471in}{1.6838in}{0pt}{}{}{Figure}{\special{language
"Scientific Word";type "GRAPHIC";maintain-aspect-ratio TRUE;display
"USEDEF";valid_file "T";width 2.3471in;height 1.6838in;depth
0pt;original-width 4.7625in;original-height 3.41in;cropleft "0";croptop
"1";cropright "1";cropbottom "0";tempfilename
'PQXXQZXS.wmf';tempfile-properties "XPR";}}In the figure we have two
objects. I measure both objects temperature. I find that they are the same.
Suppose I\ put the two objects in \textquotedblleft
contact.\textquotedblright

\FRAME{dtbpF}{2.2105in}{1.6665in}{0pt}{}{}{Figure}{\special{language
"Scientific Word";type "GRAPHIC";maintain-aspect-ratio TRUE;display
"USEDEF";valid_file "T";width 2.2105in;height 1.6665in;depth
0pt;original-width 4.5117in;original-height 3.3944in;cropleft "0";croptop
"1";cropright "1";cropbottom "0";tempfilename
'PQXXQZXT.wmf';tempfile-properties "XPR";}}

Would you expect there to be energy transferred between the two objects? For
most of us our intuition says \textquotedblleft no.\textquotedblright\ Given
this intuition, lets make some more formal definitions:

\begin{itemize}
\item Thermal Contact: Two objects are in thermal contact if they can
exchange energy due to a temperature difference

\item Thermal Equilibrium : A state where two objects would not exchange
energy by heat or electromagnetic radiation if they were placed in thermal
contact

\item Temperature: The property that determines if an object is in thermal
equilibrium with other objects (two objects in thermal equilibrium have the
same temperature)
\end{itemize}

Given all this, we can state a law of thermal dynamics: \textbf{If objects A
and B are separately in thermal equilibrium with a third object C, then A
and B are in thermal equilibrium with each other.}

Again this fits our intuition, if we have two containers of water that are
both at the same temperature as a third container of water, we expect they
all have the same temperature.

In a sense, this seems obvious, so obvious that it was always assumed
without statement. But in retrospect, it is a fundamental concept that
deserves it's place as one of the laws of thermodynamics. By the time this
was acknowledged, the names \textquotedblleft first law\textquotedblright\
and \textquotedblleft second\ law\textquotedblright\ were taken, so this
became the \emph{Zeroth law of Thermodynamics.}

\chapter{Moles, Temperature, and Phase Changes}

Our study of the motion of many things naturally starts with atoms. Our
world is built of atoms, and there are a lot of them. So let's think about
atoms (and leave collections of stars and planets in galaxies, say, for a
later course)

%TCIMACRO{%
%\TeXButton{Fundamental Concepts}{\hspace{-1.3in}{\Large Fundamental Concepts\vspace{0.25in}}}}%
%BeginExpansion
\hspace{-1.3in}{\Large Fundamental Concepts\vspace{0.25in}}%
%EndExpansion

\begin{enumerate}
\item Number Density

\item Atomic Mass

\item Atomic Mass Number

\item The Mole

\item Temperature

\item Kelvin Temperature Scale

\item Phase Changes and Phase Diagrams
\end{enumerate}

\section{Numbers of Atoms and Number Densities}

Atoms are small. They are really incredibly small. A typical atom measures a
few Angstroms across. An angstrom is $1\times 10^{-10}\unit{m}.$ That is
unbelievably small. Let's try to get an understanding of this scale. Picture
a meter stick. Then divide it up into $10000000000$ equal units. That is how
small an atoms is. Or, consider scaling our meter stick to be the size of
the continental US, that is about $4.\,\allowbreak 087\,7\times 10^{6}\unit{m%
}$. Then for our US\ sized meter stick, on this scale an atom would still
only be about half a millimeter across! About the size of a cake sprinkle.

So it is no surprise that in a typical sample of a solid we can have
something like $10^{25}$ atoms. We use the capital letter $N$ to mean the
number of atoms in a sample of matter.

Because these numbers are so large, it is sometimes convenient to talk about
how many atoms there are in a small volume of a substance. This is often
less than in the whole sample. We can say, for example, that there are $1000$
atoms per cubic micrometer of material. We call such a ratio of number of
atoms per volume a \emph{number density. }In the SI system, the units for
the number of atoms per unit volume are $1/\unit{m}^{3}.$ So number
densities can still be quite large! Number densities are more useful for
samples where the density is uniform, but we can talk about substances who's
number density changes. Unfortunately you will often see a small $n$ used to
mean number density. This over use of $n$ is confusing, so sometimes we will
write number density as $N/%
%TCIMACRO{\TeXButton{V}{\ooalign{\hfil$V$\hfil\cr\kern0.1em--\hfil\cr}}}%
%BeginExpansion
\ooalign{\hfil$V$\hfil\cr\kern0.1em--\hfil\cr}%
%EndExpansion
$ so it is obvious what we mean.

\subsection{Atomic Mass}

You probably ran into atomic mass in High School. The idea of atomic mass is
that each nucleon (proton or neutron) has about the same mass, and the
electrons don't have much mass at all, so we can sort of count nucleons and
say that an atom has a mass in units of nucleon masses. Carbon has 12
nucleons (six protons and six neutrons) so it would have a mass of $12\unit{u%
}.$ The unit $\unit{u}$ is the atomic mas unit, the mass of one nucleon.
Because of the electrons (and other effects)\ this is not an exact approach,
and it is not good enough if we attempt to do nuclear physics. But it is
close enough for us for now. So we can say that hydrogen $\left(
^{1}H\right) $ has an atomic mass of $1\unit{u}.$ If we have a molecule, we
can just add the atomic masses of each atom in the molecule together to find
the molecule's mass. Note that to be more accurate, we can use a periodic
table and find the atomic mass, or, if we need many digits of accuracy, we
can go to a table of the nucleotides. We can convert these masses to SI
units using%
\[
1\unit{u}=1.66\times 10^{-27}\unit{kg} 
\]

\subsection{Moles}

Chances are that you are familiar with the concept of a mole as well. But
experience has shown me that the mole will be new to some people. So let's
start with something we know, a dozen.

What is a dozen? Well, it is $12$ of something. We could have a dozen
doughnuts or a dozen eggs. or a dozen large white pickup trucks. It is just
an amount. It is $12$ of whatever we are counting.

A mole is like a dozen, but it is a BIG\ dozen. Is is $6.02\times 10^{23}$
of something. I\ could have $6.02\times 10^{23}$ doughnuts, or $6.02\times
10^{23}$ eggs. I would need quite a few hungry friends to eat it all. But a
mole is just a number of something. The mole is not a very practical amount
for everyday objects like doughtnuts or trucks. But think of the large
number of atoms in a sample of matter. For this situation, a mole seems like
a reasonable number to use.

Like we say I want to buy a dozen eggs, we could say we want to buy a mole
of hydrogen atoms. Since atoms are so small, packaging them in big numbers
makes sense.

This number, $6.02\times 10^{23},$ is called \emph{Avogadro's number}%
\[
N_{A}=6.02\times 10^{23} 
\]%
To find the number of dozens of a something, we take the number of items and
divide by 12.%
\[
n_{dozen}=\frac{N}{12} 
\]%
To find the number of moles of a something, we take the number of items and
divide by Avogadro's number%
\[
n_{moles}=\frac{N}{N_{A}} 
\]%
usually we will use small $n$ to be the number of moles of something. This
could be a problem because $n$ can also mean number density. So we will have
to be careful to state what we mean with our variables.

\subsection{Molar mass}

We could give a mass for a dozen eggs. If each egg has the same mass (more
or less) then the dozenal mass would be the mass of a dozen eggs and it
would be twelve times the mass of one egg. The dozenal mass of a dozen
trucks would be larger than the dozenal mass of eggs because the mass of a
single truck is larger than the mass of an egg, and the mass of a dozen
trucks is $12$ times the mass of one truck.

We can also define a molar mass, which would be the mass of $6.02\times
10^{23}$ of something. Usually, we choose the units for this molar mass to
be the mass \emph{in grams} of a mole of something.

The molar mass is equal to the numerical value of the atomic or molecular
mass. for example, the molar mass of He, with $m=4\unit{u}$ is $M=4\unit{g}/%
\unit{mol}.$ The molar mass of diatomic nitrogen would be $28\unit{g}/\unit{%
mol}$ because each nitrogen atom has an atomic mass of $m=14\unit{u}$ and we
add the two atomic masses together for the molecular mass. With this
definition of molar mass we can see that 
\[
n_{moles}=\frac{M\left( \text{in grams}\right) }{M_{mol}} 
\]

\section{Thermometers: Measuring Temperature}

%TCIMACRO{%
%\TeXButton{Question 123.8.10.1}{\marginpar {
%\hspace{-0.5in}
%\begin{minipage}[t]{1in}
%\small{Question 123.8.10.1}
%\end{minipage}
%}}}%
%BeginExpansion
\marginpar {
\hspace{-0.5in}
\begin{minipage}[t]{1in}
\small{Question 123.8.10.1}
\end{minipage}
}%
%EndExpansion
We will find that several material properties change with temperature

\begin{enumerate}
\item volume of a liquid

\item the dimensions of a solid

\item the pressure of a gas at constant volume

\item the volume of a gas at constant pressure

\item The electric resistance of a conductor

\item The color of an object
\end{enumerate}

We can use any of these to make a temperature measuring device.

Think of old fashioned thermometers, how do they work?

\FRAME{dtbpF}{2.4474in}{0.2421in}{0pt}{}{}{Figure}{\special{language
"Scientific Word";type "GRAPHIC";maintain-aspect-ratio TRUE;display
"USEDEF";valid_file "T";width 2.4474in;height 0.2421in;depth
0pt;original-width 2.4059in;original-height 0.2136in;cropleft "0";croptop
"1";cropright "1";cropbottom "0";tempfilename
'PQXXQZXU.wmf';tempfile-properties "XPR";}}Old fashioned mercury
thermometers (and their glycerine replacements) use the first property,
volume of a liquid increases with temperature. The liquid is placed in an
evacuated tube, and the amount of the tube that is filled depends on the
temperature. By placing a ruled background behind the tube or even by making
rule marks on the outside of the tube, we can find out how much the liquid
volume changed, and calculate the temperature change.

\subsection{Celsius temperature scale}

The most important thing to say about the Celsius scale is not to use it.
(The same goes double for the Fahrenheit scale). But since it is common to
see Celsius temperatures, we will talk about them.

\FRAME{dtbpF}{3.0193in}{2.6583in}{0in}{}{}{Figure}{\special{language
"Scientific Word";type "GRAPHIC";maintain-aspect-ratio TRUE;display
"USEDEF";valid_file "T";width 3.0193in;height 2.6583in;depth
0in;original-width 3.0486in;original-height 2.6805in;cropleft "0";croptop
"1";cropright "1";cropbottom "0";tempfilename
'PQXXQZXV.wmf';tempfile-properties "XPR";}}

This scale sets the zero point at the ice point of water and sets $100\unit{%
%TCIMACRO{\U{2103}}%
%BeginExpansion
{}^{\circ}{\rm C}%
%EndExpansion
}$ at the steam point of water. The scale uses $100$ evenly spaced divisions
between these points. This calibration does not work well if high accuracy
is required. There are often large discrepancies when temperatures are
measured far from the calibration points.

Worse yet, water only boils at $100\unit{%
%TCIMACRO{\U{2103}}%
%BeginExpansion
{}^{\circ}{\rm C}%
%EndExpansion
}$ at sea level, and does not always freeze at $0\unit{%
%TCIMACRO{\U{2103}}%
%BeginExpansion
{}^{\circ}{\rm C}%
%EndExpansion
}.$

Even worse, there is nothing really very $0$ish about $0\unit{%
%TCIMACRO{\U{2103}}%
%BeginExpansion
{}^{\circ}{\rm C}%
%EndExpansion
}.$ We picked it because it was a nice value for water. But we can certainly
have lower temperatures than $0\unit{%
%TCIMACRO{\U{2103}}%
%BeginExpansion
{}^{\circ}{\rm C}%
%EndExpansion
}.$\footnote{%
We often do in Rexburg from about December to March.} This makes $0\unit{%
%TCIMACRO{\U{2103}}%
%BeginExpansion
{}^{\circ}{\rm C}%
%EndExpansion
}$ more like an origin of a coordinate system. We could say that the
intersection of Main St. and Center St. is our origin, but there is noting
really zeroish about that intersection. But it turns out there \emph{is} a
natural zero point for temperature ! A better temperature scale would start
at the natural zero point. Let's see if we can make such a scale.

\subsection{The Constant-Volume Gas Thermometer and the Absolute Temperature
Scale}

We wish to define a temperature scale that is based on a physical zero
point. To do this we need a better thermometer. So here is one in the figure
below.\FRAME{dtbpF}{4.7285in}{1.9469in}{0pt}{}{}{Figure}{\special{language
"Scientific Word";type "GRAPHIC";maintain-aspect-ratio TRUE;display
"USEDEF";valid_file "T";width 4.7285in;height 1.9469in;depth
0pt;original-width 4.7897in;original-height 1.9558in;cropleft "0";croptop
"1";cropright "1";cropbottom "0";tempfilename
'PQXXR0XW.wmf';tempfile-properties "XPR";}}This device uses the rise in
pressure at a constant volume (the number 3 property from our list at the
beginning of the lecture). It works by placing the gas bulb in thermal
contact with the object who's temperature is to be measured. A scale (like a
ruler) is placed in the apparatus such that the mercury in reservoir $A$ is
at the zero point. We could take water at its ice point, for example. Then
the bulb is placed in thermal contact with another object, say water at its
steam point. The mercury will move because the pressure increases in the
bulb, and the gas tries to expand. We add mercury into reservoir B (just
when you thought you were done with U-tubes!) until the level of reservoir $%
A $ is again at the zero point. The height of the additional mercury we
added is proportional to the change in pressure, as we now know. The
Pressure for both measurements is calculated and plotted against the
temperature and a linear curve is drawn between the two pressure-temperature
points. This curve serves as a calibration for other temperatures.

Now we could fill the bulb with different gasses. If we do this, we get
different sloped lines. It seems like we should have zero temperature if we
have zero gas in the bulb. We have different lines for each gas, but if we
then extend these lines back to where $P=0$ (so that we have no gas), we
find they intersect with $P=0$ at $T=273.15\unit{%
%TCIMACRO{\U{2103}}%
%BeginExpansion
{}^{\circ}{\rm C}%
%EndExpansion
}.$ This is the basis we need for an absolute temperature scale!

\FRAME{dtbpF}{3.3492in}{1.9886in}{0in}{}{}{Figure}{\special{language
"Scientific Word";type "GRAPHIC";maintain-aspect-ratio TRUE;display
"USEDEF";valid_file "T";width 3.3492in;height 1.9886in;depth
0in;original-width 3.3847in;original-height 1.9984in;cropleft "0";croptop
"1";cropright "1";cropbottom "0";tempfilename
'PQXXR0XX.wmf';tempfile-properties "XPR";}}

We will take this common zero pressure point, $T=273.15\unit{%
%TCIMACRO{\U{2103}}%
%BeginExpansion
{}^{\circ}{\rm C}%
%EndExpansion
}$ as our zero point for our improved temperature scale, but we need another
point. We will choose the triple point of water ( the one point where water,
ice and steam exist in equilibrium; this happens at $0.01\unit{%
%TCIMACRO{\U{2103}}%
%BeginExpansion
{}^{\circ}{\rm C}%
%EndExpansion
}$ and $4.58\unit{mmHg}=$ $610.\,\allowbreak 61\unit{Pa}.$ We will call this
point $273.15$ on our new scale. Each unit of temperature will then be $%
1/273.15^{th}$ of the distance between our zero point and the water triple
point. This way the degrees in this scale are the same size as the degrees
in the Celsius scale (which is kind of nice). But this new scale that starts
at the physical zero point we will call the Kelvin Temperature scale and the
units are Kelvins ($\unit{K}$). This is the temperature system we really
should use.

\subsection{Temperature scale conversions}

We will need some conversion factors so you can take temperatures from the
Celsius and Fahrenheit scales to the Kelvin scale.%
\begin{eqnarray}
T_{F} &=&\frac{9}{5}T_{C}+32\unit{%
%TCIMACRO{\U{2109}}%
%BeginExpansion
{}^{\circ}{\rm F}%
%EndExpansion
} \\
\Delta T_{C} &=&\Delta T_{K}=\frac{5}{9}\Delta T_{F} \\
T_{C} &=&T_{K}-273.15
\end{eqnarray}

Since the Kelvin scale is not the scale your doctor or your meteorologist
uses, we should try to get some feeling for the Kelvin scale. Here are some
interesting temperatures placed on the Kelvin scale.

\FRAME{dtbpF}{2.2237in}{2.3017in}{0in}{}{}{Figure}{\special{language
"Scientific Word";type "GRAPHIC";maintain-aspect-ratio TRUE;display
"USEDEF";valid_file "T";width 2.2237in;height 2.3017in;depth
0in;original-width 2.237in;original-height 2.3168in;cropleft "0";croptop
"1";cropright "1";cropbottom "0";tempfilename
'PQXXR0XY.wmf';tempfile-properties "XPR";}}

\section{Phase Changes and Phase Diagrams}

Let's consider heating ice. You take a chunk of ice from outside on a cold
February day in Rexburg, Idaho. The ice has a temperature of about $-25\unit{%
%TCIMACRO{\U{2103}}%
%BeginExpansion
{}^{\circ}{\rm C}%
%EndExpansion
}.$\footnote{%
OK, that is an exaggeration, it very rarely gets that cold in Rexburg.} You
place it in a pan on your stove and heat it up. Then, being a physics
student, you plot the temperature as a function of time. Here is what you
get. \FRAME{dhF}{3.2292in}{2.0237in}{0pt}{}{}{Figure}{\special{language
"Scientific Word";type "GRAPHIC";maintain-aspect-ratio TRUE;display
"USEDEF";valid_file "T";width 3.2292in;height 2.0237in;depth
0pt;original-width 3.1842in;original-height 1.9847in;cropleft "0";croptop
"1";cropright "1";cropbottom "0";tempfilename
'PQXXR0XZ.wmf';tempfile-properties "XPR";}}Notice the places where you are
still heating, but the temperature does not change. These are phase changes.
The first is melting, the second is boiling.

One of the things you will notice is that the melted ice does not boil at $%
100\unit{%
%TCIMACRO{\U{2103}}%
%BeginExpansion
{}^{\circ}{\rm C}%
%EndExpansion
}$ here in Rexburg. This is because we are up so high that the air pressure
is less than $1\unit{atm}.$ In the next graph, I have plotted the melting,
and boiling point of a substance as a function of pressure. The melting
point and boiling point are not single points, but lines. This is because
melting and boiling points depend on pressure. There is also a line for the
sublimation point. Along this line, a solid will change directly into a gas.
Dry ice does this at normal room temperature and pressure. \FRAME{dhF}{%
2.3359in}{1.8931in}{0pt}{}{}{Figure}{\special{language "Scientific
Word";type "GRAPHIC";maintain-aspect-ratio TRUE;display "USEDEF";valid_file
"T";width 2.3359in;height 1.8931in;depth 0pt;original-width
3.6703in;original-height 2.9706in;cropleft "0";croptop "1";cropright
"1";cropbottom "0";tempfilename 'PQXXR0Y0.wmf';tempfile-properties "XPR";}}%
If we observe the boiling boiling line. We see that as pressure goes down,
the boiling temperature goes down. Look at $T_{1}$ and $T_{2}$ in the next
figure, where $T_{2}<T_{1}.$ Note that $P_{2}<P_{1}.$\FRAME{dhF}{1.9683in}{%
1.8222in}{0pt}{}{}{Figure}{\special{language "Scientific Word";type
"GRAPHIC";maintain-aspect-ratio TRUE;display "USEDEF";valid_file "T";width
1.9683in;height 1.8222in;depth 0pt;original-width 10.8577in;original-height
10.0474in;cropleft "0";croptop "1";cropright "1";cropbottom "0";tempfilename
'PQXXR0Y1.wmf';tempfile-properties "XPR";}}This agrees with our Rexburg
boiling point experience. We can see from the graph that melting or boiling
depends on pressure, and so does sublimation point.

The triple point is a special point where all three states are possible at
the same pressure and temperature. Ice, steam, and water will all exist in
equilibrium, for example, at the triple point of water.

The critical point is where the clear distinction between liquid and gas
stops. beyond this boiling the material is in a fluid state, but not clearly
liquid or gas.

Our diagram is rather schematic, and this is on purpose. An actual phase
diagram for water is given in the next figure.

\FRAME{dtbpFU}{4.203in}{1.9545in}{0pt}{\Qcb{More Complete Phase Diagram for
Water (Public Domain Image courtesy Karlhahn)}}{}{Figure}{\special{language
"Scientific Word";type "GRAPHIC";maintain-aspect-ratio TRUE;display
"USEDEF";valid_file "T";width 4.203in;height 1.9545in;depth
0pt;original-width 3.3589in;original-height 1.5471in;cropleft "0";croptop
"1";cropright "1";cropbottom "0";tempfilename
'PQXXR0Y2.wmf';tempfile-properties "XPR";}}Along each of the lines, we can
say that the material is in phase equilibrium, meaning that both phases from
either side of the line are possible and equally probable. Only at the
triple point are all three phases equally probable.

\chapter{Expansion of Solids and Gasses (Ideal Gas Law)}

We now have a better understanding of temperature, and we have new language
to describe systems that have large numbers of objects. We have the zeroth
law of thermodynamics. We are prepared to consider the next two laws. We
will consider thermal expansion in this lecture. We will start with solids,
but soon work our way to gasses. To model the thermal properties of gasses
we will introduce and use the Ideal Gas Law.

%TCIMACRO{%
%\TeXButton{Fundamental Concepts}{\hspace{-1.3in}{\Large Fundamental Concepts\vspace{0.25in}}}}%
%BeginExpansion
\hspace{-1.3in}{\Large Fundamental Concepts\vspace{0.25in}}%
%EndExpansion

\begin{itemize}
\item Thermal Expansion

\item Ideal Gas Law

\item Idea Gas Processes
\end{itemize}

\section{Thermal Expansion}

%TCIMACRO{%
%\TeXButton{Ring and Ball Demo}{\marginpar {
%\hspace{-0.5in}
%\begin{minipage}[t]{1in}
%\small{Ring and Ball Demo}
%\end{minipage}
%}}}%
%BeginExpansion
\marginpar {
\hspace{-0.5in}
\begin{minipage}[t]{1in}
\small{Ring and Ball Demo}
\end{minipage}
}%
%EndExpansion

We have not defined heat yet, but we kind of know what temperature is, so
suppose we increase the temperature of a substance. We could take the
mercury in a thermometer, for example. When the temperature is higher, the
mercury takes up more space. It expands. This expansion is linear in
temperature for many situations. When the expansion length is much less than
the original dimensions of the object $\left( L_{i}\right) $, we can
consider the expansion to be linear in the change in temperature $\Delta T$.
Of course we could increase the temperature above this nice linear region,
and we could get interesting effects like melting or exploding. But we will
limit our discussion to a linear change in length with temperature for now.

\subsection{Expansion Coefficient}

What does it mean to have a linear change in length with temperature? 
\begin{equation}
L_{f}=L_{i}+\alpha L_{i}\Delta T  \label{Linear Expansion}
\end{equation}%
It means that the final length of the object is equal to the initial length
of the object plus a little bit more that depends on $\Delta T$ and not $%
\Delta T^{2}$ or $\Delta T^{3}$ or some other function of $\Delta T.$ You
can see that this is the case for the equation above for thermal expansion.
We can write it next to an equation for a straight line to show the linear
nature of the equation%
\begin{eqnarray*}
y &=&b+mx \\
L_{f} &=&L_{i}+\alpha L_{i}\Delta T
\end{eqnarray*}%
If we plot $L_{f}$ vs. $\Delta T$ this would make a straight line. The slope
of the line would be $\alpha L_{i}$ so the amount of expansion depends on
how big the object was to begin with. It also depends on what material we
have. That is what the $\alpha $ tells us. Different substances expand at
different rates even if they have the same initial length. So if you have a
metal cavity of length $L_{i}$ and fill it with a laser crystal of length $%
L_{i}$ and heat them both up, you may find the crystal expands at a
different rate than the metal faseners when the temperature rises--and may
need to buy a replacement for the shattered crystal.\footnote{%
It was a very sad day when this happened to me personally.}. The coefficient 
$\alpha $ is different for every substance, because every substance expands
a little differently than other substances. We should give $\alpha $ a name.
It is called the \emph{average coefficient of linear expansion} where from
equation (\ref{Linear Expansion}) we can see that 
\begin{eqnarray*}
\alpha &\equiv &\frac{\frac{\Delta L}{L_{i}}}{\Delta T} \\
&=&\frac{\Delta L}{L_{i}\Delta T}
\end{eqnarray*}%
where $\Delta L$ is the change in length due to a temperature change $\Delta
T.$ And of course we already know $L_{i}$ is the original length of the
sample.

As long as $\Delta T$ is \textquotedblleft not too big\textquotedblright\ $%
\alpha $ is constant. We can write this as%
\[
L_{f}-L_{i}=\alpha L_{i}\left( T_{f}-T_{i}\right) 
\]%
where the subscript $f$ means final and $i$ means initial as usual. The
units of $\alpha $ are inverse temperature (unfortunately it is often in $%
\unit{%
%TCIMACRO{\U{2103}}%
%BeginExpansion
{}^{\circ}{\rm C}%
%EndExpansion
},$ but we would prefer to use $\unit{K}$). Laboratories publish tables of
average coefficients of linear expansion. A few are given in the table below%
\[
\begin{tabular}{ll}
Material & Average $\alpha \left( \unit{K}^{-1}\right) $ \\ 
Aluminum & $25\times 10^{-6}$ \\ 
Copper & $17\times 10^{-6}$ \\ 
Gold & $14\times 10^{-6}$ \\ 
Lead & $29\times 10^{-6}$ \\ 
Steel & $11\times 10^{-6}$ \\ 
Brass & $18.7\times 10^{-6}$ \\ 
quartz & $0.4\times 10^{-6}$ \\ 
Glass & $9\times 10^{-6}$ \\ 
Concrete & $12\times 10^{-6}$%
\end{tabular}%
\]

Suppose we increase the temperature of something that has a hole in it. Does
the hole increase or decrease in size? A cavity in a piece of material
expands in the same way as if the cavity were filled with the material.
Let's see that this must be true.

\subsection{Volume expansion}

%TCIMACRO{%
%\TeXButton{Don't do this in class}{\marginpar {
%\hspace{-0.5in}
%\begin{minipage}[t]{1in}
%\small{Don't do this in class}
%\end{minipage}
%}}}%
%BeginExpansion
\marginpar {
\hspace{-0.5in}
\begin{minipage}[t]{1in}
\small{Don't do this in class}
\end{minipage}
}%
%EndExpansion
suppose we have a cube with sides of length $L_{i}.$ What happens when we
increase the cube's temperature? Each side will become larger. So what
happened to the volume?%
\begin{eqnarray*}
%TCIMACRO{\TeXButton{V}{\ooalign{\hfil$V$\hfil\cr\kern0.1em--\hfil\cr}}}%
%BeginExpansion
\ooalign{\hfil$V$\hfil\cr\kern0.1em--\hfil\cr}%
%EndExpansion
_{i}+\Delta 
%TCIMACRO{\TeXButton{V}{\ooalign{\hfil$V$\hfil\cr\kern0.1em--\hfil\cr}} }%
%BeginExpansion
\ooalign{\hfil$V$\hfil\cr\kern0.1em--\hfil\cr}
%EndExpansion
&=&\left( L_{i}+\Delta L\right) \left( L_{i}+\Delta L\right) \left(
L_{i}+\Delta L\right) \\
&=&\left( L_{i}+\alpha L_{i}\Delta T\right) \left( L_{i}+\alpha L_{i}\Delta
T\right) \left( L_{i}+\alpha L_{i}\Delta T\right) \\
&=&L_{i}^{3}\left( 1+\alpha \Delta T\right) \left( 1+\alpha \Delta T\right)
\left( 1+\alpha \Delta T\right) \\
&=&L_{i}^{3}\left( \allowbreak \Delta T^{3}\alpha ^{3}+3\Delta T^{2}\alpha
^{2}+3\Delta T\alpha +1\right) \\
&=&%
%TCIMACRO{\TeXButton{V}{\ooalign{\hfil$V$\hfil\cr\kern0.1em--\hfil\cr}}}%
%BeginExpansion
\ooalign{\hfil$V$\hfil\cr\kern0.1em--\hfil\cr}%
%EndExpansion
_{i}\left( \allowbreak \Delta T^{3}\alpha ^{3}+3\Delta T^{2}\alpha
^{2}+3\Delta T\alpha +1\right)
\end{eqnarray*}%
Now let's divide by $%
%TCIMACRO{\TeXButton{V}{\ooalign{\hfil$V$\hfil\cr\kern0.1em--\hfil\cr}}}%
%BeginExpansion
\ooalign{\hfil$V$\hfil\cr\kern0.1em--\hfil\cr}%
%EndExpansion
_{i}$ 
\begin{eqnarray*}
1+\frac{\Delta 
%TCIMACRO{\TeXButton{V}{\ooalign{\hfil$V$\hfil\cr\kern0.1em--\hfil\cr}}}%
%BeginExpansion
\ooalign{\hfil$V$\hfil\cr\kern0.1em--\hfil\cr}%
%EndExpansion
}{%
%TCIMACRO{\TeXButton{V}{\ooalign{\hfil$V$\hfil\cr\kern0.1em--\hfil\cr}}}%
%BeginExpansion
\ooalign{\hfil$V$\hfil\cr\kern0.1em--\hfil\cr}%
%EndExpansion
_{i}} &=&\allowbreak \Delta T^{3}\alpha ^{3}+3\Delta T^{2}\alpha
^{2}+3\Delta T\alpha +1 \\
\frac{\Delta 
%TCIMACRO{\TeXButton{V}{\ooalign{\hfil$V$\hfil\cr\kern0.1em--\hfil\cr}}}%
%BeginExpansion
\ooalign{\hfil$V$\hfil\cr\kern0.1em--\hfil\cr}%
%EndExpansion
}{%
%TCIMACRO{\TeXButton{V}{\ooalign{\hfil$V$\hfil\cr\kern0.1em--\hfil\cr}}}%
%BeginExpansion
\ooalign{\hfil$V$\hfil\cr\kern0.1em--\hfil\cr}%
%EndExpansion
_{i}} &=&\allowbreak \Delta T^{3}\alpha ^{3}+3\Delta T^{2}\alpha
^{2}+3\Delta T\alpha
\end{eqnarray*}%
and let's make an approximation. For normal conditions, $\alpha \Delta T\ll
1 $ (good for $T<373.\,\allowbreak 15\unit{K}$ ($100\unit{%
%TCIMACRO{\U{2103}}%
%BeginExpansion
{}^{\circ}{\rm C}%
%EndExpansion
}$)). Then%
\begin{equation}
\frac{\Delta 
%TCIMACRO{\TeXButton{V}{\ooalign{\hfil$V$\hfil\cr\kern0.1em--\hfil\cr}}}%
%BeginExpansion
\ooalign{\hfil$V$\hfil\cr\kern0.1em--\hfil\cr}%
%EndExpansion
}{%
%TCIMACRO{\TeXButton{V}{\ooalign{\hfil$V$\hfil\cr\kern0.1em--\hfil\cr}}}%
%BeginExpansion
\ooalign{\hfil$V$\hfil\cr\kern0.1em--\hfil\cr}%
%EndExpansion
_{i}}\approx \allowbreak 3\alpha \Delta T
\end{equation}%
We can write this as 
\begin{eqnarray}
\Delta 
%TCIMACRO{\TeXButton{V}{\ooalign{\hfil$V$\hfil\cr\kern0.1em--\hfil\cr}} }%
%BeginExpansion
\ooalign{\hfil$V$\hfil\cr\kern0.1em--\hfil\cr}
%EndExpansion
&\approx &\allowbreak 3\alpha 
%TCIMACRO{\TeXButton{V}{\ooalign{\hfil$V$\hfil\cr\kern0.1em--\hfil\cr}}}%
%BeginExpansion
\ooalign{\hfil$V$\hfil\cr\kern0.1em--\hfil\cr}%
%EndExpansion
_{i}\Delta T \\
&=&\beta 
%TCIMACRO{\TeXButton{V}{\ooalign{\hfil$V$\hfil\cr\kern0.1em--\hfil\cr}}}%
%BeginExpansion
\ooalign{\hfil$V$\hfil\cr\kern0.1em--\hfil\cr}%
%EndExpansion
_{i}\Delta T  \nonumber
\end{eqnarray}%
where $\beta $ is called the \emph{average coefficient of volume expansion.}

%TCIMACRO{%
%\TeXButton{Question 123.9.1}{\marginpar {
%\hspace{-0.5in}
%\begin{minipage}[t]{1in}
%\small{Question 123.9.1}
%\end{minipage}
%}}}%
%BeginExpansion
\marginpar {
\hspace{-0.5in}
\begin{minipage}[t]{1in}
\small{Question 123.9.1}
\end{minipage}
}%
%EndExpansion

We could do the same for a two dimensional metal plate (in the approximation
that a plate is 2D!)

\begin{equation}
\Delta A=2\alpha A_{i}\Delta T
\end{equation}

We need to be careful that we don't use these linear expansion equations
outside of their valid range. For example, suppose we build a satellite and
launch it into a low Earth orbit (LEO). The space environment for a LEO
orbit (sun side to shade side) has a very large $\Delta T.$

\begin{equation}
\begin{tabular}{l}
$T_{sun}=120\unit{%
%TCIMACRO{\U{2103}}%
%BeginExpansion
{}^{\circ}{\rm C}%
%EndExpansion
}=393.\,\allowbreak 15\unit{K}$ \\ 
$T_{shade}=-100\unit{%
%TCIMACRO{\U{2103}}%
%BeginExpansion
{}^{\circ}{\rm C}%
%EndExpansion
}=173.\,\allowbreak 15\unit{K}$%
\end{tabular}%
\end{equation}%
This $\Delta T$ range is too large for most materials. Special low $\alpha $
materials are used to allow satellite structures to survive

But how about on Earth? Even on Earth we can have extremes

\begin{equation}
\begin{tabular}{l}
$T_{\max }=57.6\unit{%
%TCIMACRO{\U{2103}}%
%BeginExpansion
{}^{\circ}{\rm C}%
%EndExpansion
}=330.\,\allowbreak 75\unit{K}$ \\ 
$T_{\min }=-89\unit{%
%TCIMACRO{\U{2103}}%
%BeginExpansion
{}^{\circ}{\rm C}%
%EndExpansion
}=184.\,\allowbreak 15\unit{K}$%
\end{tabular}%
\end{equation}

This $\Delta T$ range would be a challenge to cover with most materials.%
\footnote{%
Expeciall metal structures and laser crystals.} So if you are going to build
a device or instrument that will travel around the globe, you need to use
special low $\alpha $ materials. Often making these materials is very
expensive and the process is a tightly held industrial secret.

\subsection{Different materials}

Looking at table of average linear expansion coefficients we find see
different $\alpha $ values for different materials. Sometimes $\alpha $
values are quite different. Suppose we take two materials with different $%
\alpha $'s and put them together. \FRAME{dtbpF}{3.4229in}{1.3189in}{0pt}{}{}{%
Figure}{\special{language "Scientific Word";type
"GRAPHIC";maintain-aspect-ratio TRUE;display "USEDEF";valid_file "T";width
3.4229in;height 1.3189in;depth 0pt;original-width 3.4601in;original-height
1.3154in;cropleft "0";croptop "1";cropright "1";cropbottom "0";tempfilename
'PQXXR0Y3.wmf';tempfile-properties "XPR";}}The figure shows a bimetallic
strip. What would happen if we heated this bimetallic strip? The two metals
will expand at different rates, forcing the bonded pair of metals to bend.

\subsection{Water is weird (and the Lord knows what He is doing!)}

%TCIMACRO{%
%\TeXButton{Question123.9.2}{\marginpar {
%\hspace{-0.5in}
%\begin{minipage}[t]{1in}
%\small{Question 123.9.2}
%\end{minipage}
%}}}%
%BeginExpansion
\marginpar {
\hspace{-0.5in}
\begin{minipage}[t]{1in}
\small{Question 123.9.2}
\end{minipage}
}%
%EndExpansion
\FRAME{dtbpFU}{2.8746in}{2.3082in}{0pt}{\Qcb{Density of water near $0\unit{%
%TCIMACRO{\U{2103}}%
%BeginExpansion
{}^{\circ}{\rm C}%
%EndExpansion
}.$ (Public Domain image PD-1923, PD-Australia)}}{}{Figure}{\special%
{language "Scientific Word";type "GRAPHIC";maintain-aspect-ratio
TRUE;display "USEDEF";valid_file "T";width 2.8746in;height 2.3082in;depth
0pt;original-width 2.8314in;original-height 2.2684in;cropleft "0";croptop
"1";cropright "1";cropbottom "0";tempfilename
'PQXXR0Y4.wmf';tempfile-properties "XPR";}}

For most substances the volume increases with increasing temperature.
Usually liquids increase in volume more than solids. But water is different.
Look at the graph of density vs. Temperature.

Just as water melts, it starts to increase in density, but then it becomes
less dense with increased temperature after about $277.\,\allowbreak 15\unit{%
K}$ ($4\unit{%
%TCIMACRO{\U{2103}}%
%BeginExpansion
{}^{\circ}{\rm C}%
%EndExpansion
}$). We expect the behavior $277.\,\allowbreak 15\unit{K}$ ($4\unit{%
%TCIMACRO{\U{2103}}%
%BeginExpansion
{}^{\circ}{\rm C}%
%EndExpansion
}$), but from $273.\,\allowbreak 15\unit{K}$ ($0\unit{%
%TCIMACRO{\U{2103}}%
%BeginExpansion
{}^{\circ}{\rm C}%
%EndExpansion
}$) to $277.\,\allowbreak 15\unit{K}$ ($4\unit{%
%TCIMACRO{\U{2103}}%
%BeginExpansion
{}^{\circ}{\rm C}%
%EndExpansion
}$) we see the volume decreases with $\Delta T.$ This is why a lake freezes
from the top down. The colder water near $273.\,\allowbreak 15\unit{K}$ ($0%
\unit{%
%TCIMACRO{\U{2103}}%
%BeginExpansion
{}^{\circ}{\rm C}%
%EndExpansion
}$) is less dense, and so before it freezes it floats to the surface. This
is great, because if it became more dense, the water would freeze and sink
to the bottom, leaving the water exposed ot the cold atmosphere to freeze
more water. The lake would fill up from the bottom with solid ice, killing
fish and water life.

\FRAME{dtbpFU}{3.3555in}{2.5918in}{0pt}{\Qcb{Lake Erie from Space. Notice
the ice covering part of the lake. (Public Domain Image courtesy NASA\
Visible Earth, http://visibleearth.nasa.gov/view\_rec.php?id=1286}}{}{Figure%
}{\special{language "Scientific Word";type "GRAPHIC";maintain-aspect-ratio
TRUE;display "USEDEF";valid_file "T";width 3.3555in;height 2.5918in;depth
0pt;original-width 5.2166in;original-height 4.0231in;cropleft "0";croptop
"1";cropright "1";cropbottom "0";tempfilename
'PQXXR0Y5.wmf';tempfile-properties "XPR";}}

%TCIMACRO{%
%\TeXButton{Question123.9.2}{\marginpar {
%\hspace{-0.5in}
%\begin{minipage}[t]{1in}
%\small{Question 123.9.2}
%\end{minipage}
%}}}%
%BeginExpansion
\marginpar {
\hspace{-0.5in}
\begin{minipage}[t]{1in}
\small{Question 123.9.2}
\end{minipage}
}%
%EndExpansion
%TCIMACRO{%
%\TeXButton{Question123.9.3}{\marginpar {
%\hspace{-0.5in}
%\begin{minipage}[t]{1in}
%\small{Question 123.9.3}
%\end{minipage}
%}}}%
%BeginExpansion
\marginpar {
\hspace{-0.5in}
\begin{minipage}[t]{1in}
\small{Question 123.9.3}
\end{minipage}
}%
%EndExpansion

\section{Quazi-static processes}

Consider the situation shown below\FRAME{dhF}{1.6215in}{2.2745in}{0pt}{}{}{%
Figure}{\special{language "Scientific Word";type
"GRAPHIC";maintain-aspect-ratio TRUE;display "USEDEF";valid_file "T";width
1.6215in;height 2.2745in;depth 0pt;original-width 1.5852in;original-height
2.2347in;cropleft "0";croptop "1";cropright "1";cropbottom "0";tempfilename
'PQXXR0Y6.wmf';tempfile-properties "XPR";}}We have a gas in one side of a
sealed container and a vacuum on the other side. A membrane separates the
two. We have a very definite temperature in the gas, and also a definite
temperature in the vacuum. But if I puncture the membrane, \FRAME{dhF}{%
1.7755in}{2.4976in}{0pt}{}{}{Figure}{\special{language "Scientific
Word";type "GRAPHIC";maintain-aspect-ratio TRUE;display "USEDEF";valid_file
"T";width 1.7755in;height 2.4976in;depth 0pt;original-width
1.7383in;original-height 2.4569in;cropleft "0";croptop "1";cropright
"1";cropbottom "0";tempfilename 'PQXXR0Y7.wmf';tempfile-properties "XPR";}}%
gas goes swirling into the vacuum. The density will not be constant as the
gas swirls. Where there is gas, we would say we have a temperature closer to 
$T_{i}.$ But where there is no gas, we have a temperature nearer $0\unit{K}.$
It takes a while for the gas to come to equilibrium. During this process we
can't say that we have a definite temperature, or definite pressure for the
system. Higher physics classes (and engineering classes) deal with the
difficult situation that exists before equilibrium is achieved. For now, we
will only look at situations that have reached an equilibrium, or are not
too far from equilibrium. We call states that change slowly so they are
never far from equilibrium, \emph{quazi-static}. The situation we have
described is called a \emph{free expansion} because the gas is allowed to
expand with no outside influences. A free expansion is \emph{not} a
quasi-static process. Let's think of how we might make a quasi-static
process happen.

\section{Reversible and Irreversible Processes}

A reversible process is one in which every point along some path is an
equilibrium state and one for which the system can be returned to its
initial state along the same path.

All natural processes are known to be irreversible. Reversible processes are
an idealization, like frictionless surfaces or massless strings. But some
real processes are good approximations to reversible processes.

Our free expansion, of course, is not a reversible process. Imagine trying
to get all the gas atoms back through the hole in the membrane!

%TCIMACRO{%
%\TeXButton{BYU Demo}{\marginpar {
%\hspace{-0.5in}
%\begin{minipage}[t]{1in}
%\small{BYU Demo}
%\end{minipage}
%}}}%
%BeginExpansion
\marginpar {
\hspace{-0.5in}
\begin{minipage}[t]{1in}
\small{BYU Demo}
\end{minipage}
}%
%EndExpansion

We can do work on the gas to achieve this (say, put a piston in the system
and apply a force). This would increase the temperature, so we would have to
let energy transfer out of the system (by heat). This would allow the system
to return to it's final state, but the surroundings will have changed. This
is not a reversible process!

A quasi-static process can be nearly reversible. \FRAME{dtbpF}{1.3569in}{%
1.3742in}{0pt}{}{}{Figure}{\special{language "Scientific Word";type
"GRAPHIC";maintain-aspect-ratio TRUE;display "USEDEF";valid_file "T";width
1.3569in;height 1.3742in;depth 0pt;original-width 4.3059in;original-height
4.3595in;cropleft "0";croptop "1";cropright "1";cropbottom "0";tempfilename
'PQXXR0Y8.wmf';tempfile-properties "XPR";}}Imagine dropping sand on a
frictionless piston to compress a gas. Then slowly removing the sand. The
system stays almost in equilibrium as it changes, so that it returns to a
new equilibrium right away. After all, you only moved one grain of sand at a
time. This process is almost reversible. Of course, you still have to do
work in moving the sand, so the outside environment is changed, so even this
process is not truly reversible. But it is a good mental model for the sort
of processes we will study in this class; slow, almost in equilibrium all
the time.

\section{Ideal Gases}

%TCIMACRO{%
%\TeXButton{Question123.9.4-6}{\marginpar {
%\hspace{-0.5in}
%\begin{minipage}[t]{1in}
%\small{Question 123.9.4-6}
%\end{minipage}
%}}}%
%BeginExpansion
\marginpar {
\hspace{-0.5in}
\begin{minipage}[t]{1in}
\small{Question 123.9.4-6}
\end{minipage}
}%
%EndExpansion

Let's consider the following data\FRAME{dhF}{2.6472in}{1.5947in}{0pt}{}{}{%
Figure}{\special{language "Scientific Word";type
"GRAPHIC";maintain-aspect-ratio TRUE;display "USEDEF";valid_file "T";width
2.6472in;height 1.5947in;depth 0pt;original-width 6.7014in;original-height
4.0248in;cropleft "0";croptop "1";cropright "1";cropbottom "0";tempfilename
'PQXXR0Y9.wmf';tempfile-properties "XPR";}}What can we say about the curve
fit?

It might look useless. But look at the region from about $-6<x<5.$ In this
region, the curve fit is not too bad. Of course, the curve fit is not the
right equation, and we know this from the regions $x<-6$ and $x>4.$ So our
researcher that collected this data does not have an exact theory that
explains the behavior of his/her system well. But could the curve fit be
useful? If all you needed was an estimate of the value of the actual
function at $x=-1,$ then the curve fit might be enough to get your work
done! This may not be terribly satisfying, but sometimes it is a useful way
to work.

Another way to think about this is that if most of the time, under usual
conditions, the phenomena we experience fall within the range $-6<x<5$ and
the curve fit equation is much simpler, it might be good enough and
convenient to use the curve fit. For example, we know that Newton's laws are
not exact. We have to use General Relativity to be accurate. But in
calculating our average speed on a trip to Idaho Falls, Newton's laws are
plenty good enough, and \emph{much} more convenient than their General
Relativistic forms.

We will use the Ideal Gas Law to study the expansion of gasses as
temperature changes. The Ideal Gas Law is an approximation very like our
curve fit we have been considering. It works pretty well under normal
conditions, and it is much easier to use than the exact law. It allows us to
gain great insight, without the mathematical difficulty of the exact
relationship.

Given all this, we should realize that there is no such thing as and ideal
gas. But in this approximation, the atoms are so weakly bound that there is
no equilibrium position between atoms. Thus there is no definite volume
defined for an ideal gas. This is generally true at standard pressure and
temperature for real gasses. So we will let $%
%TCIMACRO{\TeXButton{V}{\ooalign{\hfil$V$\hfil\cr\kern0.1em--\hfil\cr}}}%
%BeginExpansion
\ooalign{\hfil$V$\hfil\cr\kern0.1em--\hfil\cr}%
%EndExpansion
$ be a variable for gases. Let's look at where the ideal gas law came from.
It is much like our hypothetical experiment. The experimentalists gave us
some clues on how to relate $%
%TCIMACRO{\TeXButton{V}{\ooalign{\hfil$V$\hfil\cr\kern0.1em--\hfil\cr}}}%
%BeginExpansion
\ooalign{\hfil$V$\hfil\cr\kern0.1em--\hfil\cr}%
%EndExpansion
,$ $P,$ and $T.$\FRAME{dtbpF}{0.9074in}{1.931in}{0pt}{}{}{Figure}{\special%
{language "Scientific Word";type "GRAPHIC";maintain-aspect-ratio
TRUE;display "USEDEF";valid_file "T";width 0.9074in;height 1.931in;depth
0pt;original-width 1.3367in;original-height 2.8783in;cropleft "0";croptop
"1";cropright "1";cropbottom "0";tempfilename
'PQXXR0YA.wmf';tempfile-properties "XPR";}}

\subsection{What lead to the Ideal Gas Law}

Boyle found that if a system is kept at a constant temperature then 
\begin{equation}
P\propto \frac{1}{%
%TCIMACRO{\TeXButton{V}{\ooalign{\hfil$V$\hfil\cr\kern0.1em--\hfil\cr}}}%
%BeginExpansion
\ooalign{\hfil$V$\hfil\cr\kern0.1em--\hfil\cr}%
%EndExpansion
}\text{ for constant }T
\end{equation}%
\FRAME{dtbpFX}{3.1669in}{2.111in}{0pt}{}{}{Plot}{\special{language
"Scientific Word";type "MAPLEPLOT";width 3.1669in;height 2.111in;depth
0pt;display "USEDEF";plot_snapshots TRUE;mustRecompute FALSE;lastEngine
"MuPAD";animated TRUE;animationStartTime "0"; animationEndTime "10";
animationFramesPerSecond "10";xmin "0";xmax "100";animationParamMin
"290";animationParamMax "350";xviewmin "0";xviewmax "40";yviewmin
"0";yviewmax "400";viewset"XY";rangeset"X";plottype 4;labeloverrides
3;x-label "V(m^3)";y-label "P(Pa)";axesFont "Times New
Roman,12,0000000000,useDefault,normal";numpoints 100;plotstyle
"patch";axesstyle "normal";axestips FALSE;xis \TEXUX{V};animationParam
\TEXUX{t};var1name \TEXUX{$V$};animationParamList \TEXUX{$t$};function
\TEXUX{$\allowbreak 8.\,\allowbreak 314\frac{1}{V}t$};linecolor
"blue";linestyle 1;pointstyle "point";linethickness 1;lineAttributes
"Solid";var1range "0,100";animationParamRange "290,350";num-x-gridlines
50;curveColor "[flat::RGB:0x000000ff]";curveStyle "Line";animationStartTime
"0"; animationEndTime "10"; animationFramesPerSecond
"10";animationVisibleBeforeStart FALSE;rangeset"XA";function
\TEXUX{$\allowbreak 8.\,\allowbreak 314\frac{1}{V}290$};linecolor
"blue";linestyle 2;pointstyle "point";linethickness 1;lineAttributes
"Dash";var1range "0,40";animationParamRange "290,350";num-x-gridlines
100;curveColor "[flat::RGB:0x000000ff]";curveStyle "Line";VCamFile
'PTF4LI00.xvz';valid_file "T";tempfilename
'PQXXR0YB.wmf';tempfile-properties "XPR";}}That is, if you don't let the
temperature of the gas change, then a change in pressures is inversely
related to a change in the volume of the gas. The more pressure, the smaller
the volume.

Two researchers, Charles and Gay-Lussac, found that if a system is kept at
constant pressure then%
\begin{equation}
%TCIMACRO{\TeXButton{V}{\ooalign{\hfil$V$\hfil\cr\kern0.1em--\hfil\cr}}}%
%BeginExpansion
\ooalign{\hfil$V$\hfil\cr\kern0.1em--\hfil\cr}%
%EndExpansion
\propto T\text{ for constant }P
\end{equation}%
\FRAME{dtbpFX}{3.1669in}{2.111in}{0pt}{}{}{Plot}{\special{language
"Scientific Word";type "MAPLEPLOT";width 3.1669in;height 2.111in;depth
0pt;display "USEDEF";plot_snapshots TRUE;mustRecompute FALSE;lastEngine
"MuPAD";animated TRUE;animationStartTime "0"; animationEndTime "10";
animationFramesPerSecond "10";xmin "0";xmax "100";animationParamMin
"290";animationParamMax "350";xviewmin "0";xviewmax "40";yviewmin
"0";yviewmax "400";viewset"XY";rangeset"X";plottype 4;labeloverrides
3;x-label "V(m^3)";y-label "P(Pa)";axesFont "Times New
Roman,12,0000000000,useDefault,normal";numpoints 100;plotstyle
"patch";axesstyle "normal";axestips FALSE;xis \TEXUX{V};animationParam
\TEXUX{t};var1name \TEXUX{$V$};animationParamList \TEXUX{$t$};function
\TEXUX{$200$};linecolor "blue";linestyle 1;pointstyle "point";linethickness
1;lineAttributes "Solid";var1range "0,100";animationParamRange
"290,350";num-x-gridlines 50;curveColor "[flat::RGB:0x000000ff]";curveStyle
"Line";animationStartTime "0"; animationEndTime "10";
animationFramesPerSecond "10";animationVisibleBeforeStart
FALSE;rangeset"XA";VCamFile 'PTF4LT01.xvz';valid_file "T";tempfilename
'PQXXR0YC.wmf';tempfile-properties "XPR";}}The warmer the gas, the more it
will expand.

Combining these two findings gives an equation that describes ideal gasses.

\begin{equation}
P%
%TCIMACRO{\TeXButton{V}{\ooalign{\hfil$V$\hfil\cr\kern0.1em--\hfil\cr}}}%
%BeginExpansion
\ooalign{\hfil$V$\hfil\cr\kern0.1em--\hfil\cr}%
%EndExpansion
=nRT
\end{equation}%
where $n$ is the number of moles, and $R$ is called the \emph{universal gas
constant}. This is the \emph{ideal gas law}. The law came from researchers
that could not access extreme pressures or temperatures, and so for normal
conditions, it works just fine. But it will not work at all for extremely
low or height pressures and temperatures.

The variables $n,$ $%
%TCIMACRO{\TeXButton{V}{\ooalign{\hfil$V$\hfil\cr\kern0.1em--\hfil\cr}}}%
%BeginExpansion
\ooalign{\hfil$V$\hfil\cr\kern0.1em--\hfil\cr}%
%EndExpansion
,$ $P,$ and $T,$ are state variables for ideal gasses. Depending on the
units for $%
%TCIMACRO{\TeXButton{V}{\ooalign{\hfil$V$\hfil\cr\kern0.1em--\hfil\cr}}}%
%BeginExpansion
\ooalign{\hfil$V$\hfil\cr\kern0.1em--\hfil\cr}%
%EndExpansion
,$ $P,$ and $T,$ the value for $R$ may be different.%
\begin{eqnarray}
R &=&8.314\frac{\unit{J}}{\unit{mol}\unit{K}} \\
&=&0.08214\frac{\unit{atm}\unit{l}}{\unit{mol}\unit{K}}  \nonumber
\end{eqnarray}

\subsection{Alternate form of the Ideal Gas Law}

We know we can describe the number of moles as 
\begin{equation}
n=\frac{N}{N_{A}}
\end{equation}%
where $N$ is the number of molecules of gas that we have. Then we can write%
\begin{equation}
P%
%TCIMACRO{\TeXButton{V}{\ooalign{\hfil$V$\hfil\cr\kern0.1em--\hfil\cr}}}%
%BeginExpansion
\ooalign{\hfil$V$\hfil\cr\kern0.1em--\hfil\cr}%
%EndExpansion
=\frac{N}{N_{A}}RT
\end{equation}%
if we define a new constant 
\begin{equation}
k_{B}=\frac{R}{N_{A}}
\end{equation}%
then%
\begin{equation}
P%
%TCIMACRO{\TeXButton{V}{\ooalign{\hfil$V$\hfil\cr\kern0.1em--\hfil\cr}}}%
%BeginExpansion
\ooalign{\hfil$V$\hfil\cr\kern0.1em--\hfil\cr}%
%EndExpansion
=Nk_{B}RT
\end{equation}%
the constant $k_{B}$ is called \emph{Boltzmann's constant}

\begin{equation}
k_{B}=1.38\times 10^{-23}\frac{\unit{J}}{\unit{K}}
\end{equation}

%TCIMACRO{%
%\TeXButton{Fire Syringe Demo}{\marginpar {
%\hspace{-0.5in}
%\begin{minipage}[t]{1in}
%\small{Fire Syringe Demo}
%\end{minipage}
%}}}%
%BeginExpansion
\marginpar {
\hspace{-0.5in}
\begin{minipage}[t]{1in}
\small{Fire Syringe Demo}
\end{minipage}
}%
%EndExpansion

\section{Ideal Gas Special Processes}

In what we did above, I introduced a new kind of graph. I graphed $P$ vs. $%
%TCIMACRO{\TeXButton{V}{\ooalign{\hfil$V$\hfil\cr\kern0.1em--\hfil\cr}}}%
%BeginExpansion
\ooalign{\hfil$V$\hfil\cr\kern0.1em--\hfil\cr}%
%EndExpansion
.$ In the ideal gas law, there are four state variables, $n,$ $P,$ $%
%TCIMACRO{\TeXButton{V}{\ooalign{\hfil$V$\hfil\cr\kern0.1em--\hfil\cr}}}%
%BeginExpansion
\ooalign{\hfil$V$\hfil\cr\kern0.1em--\hfil\cr}%
%EndExpansion
,$ and $T.$ But if we have a sealed container so $n$ can't change, we can
describe the other three variables on one graph, so each point on the graph
will show the possible states of the gas. This seems almost too good to be
true, By plotting only $P$ and $%
%TCIMACRO{\TeXButton{V}{\ooalign{\hfil$V$\hfil\cr\kern0.1em--\hfil\cr}}}%
%BeginExpansion
\ooalign{\hfil$V$\hfil\cr\kern0.1em--\hfil\cr}%
%EndExpansion
,$ how can we get $T$ on the graph?

Well, think about the ideal gas law if $n$ is constant%
\[
\frac{P%
%TCIMACRO{\TeXButton{V}{\ooalign{\hfil$V$\hfil\cr\kern0.1em--\hfil\cr}}}%
%BeginExpansion
\ooalign{\hfil$V$\hfil\cr\kern0.1em--\hfil\cr}%
%EndExpansion
}{T}=nR 
\]%
the right hand side is constant. Now suppose we also let $T$ be constant. 
\[
P=\left( nRT\right) \frac{1}{%
%TCIMACRO{\TeXButton{V}{\ooalign{\hfil$V$\hfil\cr\kern0.1em--\hfil\cr}}}%
%BeginExpansion
\ooalign{\hfil$V$\hfil\cr\kern0.1em--\hfil\cr}%
%EndExpansion
} 
\]%
everything in the parenthesis is constant so 
\[
P\propto \frac{1}{%
%TCIMACRO{\TeXButton{V}{\ooalign{\hfil$V$\hfil\cr\kern0.1em--\hfil\cr}}}%
%BeginExpansion
\ooalign{\hfil$V$\hfil\cr\kern0.1em--\hfil\cr}%
%EndExpansion
} 
\]%
We can plot this for

\[
\begin{tabular}{ll}
$n=1\unit{mol}$ & number of moles \\ 
$R=8.314\frac{\unit{J}}{\unit{mol}\unit{K}}$ & Universal gas constant \\ 
$T=290\unit{K}$ & Temperature in Kelvins%
\end{tabular}%
\]

\FRAME{dtbpFX}{3.1669in}{2.111in}{0pt}{}{}{Plot}{\special{language
"Scientific Word";type "MAPLEPLOT";width 3.1669in;height 2.111in;depth
0pt;display "USEDEF";plot_snapshots TRUE;mustRecompute FALSE;lastEngine
"MuPAD";animated TRUE;animationStartTime "0"; animationEndTime "10";
animationFramesPerSecond "10";xmin "0";xmax "40";animationParamMin
"290";animationParamMax "350";xviewmin "0";xviewmax "40";yviewmin
"0";yviewmax "400";viewset"XY";rangeset"X";plottype 4;labeloverrides
3;x-label "V(m^3)";y-label "P(Pa)";axesFont "Times New
Roman,12,0000000000,useDefault,normal";numpoints 100;plotstyle
"patch";axesstyle "normal";axestips FALSE;xis \TEXUX{V};animationParam
\TEXUX{t};var1name \TEXUX{$V$};animationParamList \TEXUX{$t$};function
\TEXUX{$\allowbreak 8.\,\allowbreak 314\frac{1}{V}290$};linecolor
"blue";linestyle 1;pointstyle "point";linethickness 3;lineAttributes
"Solid";var1range "0,40";animationParamRange "290,350";num-x-gridlines
100;curveColor "[flat::RGB:0x000000ff]";curveStyle "Line";VCamFile
'PTF4M202.xvz';valid_file "T";tempfilename
'PQXXR0YD.wmf';tempfile-properties "XPR";}}now let's plot it again for a
different value of $T$%
\[
\begin{tabular}{ll}
$n=1\unit{mol}$ & number of moles \\ 
$R=8.314\frac{\unit{J}}{\unit{mol}\unit{K}}$ & Universal gas constant \\ 
$T=590\unit{K}$ & Temperature in Kelvins%
\end{tabular}%
\]%
\FRAME{dtbpFX}{3.1669in}{2.111in}{0pt}{}{}{Plot}{\special{language
"Scientific Word";type "MAPLEPLOT";width 3.1669in;height 2.111in;depth
0pt;display "USEDEF";plot_snapshots TRUE;mustRecompute FALSE;lastEngine
"MuPAD";animated TRUE;animationStartTime "0"; animationEndTime "10";
animationFramesPerSecond "10";xmin "0";xmax "40";animationParamMin
"290";animationParamMax "350";xviewmin "0";xviewmax "40";yviewmin
"0";yviewmax "400";viewset"XY";rangeset"X";plottype 4;labeloverrides
3;x-label "V(m^3)";y-label "P(Pa)";axesFont "Times New
Roman,12,0000000000,useDefault,normal";numpoints 100;plotstyle
"patch";axesstyle "normal";axestips FALSE;xis \TEXUX{V};animationParam
\TEXUX{t};var1name \TEXUX{$V$};animationParamList \TEXUX{$t$};function
\TEXUX{$\allowbreak 8.\,\allowbreak 314\frac{1}{V}590$};linecolor
"blue";linestyle 1;pointstyle "point";linethickness 3;lineAttributes
"Solid";var1range "0,40";animationParamRange "290,350";num-x-gridlines
100;curveColor "[flat::RGB:0x000000ff]";curveStyle "Line";VCamFile
'PTF4NI06.xvz';valid_file "T";tempfilename
'PQXXR0YE.wmf';tempfile-properties "XPR";}}Notice that the curve looks like
the same shape, but it has been shifted toward the upper right hand corner
of the graph. If we place both the $T=290$ and $T=590$ graphs on the same
plot, we can better see the shift. \FRAME{dtbpFX}{3.1669in}{2.111in}{0pt}{}{%
}{Plot}{\special{language "Scientific Word";type "MAPLEPLOT";width
3.1669in;height 2.111in;depth 0pt;display "USEDEF";plot_snapshots
TRUE;mustRecompute FALSE;lastEngine "MuPAD";animated TRUE;animationStartTime
"0"; animationEndTime "10"; animationFramesPerSecond "10";xmin "0";xmax
"40";animationParamMin "290";animationParamMax "350";xviewmin "0";xviewmax
"40";yviewmin "0";yviewmax "400";viewset"XY";rangeset"X";plottype
4;labeloverrides 3;x-label "V(m^3)";y-label "P(Pa)";axesFont "Times New
Roman,12,0000000000,useDefault,normal";numpoints 100;plotstyle
"patch";axesstyle "normal";axestips FALSE;xis \TEXUX{V};animationParam
\TEXUX{t};var1name \TEXUX{$V$};animationParamList \TEXUX{$t$};function
\TEXUX{$\allowbreak 8.\,\allowbreak 314\frac{1}{V}290$};linecolor
"blue";linestyle 1;pointstyle "point";linethickness 1;lineAttributes
"Solid";var1range "0,40";animationParamRange "290,350";num-x-gridlines
100;curveColor "[flat::RGB:0x000000ff]";curveStyle "Line";function
\TEXUX{$\allowbreak 8.\,\allowbreak 314\frac{1}{V}590$};linecolor
"blue";linestyle 1;pointstyle "point";linethickness 3;lineAttributes
"Solid";var1range "0,40";animationParamRange "290,350";num-x-gridlines
100;curveColor "[flat::RGB:0x000000ff]";curveStyle "Line";VCamFile
'PTF4N805.xvz';valid_file "T";tempfilename
'PQXXR0YF.wmf';tempfile-properties "XPR";}}The dark curve is our new pot.
The light curve is the old plot for comparison. The curves look a lot alike.
They are essentially the same shape. But their location (how far they are
from the origin) depends on the temperature, with warmer temperatures moving
to the upper right hand part off the graph. So our $P$ vs. $%
%TCIMACRO{\TeXButton{V}{\ooalign{\hfil$V$\hfil\cr\kern0.1em--\hfil\cr}}}%
%BeginExpansion
\ooalign{\hfil$V$\hfil\cr\kern0.1em--\hfil\cr}%
%EndExpansion
$ graph can tell us something about temperature.

Knowing how to interpret a $P%
%TCIMACRO{\TeXButton{V}{\ooalign{\hfil$V$\hfil\cr\kern0.1em--\hfil\cr}}}%
%BeginExpansion
\ooalign{\hfil$V$\hfil\cr\kern0.1em--\hfil\cr}%
%EndExpansion
\ $diagram, we can now look at some simple processes that have special
properties.

\subsection{Constant volume process}

%TCIMACRO{%
%\TeXButton{Question 123.9.12}{\marginpar {
%\hspace{-0.5in}
%\begin{minipage}[t]{1in}
%\small{Question 123.9.12}
%\end{minipage}
%}}}%
%BeginExpansion
\marginpar {
\hspace{-0.5in}
\begin{minipage}[t]{1in}
\small{Question 123.9.12}
\end{minipage}
}%
%EndExpansion
%TCIMACRO{%
%\TeXButton{Question 123.9.13}{\marginpar {
%\hspace{-0.5in}
%\begin{minipage}[t]{1in}
%\small{Question 123.9.13}
%\end{minipage}
%}}}%
%BeginExpansion
\marginpar {
\hspace{-0.5in}
\begin{minipage}[t]{1in}
\small{Question 123.9.13}
\end{minipage}
}%
%EndExpansion
Suppose we have a process that takes us between two states on a $P%
%TCIMACRO{\TeXButton{V}{\ooalign{\hfil$V$\hfil\cr\kern0.1em--\hfil\cr}}}%
%BeginExpansion
\ooalign{\hfil$V$\hfil\cr\kern0.1em--\hfil\cr}%
%EndExpansion
\ $diagram as shown in the next figure. \FRAME{dtbpFX}{1.9545in}{1.3041in}{%
0pt}{}{}{Plot}{\special{language "Scientific Word";type "MAPLEPLOT";width
1.9545in;height 1.3041in;depth 0pt;display "USEDEF";plot_snapshots
TRUE;mustRecompute FALSE;lastEngine "MuPAD";xmin "2";xmax "10";xviewmin
"2";xviewmax "10";yviewmin "100";yviewmax
"900";viewset"XY";rangeset"X";plottype 4;labeloverrides 3;x-label
"V(m^3)";y-label "P(Pa)";axesFont "Times New
Roman,12,0000000000,useDefault,normal";numpoints 100;plotstyle
"patch";axesstyle "normal";axestips FALSE;xis \TEXUX{V};var1name
\TEXUX{$V$};function
\TEXUX{$\MATRIX{2,2}{c}\VR{,,c,,,}{,,c,,,}{,,,,,}\HR{,,}\CELL{6}\CELL{%
\allowbreak 278.\,\allowbreak 29}\CELL{6}\CELL{\allowbreak 775.\,\allowbreak
28}$};linecolor "blue";linestyle 1;pointstyle "point";linethickness
1;lineAttributes "Solid";curveColor "[flat::RGB:0x000000ff]";curveStyle
"Line";function
\TEXUX{$\MATRIX{2,2}{c}\VR{,,c,,,}{,,c,,,}{,,,,,}\HR{,,}\CELL{6}\CELL{%
\allowbreak 278.\,\allowbreak 29}\CELL{6}\CELL{\allowbreak 775.\,\allowbreak
28}$};linecolor "blue";linestyle 1;pointplot TRUE;pointstyle
"diamond";linethickness 1;lineAttributes "Solid";curveColor
"[flat::RGB:0x000000ff]";curveStyle "Point";VCamFile
'PTF4MB03.xvz';valid_file "T";tempfilename
'PQXXR0YG.wmf';tempfile-properties "XPR";}}We start at one pressure, say $%
P=275\unit{Pa},$ and end at another pressure, say, $P=775\unit{Pa}.$ In this
process, notice that the volume does not change. Such a process is called 
\emph{isochoric}. Would we predict that the temperature went up or down in
taking this path? Well, we went from the lower to the upper part of the
graph, so we expect that the temperature went up. Starting again with the
ideal gas law%
\[
P=\left( \frac{nR}{%
%TCIMACRO{\TeXButton{V}{\ooalign{\hfil$V$\hfil\cr\kern0.1em--\hfil\cr}}}%
%BeginExpansion
\ooalign{\hfil$V$\hfil\cr\kern0.1em--\hfil\cr}%
%EndExpansion
}\right) T 
\]%
we can see that when $%
%TCIMACRO{\TeXButton{V}{\ooalign{\hfil$V$\hfil\cr\kern0.1em--\hfil\cr}}}%
%BeginExpansion
\ooalign{\hfil$V$\hfil\cr\kern0.1em--\hfil\cr}%
%EndExpansion
$ is constant, $P$ goes like $T,$ so if $P$ went up, so did the temperature.

But how would create such a situation? Consider what would happen\footnote{%
Only consider this, because it would be irresponsible and dangerous (and
illegal in some states) to do this!} if you placed an aerosol can in a fire.
The aerosol can is rigid, so the volume can't change. But the fire will make
the temperature change, and the pressure will change--that is--until the can
explodes.

\subsection{Constant Pressure process}

%TCIMACRO{%
%\TeXButton{Question 123.9.14}{\marginpar {
%\hspace{-0.5in}
%\begin{minipage}[t]{1in}
%\small{Question 123.9.14}
%\end{minipage}
%}}}%
%BeginExpansion
\marginpar {
\hspace{-0.5in}
\begin{minipage}[t]{1in}
\small{Question 123.9.14}
\end{minipage}
}%
%EndExpansion
%TCIMACRO{%
%\TeXButton{Question 123.9.13}{\marginpar {
%\hspace{-0.5in}
%\begin{minipage}[t]{1in}
%\small{Question 123.9.13}
%\end{minipage}
%}}}%
%BeginExpansion
\marginpar {
\hspace{-0.5in}
\begin{minipage}[t]{1in}
\small{Question 123.9.13}
\end{minipage}
}%
%EndExpansion
Consider the next $P%
%TCIMACRO{\TeXButton{V}{\ooalign{\hfil$V$\hfil\cr\kern0.1em--\hfil\cr}}}%
%BeginExpansion
\ooalign{\hfil$V$\hfil\cr\kern0.1em--\hfil\cr}%
%EndExpansion
\ $diagram.

\FRAME{dtbpFX}{2.2883in}{1.5264in}{0pt}{}{}{Plot}{\special{language
"Scientific Word";type "MAPLEPLOT";width 2.2883in;height 1.5264in;depth
0pt;display "USEDEF";plot_snapshots TRUE;mustRecompute FALSE;lastEngine
"MuPAD";xmin "2";xmax "10";xviewmin "2";xviewmax "10";yviewmin
"100";yviewmax "900";viewset"XY";rangeset"X";plottype 4;labeloverrides
3;x-label "V(m^3)";y-label "P(Pa)";axesFont "Times New
Roman,12,0000000000,useDefault,normal";numpoints 100;plotstyle
"patch";axesstyle "normal";axestips FALSE;xis \TEXUX{V};var1name
\TEXUX{$V$};function
\TEXUX{$\MATRIX{2,2}{c}\VR{,,c,,,}{,,c,,,}{,,,,,}\HR{,,}\CELL{4}\CELL{%
\allowbreak 500}\CELL{8}\CELL{500}$};linecolor "green";linestyle
1;pointstyle "circle";linethickness 1;lineAttributes "Solid";curveColor
"[flat::RGB:0x00008000]";curveStyle "Line";function
\TEXUX{$\MATRIX{2,2}{c}\VR{,,c,,,}{,,c,,,}{,,,,,}\HR{,,}\CELL{4}\CELL{%
\allowbreak 500}\CELL{8}\CELL{500}$};linecolor "green";linestyle 1;pointplot
TRUE;pointstyle "circle";linethickness 1;lineAttributes "Solid";curveColor
"[flat::RGB:0x00008000]";curveStyle "Point";VCamFile
'PTF4MC04.xvz';valid_file "T";tempfilename
'PQXXR0YH.wmf';tempfile-properties "XPR";}}We can see that in this process
the pressure is not changing. If we go from left to right along this path,
will the temperature rise? We know that going to the right means higher
temperature on a $P%
%TCIMACRO{\TeXButton{V}{\ooalign{\hfil$V$\hfil\cr\kern0.1em--\hfil\cr}}}%
%BeginExpansion
\ooalign{\hfil$V$\hfil\cr\kern0.1em--\hfil\cr}%
%EndExpansion
$ diagram, so yes it will.

The process represented by this type of path on a $P%
%TCIMACRO{\TeXButton{V}{\ooalign{\hfil$V$\hfil\cr\kern0.1em--\hfil\cr}}}%
%BeginExpansion
\ooalign{\hfil$V$\hfil\cr\kern0.1em--\hfil\cr}%
%EndExpansion
$ diagram is called an \emph{isobaric} process. How would we produce such a
situation? Consider a piston in a cylinder. The cylinder has a gas inside,
and air on the outside\FRAME{dhF}{1.8308in}{1.5212in}{0pt}{}{}{Figure}{%
\special{language "Scientific Word";type "GRAPHIC";maintain-aspect-ratio
TRUE;display "USEDEF";valid_file "T";width 1.8308in;height 1.5212in;depth
0pt;original-width 1.7936in;original-height 1.4849in;cropleft "0";croptop
"1";cropright "1";cropbottom "0";tempfilename
'PQXXR0YI.wmf';tempfile-properties "XPR";}}The piston is free to move up or
down. The forces acting on the piston are shown. The weight of the piston
pulls it downward, the force due to pressure of the gas, $F_{p,gas}$ pushes
up, and the air pressure $F_{p,air}$ pushes down. The net force will be 
\[
\Sigma F_{y}=ma_{y}=F_{p,gas}-F_{p,air}-mg 
\]%
where the mass of the piston is $m.$ We can write this as 
\[
ma_{y}=P_{gas}A-P_{air}A-mg 
\]%
If the piston is not accelerating , then 
\[
P_{gas}=P_{air}+\frac{mg}{A} 
\]%
and we find that the pressure of the gas inside is slightly larger than the
outside air pressure. What would happen if we heated the gas? 
\[
P%
%TCIMACRO{\TeXButton{V}{\ooalign{\hfil$V$\hfil\cr\kern0.1em--\hfil\cr}}}%
%BeginExpansion
\ooalign{\hfil$V$\hfil\cr\kern0.1em--\hfil\cr}%
%EndExpansion
=NRT 
\]%
would suggest that $P$ might change, but since the piston is free to move,
what happens is that the force $F_{p,gas}$ increases when $P_{gas}$
increases, and the piston accelerates upward. It stops when the forces are
again balanced. But that means that $P_{gas}$ will only change for a moment,
and then we will achieve equilibrium at the same old value of $P_{gas}.$ But
something has changed, the volume has increased.

\subsection{Constant Temperature Process}

We have already met the constant temperature process.\FRAME{dtbpFX}{1.9095in%
}{1.2739in}{0pt}{}{}{Plot}{\special{language "Scientific Word";type
"MAPLEPLOT";width 1.9095in;height 1.2739in;depth 0pt;display
"USEDEF";plot_snapshots TRUE;mustRecompute FALSE;lastEngine "MuPAD";animated
TRUE;animationStartTime "0"; animationEndTime "10"; animationFramesPerSecond
"10";xmin "0";xmax "40";animationParamMin "290";animationParamMax
"350";xviewmin "0";xviewmax "40";yviewmin "0";yviewmax
"400";viewset"XY";rangeset"X";plottype 4;labeloverrides 3;x-label
"V(m^3)";y-label "P(Pa)";axesFont "Times New
Roman,12,0000000000,useDefault,normal";numpoints 100;plotstyle
"patch";axesstyle "normal";axestips FALSE;xis \TEXUX{V};animationParam
\TEXUX{x};var1name \TEXUX{$V$};animationParamList \TEXUX{$x$};function
\TEXUX{$8.\,\allowbreak 314\frac{1}{V}590$};linecolor "blue";linestyle
1;pointstyle "point";linethickness 3;lineAttributes "Solid";var1range
"0,40";animationParamRange "290,350";num-x-gridlines 100;curveColor
"[flat::RGB:0x000000ff]";curveStyle "Line";VCamFile
'PTH3D007.xvz';valid_file "T";tempfilename
'PQXXR0YJ.wmf';tempfile-properties "XPR";}}How do we make such a path
happen? Consider our piston again.\FRAME{dhF}{2.5434in}{1.6328in}{0pt}{}{}{%
Figure}{\special{language "Scientific Word";type
"GRAPHIC";maintain-aspect-ratio TRUE;display "USEDEF";valid_file "T";width
2.5434in;height 1.6328in;depth 0pt;original-width 2.5028in;original-height
1.5965in;cropleft "0";croptop "1";cropright "1";cropbottom "0";tempfilename
'PQXXR0YK.wmf';tempfile-properties "XPR";}}This time, let's put it in a vat
of ice water. We can compress the gas by pushing on the piston. But if the
temperature is changed by compressing the gas, it is quickly changed back by
the ice water.

\chapter{Work in Thermodynamic Processes}

Armed with the Idea Gas Law and our (somewhat) better understanding it is
finally time to study the next law of thermodynamics. We will use the ideas
of energy to do this. In PH121 we loved energy because the ideas of energy
made problems easier to solve. The same is true here. We will start this
lecture by thinking about the energy of large numbers of objects (atoms and
molecules).

%TCIMACRO{%
%\TeXButton{Fundamental Concepts}{\hspace{-1.3in}{\Large Fundamental Concepts\vspace{0.25in}}}}%
%BeginExpansion
\hspace{-1.3in}{\Large Fundamental Concepts\vspace{0.25in}}%
%EndExpansion

\begin{itemize}
\item $\Delta E_{int}$ is an invariant quantity in thermodynamics

\item Work in thermodynamic processes can be described as the integral $%
w_{int}=\int_{%
%TCIMACRO{\TeXButton{V}{\ooalign{\hfil$V$\hfil\cr\kern0.1em--\hfil\cr}}}%
%BeginExpansion
\ooalign{\hfil$V$\hfil\cr\kern0.1em--\hfil\cr}%
%EndExpansion
_{i}}^{%
%TCIMACRO{\TeXButton{V}{\ooalign{\hfil$V$\hfil\cr\kern0.1em--\hfil\cr}}}%
%BeginExpansion
\ooalign{\hfil$V$\hfil\cr\kern0.1em--\hfil\cr}%
%EndExpansion
_{f}}Pd%
%TCIMACRO{\TeXButton{V}{\ooalign{\hfil$V$\hfil\cr\kern0.1em--\hfil\cr}}}%
%BeginExpansion
\ooalign{\hfil$V$\hfil\cr\kern0.1em--\hfil\cr}%
%EndExpansion
$

\item $Q$ is a transfer of transfer of internal energy, and it is called
\textquotedblleft heat\textquotedblright

\item $\Delta E_{int}=Q+w_{int}$ is the first law of thermodynamics
\end{itemize}

\section{Conservation of Energy: First Law of Thermodynamics}

%TCIMACRO{%
%\TeXButton{Question 123.10.1}{\marginpar {
%\hspace{-0.5in}
%\begin{minipage}[t]{1in}
%\small{Question 123.10.1}
%\end{minipage}
%}}}%
%BeginExpansion
\marginpar {
\hspace{-0.5in}
\begin{minipage}[t]{1in}
\small{Question 123.10.1}
\end{minipage}
}%
%EndExpansion
Remember when we last studied energy in PH 121. We used the work energy
theorem%
\[
\Delta K=w 
\]%
and expanded the work side to show the different types of work explicitly.
We have work done by conservative forces $w_{c}$ and work done by
non-concervitave forces (dissipative forces like friction) $w_{nc}$.%
\[
\Delta K=w_{c}+w_{nc} 
\]%
and we could further divide up the work by categorizing the source of the
work, say, work done by the force of gravity, $w_{g},$ or work done by a
spring force $w_{s}.$%
\[
\Delta K=w_{g}+w_{s}+w_{nc} 
\]%
and we could go on in more complicated systems listing all the different
kinds of work.

We relabeled these different types of work as potential energies. For
example,%
\begin{eqnarray*}
\Delta U_{g} &=&-w_{g} \\
\Delta U_{s} &=&-w_{s}
\end{eqnarray*}%
and rewrote the work energy theorem%
\[
\Delta K+\Delta U_{g}+\Delta U_{s}=w_{nc} 
\]%
This gave us a way to think of mechanical energy in a system. But we realize
that we have not covered all types of energy yet. Suppose we consider your
car. You put gas in the car. This is a source of energy, \emph{a chemical
potential energy}, $U_{chem}$. So we should really include it.%
\[
\Delta K+\Delta U_{g}+\Delta U_{s}+\Delta U_{chem}=w_{nc} 
\]%
and if your car is a DeLorean that has been modified to use nuclear energy
to run, then you might have a nuclear potential energy%
\[
\Delta K+\Delta U_{g}+\Delta U_{s}+\Delta U_{chem}+\Delta U_{nuclear}=w_{nc} 
\]%
and back in PH121 we called the non-conservative work $w_{nc}=\Delta E_{th}$
because friction like forces dissipate mechanical energy by turning them
into thermal energy. Then 
\[
\Delta K+\Delta U_{g}+\Delta U_{s}+\Delta U_{chem}+\Delta U_{nuclear}=\Delta
E_{th} 
\]

The quantities $\Delta U_{chem},$ $\Delta U_{nuclear}$, and $\Delta E_{th}$
are different than the mechanical energies we studied before. They have to
do with the internal workings of the material, itself. We will call an
amount of energy that comes from the material, itself \emph{internal energy.}
If we think about it, we might expect that temperature has something to do
with an internal energy of the object. And we would be right.

\subsection{Work and Internal energy}

Our potential energy terms are all derived from work.%
%TCIMACRO{%
%\TeXButton{Question 123.10.2}{\marginpar {
%\hspace{-0.5in}
%\begin{minipage}[t]{1in}
%\small{Question 123.10.2}
%\end{minipage}
%}}}%
%BeginExpansion
\marginpar {
\hspace{-0.5in}
\begin{minipage}[t]{1in}
\small{Question 123.10.2}
\end{minipage}
}%
%EndExpansion
%TCIMACRO{%
%\TeXButton{Question 123.10.2.5}{\marginpar {
%\hspace{-0.5in}
%\begin{minipage}[t]{1in}
%\small{Question 123.10.2.5}
%\end{minipage}
%}}}%
%BeginExpansion
\marginpar {
\hspace{-0.5in}
\begin{minipage}[t]{1in}
\small{Question 123.10.2.5}
\end{minipage}
}%
%EndExpansion
Long ago we found that the work done on an object may depend on the path the
object takes. We found work using 
\begin{equation}
w=\int_{x_{i}}^{x_{f}}\overrightarrow{F}\cdot \overrightarrow{dr}
\end{equation}%
and the work was the area under the force vs. displacement curve. But this
was work done moving an object as a whole. Now we can do work on an object
by compressing it, for example. This is different from moving the whole
object around. Let's call this new kind of work \emph{internal work. }This
new kind of work should be similar in form to regular mechanical work. We
expect it to depend on the details of how the internal work is done.

For mechanical work we defined the potential energy, $U,$ as 
\begin{equation}
w=-\Delta U
\end{equation}

%TCIMACRO{%
%\TeXButton{Question 123.10.3}{\marginpar {
%\hspace{-0.5in}
%\begin{minipage}[t]{1in}
%\small{Question 123.10.3}
%\end{minipage}
%}}}%
%BeginExpansion
\marginpar {
\hspace{-0.5in}
\begin{minipage}[t]{1in}
\small{Question 123.10.3}
\end{minipage}
}%
%EndExpansion
Using this concept, we showed that mechanical energy 
\begin{equation}
E_{mech}=K+U
\end{equation}%
was often conserved, when $E_{thermal}=0$ where $E_{mech}$ was the total
mechanic energy. Remember we solved a lot of problems back in PH121 by
knowing that for a closed system, $E_{mech}$ did not change. In other words, 
$E_{mech}$ was independent of path we take on our $F$ vs. $x$ diagram. This
makes $E_{mech}$ much easier to work with. A path independent quantity makes
problems easier.

We want another variable to work with that removes the difficulty of
worrying about the details of how the motion happens (the path). But we
can't use the kinetic energy for this. The kinetic energy is part of the
mechanical work of the object as a whole, and does not affect the internal
workings of the object.

Frictional forces do affect the object. Think of rubbing your hands
together. There is mechanical work done by your muscles. There is kinetic
energy while your hands move. But your hands heat up. This higher
temperature is still there for a while after your hands stop moving.
Eventually the temperature of your hands returns to normal. We expect that
this is a transfer of energy, since dissipative forces take energy out of
the system. We will call the movement of energy out of a system
\textquotedblleft \emph{heat.}\textquotedblright\ We give it the symbol $Q$
in this class.

Note that this is a very strange and specific use of the word
\textquotedblleft heat.\textquotedblright\ For us heat is a transfer of
energy. It is not the amount of energy in a body. We will call the amount of
energy that is in the body the \emph{internal energy} and give it the symbol 
$E_{int}.$ But \textquotedblleft heat\textquotedblright\ is a \emph{transfer
of internal energy}. It does not make sense to ask how much heat an object
has. An object can have internal energy, but heat is a transfer of that
energy to something else.

The internal energy would depend on how much energy is transferred into the
object. Say, we have a large metal block and we set it on a fire. The fire
would transfer energy to the block. We could also do work on the block to
increase the internal energy of the block, say, by compressing the block.

%TCIMACRO{%
%\TeXButton{Question 123.10.4}{\marginpar {
%\hspace{-0.5in}
%\begin{minipage}[t]{1in}
%\small{Question 123.10.4}
%\end{minipage}
%}}}%
%BeginExpansion
\marginpar {
\hspace{-0.5in}
\begin{minipage}[t]{1in}
\small{Question 123.10.4}
\end{minipage}
}%
%EndExpansion
Now we are ready for our path independent quantity for internal energy. We
define the change in internal energy as%
\[
\Delta E_{int}=Q+w_{int} 
\]

As with the mechanical energy, the change in internal energy does not depend
on the details of how the change happens (what we are calling
\textquotedblleft path\textquotedblright ). The change in energy is used
because $Q$ and $w$ are transfer variables, so they \emph{do} depend on the
details of the transfer. For small changes we can write%
\[
\Delta E_{int}=\Delta Q+\Delta w_{int} 
\]%
or for very small changes, we could replace the $\Delta $'s with small $d$'s

\[
dE_{int}=dQ+dw_{int} 
\]%
but be careful! these are not true differentials because $dw$ depends on the
path! Integration of $dw$ is trick and you would have to think about it
carefully. The relationship 
\begin{equation}
\Delta E_{int}=Q+w_{int}
\end{equation}%
is known as the \emph{first law of thermodynamics.} This will be the subject
of our next lecture.

Let's look at some systems and think about their internal energies.

\subsection{An Isolated System}

An isolated system is a system separated from the rest of the universe.
Nothing can act on it. When a system is isolated, we have%
\begin{equation}
w_{int}=0
\end{equation}%
because no work can be done on an isolated system and 
\begin{equation}
Q=0
\end{equation}%
because no heat can flow to or from an isolated system so%
\begin{equation}
\Delta E_{int}=w_{int}+Q=0
\end{equation}

\subsection{A Cyclic Process}

Suppose we have a process, something that is done to a sample of matter, say
a gas. Further suppose that our process repeats over and over again. We call
such a process a \emph{cyclic} process. This is like the pistons in your car
going through the same motion over and over again to compress the gas in
your engine cylinders. 
%TCIMACRO{%
%\TeXButton{Question 123.10.5}{\marginpar {
%\hspace{-0.5in}
%\begin{minipage}[t]{1in}
%\small{Question 123.10.5}
%\end{minipage}
%}}}%
%BeginExpansion
\marginpar {
\hspace{-0.5in}
\begin{minipage}[t]{1in}
\small{Question 123.10.5}
\end{minipage}
}%
%EndExpansion
Let's practice with the idea of internal energy, work, heat and PV\ diagrams
for such a process. If a process repeats, it starts and ends in the same
state so then 
\begin{equation}
\Delta E_{int}=0
\end{equation}%
and%
\begin{equation}
Q=-w_{int}
\end{equation}

%TCIMACRO{%
%\TeXButton{Question 123.10.6}{\marginpar {
%\hspace{-0.5in}
%\begin{minipage}[t]{1in}
%\small{Question 123.10.6}
%\end{minipage}
%}}}%
%BeginExpansion
\marginpar {
\hspace{-0.5in}
\begin{minipage}[t]{1in}
\small{Question 123.10.6}
\end{minipage}
}%
%EndExpansion

\section{Work and Thermodynamic Processes}

Consider the work done by the piston as it compresses the gas. The amount of
work must be equal to the amount of energy transfer by heat. We want this to
be true! Otherwise you could melt your engine. We want an engine in a car
(or train, or airplane) to be a cyclic process. This explains why your car
engine gets hot. Work is being done in the cylinders in your engine, but to
make the process cyclic some energy must be lost by heat (though there are
details of car engines that we won't tackle until later).

\FRAME{dtbpF}{2.1483in}{2.0294in}{0in}{}{}{Figure}{\special{language
"Scientific Word";type "GRAPHIC";maintain-aspect-ratio TRUE;display
"USEDEF";valid_file "T";width 2.1483in;height 2.0294in;depth
0in;original-width 2.1607in;original-height 2.0392in;cropleft "0";croptop
"1";cropright "1";cropbottom "0";tempfilename
'PQXXR0YL.wmf';tempfile-properties "XPR";}}

Lets take a close look at the gas within the piston. At equilibrium we know 
\[
P=\frac{F_{P}}{A} 
\]%
or%
\[
F_{P}=PA 
\]%
This force is the force of the gas pushing on the piston. We could label it 
\[
F_{gp}=F_{P} 
\]
Notice that the gas pushes to the right. If we apply an external force to
the piston slowly so as to allow the system to remain in thermal equilibrium
as we go, then 
\[
\overrightarrow{F}_{ex}=-F_{ex}\hat{\imath} 
\]%
because we are pushing in the $-x$-direction. Again, we will move the piston
slowly so we don't violate our quasi-static rule. As we move the piston the
gas inside the piston adjusts in pressure until we are again in equilibrium.
So 
\[
F_{ex}=F_{gp} 
\]%
but the pressure force $F_{gp}$ pushes the opposite direction 
\begin{eqnarray*}
\overrightarrow{F}_{gp} &=&+F_{ex}\hat{\imath} \\
&=&PA\hat{\imath}
\end{eqnarray*}%
We should also note that since neither the piston nor the gas are
accelerating, Newton's third law tells us that the force of the gas on the
piston and the force of the piston on the gas are equal in magnitude but
opposite in direction%
\[
F_{gp}=F_{pg} 
\]%
\[
\overrightarrow{F}_{gp}=-F_{pg}\hat{\imath} 
\]
When we move the piston to the left, we have a displacement%
\[
\overrightarrow{dr}=-dx\hat{\imath} 
\]%
so the work done \emph{on the gas by the piston would be }%
\[
w_{gp}=\int_{x_{i}}^{x_{f}}\overrightarrow{F}_{ex}\cdot d\overrightarrow{r} 
\]%
Remembering how to use a dot product we can write this as 
\[
w_{gp}=\int_{x_{i}}^{x_{f}}F_{ex}dx\cos \theta _{Fx} 
\]%
It is worth remembering that in this formula $F_{p}$ and $dx$ are
magnitudes. Let's write this as%
\[
w_{gp}=\int_{x_{i}}^{x_{f}}\left\vert F_{ex}\right\vert \left\vert
dx\right\vert \cos \theta _{Fx} 
\]%
to remind us. Substituting in $\left\vert F_{ex}\right\vert $ for $%
\left\vert F_{gp}\right\vert $ we have 
\begin{eqnarray*}
w_{gp} &=&\int_{x_{i}}^{x_{f}}\left\vert F_{gp}\right\vert \left\vert
dx\right\vert \cos \theta _{Fx} \\
&=&\int_{x_{i}}^{x_{f}}\left\vert PA\right\vert \left\vert dx\right\vert
\cos \theta _{Fx}
\end{eqnarray*}%
and the angle between $F_{gp}$ and $dx$ is $0$ (they are both to the left) so

\[
w_{gp}=\int_{x_{i}}^{x_{f}}\left\vert PA\right\vert \left\vert dx\right\vert 
\]%
where, to get the total work done on the piston we are integrating over the
entire path of the piston travels. It is worth noting that $P$ can only be
positive, and the same is true for $A$. Both $P$ and $A$ are constant, since
we are using our quazi-static process and allowing the pressure to equalize
as we go. So we could write the work done on the piston by the gas as 
\[
w_{gp}=PA\int_{x_{i}}^{x_{f}}\left\vert dx\right\vert 
\]%
but now we need to consider $\left\vert dx\right\vert .$ We know that $dx$
is a small displacement, and displacements can be negative. And in our case,
as the piston goes to the left, $dx$ certainly is negative. But no matter,
we don't just have $dx,$ we have $\left\vert dx\right\vert $. So the work is
positive just as it should be.

But here we can be a little bit clever. If we add in a minus sign so that 
\[
w_{gp}=-PA\int_{x_{i}}^{x_{f}}dx 
\]%
our added minus sign will cancel with the minus sign hiding in the $dx,$
still giving us positive work (think of what we did with springs in PH121 to
deal with $S=-k\left( x-x_{o}\right) $ in work integrals). Now we don't have
to worry about integrating with absolute value signs.

Remember that we are using an isobaric process in compressing the gas That
is, we keep the pressure the same but allow the temperature to change, so
that the volume and temperature change proportionally $%
%TCIMACRO{\TeXButton{V}{\ooalign{\hfil$V$\hfil\cr\kern0.1em--\hfil\cr}}}%
%BeginExpansion
\ooalign{\hfil$V$\hfil\cr\kern0.1em--\hfil\cr}%
%EndExpansion
=(nRT/P)$) So our integral is just 
\begin{eqnarray*}
w_{gp} &=&-PA\int_{x_{i}}^{x_{f}}dx \\
&=&-PA\left( x_{f}-x_{i}\right)
\end{eqnarray*}%
We can rewrite this result using what we know about the volume of the gas 
\[
%TCIMACRO{\TeXButton{V}{\ooalign{\hfil$V$\hfil\cr\kern0.1em--\hfil\cr}}}%
%BeginExpansion
\ooalign{\hfil$V$\hfil\cr\kern0.1em--\hfil\cr}%
%EndExpansion
=A\Delta x 
\]%
so the work done by the gas pushing on the piston is 
\begin{eqnarray*}
w_{gp} &=&-P\left( Ax_{f}-Ax_{i}\right) \\
&=&-P\left( 
%TCIMACRO{\TeXButton{V}{\ooalign{\hfil$V$\hfil\cr\kern0.1em--\hfil\cr}}}%
%BeginExpansion
\ooalign{\hfil$V$\hfil\cr\kern0.1em--\hfil\cr}%
%EndExpansion
_{f}-%
%TCIMACRO{\TeXButton{V}{\ooalign{\hfil$V$\hfil\cr\kern0.1em--\hfil\cr}}}%
%BeginExpansion
\ooalign{\hfil$V$\hfil\cr\kern0.1em--\hfil\cr}%
%EndExpansion
_{i}\right)
\end{eqnarray*}%
This gives us an idea. We could write the quantity $Adx$ as $d%
%TCIMACRO{\TeXButton{V}{\ooalign{\hfil$V$\hfil\cr\kern0.1em--\hfil\cr}}}%
%BeginExpansion
\ooalign{\hfil$V$\hfil\cr\kern0.1em--\hfil\cr}%
%EndExpansion
$ and change the limits of integration from $x_{f}$ and $x_{i}$ to $%
%TCIMACRO{\TeXButton{V}{\ooalign{\hfil$V$\hfil\cr\kern0.1em--\hfil\cr}}}%
%BeginExpansion
\ooalign{\hfil$V$\hfil\cr\kern0.1em--\hfil\cr}%
%EndExpansion
_{f}$ and $%
%TCIMACRO{\TeXButton{V}{\ooalign{\hfil$V$\hfil\cr\kern0.1em--\hfil\cr}}}%
%BeginExpansion
\ooalign{\hfil$V$\hfil\cr\kern0.1em--\hfil\cr}%
%EndExpansion
_{i}.$ Then our integral would be 
\begin{eqnarray*}
w_{gp} &=&-\int_{x_{i}}^{x_{f}}PAdx \\
&=&-\int_{%
%TCIMACRO{\TeXButton{V}{\ooalign{\hfil$V$\hfil\cr\kern0.1em--\hfil\cr}}}%
%BeginExpansion
\ooalign{\hfil$V$\hfil\cr\kern0.1em--\hfil\cr}%
%EndExpansion
_{i}}^{%
%TCIMACRO{\TeXButton{V}{\ooalign{\hfil$V$\hfil\cr\kern0.1em--\hfil\cr}}}%
%BeginExpansion
\ooalign{\hfil$V$\hfil\cr\kern0.1em--\hfil\cr}%
%EndExpansion
_{f}}Pd%
%TCIMACRO{\TeXButton{V}{\ooalign{\hfil$V$\hfil\cr\kern0.1em--\hfil\cr}}}%
%BeginExpansion
\ooalign{\hfil$V$\hfil\cr\kern0.1em--\hfil\cr}%
%EndExpansion
\end{eqnarray*}%
This gives the same answer for the work done on the piston for our isobaric
case%
\[
w_{gp}=-P\left( 
%TCIMACRO{\TeXButton{V}{\ooalign{\hfil$V$\hfil\cr\kern0.1em--\hfil\cr}}}%
%BeginExpansion
\ooalign{\hfil$V$\hfil\cr\kern0.1em--\hfil\cr}%
%EndExpansion
_{f}-%
%TCIMACRO{\TeXButton{V}{\ooalign{\hfil$V$\hfil\cr\kern0.1em--\hfil\cr}}}%
%BeginExpansion
\ooalign{\hfil$V$\hfil\cr\kern0.1em--\hfil\cr}%
%EndExpansion
_{i}\right) 
\]

But this is the amount of work that the piston does on the gas. Is the gas
pushing back on the piston? Of course it is! And that push is also a force, $%
F_{pg}$, so it also does work. But let's review what we know about work from
PH121.

Suppose we have two people exerting a force on a box. One is a professional
American football player, The other is a six-year-old. \FRAME{dtbpF}{2.8158in%
}{1.8524in}{0pt}{}{}{Figure}{\special{language "Scientific Word";type
"GRAPHIC";maintain-aspect-ratio TRUE;display "USEDEF";valid_file "T";width
2.8158in;height 1.8524in;depth 0pt;original-width 3.4082in;original-height
2.2329in;cropleft "0";croptop "1";cropright "1";cropbottom "0";tempfilename
'PRX6460D.wmf';tempfile-properties "XPR";}}They push in opposite directions.
How much work is being done? To answer this, let's consider a free-body
diagram.\FRAME{dtbpF}{1.8291in}{1.8377in}{0pt}{}{}{Figure}{\special{language
"Scientific Word";type "GRAPHIC";maintain-aspect-ratio TRUE;display
"USEDEF";valid_file "T";width 1.8291in;height 1.8377in;depth
0pt;original-width 1.7919in;original-height 1.8005in;cropleft "0";croptop
"1";cropright "1";cropbottom "0";tempfilename
'PRX64N0E.wmf';tempfile-properties "XPR";}}where the subscript $B$ is for
the box and the subscript $6$ is for the six-year-old and $L$ is for the
football player (In this case, a (L)inebacker). We can see that there will
be a net force.%
\[
F_{net_{s}}=N_{BL}-N_{B6}-f_{BF} 
\]%
But which way will the box go? We could guess that the football player will
push the box \emph{and} the six-year-old to the left. So $\Delta s$ will be
negative. Then the work done on the box by the football player will be 
\begin{eqnarray*}
w_{BL} &=&\int_{s_{i}}^{s_{f}}N_{BL}ds \\
&=&N_{BL}\Delta s
\end{eqnarray*}%
but the work done by the child would be 
\begin{eqnarray*}
w_{B6} &=&\int_{s_{i}}^{s_{f}}-N_{B6}ds \\
&=&-N_{BM}\Delta s
\end{eqnarray*}%
The child's work is negative! What can that mean? Well, for starters, the
child's force can't be the force that is making the box move. In fact, the
child's force is another obstacle that the linebacker's force must overcome
to make the box move. This means that the linebacker must do enough work
(push hard enough) to make the box go, \emph{and} to overcome the backward
push of the child.

But that is just our case with the gas and piston The gas pressure force 
\emph{is} pushing on the piston. But the gas pressure force on the piston is
losing ground (the external force is pushing the piston to the left). So we
would expect the work done by the gas on the piston to be negative.

Since we assumed a quasi-static process, the two forces must always be
nearly the same, 
\[
F_{ex}=F_{pg} 
\]%
so the amount of work must be nearly the same, but the sign must be negative
for the work on the gas due to the piston's push. We wrote the work done by
the piston on the gas as 
\[
w_{gp}=-\int_{%
%TCIMACRO{\TeXButton{V}{\ooalign{\hfil$V$\hfil\cr\kern0.1em--\hfil\cr}}}%
%BeginExpansion
\ooalign{\hfil$V$\hfil\cr\kern0.1em--\hfil\cr}%
%EndExpansion
_{i}}^{%
%TCIMACRO{\TeXButton{V}{\ooalign{\hfil$V$\hfil\cr\kern0.1em--\hfil\cr}}}%
%BeginExpansion
\ooalign{\hfil$V$\hfil\cr\kern0.1em--\hfil\cr}%
%EndExpansion
_{f}}Pd%
%TCIMACRO{\TeXButton{V}{\ooalign{\hfil$V$\hfil\cr\kern0.1em--\hfil\cr}}}%
%BeginExpansion
\ooalign{\hfil$V$\hfil\cr\kern0.1em--\hfil\cr}%
%EndExpansion
\]%
Then the work done by the gas on the piston must be 
\begin{eqnarray*}
w_{pg} &=&-\left( -\int_{%
%TCIMACRO{\TeXButton{V}{\ooalign{\hfil$V$\hfil\cr\kern0.1em--\hfil\cr}}}%
%BeginExpansion
\ooalign{\hfil$V$\hfil\cr\kern0.1em--\hfil\cr}%
%EndExpansion
_{i}}^{%
%TCIMACRO{\TeXButton{V}{\ooalign{\hfil$V$\hfil\cr\kern0.1em--\hfil\cr}}}%
%BeginExpansion
\ooalign{\hfil$V$\hfil\cr\kern0.1em--\hfil\cr}%
%EndExpansion
_{f}}Pd%
%TCIMACRO{\TeXButton{V}{\ooalign{\hfil$V$\hfil\cr\kern0.1em--\hfil\cr}}}%
%BeginExpansion
\ooalign{\hfil$V$\hfil\cr\kern0.1em--\hfil\cr}%
%EndExpansion
\right) \\
&=&\int_{%
%TCIMACRO{\TeXButton{V}{\ooalign{\hfil$V$\hfil\cr\kern0.1em--\hfil\cr}}}%
%BeginExpansion
\ooalign{\hfil$V$\hfil\cr\kern0.1em--\hfil\cr}%
%EndExpansion
_{i}}^{%
%TCIMACRO{\TeXButton{V}{\ooalign{\hfil$V$\hfil\cr\kern0.1em--\hfil\cr}}}%
%BeginExpansion
\ooalign{\hfil$V$\hfil\cr\kern0.1em--\hfil\cr}%
%EndExpansion
_{f}}Pd%
%TCIMACRO{\TeXButton{V}{\ooalign{\hfil$V$\hfil\cr\kern0.1em--\hfil\cr}}}%
%BeginExpansion
\ooalign{\hfil$V$\hfil\cr\kern0.1em--\hfil\cr}%
%EndExpansion
\end{eqnarray*}%
that is, the work done on the gas by the piston is the negative of the work
done on the piston by the gas.

We are principally interested in what happens to the gas, (we studied
objects like the piston in PH121) so the first equation is what we are
looking for. This is the internal work done on the gas!

\begin{equation}
w_{gp}=w_{int}=-\int_{%
%TCIMACRO{\TeXButton{V}{\ooalign{\hfil$V$\hfil\cr\kern0.1em--\hfil\cr}}}%
%BeginExpansion
\ooalign{\hfil$V$\hfil\cr\kern0.1em--\hfil\cr}%
%EndExpansion
_{i}}^{%
%TCIMACRO{\TeXButton{V}{\ooalign{\hfil$V$\hfil\cr\kern0.1em--\hfil\cr}}}%
%BeginExpansion
\ooalign{\hfil$V$\hfil\cr\kern0.1em--\hfil\cr}%
%EndExpansion
_{f}}Pd%
%TCIMACRO{\TeXButton{V}{\ooalign{\hfil$V$\hfil\cr\kern0.1em--\hfil\cr}}}%
%BeginExpansion
\ooalign{\hfil$V$\hfil\cr\kern0.1em--\hfil\cr}%
%EndExpansion
\end{equation}

%TCIMACRO{%
%\TeXButton{Question 123.10.6}{\marginpar {
%\hspace{-0.5in}
%\begin{minipage}[t]{1in}
%\small{Question 123.10.6}
%\end{minipage}
%}}}%
%BeginExpansion
\marginpar {
\hspace{-0.5in}
\begin{minipage}[t]{1in}
\small{Question 123.10.6}
\end{minipage}
}%
%EndExpansion

This looks easy enough, but to integrate we must know how $P$ depends on $%
%TCIMACRO{\TeXButton{V}{\ooalign{\hfil$V$\hfil\cr\kern0.1em--\hfil\cr}}}%
%BeginExpansion
\ooalign{\hfil$V$\hfil\cr\kern0.1em--\hfil\cr}%
%EndExpansion
$ during the whole process. It was simple for an isobaric process, but it
could be much harder for other processes. We can plot $P$ vs. $%
%TCIMACRO{\TeXButton{V}{\ooalign{\hfil$V$\hfil\cr\kern0.1em--\hfil\cr}}}%
%BeginExpansion
\ooalign{\hfil$V$\hfil\cr\kern0.1em--\hfil\cr}%
%EndExpansion
$ to see how it varies for a process. We would expect the integral to be the
area under a $P%
%TCIMACRO{\TeXButton{V}{\ooalign{\hfil$V$\hfil\cr\kern0.1em--\hfil\cr}}}%
%BeginExpansion
\ooalign{\hfil$V$\hfil\cr\kern0.1em--\hfil\cr}%
%EndExpansion
$ curve. This is very like mechanical work, which was the area under the $F$-%
$x$ curve.\FRAME{dhF}{2.7951in}{2.2883in}{0pt}{}{}{Figure}{\special{language
"Scientific Word";type "GRAPHIC";maintain-aspect-ratio TRUE;display
"USEDEF";valid_file "T";width 2.7951in;height 2.2883in;depth
0pt;original-width 2.7527in;original-height 2.2485in;cropleft "0";croptop
"1";cropright "1";cropbottom "0";tempfilename
'PQXXR0YO.wmf';tempfile-properties "XPR";}}

We can make a general statement: The work done on a gas in a quasi-static
process that takes the gas from an initial state to a final state is the
negative of the area under the curve on a PV diagram evaluated between the
initial and final states.

Unfortunately, the work done depends on the shape of the curve on the $P%
%TCIMACRO{\TeXButton{V}{\ooalign{\hfil$V$\hfil\cr\kern0.1em--\hfil\cr}}}%
%BeginExpansion
\ooalign{\hfil$V$\hfil\cr\kern0.1em--\hfil\cr}%
%EndExpansion
$ diagram. Several are shown below. Each has a different amount of work done
(notice the different areas under the curves), but the same initial and
final points.

\FRAME{dhF}{3.4999in}{1.0438in}{0pt}{}{}{Figure}{\special{language
"Scientific Word";type "GRAPHIC";maintain-aspect-ratio TRUE;display
"USEDEF";valid_file "T";width 3.4999in;height 1.0438in;depth
0pt;original-width 6.5337in;original-height 1.9285in;cropleft "0";croptop
"1";cropright "1";cropbottom "0";tempfilename
'PQXXR0YP.wmf';tempfile-properties "XPR";}}

%TCIMACRO{%
%\TeXButton{Question 123.10.7}{\marginpar {
%\hspace{-0.5in}
%\begin{minipage}[t]{1in}
%\small{Question 123.10.7}
%\end{minipage}
%}}}%
%BeginExpansion
\marginpar {
\hspace{-0.5in}
\begin{minipage}[t]{1in}
\small{Question 123.10.7}
\end{minipage}
}%
%EndExpansion

Notice that we have calculated the amount of work done on the gas. You might
be wondering if that is what we want to build car or jet engines. Don't we
want the gas to expand to push the piston and make the car go? And of
course, you are right in your thinking. Eventually we want to know the work
done by the gas on the piston, $w_{pg}.$ But for now, we will look at $%
w_{gp}.$ But from what we have done we know these two works are very related
for a quasi-static process. 
\[
w_{pg}=-w_{gp} 
\]

\section{Special processes and work}

Let's use our definition of internal work and apply it to ideal gasses. We
have three special processes we have studied. Let's find the work done in
each process.

\subsection{Isobaric process}

\FRAME{dtbpFX}{2.0453in}{1.3647in}{0pt}{}{}{Plot}{\special{language
"Scientific Word";type "MAPLEPLOT";width 2.0453in;height 1.3647in;depth
0pt;display "USEDEF";plot_snapshots TRUE;mustRecompute FALSE;lastEngine
"MuPAD";xmin "2";xmax "10";xviewmin "2";xviewmax "10";yviewmin
"100";yviewmax "900";viewset"XY";rangeset"X";plottype 4;labeloverrides
3;x-label "V(m^3)";y-label "P(Pa)";axesFont "Times New
Roman,12,0000000000,useDefault,normal";numpoints 100;plotstyle
"patch";axesstyle "normal";axestips FALSE;xis \TEXUX{V};var1name
\TEXUX{$V$};function
\TEXUX{$\MATRIX{2,2}{c}\VR{,,c,,,}{,,c,,,}{,,,,,}\HR{,,}\CELL{4}\CELL{%
\allowbreak 500}\CELL{8}\CELL{500}$};linecolor "green";linestyle
1;pointstyle "circle";linethickness 1;lineAttributes "Solid";curveColor
"[flat::RGB:0x00008000]";curveStyle "Line";function
\TEXUX{$\MATRIX{2,2}{c}\VR{,,c,,,}{,,c,,,}{,,,,,}\HR{,,}\CELL{4}\CELL{%
\allowbreak 500}\CELL{8}\CELL{500}$};linecolor "green";linestyle 1;pointplot
TRUE;pointstyle "circle";linethickness 1;lineAttributes "Solid";curveColor
"[flat::RGB:0x00008000]";curveStyle "Point";VCamFile
'PTNWB101.xvz';valid_file "T";tempfilename
'PQXXR0YQ.wmf';tempfile-properties "XPR";}}

For an isobaric process, we can see that our integral becomes very simple 
\begin{eqnarray*}
w_{int} &=&-\int_{%
%TCIMACRO{\TeXButton{V}{\ooalign{\hfil$V$\hfil\cr\kern0.1em--\hfil\cr}}}%
%BeginExpansion
\ooalign{\hfil$V$\hfil\cr\kern0.1em--\hfil\cr}%
%EndExpansion
_{i}}^{%
%TCIMACRO{\TeXButton{V}{\ooalign{\hfil$V$\hfil\cr\kern0.1em--\hfil\cr}}}%
%BeginExpansion
\ooalign{\hfil$V$\hfil\cr\kern0.1em--\hfil\cr}%
%EndExpansion
_{f}}Pd%
%TCIMACRO{\TeXButton{V}{\ooalign{\hfil$V$\hfil\cr\kern0.1em--\hfil\cr}} }%
%BeginExpansion
\ooalign{\hfil$V$\hfil\cr\kern0.1em--\hfil\cr}
%EndExpansion
\\
&=&-P\int_{%
%TCIMACRO{\TeXButton{V}{\ooalign{\hfil$V$\hfil\cr\kern0.1em--\hfil\cr}}}%
%BeginExpansion
\ooalign{\hfil$V$\hfil\cr\kern0.1em--\hfil\cr}%
%EndExpansion
_{i}}^{%
%TCIMACRO{\TeXButton{V}{\ooalign{\hfil$V$\hfil\cr\kern0.1em--\hfil\cr}}}%
%BeginExpansion
\ooalign{\hfil$V$\hfil\cr\kern0.1em--\hfil\cr}%
%EndExpansion
_{f}}d%
%TCIMACRO{\TeXButton{V}{\ooalign{\hfil$V$\hfil\cr\kern0.1em--\hfil\cr}} }%
%BeginExpansion
\ooalign{\hfil$V$\hfil\cr\kern0.1em--\hfil\cr}
%EndExpansion
\\
&=&-P\left( 
%TCIMACRO{\TeXButton{V}{\ooalign{\hfil$V$\hfil\cr\kern0.1em--\hfil\cr}}}%
%BeginExpansion
\ooalign{\hfil$V$\hfil\cr\kern0.1em--\hfil\cr}%
%EndExpansion
_{f}-%
%TCIMACRO{\TeXButton{V}{\ooalign{\hfil$V$\hfil\cr\kern0.1em--\hfil\cr}}}%
%BeginExpansion
\ooalign{\hfil$V$\hfil\cr\kern0.1em--\hfil\cr}%
%EndExpansion
_{i}\right)
\end{eqnarray*}%
where $P$ is constant.

\subsection{Isochoric (isovolumetric) process}

\FRAME{dtbpFX}{1.9545in}{1.3041in}{0pt}{}{}{Plot}{\special{language
"Scientific Word";type "MAPLEPLOT";width 1.9545in;height 1.3041in;depth
0pt;display "USEDEF";plot_snapshots TRUE;mustRecompute FALSE;lastEngine
"MuPAD";xmin "-5";xmax "5";xviewmin "2";xviewmax "10";yviewmin
"100";yviewmax "900";viewset"XY";rangeset"X";plottype 4;labeloverrides
3;x-label "V(m^3)";y-label "P(Pa)";axesFont "Times New
Roman,12,0000000000,useDefault,normal";numpoints 100;plotstyle
"patch";axesstyle "normal";axestips FALSE;xis \TEXUX{V};var1name
\TEXUX{$V$};function
\TEXUX{$\MATRIX{2,2}{c}\VR{,,c,,,}{,,c,,,}{,,,,,}\HR{,,}\CELL{6}\CELL{%
\allowbreak 278.\,\allowbreak 29}\CELL{6}\CELL{\allowbreak 775.\,\allowbreak
28}$};linecolor "blue";linestyle 1;pointstyle "point";linethickness
1;lineAttributes "Solid";curveColor "[flat::RGB:0x000000ff]";curveStyle
"Line";function
\TEXUX{$\MATRIX{2,2}{c}\VR{,,c,,,}{,,c,,,}{,,,,,}\HR{,,}\CELL{6}\CELL{%
\allowbreak 278.\,\allowbreak 29}\CELL{6}\CELL{\allowbreak 775.\,\allowbreak
28}$};linecolor "blue";linestyle 1;pointplot TRUE;pointstyle
"diamond";linethickness 1;lineAttributes "Solid";curveColor
"[flat::RGB:0x000000ff]";curveStyle "Point";VCamFile
'PTNWBX02.xvz';valid_file "T";tempfilename
'PQXXR1YR.wmf';tempfile-properties "XPR";}}In this case the integral is even
easier. The volume does not change. The area under the curve is zero.%
\begin{eqnarray*}
w_{int} &=&-\int_{%
%TCIMACRO{\TeXButton{V}{\ooalign{\hfil$V$\hfil\cr\kern0.1em--\hfil\cr}}}%
%BeginExpansion
\ooalign{\hfil$V$\hfil\cr\kern0.1em--\hfil\cr}%
%EndExpansion
_{i}}^{%
%TCIMACRO{\TeXButton{V}{\ooalign{\hfil$V$\hfil\cr\kern0.1em--\hfil\cr}}}%
%BeginExpansion
\ooalign{\hfil$V$\hfil\cr\kern0.1em--\hfil\cr}%
%EndExpansion
_{f}}Pd%
%TCIMACRO{\TeXButton{V}{\ooalign{\hfil$V$\hfil\cr\kern0.1em--\hfil\cr}} }%
%BeginExpansion
\ooalign{\hfil$V$\hfil\cr\kern0.1em--\hfil\cr}
%EndExpansion
\\
&=&0
\end{eqnarray*}%
This means no work is done in an isovolumetric process. This makes some
sense. We have neither compressed the gas, nor have we expanded the gas.
Think of $\overrightarrow{F}\cdot d\overrightarrow{r}.$ Since $d\mathbf{r}$
is zero, the work must be zero.

\subsection{Isothermal process}

\FRAME{dtbpFX}{2.2087in}{1.4719in}{0pt}{}{}{Plot}{\special{language
"Scientific Word";type "MAPLEPLOT";width 2.2087in;height 1.4719in;depth
0pt;display "USEDEF";plot_snapshots TRUE;mustRecompute FALSE;lastEngine
"MuPAD";animated TRUE;animationStartTime "0"; animationEndTime "10";
animationFramesPerSecond "10";xmin "0";xmax "100";animationParamMin
"290";animationParamMax "350";xviewmin "0";xviewmax "40";yviewmin
"0";yviewmax "400";viewset"XY";rangeset"X";plottype 4;labeloverrides
3;x-label "V(m^3)";y-label "P(Pa)";axesFont "Times New
Roman,12,0000000000,useDefault,normal";numpoints 100;plotstyle
"patch";axesstyle "normal";axestips FALSE;xis \TEXUX{V};animationParam
\TEXUX{t};var1name \TEXUX{$V$};animationParamList \TEXUX{$t$};function
\TEXUX{$\allowbreak 8.\,\allowbreak 314\frac{1}{V}t$};linecolor
"blue";linestyle 1;pointstyle "point";linethickness 1;lineAttributes
"Solid";var1range "0,100";animationParamRange "290,350";num-x-gridlines
50;curveColor "[flat::RGB:0x000000ff]";curveStyle "Line";animationStartTime
"0"; animationEndTime "10"; animationFramesPerSecond
"10";animationVisibleBeforeStart FALSE;rangeset"XA";function
\TEXUX{$\allowbreak 8.\,\allowbreak 314\frac{1}{V}290$};linecolor
"blue";linestyle 2;pointstyle "point";linethickness 1;lineAttributes
"Dash";var1range "0,40";animationParamRange "290,350";num-x-gridlines
100;curveColor "[flat::RGB:0x000000ff]";curveStyle "Line";VCamFile
'PTNWC303.xvz';valid_file "T";tempfilename
'PQXXR1YS.wmf';tempfile-properties "XPR";}}

For an isothermal process, the temperature does not change, but both the
volume and the pressure change. We can rewrite the ideal gas law as

\begin{equation}
P=\frac{nRT}{%
%TCIMACRO{\TeXButton{V}{\ooalign{\hfil$V$\hfil\cr\kern0.1em--\hfil\cr}}}%
%BeginExpansion
\ooalign{\hfil$V$\hfil\cr\kern0.1em--\hfil\cr}%
%EndExpansion
}
\end{equation}

Performing our work integral for a gas going from some initial volume $%
%TCIMACRO{\TeXButton{V}{\ooalign{\hfil$V$\hfil\cr\kern0.1em--\hfil\cr}}}%
%BeginExpansion
\ooalign{\hfil$V$\hfil\cr\kern0.1em--\hfil\cr}%
%EndExpansion
_{i}$ to a final volume, $%
%TCIMACRO{\TeXButton{V}{\ooalign{\hfil$V$\hfil\cr\kern0.1em--\hfil\cr}}}%
%BeginExpansion
\ooalign{\hfil$V$\hfil\cr\kern0.1em--\hfil\cr}%
%EndExpansion
_{f}.$%
\begin{eqnarray}
w_{int} &=&-\int_{%
%TCIMACRO{\TeXButton{V}{\ooalign{\hfil$V$\hfil\cr\kern0.1em--\hfil\cr}}}%
%BeginExpansion
\ooalign{\hfil$V$\hfil\cr\kern0.1em--\hfil\cr}%
%EndExpansion
_{i}}^{%
%TCIMACRO{\TeXButton{V}{\ooalign{\hfil$V$\hfil\cr\kern0.1em--\hfil\cr}}}%
%BeginExpansion
\ooalign{\hfil$V$\hfil\cr\kern0.1em--\hfil\cr}%
%EndExpansion
_{f}}Pd%
%TCIMACRO{\TeXButton{V}{\ooalign{\hfil$V$\hfil\cr\kern0.1em--\hfil\cr}} }%
%BeginExpansion
\ooalign{\hfil$V$\hfil\cr\kern0.1em--\hfil\cr}
%EndExpansion
\\
&=&-\int_{%
%TCIMACRO{\TeXButton{V}{\ooalign{\hfil$V$\hfil\cr\kern0.1em--\hfil\cr}}}%
%BeginExpansion
\ooalign{\hfil$V$\hfil\cr\kern0.1em--\hfil\cr}%
%EndExpansion
_{i}}^{%
%TCIMACRO{\TeXButton{V}{\ooalign{\hfil$V$\hfil\cr\kern0.1em--\hfil\cr}}}%
%BeginExpansion
\ooalign{\hfil$V$\hfil\cr\kern0.1em--\hfil\cr}%
%EndExpansion
_{f}}\frac{nRT}{%
%TCIMACRO{\TeXButton{V}{\ooalign{\hfil$V$\hfil\cr\kern0.1em--\hfil\cr}}}%
%BeginExpansion
\ooalign{\hfil$V$\hfil\cr\kern0.1em--\hfil\cr}%
%EndExpansion
}d%
%TCIMACRO{\TeXButton{V}{\ooalign{\hfil$V$\hfil\cr\kern0.1em--\hfil\cr}} }%
%BeginExpansion
\ooalign{\hfil$V$\hfil\cr\kern0.1em--\hfil\cr}
%EndExpansion
\\
&=&-nRT\int_{%
%TCIMACRO{\TeXButton{V}{\ooalign{\hfil$V$\hfil\cr\kern0.1em--\hfil\cr}}}%
%BeginExpansion
\ooalign{\hfil$V$\hfil\cr\kern0.1em--\hfil\cr}%
%EndExpansion
_{i}}^{%
%TCIMACRO{\TeXButton{V}{\ooalign{\hfil$V$\hfil\cr\kern0.1em--\hfil\cr}}}%
%BeginExpansion
\ooalign{\hfil$V$\hfil\cr\kern0.1em--\hfil\cr}%
%EndExpansion
_{f}}\frac{1}{%
%TCIMACRO{\TeXButton{V}{\ooalign{\hfil$V$\hfil\cr\kern0.1em--\hfil\cr}}}%
%BeginExpansion
\ooalign{\hfil$V$\hfil\cr\kern0.1em--\hfil\cr}%
%EndExpansion
}d%
%TCIMACRO{\TeXButton{V}{\ooalign{\hfil$V$\hfil\cr\kern0.1em--\hfil\cr}} }%
%BeginExpansion
\ooalign{\hfil$V$\hfil\cr\kern0.1em--\hfil\cr}
%EndExpansion
\\
&=&-nRT\left[ \ln \frac{%
%TCIMACRO{\TeXButton{V}{\ooalign{\hfil$V$\hfil\cr\kern0.1em--\hfil\cr}}}%
%BeginExpansion
\ooalign{\hfil$V$\hfil\cr\kern0.1em--\hfil\cr}%
%EndExpansion
_{f}}{%
%TCIMACRO{\TeXButton{V}{\ooalign{\hfil$V$\hfil\cr\kern0.1em--\hfil\cr}}}%
%BeginExpansion
\ooalign{\hfil$V$\hfil\cr\kern0.1em--\hfil\cr}%
%EndExpansion
_{i}}\right]
\end{eqnarray}%
then for an isothermal process%
\begin{equation}
w_{int}=nRT\left[ \ln \frac{%
%TCIMACRO{\TeXButton{V}{\ooalign{\hfil$V$\hfil\cr\kern0.1em--\hfil\cr}}}%
%BeginExpansion
\ooalign{\hfil$V$\hfil\cr\kern0.1em--\hfil\cr}%
%EndExpansion
_{i}}{%
%TCIMACRO{\TeXButton{V}{\ooalign{\hfil$V$\hfil\cr\kern0.1em--\hfil\cr}}}%
%BeginExpansion
\ooalign{\hfil$V$\hfil\cr\kern0.1em--\hfil\cr}%
%EndExpansion
_{f}}\right]
\end{equation}

So if $%
%TCIMACRO{\TeXButton{V}{\ooalign{\hfil$V$\hfil\cr\kern0.1em--\hfil\cr}}}%
%BeginExpansion
\ooalign{\hfil$V$\hfil\cr\kern0.1em--\hfil\cr}%
%EndExpansion
_{f}>%
%TCIMACRO{\TeXButton{V}{\ooalign{\hfil$V$\hfil\cr\kern0.1em--\hfil\cr}}}%
%BeginExpansion
\ooalign{\hfil$V$\hfil\cr\kern0.1em--\hfil\cr}%
%EndExpansion
_{i}$ the work is negative and if $%
%TCIMACRO{\TeXButton{V}{\ooalign{\hfil$V$\hfil\cr\kern0.1em--\hfil\cr}}}%
%BeginExpansion
\ooalign{\hfil$V$\hfil\cr\kern0.1em--\hfil\cr}%
%EndExpansion
_{f}<%
%TCIMACRO{\TeXButton{V}{\ooalign{\hfil$V$\hfil\cr\kern0.1em--\hfil\cr}}}%
%BeginExpansion
\ooalign{\hfil$V$\hfil\cr\kern0.1em--\hfil\cr}%
%EndExpansion
_{i}$ the work is positive.

\subsection{Path Dependence}

We should do a problem that shows the path dependence of thermodynamics
work. Notice that how we compress the gas (quickly vs. slowly, etc.) makes a
difference on our PV diagram. For example, quick actions would not let
energy leave by heat, so temperature would rise. Slow actions would allow
enough time for energy to leave, allowing the temperature to stay the same.
So when we say there is a \textquotedblleft path
dependence\textquotedblright\ for thermodynamic work, it is equivalent to
saying we have different amounts of work for \textquotedblleft different
paths\textquotedblright\ on a PV-diagram. The PV-diagram shows the different
physical paths as different lines on the graph. Here is an example.

Suppose we change $1\unit{mol}$ of an ideal gas from one state to another by
two different paths. The two paths are shown in the next figure.\FRAME{dtbpF%
}{3.0894in}{2.0676in}{0pt}{}{}{Figure}{\special{language "Scientific
Word";type "GRAPHIC";maintain-aspect-ratio TRUE;display "USEDEF";valid_file
"T";width 3.0894in;height 2.0676in;depth 0pt;original-width
3.1195in;original-height 2.0782in;cropleft "0";croptop "1";cropright
"1";cropbottom "0";tempfilename 'PQXXR1YT.wmf';tempfile-properties "XPR";}}

We recognize the blue path as a combination of an isobaric and an isothermal
process. So for the blue path, the work done will be a combination of%
\begin{eqnarray*}
w_{isobaric} &=&-P\left( 
%TCIMACRO{\TeXButton{V}{\ooalign{\hfil$V$\hfil\cr\kern0.1em--\hfil\cr}}}%
%BeginExpansion
\ooalign{\hfil$V$\hfil\cr\kern0.1em--\hfil\cr}%
%EndExpansion
_{f}-%
%TCIMACRO{\TeXButton{V}{\ooalign{\hfil$V$\hfil\cr\kern0.1em--\hfil\cr}}}%
%BeginExpansion
\ooalign{\hfil$V$\hfil\cr\kern0.1em--\hfil\cr}%
%EndExpansion
_{i}\right) \\
w_{isothermal} &=&nRT\left[ \ln \frac{%
%TCIMACRO{\TeXButton{V}{\ooalign{\hfil$V$\hfil\cr\kern0.1em--\hfil\cr}}}%
%BeginExpansion
\ooalign{\hfil$V$\hfil\cr\kern0.1em--\hfil\cr}%
%EndExpansion
_{i}}{%
%TCIMACRO{\TeXButton{V}{\ooalign{\hfil$V$\hfil\cr\kern0.1em--\hfil\cr}}}%
%BeginExpansion
\ooalign{\hfil$V$\hfil\cr\kern0.1em--\hfil\cr}%
%EndExpansion
_{f}}\right]
\end{eqnarray*}

Then%
\begin{eqnarray*}
w_{blue} &=&-P\left( 
%TCIMACRO{\TeXButton{V}{\ooalign{\hfil$V$\hfil\cr\kern0.1em--\hfil\cr}}}%
%BeginExpansion
\ooalign{\hfil$V$\hfil\cr\kern0.1em--\hfil\cr}%
%EndExpansion
_{f}-%
%TCIMACRO{\TeXButton{V}{\ooalign{\hfil$V$\hfil\cr\kern0.1em--\hfil\cr}}}%
%BeginExpansion
\ooalign{\hfil$V$\hfil\cr\kern0.1em--\hfil\cr}%
%EndExpansion
_{i}\right) +nRT\left[ \ln \frac{%
%TCIMACRO{\TeXButton{V}{\ooalign{\hfil$V$\hfil\cr\kern0.1em--\hfil\cr}}}%
%BeginExpansion
\ooalign{\hfil$V$\hfil\cr\kern0.1em--\hfil\cr}%
%EndExpansion
_{i}}{%
%TCIMACRO{\TeXButton{V}{\ooalign{\hfil$V$\hfil\cr\kern0.1em--\hfil\cr}}}%
%BeginExpansion
\ooalign{\hfil$V$\hfil\cr\kern0.1em--\hfil\cr}%
%EndExpansion
_{f}}\right] \\
&=&-378.29\unit{Pa}\left( 7.\,\allowbreak 538\,4\unit{m}^{3}-6\unit{m}%
^{3}\right) \\
&&+\left( 1\right) \left( 8.314\frac{\unit{J}}{\unit{mol}\unit{K}}\right)
\left( 343\unit{K}\right) \left[ \ln \frac{7.\,\allowbreak 538\,4\unit{m}^{3}%
}{6\unit{m}^{3}}\right] \\
&=&\allowbreak 650.\,\allowbreak 9\unit{J}
\end{eqnarray*}

But the red path is an isovolumetric path. We know that for isovolumetric
paths the work is zero!%
\begin{eqnarray*}
w_{red} &=&-\int_{%
%TCIMACRO{\TeXButton{V}{\ooalign{\hfil$V$\hfil\cr\kern0.1em--\hfil\cr}}}%
%BeginExpansion
\ooalign{\hfil$V$\hfil\cr\kern0.1em--\hfil\cr}%
%EndExpansion
_{1}}^{%
%TCIMACRO{\TeXButton{V}{\ooalign{\hfil$V$\hfil\cr\kern0.1em--\hfil\cr}}}%
%BeginExpansion
\ooalign{\hfil$V$\hfil\cr\kern0.1em--\hfil\cr}%
%EndExpansion
_{2}}Pd%
%TCIMACRO{\TeXButton{V}{\ooalign{\hfil$V$\hfil\cr\kern0.1em--\hfil\cr}} }%
%BeginExpansion
\ooalign{\hfil$V$\hfil\cr\kern0.1em--\hfil\cr}
%EndExpansion
\\
&=&0
\end{eqnarray*}%
and we can see that zero is very different than $651\unit{J}$! Indeed the
work for each PV-diagram path is different. But why?

The red path is what we would get if we put a sealed strong container in a
fire. The fire would provide energy to the gas in the container by heat. But
the strong walls of the container would keep the gas from expanding. The
pressure would rise and the temperature as well. But for the blue path we
envision a movable piston in a cylinder of gas. when the fire adds energy by
heat, the piston rises, keeping the pressure the same. The temperature does
go up. Then we remove the fire, and slowly compress the gas back to its
original volume. By doing this slowly the work we do adds a little bit of
energy to the gas, but the energy can leave by heat, so the temperature of
the gas stays the same.

Note that these are really different processes! So it is no wonder that the
internal work done is different. And we see all of this with our PV-diagram!

\chapter{Heat and the First Law of Thermodynamics}

Last lecture we learned about work and the first law of thermodynamics. But
in the first law there is the energy transfer by heat, $Q.$ We need to
understand this better. That is the subject of today's lecture.

%TCIMACRO{%
%\TeXButton{Fundamental Concepts}{\hspace{-1.3in}{\Large Fundamental Concepts\vspace{0.25in}}}}%
%BeginExpansion
\hspace{-1.3in}{\Large Fundamental Concepts\vspace{0.25in}}%
%EndExpansion

\begin{itemize}
\item Doing work on a system can change the system's temperature. This is
called the mechanical equivalent of heat.

\item There is a sign convention for energy transfer. If energy leaves it is
negative, if it comes into the system it is positive.

\item In thermodynamics, we try to find $\Delta E_{int},$ $Q,$ and $w$ for a
process. We use the ideal gas law and the first law of thermodynamics to do
this.

\item We have special processes for which it is easy to find $\Delta
E_{int}, $ $Q,$ and $w$

\item We are still missing something.
\end{itemize}

\section{Heat and Internal Energy}

%TCIMACRO{%
%\TeXButton{Question 123.11.1}{\marginpar {
%\hspace{-0.5in}
%\begin{minipage}[t]{1in}
%\small{Question 123.11.1}
%\end{minipage}
%}}}%
%BeginExpansion
\marginpar {
\hspace{-0.5in}
\begin{minipage}[t]{1in}
\small{Question 123.11.1}
\end{minipage}
}%
%EndExpansion
We learned in PH 121 what energy was. We studied work and potential energy,
and we studied kinetic energy. But these were always the energy \emph{of} an
object. We described these energies in terms of the motion of the center of
mass of an object using the particle model. Now we know that there is energy 
\emph{in} an object. This is different. Particles don't have internal parts,
but real things do. So we need to go beyond the particle model here. We will
call the energy associated with the inside parts of an object \emph{internal
energy.} And this internal energy is associated with thermal energy.

To study this, Let's refine our definition of internal energy by saying that
it is all the energy of a system that is associated with its microscopic
components -- atoms and molecules -- when viewed form a reference frame at
rest with respect to the center of mass of the system. Any motion of the
center of mass is our PH121 mechanical energy. So we want the motion of the
parts of the object with respect to the center of mass. Let's review center
of mass briefly.

We learned about center of mass in PH 121. 
\begin{equation}
r_{cm}=\frac{1}{M}\int \mathbf{r}dm
\end{equation}%
\FRAME{dtbpF}{2.0968in}{2.3407in}{0pt}{}{}{Figure}{\special{language
"Scientific Word";type "GRAPHIC";maintain-aspect-ratio TRUE;display
"USEDEF";valid_file "T";width 2.0968in;height 2.3407in;depth
0pt;original-width 2.1075in;original-height 2.3567in;cropleft "0";croptop
"1";cropright "1";cropbottom "0";tempfilename
'PQXXR1YU.wmf';tempfile-properties "XPR";}}For groups of molecules, we
remember 
\begin{equation}
\overrightarrow{r}_{cm}=\frac{\dsum\limits_{i}m_{i}\overrightarrow{\mathbf{r}%
}_{i}}{M}
\end{equation}%
where 
\begin{equation}
M=\dsum\limits_{i}m_{i}
\end{equation}

You might remember this in one dimension for two masses%
\begin{equation}
x_{cm}=\frac{m_{1}x_{1}+m_{2}x_{2}}{m_{1}+m_{2}}
\end{equation}

Now suppose we let all of the masses involved be tied together with
spring-like forces. These spring-like forces would allow the masses to
oscillate a little. That would change the $\mathbf{r}_{i}$ involved in our
center of mass calculation. But if the oscillations were random, the actual
center of mass would not change. This is the kind of motion involved in
internal energy.

By making all the little molecules move, we have changed the energy state of
the object, but we have not moved it, so the mechanical energy has not
changed. The energy we have described is the thermal internal energy (there
is still the nuclear internal energy, but that is a subject for PH279).
Let's consider forms of internal energy in more detail in this lecture.

\section{Heat}

We have mentioned heat, and given it a symbol $Q,$ but in this lecture we
want to refine our definition of heat. Heat is defined as the transfer of
energy across the boundary of a system due to a temperature difference
between the system and its surroundings. When we \textquotedblleft
heat\textquotedblright\ a substance, we add energy from it's surroundings.
An object could gain or lose energy \emph{by heat}. We will use the word
\textquotedblleft heat\textquotedblright\ for the transfer of energy no
matter which direction the energy is flowing. This is a little like using
the word \textquotedblleft acceleration\textquotedblright\ no matter whether
we are speeding up or slowing down.

In the dim past, scientists thought heat was a fluid (called caloric). This
model of heat is not correct (science changes!). We now define heat as a
transfer of energy. But because of this history, we have some left-over
names that are not very descriptive of the modern ideas they represent.
Examples are \emph{latent heat} and \emph{heat capacity}.

The thing to remember is that a system \emph{has} a temperature but it can 
\emph{give or receive} energy by heat.

\subsection{Units}

There are several units for heat. The calorie (cal) is the amount of energy
transfer necessary to raise the temperature of $1\unit{g}$ of water from $%
14.5\unit{%
%TCIMACRO{\U{2103}}%
%BeginExpansion
{}^{\circ}{\rm C}%
%EndExpansion
}$ ($287.\,\allowbreak 65\unit{K}$) to $15.5\unit{%
%TCIMACRO{\U{2103}}%
%BeginExpansion
{}^{\circ}{\rm C}%
%EndExpansion
}$ ($288.\,\allowbreak 65\unit{K}$). This is somewhat arbitrary, but it
works for many things, especially biological systems. The British thermal
unit (Btu) is the unit used by refrigeration and heating contractors. One $%
\unit{Btu}$ is the amount of energy transfer necessary to raise the
temperature of $1\unit{lb}$ of water from $63\unit{%
%TCIMACRO{\U{2109}}%
%BeginExpansion
{}^{\circ}{\rm F}%
%EndExpansion
}$ $(290.\,\allowbreak 37\unit{K})$ to $64\unit{%
%TCIMACRO{\U{2109}}%
%BeginExpansion
{}^{\circ}{\rm F}%
%EndExpansion
}$ ($290.\,\allowbreak 93\unit{K}$). This is an enormous amount of energy
being transferred! If you wish to heat an entire building, this is a fine
unit, but for us it is a bit large. In the SI system we already have a unit
for energy, the Joule. We will try to stick to this nice, medium sized,
unit. But let's see where it came from.

\subsection{Mechanical equivalent of heat}

\FRAME{dhFU}{1.8503in}{2.2095in}{0pt}{\Qcb{James Prescott Joule. (Image in
the Public Domain)}}{}{Figure}{\special{language "Scientific Word";type
"GRAPHIC";maintain-aspect-ratio TRUE;display "USEDEF";valid_file "T";width
1.8503in;height 2.2095in;depth 0pt;original-width 4.6985in;original-height
5.6204in;cropleft "0";croptop "1";cropright "1";cropbottom "0";tempfilename
'PQXXR1YV.wmf';tempfile-properties "XPR";}}

the name \textquotedblleft Joule\textquotedblright\ is a person's name,
Joule was an early researcher and is credited with showing that heat
transfer is a transfer of energy. Joule was able to build a device to prove
that mechanical energy could be converted to heat energy. His device is
shown in the picture.\FRAME{dhFU}{1.1815in}{1.7651in}{0pt}{\Qcb{Joule
Apparatus (Image in the Public Domain courtesy Gaius Cornelius)}}{}{Figure}{%
\special{language "Scientific Word";type "GRAPHIC";maintain-aspect-ratio
TRUE;display "USEDEF";valid_file "T";width 1.1815in;height 1.7651in;depth
0pt;original-width 2.2667in;original-height 3.4004in;cropleft "0";croptop
"1";cropright "1";cropbottom "0";tempfilename
'PQXXR1YW.wmf';tempfile-properties "XPR";}}

The paddle device was placed in the drum that is sitting behind it. The
crank was operated by strings attached to it that were pulled by masses tied
to the other end of the string (see the next figure). When the masses were
released, the paddle wheel stirred a liquid. The friction from the paddle
wheel stirring the liquid raised the temperature of the liquid.\FRAME{dhF}{%
3.2707in}{2.7908in}{0pt}{}{}{Figure}{\special{language "Scientific
Word";type "GRAPHIC";maintain-aspect-ratio TRUE;display "USEDEF";valid_file
"T";width 3.2707in;height 2.7908in;depth 0pt;original-width
3.2258in;original-height 2.7484in;cropleft "0";croptop "1";cropright
"1";cropbottom "0";tempfilename 'PQXXR1YX.wmf';tempfile-properties "XPR";}}%
The masses would loose energy equal to 
\begin{equation}
\Delta U=2mgh
\end{equation}%
assuming the masses were equal. By carefully measuring the temperature of
the water, Joule found that 
\begin{equation}
\Delta U=C\Delta T
\end{equation}%
with the proportionality constant approximately 
\begin{equation}
C=4.18\frac{\unit{J}}{\unit{g}}\unit{%
%TCIMACRO{\U{2103}}%
%BeginExpansion
{}^{\circ}{\rm C}%
%EndExpansion
}
\end{equation}%
which means that $4.18\unit{J}$ of mechanical energy raises the temperature
of $1\unit{g}$ of water by $1\unit{%
%TCIMACRO{\U{2103}}%
%BeginExpansion
{}^{\circ}{\rm C}%
%EndExpansion
}.$ (Here $C$ is a generic symbol for a constant). We have more precise
measurements now that set this value at $4.186\frac{\unit{J}}{\unit{g}}\unit{%
%TCIMACRO{\U{2103}}%
%BeginExpansion
{}^{\circ}{\rm C}%
%EndExpansion
}.$ But you can see that Joule did very good for working with what amounts
to a fancy butter churn in the 1800's!

Notice that we are considering raising the temperature of $1\unit{g}$ of
water by $1\unit{%
%TCIMACRO{\U{2103}}%
%BeginExpansion
{}^{\circ}{\rm C}%
%EndExpansion
}.$ This is the basis of the unit called the calorie. Using the same $4.5%
\unit{%
%TCIMACRO{\U{2103}}%
%BeginExpansion
{}^{\circ}{\rm C}%
%EndExpansion
}$ ($287.\,\allowbreak 65\unit{K}$) to $15.5\unit{%
%TCIMACRO{\U{2103}}%
%BeginExpansion
{}^{\circ}{\rm C}%
%EndExpansion
}$ ($288.\,\allowbreak 65\unit{K}$) change in temperature, we realize%
\begin{equation}
1\unit{cal}=4.186\unit{J}
\end{equation}%
which gives us a convenient way to convert from calories to Jules. But more
importantly, it shows that heat is really tied to energy, not to some
substance. Work can raise a temperature in a system just like
\textquotedblleft heating\textquotedblright\ the system with a fire or stove.

Because of this experiment by Joule, this idea that work can be turned into
thermal energy is known as the \emph{mechanical equivalent of heat}.

\section{Heat and systems}

Heat is a transfer of energy, so it can't be a property of the system,
itself. It is an interaction of the system with its environment. This means
that heat is not a state variable. It will depend on process path on a
PV-diagram, like work does. Only the combination 
\[
\Delta E_{int}=Q+w 
\]%
is path dependent.

It is customary to refer to energy gained from an environment as positive
heat, and energy lost to the environment as negative heat.\FRAME{dtbpF}{%
2.095in}{2.1154in}{0in}{}{}{Figure}{\special{language "Scientific Word";type
"GRAPHIC";maintain-aspect-ratio TRUE;display "USEDEF";valid_file "T";width
2.095in;height 2.1154in;depth 0in;original-width 2.1066in;original-height
2.127in;cropleft "0";croptop "1";cropright "1";cropbottom "0";tempfilename
'PQXXR1YY.wmf';tempfile-properties "XPR";}} This is a sign convention. We
could totally define this the other way. But it is useful for everyone to
agree to one system of signs, and the system used in physics today is that
energy is negative when it leaves a system, and positive when it enters a
system. We will need to remember this for our problems.

It is useful to compare the sign convention for heat with that we built for
work (on the gas).

\[
\begin{tabular}{|c|c|c|}
\hline
& \textbf{Work} & \textbf{Heat} \\ \hline
\begin{tabular}{l}
{\small Interaction} \\ 
{\small mechanism}%
\end{tabular}
& 
\begin{tabular}{l}
{\small Mechanical} \\ 
\end{tabular}
& 
\begin{tabular}{l}
{\small Thermal} \\ 
\end{tabular}
\\ \hline
\begin{tabular}{l}
{\small Process} \\ 
\end{tabular}
& 
\begin{tabular}{l}
{\small Macroscopic forces} \\ 
{\small acting on the system}%
\end{tabular}
& 
\begin{tabular}{l}
{\small Microscopic collisions} \\ 
{\small between gas particles in the system}%
\end{tabular}
\\ \hline
\begin{tabular}{l}
{\small Process} \\ 
{\small Requires}%
\end{tabular}
& 
\begin{tabular}{l}
{\small External force} \\ 
{\small and displacement}%
\end{tabular}
& 
\begin{tabular}{l}
{\small Temperature } \\ 
{\small Difference}%
\end{tabular}
\\ \hline
\begin{tabular}{l}
{\small Positive } \\ 
{\small when}%
\end{tabular}
& 
\begin{tabular}{l}
{\small A gas is compressed: mechanical} \\ 
{\small energy is transferred in to system}%
\end{tabular}
& 
\begin{tabular}{l}
{\small The environment is at a higher } \\ 
{\small temperature than the system}%
\end{tabular}
\\ \hline
\begin{tabular}{l}
{\small Negative} \\ 
{\small when}%
\end{tabular}
& 
\begin{tabular}{l}
{\small A gas expands: mechanical} \\ 
{\small energy is transferred out of the system}%
\end{tabular}
& 
\begin{tabular}{l}
{\small The environment is at a lower } \\ 
{\small temperature than the system}%
\end{tabular}
\\ \hline
\begin{tabular}{l}
{\small Equilibrium} \\ 
{\small condition}%
\end{tabular}
& 
\begin{tabular}{l}
{\small No net force or torque} \\ 
{\small on the system}%
\end{tabular}
& 
\begin{tabular}{l}
{\small System and environment are } \\ 
{\small at the same temperature}%
\end{tabular}
\\ \hline
\end{tabular}%
\]

Just like with work, the definition of the word \textquotedblleft
heat\textquotedblright\ takes some practice to get used to. 
%TCIMACRO{%
%\TeXButton{Question 123.11.2}{\marginpar {
%\hspace{-0.5in}
%\begin{minipage}[t]{1in}
%\small{Question 123.11.2}
%\end{minipage}
%}}}%
%BeginExpansion
\marginpar {
\hspace{-0.5in}
\begin{minipage}[t]{1in}
\small{Question 123.11.2}
\end{minipage}
}%
%EndExpansion
%TCIMACRO{%
%\TeXButton{Question 123.11.3}{\marginpar {
%\hspace{-0.5in}
%\begin{minipage}[t]{1in}
%\small{Question 123.11.3}
%\end{minipage}
%}}}%
%BeginExpansion
\marginpar {
\hspace{-0.5in}
\begin{minipage}[t]{1in}
\small{Question 123.11.3}
\end{minipage}
}%
%EndExpansion
%TCIMACRO{%
%\TeXButton{Question 123.11.4}{\marginpar {
%\hspace{-0.5in}
%\begin{minipage}[t]{1in}
%\small{Question 123.11.4}
%\end{minipage}
%}}}%
%BeginExpansion
\marginpar {
\hspace{-0.5in}
\begin{minipage}[t]{1in}
\small{Question 123.11.4}
\end{minipage}
}%
%EndExpansion

It is really important to not use the word \textquotedblleft
heat\textquotedblright\ as we use the word with \textquotedblleft
temperature.\textquotedblright\ Temperature is proportional to the internal
energy, 
\[
E_{int}\propto T 
\]%
as we will show soon. But $Q$ is a transfer of energy. A gas can
\textquotedblleft have\textquotedblright\ a temperature, but it
\textquotedblleft experiences\textquotedblright\ a transfer of energy. That
is why it shows up in the first law as a mechanism for the change in
internal energy.%
\[
\Delta E_{int}=Q+w 
\]%
This is not intuitive because of our left-over language from the caloric
theory. So we must continually be on our guard, because poor language leads
to poor decisions in experiments and engineering designs.

It would help if we all said things like \textquotedblleft to do
heat\textquotedblright\ instead of \textquotedblleft heat\textquotedblright\
when we are warming things. That is what we do with work. We
\textquotedblleft do work\textquotedblright\ so we should \textquotedblleft
do heat,\textquotedblright\ and this may be a good way to think about it,
even though to say \textquotedblleft do heat\textquotedblright\ like this
would make our roommates wonder.

%TCIMACRO{%
%\TeXButton{Question 123.11.5}{\marginpar {
%\hspace{-0.5in}
%\begin{minipage}[t]{1in}
%\small{Question 123.11.5}
%\end{minipage}
%}}}%
%BeginExpansion
\marginpar {
\hspace{-0.5in}
\begin{minipage}[t]{1in}
\small{Question 123.11.5}
\end{minipage}
}%
%EndExpansion
We now know that $\Delta E_{int}$ is a state variable. It does not depend on
path. But it is just one of many state variable for a gas. For ideal gasses,
we have 
\[
P,%
%TCIMACRO{\TeXButton{V}{\ooalign{\hfil$V$\hfil\cr\kern0.1em--\hfil\cr}}}%
%BeginExpansion
\ooalign{\hfil$V$\hfil\cr\kern0.1em--\hfil\cr}%
%EndExpansion
,n,T,\Delta E_{int} 
\]%
all as state variables. It takes all of these quantities to describe an
ideal gas.

For example, we know that%
\[
E_{int}\propto T 
\]%
but this does not tell us $\Delta E_{int}.$ The quantity $\Delta E_{int}$
only tells us about the change in $E_{int}$. We cannot tell what path was
used to accomplish $\Delta E_{int}.$ So our first law does not tell us $P,$ $%
%TCIMACRO{\TeXButton{V}{\ooalign{\hfil$V$\hfil\cr\kern0.1em--\hfil\cr}}}%
%BeginExpansion
\ooalign{\hfil$V$\hfil\cr\kern0.1em--\hfil\cr}%
%EndExpansion
,$ or $n$ either. We will have to use the first law in cooperation with the
ideal gas law.

\section{Special processes}

Armed with both the ideal gas law, and the first law of thermodynamics,
let's review our special processes--and maybe gain another! Our goal is to
see how these processes relate to the first law.

\subsection{Constant Temperature Process}

For an isothermal process\FRAME{dhF}{2.4569in}{1.6293in}{0pt}{}{}{Figure}{%
\special{language "Scientific Word";type "GRAPHIC";maintain-aspect-ratio
TRUE;display "USEDEF";valid_file "T";width 2.4569in;height 1.6293in;depth
0pt;original-width 4.0318in;original-height 2.6654in;cropleft "0";croptop
"1";cropright "1";cropbottom "0";tempfilename
'PQXXR1YZ.wmf';tempfile-properties "XPR";}}we expect that 
\[
\Delta E_{int}\propto \Delta T=0 
\]%
The work we found to be%
\begin{eqnarray*}
W &=&-\int Pd%
%TCIMACRO{\TeXButton{V}{\ooalign{\hfil$V$\hfil\cr\kern0.1em--\hfil\cr}} }%
%BeginExpansion
\ooalign{\hfil$V$\hfil\cr\kern0.1em--\hfil\cr}
%EndExpansion
\\
&=&nRT\left[ \ln \frac{%
%TCIMACRO{\TeXButton{V}{\ooalign{\hfil$V$\hfil\cr\kern0.1em--\hfil\cr}}}%
%BeginExpansion
\ooalign{\hfil$V$\hfil\cr\kern0.1em--\hfil\cr}%
%EndExpansion
_{i}}{%
%TCIMACRO{\TeXButton{V}{\ooalign{\hfil$V$\hfil\cr\kern0.1em--\hfil\cr}}}%
%BeginExpansion
\ooalign{\hfil$V$\hfil\cr\kern0.1em--\hfil\cr}%
%EndExpansion
_{f}}\right]
\end{eqnarray*}%
and from the first law%
\begin{eqnarray*}
\Delta E_{int} &=&Q+w \\
0 &=&Q+nRT\left[ \ln \frac{%
%TCIMACRO{\TeXButton{V}{\ooalign{\hfil$V$\hfil\cr\kern0.1em--\hfil\cr}}}%
%BeginExpansion
\ooalign{\hfil$V$\hfil\cr\kern0.1em--\hfil\cr}%
%EndExpansion
_{i}}{%
%TCIMACRO{\TeXButton{V}{\ooalign{\hfil$V$\hfil\cr\kern0.1em--\hfil\cr}}}%
%BeginExpansion
\ooalign{\hfil$V$\hfil\cr\kern0.1em--\hfil\cr}%
%EndExpansion
_{f}}\right]
\end{eqnarray*}%
so%
\[
Q=nRT\left[ \ln \frac{%
%TCIMACRO{\TeXButton{V}{\ooalign{\hfil$V$\hfil\cr\kern0.1em--\hfil\cr}}}%
%BeginExpansion
\ooalign{\hfil$V$\hfil\cr\kern0.1em--\hfil\cr}%
%EndExpansion
_{f}}{%
%TCIMACRO{\TeXButton{V}{\ooalign{\hfil$V$\hfil\cr\kern0.1em--\hfil\cr}}}%
%BeginExpansion
\ooalign{\hfil$V$\hfil\cr\kern0.1em--\hfil\cr}%
%EndExpansion
_{i}}\right] 
\]

Looking at an isothermal process can help us understand our other processes.
Suppose we have two different isothermal processes, one at $100\unit{K}$ and
one at $300\unit{K}.$ The isotherms look like this on a $P%
%TCIMACRO{\TeXButton{V}{\ooalign{\hfil$V$\hfil\cr\kern0.1em--\hfil\cr}}}%
%BeginExpansion
\ooalign{\hfil$V$\hfil\cr\kern0.1em--\hfil\cr}%
%EndExpansion
$ diagram: \FRAME{dtbpF}{2.7612in}{1.8316in}{0pt}{}{}{Figure}{\special%
{language "Scientific Word";type "GRAPHIC";maintain-aspect-ratio
TRUE;display "USEDEF";valid_file "T";width 2.7612in;height 1.8316in;depth
0pt;original-width 2.7851in;original-height 1.8378in;cropleft "0";croptop
"1";cropright "1";cropbottom "0";tempfilename
'PQXXR1Z0.wmf';tempfile-properties "XPR";}}Every point on the $300\unit{K}$
line has the same temperature. Every point on the $100\unit{K}$ line has the
same temperature. We can see that points on the $P%
%TCIMACRO{\TeXButton{V}{\ooalign{\hfil$V$\hfil\cr\kern0.1em--\hfil\cr}}}%
%BeginExpansion
\ooalign{\hfil$V$\hfil\cr\kern0.1em--\hfil\cr}%
%EndExpansion
\ $diagram that are closer to the origin must have lower temperatures. We
can see the points farther from the origin have higher temperatures. \FRAME{%
dtbpF}{2.8091in}{1.8636in}{0pt}{}{}{Figure}{\special{language "Scientific
Word";type "GRAPHIC";maintain-aspect-ratio TRUE;display "USEDEF";valid_file
"T";width 2.8091in;height 1.8636in;depth 0pt;original-width
2.8339in;original-height 1.8707in;cropleft "0";croptop "1";cropright
"1";cropbottom "0";tempfilename 'PQXXR1Z1.wmf';tempfile-properties "XPR";}}

\subsection{Constant volume process}

%TCIMACRO{%
%\TeXButton{Question 123.11.6}{\marginpar {
%\hspace{-0.5in}
%\begin{minipage}[t]{1in}
%\small{Question 123.11.6}
%\end{minipage}
%}}}%
%BeginExpansion
\marginpar {
\hspace{-0.5in}
\begin{minipage}[t]{1in}
\small{Question 123.11.6}
\end{minipage}
}%
%EndExpansion
We know an isochoric process takes us between two states on a $P%
%TCIMACRO{\TeXButton{V}{\ooalign{\hfil$V$\hfil\cr\kern0.1em--\hfil\cr}}}%
%BeginExpansion
\ooalign{\hfil$V$\hfil\cr\kern0.1em--\hfil\cr}%
%EndExpansion
\ $diagram as shown in the next figure. \FRAME{dhF}{2.6861in}{1.8948in}{0pt}{%
}{}{Figure}{\special{language "Scientific Word";type
"GRAPHIC";maintain-aspect-ratio TRUE;display "USEDEF";valid_file "T";width
2.6861in;height 1.8948in;depth 0pt;original-width 5.1439in;original-height
3.6227in;cropleft "0";croptop "1";cropright "1";cropbottom "0";tempfilename
'PQXXR1Z2.wmf';tempfile-properties "XPR";}}We can see that the temperature
will change, and since 
\[
E_{int}\propto T 
\]%
we expect that 
\[
\Delta E_{int}\propto \Delta T 
\]%
so we expect 
\[
\Delta E_{int}\neq 0 
\]%
The work done is 
\[
W=-\int Pd%
%TCIMACRO{\TeXButton{V}{\ooalign{\hfil$V$\hfil\cr\kern0.1em--\hfil\cr}}}%
%BeginExpansion
\ooalign{\hfil$V$\hfil\cr\kern0.1em--\hfil\cr}%
%EndExpansion
\]%
but the volume does not change, so 
\[
W=0 
\]%
from the first law, we can see that 
\begin{eqnarray*}
\Delta E_{int} &=&Q+w \\
&=&Q+0
\end{eqnarray*}%
so 
\[
Q=\Delta E_{int} 
\]

\subsection{Constant Pressure process}

Consider the next $P%
%TCIMACRO{\TeXButton{V}{\ooalign{\hfil$V$\hfil\cr\kern0.1em--\hfil\cr}}}%
%BeginExpansion
\ooalign{\hfil$V$\hfil\cr\kern0.1em--\hfil\cr}%
%EndExpansion
\ $diagram.

\FRAME{dhF}{3.0009in}{2.047in}{0pt}{}{}{Figure}{\special{language
"Scientific Word";type "GRAPHIC";maintain-aspect-ratio TRUE;display
"USEDEF";valid_file "T";width 3.0009in;height 2.047in;depth
0pt;original-width 5.5893in;original-height 3.8026in;cropleft "0";croptop
"1";cropright "1";cropbottom "0";tempfilename
'PQXXR1Z3.wmf';tempfile-properties "XPR";}}Again we see that the temperature
has changed, so 
\[
\Delta E_{int}\propto \Delta T 
\]%
and we expect that 
\[
\Delta E_{int}\neq 0 
\]%
The work done is 
\begin{eqnarray*}
W &=&-\int Pd%
%TCIMACRO{\TeXButton{V}{\ooalign{\hfil$V$\hfil\cr\kern0.1em--\hfil\cr}} }%
%BeginExpansion
\ooalign{\hfil$V$\hfil\cr\kern0.1em--\hfil\cr}
%EndExpansion
\\
&=&P\left( 
%TCIMACRO{\TeXButton{V}{\ooalign{\hfil$V$\hfil\cr\kern0.1em--\hfil\cr}}}%
%BeginExpansion
\ooalign{\hfil$V$\hfil\cr\kern0.1em--\hfil\cr}%
%EndExpansion
_{2}-%
%TCIMACRO{\TeXButton{V}{\ooalign{\hfil$V$\hfil\cr\kern0.1em--\hfil\cr}}}%
%BeginExpansion
\ooalign{\hfil$V$\hfil\cr\kern0.1em--\hfil\cr}%
%EndExpansion
_{1}\right)
\end{eqnarray*}%
From the first law 
\begin{eqnarray*}
\Delta E_{int} &=&Q+w \\
&=&Q-P\left( 
%TCIMACRO{\TeXButton{V}{\ooalign{\hfil$V$\hfil\cr\kern0.1em--\hfil\cr}}}%
%BeginExpansion
\ooalign{\hfil$V$\hfil\cr\kern0.1em--\hfil\cr}%
%EndExpansion
_{2}-%
%TCIMACRO{\TeXButton{V}{\ooalign{\hfil$V$\hfil\cr\kern0.1em--\hfil\cr}}}%
%BeginExpansion
\ooalign{\hfil$V$\hfil\cr\kern0.1em--\hfil\cr}%
%EndExpansion
_{1}\right)
\end{eqnarray*}%
so 
\[
Q=\Delta E_{int}+P\left( 
%TCIMACRO{\TeXButton{V}{\ooalign{\hfil$V$\hfil\cr\kern0.1em--\hfil\cr}}}%
%BeginExpansion
\ooalign{\hfil$V$\hfil\cr\kern0.1em--\hfil\cr}%
%EndExpansion
_{2}-%
%TCIMACRO{\TeXButton{V}{\ooalign{\hfil$V$\hfil\cr\kern0.1em--\hfil\cr}}}%
%BeginExpansion
\ooalign{\hfil$V$\hfil\cr\kern0.1em--\hfil\cr}%
%EndExpansion
_{1}\right) 
\]%
but we don't have enough information to obtain a numerical value. We must be
missing something!

\subsection{Adiabatic process}

We have a process where $w=0.$ It seems reasonable that we would also have a
process where $Q=0.$ But none of our special process do that so far. Let's
invent a new process such that $Q=0$ and call it by the name \emph{adiabatic}%
. The PV diagram looks like this.\FRAME{dtbpFX}{2.2087in}{1.4719in}{0pt}{}{}{%
Plot}{\special{language "Scientific Word";type "MAPLEPLOT";width
2.2087in;height 1.4719in;depth 0pt;display "USEDEF";plot_snapshots
TRUE;mustRecompute FALSE;lastEngine "MuPAD";animated TRUE;animationStartTime
"0"; animationEndTime "10"; animationFramesPerSecond "10";xmin "0";xmax
"100";animationParamMin "290";animationParamMax "350";xviewmin "0";xviewmax
"40";yviewmin "0";yviewmax "400";viewset"XY";rangeset"X";plottype
4;labeloverrides 3;x-label "V(m^3)";y-label "P(Pa)";axesFont "Times New
Roman,12,0000000000,useDefault,normal";numpoints 100;plotstyle
"patch";axesstyle "normal";axestips FALSE;xis \TEXUX{V};animationParam
\TEXUX{t};var1name \TEXUX{$V$};animationParamList \TEXUX{$t$};function
\TEXUX{$\allowbreak 7500\frac{1}{V^{1.67}}$};linecolor "blue";linestyle
1;pointstyle "point";linethickness 3;lineAttributes "Solid";var1range
"0,100";animationParamRange "290,350";num-x-gridlines 50;curveColor
"[flat::RGB:0x000000ff]";curveStyle "Line";animationStartTime "0";
animationEndTime "10"; animationFramesPerSecond
"10";animationVisibleBeforeStart FALSE;rangeset"XA";function
\TEXUX{$\allowbreak 8.\,\allowbreak 314\frac{1}{V}290$};linecolor
"red";linestyle 2;pointstyle "point";linethickness 1;lineAttributes
"Dash";var1range "0,40";animationParamRange "290,350";num-x-gridlines
100;curveColor "[flat::RGB:0x00ff0000]";curveStyle "Line";function
\TEXUX{$8.\,\allowbreak 314\frac{1}{V}100$};linecolor "red";linestyle
2;pointstyle "point";linethickness 1;lineAttributes "Dash";var1range
"0,40";animationParamRange "290,350";num-x-gridlines 100;curveColor
"[flat::RGB:0x00ff0000]";curveStyle "Line";VCamFile
'PTQ0V800.xvz';valid_file "T";tempfilename
'PQXXR1Z4.wmf';tempfile-properties "XPR";}}

The blue dashed lines are isotherms. Notice that the temperature is changing
along the adiabatic path. That implies that 
\[
\Delta E_{int}\propto \Delta T\neq 0 
\]%
but we defined this path such that 
\[
Q=0 
\]%
so by the first law%
\begin{eqnarray*}
\Delta E_{int} &=&Q+w \\
&=&0+w \\
\Delta E_{int} &=&w
\end{eqnarray*}%
We again don't have enough information to tell what work was done. We
suspect there is something missing. We will start to fill in the missing
piece in the next lecture.

\chapter{Thermal Properties of Matter}

Last lecture we found we can't quite finish the job of finding $\Delta
E_{int},$ $Q,$ and $w$ for our special thermodynamic processes. The problem
is that although we have a strategy for finding $w,$ we don't have such a
strategy for Q. We will start to fill that void in this lecture.

%TCIMACRO{%
%\TeXButton{Fundamental Concepts}{\hspace{-1.3in}{\Large Fundamental Concepts\vspace{0.25in}}}}%
%BeginExpansion
\hspace{-1.3in}{\Large Fundamental Concepts\vspace{0.25in}}%
%EndExpansion

\begin{enumerate}
\item Heat capacity relates energy transfer by heat to temperature change in
solids and liquids

\item Specific heat is a heat capacity per unit mass

\item A phase change is changing from solid to liquid, or from liquid to gas
for a sample of a material

\item Calorimetry is the use of thermal conservation of energy to study a
sample of material
\end{enumerate}

%TCIMACRO{%
%\TeXButton{Question 123.12.1}{\marginpar {
%\hspace{-0.5in}
%\begin{minipage}[t]{1in}
%\small{Question 123.12.1}
%\end{minipage}
%}}}%
%BeginExpansion
\marginpar {
\hspace{-0.5in}
\begin{minipage}[t]{1in}
\small{Question 123.12.1}
\end{minipage}
}%
%EndExpansion
%TCIMACRO{%
%\TeXButton{Question 123.12.1.1}{\marginpar {
%\hspace{-0.5in}
%\begin{minipage}[t]{1in}
%\small{Question 123.12.1.1}
%\end{minipage}
%}}}%
%BeginExpansion
\marginpar {
\hspace{-0.5in}
\begin{minipage}[t]{1in}
\small{Question 123.12.1.1}
\end{minipage}
}%
%EndExpansion
Our goal in this lecture is to find the missing piece that allows us to find 
$\Delta E_{int},$ $W,$ and $Q$ for our systems--at least for our special
processes. We have been looking at ideal gases, but let's go back to solids
and liquids for a moment and see if we can gain a clue experimentally from
these simpler cases.

Suppose you are heating something on your stove, say, a sample of steak. 
\FRAME{dtbpF}{1.6134in}{1.9451in}{0in}{}{}{Figure}{\special{language
"Scientific Word";type "GRAPHIC";maintain-aspect-ratio TRUE;display
"USEDEF";valid_file "T";width 1.6134in;height 1.9451in;depth
0in;original-width 1.6152in;original-height 1.954in;cropleft "0";croptop
"1";cropright "1";cropbottom "0";tempfilename
'PQXXR1Z5.wmf';tempfile-properties "XPR";}}You would expect that as energy
is transferred from the hot stove to the cold beef, the temperature of the
steak would increase. It makes sense that the more energy you transfer, the
more the temperature will rise. Gasses are more complicated, because their
volume will change, but the volume of solids does not change much with
temperature (we know that it changes a little) so we don't have as many
complications as we would with a gas.\FRAME{dhF}{2.1802in}{1.6466in}{0pt}{}{%
}{Figure}{\special{language "Scientific Word";type
"GRAPHIC";maintain-aspect-ratio TRUE;display "USEDEF";valid_file "T";width
2.1802in;height 1.6466in;depth 0pt;original-width 2.1404in;original-height
1.6103in;cropleft "0";croptop "1";cropright "1";cropbottom "0";tempfilename
'PQXXR1Z6.wmf';tempfile-properties "XPR";}}

We could guess that the temperature increase might be linear in the amount
of energy transfer we give to the solid.%
\[
\Delta T\propto \Delta E_{int} 
\]%
where our only form of increasing the energy here is $Q$, since the change
in volume of the stake sample is small enough that any work, $w,$ is
negligible. Since $P$ and $%
%TCIMACRO{\TeXButton{V}{\ooalign{\hfil$V$\hfil\cr\kern0.1em--\hfil\cr}}}%
%BeginExpansion
\ooalign{\hfil$V$\hfil\cr\kern0.1em--\hfil\cr}%
%EndExpansion
$ are not changing much, 
\[
\Delta T\propto Q 
\]

Experimentalists did this experiment and found that for many temperature
regions this is true. Of course, we could over heat our sample\FRAME{dtbpF}{%
2.7612in}{2.0019in}{0pt}{}{}{Figure}{\special{language "Scientific
Word";type "GRAPHIC";maintain-aspect-ratio TRUE;display "USEDEF";valid_file
"T";width 2.7612in;height 2.0019in;depth 0pt;original-width
2.7851in;original-height 2.0117in;cropleft "0";croptop "1";cropright
"1";cropbottom "0";tempfilename 'PQXXR1Z7.wmf';tempfile-properties "XPR";}}%
changing it's form (from stake to charcoal, and smoke), but then the sample
is no longer the same substance. In this case $\Delta T$ is no longer
proportional to $Q.$ But as long as we don't change the composition of the
substance (we stay in the linear regime) , we can say that $\Delta T$ is
proportional to $Q.$

Physicists don't like proportionality signs, so they invented a constant
with the right units that would make this an equality%
\[
\Delta T=\frac{1}{K}Q 
\]%
But, they found that the internal structure of the solid mattered. There was
a different constant, $K,$ for every material.

They called the constant, $K,$ the \emph{heat capacity} of the material. The
name is another throwback to the past, but for us it means the constant of
proportionality between $Q$ and $\Delta T$ that holds all the specific
material properties of that substance that change the slope of the $\Delta T$
vs. $Q$ graph.

Of course, the more material we have in our sample, the longer it takes to
heat it up. If we have a roast, it takes longer to cook than a small steak.
So it is customary to define a heat capacity per unit mass, this is called
the \emph{specific heat} of the material.%
\[
\Delta T=\frac{1}{mc}Q 
\]%
and we often rearrange this equation to give 
\[
Q=mc\Delta T 
\]%
As an example, it takes $4190\unit{J}$ to raise $1\unit{kg}$ of water by $1%
\unit{K}.$ So 
\[
4190\unit{J}=\left( 1\unit{kg}\right) \left( c\right) \left( 1\unit{K}%
\right) 
\]%
solving for $c$ yields 
\[
c_{water}=4190\frac{\unit{J}}{\unit{kg}\unit{K}} 
\]%
Which is familiar to us from Joule's work.

Here are a few more substances and their specific heat values.%
\[
\begin{tabular}{|l|l|l|}
\hline
\textbf{Substance} & $c\left( \frac{\unit{J}}{\unit{kg}\unit{K}}\right) $ & $%
c\left( \frac{\unit{J}}{\unit{mol}\unit{K}}\right) $ \\ \hline
Aluminum & $900$ & $24.3$ \\ \hline
Copper & $385$ & $24.4$ \\ \hline
Iron & $449$ & $25.1$ \\ \hline
Gold & $129$ & $25.4$ \\ \hline
Lead & $128$ & $26.5$ \\ \hline
Ice & $2090$ & $37.6$ \\ \hline
Mercury & $140$ & $28.1$ \\ \hline
Water & $4190$ & $75.4$ \\ \hline
\end{tabular}%
\]%
Notice that things that heat up or cool down quickly have small specific
heats. Things that heat up slowly or cool slowly have large specific heats.
Water in any of it's phases has large values of specific heat. This is why
living next to a lake or the ocean keeps your temperatures moderate. It
takes quite a lot of energy transfer to heat up or cool off the water, so
the temperature of the water does not change much. This is why Buffalo
temperatures lower when the lake freezes. This is also the source of lake or
ocean breezes. In the morning, the land changes temperature more quickly
than the water. The warmer air over the land becomes less dense and rises,
leaving a lower pressure area. This causes a cool breeze to form from the
sea to the land.

Conversely, in a large city, the concrete and blacktop heat up much quicker,
so we experience increases in daytime temperature as a city grows.

A specific heat is a per-unit-mass property, but we could just as well
define specific heat as a per-unit-mole quantity. Then the specific heat
would be the amount of energy it takes to increase the temperature of a mole
of material. This incarnation of the idea of specific heat is called a \emph{%
molar specific heat,} and we will give it the symbol $C.$%
\[
Q=nC\Delta T 
\]

It is worth noting that the molar heat capacities given in the last table
are remarkably alike. Except for water, which is a compound, they are all
about $25\unit{J}/\left( \unit{mol}\unit{K}\right) .$ This gives us a clue
towards understanding what is happening microscopically, but we will have to
wait to investigate this clue until a further lecture.

WARNING: Specific heat values are really not constant. They do vary with
temperature a little. We know from our study of thermal expansion that the
volume of a sample of a material will change a small amount as it is heated
or cooled. There is a small amount of work done in changing the volume of
the material. Not all of the energy will go into changing the temperature.
So our heat capacity will change a little as the volume changes. But if the
change in temperature is not too big, we can treat heat capacities for
solids and liquids as constants. Oh, but for gasses...we know the volume
might change quite a lot as the temperature changes for a gas, so we will
find that this analysis needs a few changes for gasses. We will study this
is a later lecture.

Note that water has a very high heat capacity. This explains our lake effect
that we discussed (areas near lakes do not get as cold in the winter). When
the water changes temperature, $\Delta T,$ the air around the lake receives
a large $Q.$

\section{Phase changes and heat of transformation}

%TCIMACRO{%
%\TeXButton{Question 123.12.2}{\marginpar {
%\hspace{-0.5in}
%\begin{minipage}[t]{1in}
%\small{Question 123.12.2}
%\end{minipage}
%}}}%
%BeginExpansion
\marginpar {
\hspace{-0.5in}
\begin{minipage}[t]{1in}
\small{Question 123.12.2}
\end{minipage}
}%
%EndExpansion
%TCIMACRO{%
%\TeXButton{Question 123.12.3}{\marginpar {
%\hspace{-0.5in}
%\begin{minipage}[t]{1in}
%\small{Question 123.12.3}
%\end{minipage}
%}}}%
%BeginExpansion
\marginpar {
\hspace{-0.5in}
\begin{minipage}[t]{1in}
\small{Question 123.12.3}
\end{minipage}
}%
%EndExpansion
Recall the graph of temperature vs. time for heating a chunk of Rexburg ice. 
\FRAME{dhF}{3.2595in}{2.0202in}{0pt}{}{}{Figure}{\special{language
"Scientific Word";type "GRAPHIC";maintain-aspect-ratio TRUE;display
"USEDEF";valid_file "T";width 3.2595in;height 2.0202in;depth
0pt;original-width 4.8101in;original-height 2.9706in;cropleft "0";croptop
"1";cropright "1";cropbottom "0";tempfilename
'PQXXR1Z8.wmf';tempfile-properties "XPR";}}Here is the graph again, but this
time the horizontal axis is labeled \textquotedblleft Energy
added\textquotedblright\ because we realize that we are adding internal
energy as we wait and watch the ice. \FRAME{dtbpF}{3.5267in}{2.6498in}{0pt}{%
}{}{Figure}{\special{language "Scientific Word";type
"GRAPHIC";maintain-aspect-ratio TRUE;display "USEDEF";valid_file "T";width
3.5267in;height 2.6498in;depth 0pt;original-width 5.0004in;original-height
3.7498in;cropleft "0";croptop "1";cropright "1";cropbottom "0";tempfilename
'PQXXR1Z9.wmf';tempfile-properties "XP";}}We see regions of the graph where
the ice changes temperature, and we can understand that during these times
the temperature gain will be proportional to the amount of energy
transferred by heat.%
\[
\text{Slope}=\frac{\Delta T}{Q}=\frac{1}{Mc} 
\]

But what about the horizontal parts of the graph? We identified these before
as phase changes. The slope is zero during the melting and boiling times.

We have an amount of energy transferred by heat. You might guess that how
much energy it takes for the phase change to completely change from one
state to the other depends on the strength of the bonds of the material. The
added energy is going into breaking those bonds. So the amount of energy it
takes to, say, melt different substances will be different. From experience
we know that it takes more energy to heat large amounts of material than
small amounts of material. We can express this mathematically as 
\[
Q=\pm LM 
\]%
where $L$ is the symbol we give to the constant that expresses the how
easily the material melts or freezes. We need at least two of these, one for
melting/freezing and one for boiling/condensing. Then 
\begin{eqnarray*}
Q &=&\pm L_{f}M\qquad \text{melt/freeze} \\
Q &=&\pm L_{v}M\qquad \text{boil/condense}
\end{eqnarray*}%
where we call these two constants 
\[
\begin{tabular}{ll}
$L_{f}$ & heat of fusion \\ 
$L_{v}$ & heat of vaporization%
\end{tabular}%
\]%
generically we call each of these a \emph{heat of transformation} but
generally they are still referred to by their old title of \textquotedblleft
Latent heat.\textquotedblright

\[
\begin{tabular}{|l|l|l|l|l|}
\hline
\textbf{Substance} & $T_{m}\left( \unit{%
%TCIMACRO{\U{2103}}%
%BeginExpansion
{}^{\circ}{\rm C}%
%EndExpansion
}\right) $ & $L_{f}\left( \frac{\unit{J}}{\unit{kg}}\right) $ & $T_{m}\left( 
\unit{%
%TCIMACRO{\U{2103}}%
%BeginExpansion
{}^{\circ}{\rm C}%
%EndExpansion
}\right) $ & $L_{v}\left( \frac{\unit{J}}{\unit{kg}}\right) $ \\ \hline
Nitrogen $\left( N_{2}\right) $ & $-210$ & $0.26\times 10^{5}$ & $-196$ & $%
1.99\times 10^{5}$ \\ \hline
Ethyl alcohol & $-114$ & $1.09\times 10^{5}$ & $78$ & $8.79\times 10^{5}$ \\ 
\hline
Mercury & $-39$ & $0.11\times 10^{5}$ & $357$ & $2.96\times 10^{5}$ \\ \hline
Water & $0$ & $3.33\times 10^{5}$ & $100$ & $22.6\times 10^{5}$ \\ \hline
Lead & $328$ & $0.25\times 10^{5}$ & $1750$ & $8.58\times 10^{5}$ \\ \hline
\end{tabular}%
\]

Notice that our equations each have a \textquotedblleft $\pm $%
\textquotedblright\ sign in them. We must supply the sign by context.
Remember that energy that leaves is negative and energy that is gained by
the system is positive. Also notice that so long as we warm the ice
quasi-statically, that the ice completely melts before the temperature of
the water starts to rise.

Let's do a problem. Suppose we want to increase the temperature of $1\unit{g}%
=\allowbreak 0.001\,\unit{kg}$ of ice at $-30.0\unit{%
%TCIMACRO{\U{2103}}%
%BeginExpansion
{}^{\circ}{\rm C}%
%EndExpansion
}$ and change it to steam at $120.0\unit{%
%TCIMACRO{\U{2103}}%
%BeginExpansion
{}^{\circ}{\rm C}%
%EndExpansion
}.$ A graph of temperature vs. heat energy is shown. Let's take it one piece
at a time.\FRAME{dhF}{3.1176in}{2.1093in}{0pt}{}{}{Figure}{\special{language
"Scientific Word";type "GRAPHIC";maintain-aspect-ratio TRUE;display
"USEDEF";valid_file "T";width 3.1176in;height 2.1093in;depth
0pt;original-width 5.4916in;original-height 3.7057in;cropleft "0";croptop
"1";cropright "1";cropbottom "0";tempfilename
'PQXXR1ZA.wmf';tempfile-properties "XPR";}}

Part A: Warming up the ice to $0\unit{%
%TCIMACRO{\U{2103}}%
%BeginExpansion
{}^{\circ}{\rm C}%
%EndExpansion
}.$

This part acts as we learned in the last section. We have 
\begin{eqnarray*}
Q &=&m_{i}c_{i}\Delta T \\
&=&\left( 0.001\,\unit{kg}\right) \left( 2090\frac{\unit{J}}{\unit{kg}\unit{%
%TCIMACRO{\U{2103}}%
%BeginExpansion
{}^{\circ}{\rm C}%
%EndExpansion
}}\right) \left( 0\unit{%
%TCIMACRO{\U{2103}}%
%BeginExpansion
{}^{\circ}{\rm C}%
%EndExpansion
}+30\unit{%
%TCIMACRO{\U{2103}}%
%BeginExpansion
{}^{\circ}{\rm C}%
%EndExpansion
}\right) \\
&=&62.\,\allowbreak 7\unit{J}
\end{eqnarray*}%
We can get away with using $\unit{%
%TCIMACRO{\U{2103}}%
%BeginExpansion
{}^{\circ}{\rm C}%
%EndExpansion
}$ because we only need $\Delta T$ and the divisions of the Celsius and
Kelvin scales are the same size.

Part B: Melting ice

Now the ice changes to water. We have a heat of transformation. To find out
what happens we use the equation%
\begin{eqnarray*}
Q &=&m_{i}L_{f} \\
&=&\left( 0.001\,\unit{kg}\right) \left( 3.33\times 10^{5}\frac{\unit{J}}{%
\unit{kg}}\right) \\
&=&\allowbreak 333.0\unit{J}
\end{eqnarray*}

The total energy so far is $62.\,\allowbreak 7\unit{J}+330.0\unit{J}%
=\allowbreak 392.\,\allowbreak 7\unit{J}$

Part C: Warming the melted ice (water)

Again we have the normal case where we can use the specific heat of water
(now that the ice has melted)

\begin{eqnarray*}
Q &=&m_{w}c_{w}\Delta T \\
&=&\left( 1\unit{g}\right) \left( 4190\frac{\unit{J}}{\unit{kg}\unit{%
%TCIMACRO{\U{2103}}%
%BeginExpansion
{}^{\circ}{\rm C}%
%EndExpansion
}}\right) \left( 100\unit{%
%TCIMACRO{\U{2103}}%
%BeginExpansion
{}^{\circ}{\rm C}%
%EndExpansion
}-0\unit{%
%TCIMACRO{\U{2103}}%
%BeginExpansion
{}^{\circ}{\rm C}%
%EndExpansion
}\right) \\
&=&\allowbreak 419.0\unit{J}
\end{eqnarray*}

The total energy so far is $62.\,\allowbreak 7\unit{J}+333.0\unit{J}%
+419.\,\allowbreak 0\unit{J}=\allowbreak 814.\,\allowbreak 7\unit{J}$

Part D: Boiling the melted ice (water)

Now we convert the water to steam. 
\begin{eqnarray*}
Q &=&m_{w}L_{v} \\
&=&\left( 0.001\,\unit{kg}\right) \left( 2.26\times 10^{6}\frac{\unit{J}}{%
\unit{kg}}\right) \\
&=&\allowbreak 2260.0\unit{J}
\end{eqnarray*}

The total energy so far is $62.\,\allowbreak 7\unit{J}+333.0\unit{J}%
+419.\,\allowbreak 0\unit{J}+2260.0\unit{J}=\allowbreak 3074.\,\allowbreak 7%
\unit{J}$

Part E: Warming the steam

Now we have steam, and can use the specific heat of steam (we don't usually
do this! we usually use the ideal gas law when it becomes a gas!)%
\begin{eqnarray*}
Q &=&m_{s}c_{s}\Delta T \\
&=&\left( 0.001\,\unit{kg}\right) \left( 2010\frac{\unit{J}}{\unit{kg}\unit{%
%TCIMACRO{\U{2103}}%
%BeginExpansion
{}^{\circ}{\rm C}%
%EndExpansion
}}\right) \left( 120\unit{%
%TCIMACRO{\U{2103}}%
%BeginExpansion
{}^{\circ}{\rm C}%
%EndExpansion
}-100\unit{%
%TCIMACRO{\U{2103}}%
%BeginExpansion
{}^{\circ}{\rm C}%
%EndExpansion
}\right) \\
&=&\allowbreak 40.\,\allowbreak 2\unit{J}
\end{eqnarray*}%
The total energy is $62.\,\allowbreak 7\unit{J}+333.0\unit{J}+419\unit{J}%
+2260.0\unit{J}+40.\,\allowbreak 2\unit{J}=\allowbreak 3114.\,\allowbreak 9%
\unit{J}$

\section{Calorimetry}

%TCIMACRO{%
%\TeXButton{Question 123.12.4}{\marginpar {
%\hspace{-0.5in}
%\begin{minipage}[t]{1in}
%\small{Question 123.12.4}
%\end{minipage}
%}}}%
%BeginExpansion
\marginpar {
\hspace{-0.5in}
\begin{minipage}[t]{1in}
\small{Question 123.12.4}
\end{minipage}
}%
%EndExpansion
%TCIMACRO{%
%\TeXButton{Question 123.12.5}{\marginpar {
%\hspace{-0.5in}
%\begin{minipage}[t]{1in}
%\small{Question 123.12.5}
%\end{minipage}
%}}}%
%BeginExpansion
\marginpar {
\hspace{-0.5in}
\begin{minipage}[t]{1in}
\small{Question 123.12.5}
\end{minipage}
}%
%EndExpansion
We can now calculate results for many common experiences. We have all
experienced eating something with a large temperature. The solution is to
drink something cold. This lowers the temperature. We would like to be able
to quantify this. But our mouth is not a great device to use to do
quantitative analysis on temperature changes. Let's consider a more suitable
apparatus.

Suppose we have an insulated container with a thermometer

\FRAME{dtbpF}{3.6659in}{1.5567in}{0pt}{}{}{Figure}{\special{language
"Scientific Word";type "GRAPHIC";maintain-aspect-ratio TRUE;display
"USEDEF";valid_file "T";width 3.6659in;height 1.5567in;depth
0pt;original-width 9.2033in;original-height 3.8925in;cropleft "0";croptop
"1";cropright "1";cropbottom "0";tempfilename
'PQXXR1ZB.wmf';tempfile-properties "XPR";}}Inside of this container let's
place a known mass of water and measure the temperature, $T_{w}$. Then we
can introduce our sample of hot material. Suppose we know that the hot
material has a temperature $T_{x}.$

Let's assume that the container walls are perfectly insulating, then knowing
the sample temperature $T_{x}$ and the water temperature $T_{w}$ with $%
T_{w}<T_{x}$ we can find $c$ or $K.$(or even $C$). We recognize for this
ideal system no energy can leave, because of the insulation. So 
\begin{equation}
Q_{\text{cold}}=-Q_{\text{hot}}
\end{equation}%
or energy would not be conserved. Then for the water%
\[
Q_{w}=m_{w}c_{w}\left( T_{f}-T_{w}\right) 
\]%
and for the sample 
\[
Q_{x}=m_{x}c_{x}\left( T_{f}-T_{x}\right) 
\]%
so, using energy conservation%
\begin{eqnarray*}
-Q_{x} &=&Q_{w} \\
-m_{x}c_{x}\left( T_{f}-T_{x}\right) &=&m_{w}c_{w}\left( T_{f}-T_{w}\right)
\end{eqnarray*}%
and we can solve for $c_{x}$%
\begin{equation}
c_{x}=-\frac{m_{w}c_{w}\left( T_{f}-T_{w}\right) }{m_{x}\left(
T_{f}-T_{x}\right) }
\end{equation}

Note the sign convention! If energy leaves the object, the value of $Q$ is
negative. Our denominator is, indeed, negative because for the sample $T_{f}$
is less than $T_{x}.$

Let's try a second problem. Suppose we have $0.5\unit{kg}$ of ice at $0\unit{%
%TCIMACRO{\U{2103}}%
%BeginExpansion
{}^{\circ}{\rm C}%
%EndExpansion
}$ and $1\unit{kg}$ of water at $50\unit{%
%TCIMACRO{\U{2103}}%
%BeginExpansion
{}^{\circ}{\rm C}%
%EndExpansion
}.$ What is the final temperature?

We will need to know $L_{f}=3.33\times 10^{5}\frac{\unit{J}}{\unit{kg}}$ and 
$c_{w}=4190\frac{\unit{J}}{\unit{kg}\unit{K}}$. Again

\[
Q_{\text{cold}}=-Q_{\text{hot}} 
\]%
we identify the hot thing as the water and the cold thing as the ice. The
water only changes temperature, but the ice experiences a phase change 
\begin{eqnarray*}
Q_{\text{ice}} &=&-Q_{\text{water}} \\
L_{f}M+m_{i}c_{w}\left( T_{f}-T_{0}\right) &=&-m_{w}c_{w}\left(
T_{f}-T_{w}\right)
\end{eqnarray*}%
The final temperature will be the same for both%
\[
L_{f}M+m_{i}c_{w}T_{f}-m_{i}c_{w}T_{0}=-m_{w}c_{w}T_{f}+m_{w}c_{w}T_{w} 
\]%
\[
L_{f}M-m_{w}c_{i}T_{0}-m_{w}c_{w}T_{w}=-m_{w}c_{w}T_{f}-m_{i}c_{w}T_{f} 
\]%
\[
L_{f}M-\left( m_{i}c_{w}T_{0}+m_{w}c_{w}T_{w}\right) =-\left(
m_{w}c_{w}+m_{i}c_{w}\right) T_{f} 
\]%
\[
T_{f}=\frac{\left( m_{i}c_{w}T_{0}+m_{w}c_{w}T_{w}\right) -L_{f}M}{\left(
m_{w}c_{w}+m_{w}c_{w}\right) } 
\]%
or$\allowbreak $%
\begin{eqnarray*}
T_{f} &=&\frac{\left( \left( 1\unit{kg}\right) \left( 4190\frac{\unit{J}}{%
\unit{kg}\unit{K}}\right) \left( 273\unit{K}\right) +\left( 1\unit{kg}%
\right) \left( 4190\frac{\unit{J}}{\unit{kg}\unit{K}}\right) \left(
273+50\right) \unit{K}\right) -\left( 3.33\times 10^{5}\frac{\unit{J}}{\unit{%
kg}}\right) \left( 1\unit{kg}\right) }{\left( \left( 1\unit{kg}\right)
\left( 4190\frac{\unit{J}}{\unit{kg}\unit{K}}\right) +\left( 1\unit{kg}%
\right) \left( 4190\frac{\unit{J}}{\unit{kg}\unit{K}}\right) \right) } \\
&=&258.\,\allowbreak 26\unit{K}
\end{eqnarray*}%
Now let's ask, is this reasonable?

The answer is no! Let's see why. We should have checked in advance to see if
all the ice melts. It would take 
\[
Q=m_{i}L_{f}=\left( 3.33\times 10^{5}\frac{\unit{J}}{\unit{kg}}\right)
\left( 0.5\unit{kg}\right) =1.\,\allowbreak 665\times 10^{5}\unit{J} 
\]
to melt all the ice. We have available from the warm water 
\begin{eqnarray*}
Q &=&m_{w}C_{w}\left( T_{f}-T_{w}\right) \\
&=&\left( 1\unit{kg}\right) \left( 4190\frac{\unit{J}}{\unit{kg}\unit{K}}%
\right) \left( 273\unit{K}-323\unit{K}\right) \\
&=&\allowbreak -209\,500\unit{J}
\end{eqnarray*}%
that is, the warmer water can provide $209.5\unit{J}.$ But since $209.5\unit{%
J}<1.\,\allowbreak 665\times 10^{5}\unit{J}$ the warm water does not have
enough energy available to melt all the ice. Once the water is all at $0%
\unit{%
%TCIMACRO{\U{2103}}%
%BeginExpansion
{}^{\circ}{\rm C}%
%EndExpansion
},$ the ice and water are in thermal equilibrium. No energy will be
transferred. So our final temperature of the mixture is $273\unit{K}$ or $0%
\unit{%
%TCIMACRO{\U{2103}}%
%BeginExpansion
{}^{\circ}{\rm C}%
%EndExpansion
}.$

Of course if our system of ice and water isn't isolated, eventually all the
ice will melt. But that would be due to a transfer of energy by heat from
the outside environment.

This problem is a bit of a trick, but such a situation can really happen. So
it is really important that we stop to consider whether our numerical
answers are reasonable once we have done a calorimetry problem.

We have filled in the missing piece of information for solids and liquids.
We can say that so long as we don't change the structure of the material, 
\[
Q=Mc\Delta T 
\]%
or 
\[
Q=nC\Delta T 
\]%
so we can find $Q$, $w$, and $\Delta E_{int}$ for solids and liquids But we
said gasses will be more difficult. We will take up the relationship between
energy transfer by heat and the change in temperature for gasses in our next
lecture.

\chapter{Specific Heat of Gases and Heat Transfer Mechanisms}

Last lecture we found a way to find $Q$ for solids and liquids, $Q=Mc\Delta
T $ or $Q=nC\Delta T.$ We need to do this for gasses so we can complete our
calculations of the terms in the first law of thermodynamics. We will take
on this task in this lecture.

%TCIMACRO{%
%\TeXButton{Fundamental Concepts}{\hspace{-1.3in}{\Large Fundamental Concepts\vspace{0.25in}}}}%
%BeginExpansion
\hspace{-1.3in}{\Large Fundamental Concepts\vspace{0.25in}}%
%EndExpansion

\begin{enumerate}
\item There are two molar specific heats for gasses, $C_{V}$ and $C_{P}.$

\item The two molar specific heats are related $C_{P}=C_{V}+R$

\item The ratio of the molar specific heats is $\gamma =C_{P}/C_{V}$

\item The change in internal energy can be expressed as $\Delta
E_{int}=nC_{V}\Delta T$

\item For adiabatic processes we have two new equations $P_{i}%
%TCIMACRO{\TeXButton{V}{\ooalign{\hfil$V$\hfil\cr\kern0.1em--\hfil\cr}}}%
%BeginExpansion
\ooalign{\hfil$V$\hfil\cr\kern0.1em--\hfil\cr}%
%EndExpansion
_{i}^{\gamma }=P_{f}%
%TCIMACRO{\TeXButton{V}{\ooalign{\hfil$V$\hfil\cr\kern0.1em--\hfil\cr}}}%
%BeginExpansion
\ooalign{\hfil$V$\hfil\cr\kern0.1em--\hfil\cr}%
%EndExpansion
_{f}^{\gamma }$ and $T_{f}%
%TCIMACRO{\TeXButton{V}{\ooalign{\hfil$V$\hfil\cr\kern0.1em--\hfil\cr}}}%
%BeginExpansion
\ooalign{\hfil$V$\hfil\cr\kern0.1em--\hfil\cr}%
%EndExpansion
_{f}^{\gamma -1}=T_{i}%
%TCIMACRO{\TeXButton{V}{\ooalign{\hfil$V$\hfil\cr\kern0.1em--\hfil\cr}}}%
%BeginExpansion
\ooalign{\hfil$V$\hfil\cr\kern0.1em--\hfil\cr}%
%EndExpansion
_{i}^{\gamma -1}$

\item For insolation the power transferred through the insolation material
is given by $\mathcal{P=}\frac{Q}{\Delta t}=\frac{A\left( T_{h}-T_{c}\right) 
}{\dsum\limits_{i}R_{i}}$

\item The power transferred by radiation is given by $\mathcal{P=}\frac{Q}{%
\Delta t}=\sigma AeT^{4}$
\end{enumerate}

\section{Molar Specific Heat of an Ideal Gas}

%TCIMACRO{%
%\TeXButton{Question 123.13.1}{\marginpar {
%\hspace{-0.5in}
%\begin{minipage}[t]{1in}
%\small{Question 123.13.1}
%\end{minipage}
%}}}%
%BeginExpansion
\marginpar {
\hspace{-0.5in}
\begin{minipage}[t]{1in}
\small{Question 123.13.1}
\end{minipage}
}%
%EndExpansion
Remember we said that we would not find gases in specific heat tables? We
did find steam. But other gases were missing. The reason is that there is
not one unique value for the energy transfer by heat associated with a
single temperature change of an ideal gas. We know that $Q$ is path
dependent for gasses! But we can still use our formula if we make some
limiting assumptions. We will limit ourselves to either isobaric or
isovolumetric processes. For these specific cases we can write

\begin{equation}
Q=nC_{V}\Delta T\qquad \text{isovolumetric}
\end{equation}

\begin{equation}
Q=nC_{P}\Delta T\qquad \text{isobaric}
\end{equation}

Note that these are molar specific heats. The quantity $n$ is the number of
moles, not the mass. We won't define a regular specific heat for gasses.

This may seem very limiting, but think of our goal. We want to be able to
consider $P,$ $%
%TCIMACRO{\TeXButton{V}{\ooalign{\hfil$V$\hfil\cr\kern0.1em--\hfil\cr}}}%
%BeginExpansion
\ooalign{\hfil$V$\hfil\cr\kern0.1em--\hfil\cr}%
%EndExpansion
,$ $n,$ $T,$ $\Delta E_{int}$ and $Q$ and $W$ for each of our specific
processes. These equations fill in the gap for two of those processes!

\subsection{ Molar Specific Heat for Constant pressure}

Let's take an example. Think again of adding energy by heat to a system at
constant pressure. Suppose we know $P,$ $n,$ and $%
%TCIMACRO{\TeXButton{V}{\ooalign{\hfil$V$\hfil\cr\kern0.1em--\hfil\cr}}}%
%BeginExpansion
\ooalign{\hfil$V$\hfil\cr\kern0.1em--\hfil\cr}%
%EndExpansion
_{i}$ and $T_{i}$ for the initial state, and $%
%TCIMACRO{\TeXButton{V}{\ooalign{\hfil$V$\hfil\cr\kern0.1em--\hfil\cr}}}%
%BeginExpansion
\ooalign{\hfil$V$\hfil\cr\kern0.1em--\hfil\cr}%
%EndExpansion
_{f}$ and $T_{f}$ for the final state. We want to know $\Delta E_{int}$, $Q,$
and $w$ to finish our description of the process. \FRAME{dtbpF}{1.907in}{%
2.7691in}{0pt}{}{}{Figure}{\special{language "Scientific Word";type
"GRAPHIC";maintain-aspect-ratio TRUE;display "USEDEF";valid_file "T";width
1.907in;height 2.7691in;depth 0pt;original-width 1.915in;original-height
2.7931in;cropleft "0";croptop "1";cropright "1";cropbottom "0";tempfilename
'PQXXR1ZC.wmf';tempfile-properties "XPR";}}We now know that 
\[
Q=nC_{P}\Delta T 
\]%
and we also know the work because we have calculated work before for an
isobaric process 
\[
w=-P\left( 
%TCIMACRO{\TeXButton{V}{\ooalign{\hfil$V$\hfil\cr\kern0.1em--\hfil\cr}}}%
%BeginExpansion
\ooalign{\hfil$V$\hfil\cr\kern0.1em--\hfil\cr}%
%EndExpansion
_{f}-%
%TCIMACRO{\TeXButton{V}{\ooalign{\hfil$V$\hfil\cr\kern0.1em--\hfil\cr}}}%
%BeginExpansion
\ooalign{\hfil$V$\hfil\cr\kern0.1em--\hfil\cr}%
%EndExpansion
_{i}\right) 
\]%
so from the first law%
\[
\Delta E_{int}=nC_{P}\Delta T-P\left( 
%TCIMACRO{\TeXButton{V}{\ooalign{\hfil$V$\hfil\cr\kern0.1em--\hfil\cr}}}%
%BeginExpansion
\ooalign{\hfil$V$\hfil\cr\kern0.1em--\hfil\cr}%
%EndExpansion
_{f}-%
%TCIMACRO{\TeXButton{V}{\ooalign{\hfil$V$\hfil\cr\kern0.1em--\hfil\cr}}}%
%BeginExpansion
\ooalign{\hfil$V$\hfil\cr\kern0.1em--\hfil\cr}%
%EndExpansion
_{i}\right) 
\]%
and we have achieved our goal! We know $Q$, $w,$ and $\Delta E_{int}.$

\subsection{C$_{V}$ for a monotonic gas}

Let's try another example, this time for an isochoric process. So this time
we know $n,$ and $%
%TCIMACRO{\TeXButton{V}{\ooalign{\hfil$V$\hfil\cr\kern0.1em--\hfil\cr}}}%
%BeginExpansion
\ooalign{\hfil$V$\hfil\cr\kern0.1em--\hfil\cr}%
%EndExpansion
$ and $P_{i},$ and $T_{i}$ for the initial state and $P_{f}$ and $T_{f}$ for
the final state. We wish to calculate $Q$, $w,$ and $\Delta E_{int}$ for
this process. We have calculated the work for an isochoric state before!

\[
W=0 
\]%
so from the first law%
\[
\Delta E_{int}=Q 
\]%
now we can add%
\[
\Delta E_{int}=nC_{V}\Delta T 
\]

Again we have achieved our goal. We have found $Q$, $w,$ and $\Delta E_{int}$%
.

\section{Relationship between $C_{V}$ and $C_{P}$}

%TCIMACRO{%
%\TeXButton{Question 123.13.2}{\marginpar {
%\hspace{-0.5in}
%\begin{minipage}[t]{1in}
%\small{Question 123.13.2}
%\end{minipage}
%}}}%
%BeginExpansion
\marginpar {
\hspace{-0.5in}
\begin{minipage}[t]{1in}
\small{Question 123.13.2}
\end{minipage}
}%
%EndExpansion
We noticed before that 
\[
\Delta E_{int}\propto \Delta T 
\]%
This is really a key relationship in thermodynamics. We have not found the
constant of proportionality yet (we will soon!). But this has a profound
meaning that we can begin to use here. In the next figure, we have two
isotherms (blue lines if you are seeing this in color)\FRAME{dtbpF}{3.618in}{%
2.4268in}{0pt}{}{}{Figure}{\special{language "Scientific Word";type
"GRAPHIC";maintain-aspect-ratio TRUE;display "USEDEF";valid_file "T";width
3.618in;height 2.4268in;depth 0pt;original-width 3.6579in;original-height
2.4445in;cropleft "0";croptop "1";cropright "1";cropbottom "0";tempfilename
'PQXXR1ZD.wmf';tempfile-properties "XPR";}}There is also an isovolumetric
path marked (red) and an isobaric path (green) Notice that these two paths
take the systems between the same temperature change. $\Delta T.$ Then we
expect they will have the same internal energy change, $\Delta E_{int}.$ For
the isobaric path, we find that 
\begin{eqnarray*}
Q_{P} &=&\Delta E_{int}+P\Delta 
%TCIMACRO{\TeXButton{V}{\ooalign{\hfil$V$\hfil\cr\kern0.1em--\hfil\cr}} }%
%BeginExpansion
\ooalign{\hfil$V$\hfil\cr\kern0.1em--\hfil\cr}
%EndExpansion
\\
&=&\Delta E_{int}+P\left( 
%TCIMACRO{\TeXButton{V}{\ooalign{\hfil$V$\hfil\cr\kern0.1em--\hfil\cr}}}%
%BeginExpansion
\ooalign{\hfil$V$\hfil\cr\kern0.1em--\hfil\cr}%
%EndExpansion
_{f}-%
%TCIMACRO{\TeXButton{V}{\ooalign{\hfil$V$\hfil\cr\kern0.1em--\hfil\cr}}}%
%BeginExpansion
\ooalign{\hfil$V$\hfil\cr\kern0.1em--\hfil\cr}%
%EndExpansion
_{i}\right)
\end{eqnarray*}%
For the isochoric path we have just 
\begin{eqnarray*}
Q_{V} &=&\Delta E_{int} \\
&=&\Delta E_{int}
\end{eqnarray*}%
So the energy transfer by heat must supply both the internal energy change
and the loss due to work for the isobaric path but only needs to supply the
change in internal energy for the isochoric path.. Therefore, for the same $%
n $ and $\Delta T$ 
\[
Q_{\text{constant }P}>Q_{\text{constant }V} 
\]%
Since 
\[
\Delta E_{int}=nC_{V}\Delta T 
\]%
for the isochoric path, and $\Delta E_{int}$ is the same for both paths,
then we can set our two equations for $\Delta E_{int}$ equal to each other
and write 
\[
nC_{P}\Delta T=nC_{V}\Delta T+P\left( 
%TCIMACRO{\TeXButton{V}{\ooalign{\hfil$V$\hfil\cr\kern0.1em--\hfil\cr}}}%
%BeginExpansion
\ooalign{\hfil$V$\hfil\cr\kern0.1em--\hfil\cr}%
%EndExpansion
_{f}-%
%TCIMACRO{\TeXButton{V}{\ooalign{\hfil$V$\hfil\cr\kern0.1em--\hfil\cr}}}%
%BeginExpansion
\ooalign{\hfil$V$\hfil\cr\kern0.1em--\hfil\cr}%
%EndExpansion
_{i}\right) 
\]

which gives 
\[
C_{P}=C_{V}+\frac{P\left( 
%TCIMACRO{\TeXButton{V}{\ooalign{\hfil$V$\hfil\cr\kern0.1em--\hfil\cr}}}%
%BeginExpansion
\ooalign{\hfil$V$\hfil\cr\kern0.1em--\hfil\cr}%
%EndExpansion
_{f}-%
%TCIMACRO{\TeXButton{V}{\ooalign{\hfil$V$\hfil\cr\kern0.1em--\hfil\cr}}}%
%BeginExpansion
\ooalign{\hfil$V$\hfil\cr\kern0.1em--\hfil\cr}%
%EndExpansion
_{i}\right) }{n\Delta T} 
\]

and therefore%
\[
C_{P}>C_{V} 
\]

We can go further, since, for an idea gas, 
\[
P%
%TCIMACRO{\TeXButton{V}{\ooalign{\hfil$V$\hfil\cr\kern0.1em--\hfil\cr}}}%
%BeginExpansion
\ooalign{\hfil$V$\hfil\cr\kern0.1em--\hfil\cr}%
%EndExpansion
=nRT 
\]%
$\allowbreak $

then

\[
P\Delta 
%TCIMACRO{\TeXButton{V}{\ooalign{\hfil$V$\hfil\cr\kern0.1em--\hfil\cr}}}%
%BeginExpansion
\ooalign{\hfil$V$\hfil\cr\kern0.1em--\hfil\cr}%
%EndExpansion
=nR\Delta T 
\]%
so%
\begin{eqnarray*}
C_{P} &=&C_{V}+\frac{P\Delta 
%TCIMACRO{\TeXButton{V}{\ooalign{\hfil$V$\hfil\cr\kern0.1em--\hfil\cr}}}%
%BeginExpansion
\ooalign{\hfil$V$\hfil\cr\kern0.1em--\hfil\cr}%
%EndExpansion
}{n\Delta T} \\
&=&C_{V}+\frac{nR\Delta T}{n\Delta T} \\
&=&C_{V}+R
\end{eqnarray*}%
This is a big deal! 
\begin{equation}
C_{P}=C_{V}+R
\end{equation}%
for an ideal gas. So if we know either $C_{P}$ or $C_{V}$ for a gas, we can
always find the other. This works pretty well in practice for monotonic
gasses (and even for diatomic gasses) at standard temperature and pressure.
Not only can we find $Q_{V}$ and $Q_{P}$ with molar specific heats, but if
we know one molar specific heat, we can easily calculate the other!

\section{Isothermal process and $Q_{V}$}

We now have a powerful new equation to add to our isovolumetric and isobaric
processes. But what about an isothermal process?\FRAME{dtbpF}{3.8193in}{%
2.5359in}{0pt}{}{}{Figure}{\special{language "Scientific Word";type
"GRAPHIC";maintain-aspect-ratio TRUE;display "USEDEF";valid_file "T";width
3.8193in;height 2.5359in;depth 0pt;original-width 3.8637in;original-height
2.5554in;cropleft "0";croptop "1";cropright "1";cropbottom "0";tempfilename
'PQXXR1ZE.wmf';tempfile-properties "XPR";}}

Suppose we know $T$ and $n,$ and that we know $P_{i}$ and $%
%TCIMACRO{\TeXButton{V}{\ooalign{\hfil$V$\hfil\cr\kern0.1em--\hfil\cr}}}%
%BeginExpansion
\ooalign{\hfil$V$\hfil\cr\kern0.1em--\hfil\cr}%
%EndExpansion
_{i}$ and $P_{f}$ and $%
%TCIMACRO{\TeXButton{V}{\ooalign{\hfil$V$\hfil\cr\kern0.1em--\hfil\cr}}}%
%BeginExpansion
\ooalign{\hfil$V$\hfil\cr\kern0.1em--\hfil\cr}%
%EndExpansion
_{f}.$ Can we find $\Delta E_{int,}$ $Q$, and $w$?

We already know how to find $w$%
\[
w=-nRT\left[ \ln \frac{%
%TCIMACRO{\TeXButton{V}{\ooalign{\hfil$V$\hfil\cr\kern0.1em--\hfil\cr}}}%
%BeginExpansion
\ooalign{\hfil$V$\hfil\cr\kern0.1em--\hfil\cr}%
%EndExpansion
_{f}}{%
%TCIMACRO{\TeXButton{V}{\ooalign{\hfil$V$\hfil\cr\kern0.1em--\hfil\cr}}}%
%BeginExpansion
\ooalign{\hfil$V$\hfil\cr\kern0.1em--\hfil\cr}%
%EndExpansion
_{i}}\right] 
\]%
and we know $\Delta E_{int}=0,$ so 
\begin{eqnarray*}
Q &=&-w \\
&=&nRT\left[ \ln \frac{%
%TCIMACRO{\TeXButton{V}{\ooalign{\hfil$V$\hfil\cr\kern0.1em--\hfil\cr}}}%
%BeginExpansion
\ooalign{\hfil$V$\hfil\cr\kern0.1em--\hfil\cr}%
%EndExpansion
_{f}}{%
%TCIMACRO{\TeXButton{V}{\ooalign{\hfil$V$\hfil\cr\kern0.1em--\hfil\cr}}}%
%BeginExpansion
\ooalign{\hfil$V$\hfil\cr\kern0.1em--\hfil\cr}%
%EndExpansion
_{i}}\right]
\end{eqnarray*}%
and we did not need our molar specific heat equations! We have $\Delta
E_{int,}$ $Q$, and $w$ for three of our special processes. This completes
three of our four special processes!

\section{Adiabatic Processes for and Ideal Gas}

With a complete set of equations for three of our four special processes, we
should now review our new special process, the adiabatic process. It would
be nice to be able to calculate $\Delta E_{int},$ $Q,$ and $w.$ for this
process. For and adiabatic process we have

\begin{enumerate}
\item No heat transfer $Q=0$

\item $\Delta E_{int}=w$
\end{enumerate}

It might see like we are home free, $Q=0$! We don't need our molar specific
heat equations for an adiabatic process either.

But we don't yet know how to find $w$ or $\Delta E_{int}$ for an adiabatic
process. We need to fix this problem. We could do this by finding an
expression for the pressure as a function of volume along the adiabatic
path, then using 
\[
w=-\int_{%
%TCIMACRO{\TeXButton{V}{\ooalign{\hfil$V$\hfil\cr\kern0.1em--\hfil\cr}}}%
%BeginExpansion
\ooalign{\hfil$V$\hfil\cr\kern0.1em--\hfil\cr}%
%EndExpansion
_{i}}^{%
%TCIMACRO{\TeXButton{V}{\ooalign{\hfil$V$\hfil\cr\kern0.1em--\hfil\cr}}}%
%BeginExpansion
\ooalign{\hfil$V$\hfil\cr\kern0.1em--\hfil\cr}%
%EndExpansion
_{f}}Pd%
%TCIMACRO{\TeXButton{V}{\ooalign{\hfil$V$\hfil\cr\kern0.1em--\hfil\cr}}}%
%BeginExpansion
\ooalign{\hfil$V$\hfil\cr\kern0.1em--\hfil\cr}%
%EndExpansion
\]

Since%
\begin{eqnarray*}
\Delta E_{int} &=&w \\
&=&-\int Pd%
%TCIMACRO{\TeXButton{V}{\ooalign{\hfil$V$\hfil\cr\kern0.1em--\hfil\cr}}}%
%BeginExpansion
\ooalign{\hfil$V$\hfil\cr\kern0.1em--\hfil\cr}%
%EndExpansion
\end{eqnarray*}%
this would give us both $w$ and $\Delta E_{int}$ . But it is not obvious how
to find $P$ as a function of $%
%TCIMACRO{\TeXButton{V}{\ooalign{\hfil$V$\hfil\cr\kern0.1em--\hfil\cr}}}%
%BeginExpansion
\ooalign{\hfil$V$\hfil\cr\kern0.1em--\hfil\cr}%
%EndExpansion
$ for this path. We need another strategy.

Let's recall that an adiabatic process goes from one temperature to another
temperature, and 
\[
\Delta E_{int}\varpropto \Delta T 
\]%
\FRAME{dtbpFX}{2.2087in}{1.4719in}{0pt}{}{}{Plot}{\special{language
"Scientific Word";type "MAPLEPLOT";width 2.2087in;height 1.4719in;depth
0pt;display "USEDEF";plot_snapshots TRUE;mustRecompute FALSE;lastEngine
"MuPAD";animated TRUE;animationStartTime "0"; animationEndTime "10";
animationFramesPerSecond "10";xmin "0";xmax "100";animationParamMin
"290";animationParamMax "350";xviewmin "0";xviewmax "40";yviewmin
"0";yviewmax "400";viewset"XY";rangeset"X";plottype 4;labeloverrides
3;x-label "V(m^3)";y-label "P(Pa)";axesFont "Times New
Roman,12,0000000000,useDefault,normal";numpoints 100;plotstyle
"patch";axesstyle "normal";axestips FALSE;xis \TEXUX{V};animationParam
\TEXUX{t};var1name \TEXUX{$V$};animationParamList \TEXUX{$t$};function
\TEXUX{$\allowbreak 7500\frac{1}{V^{1.67}}$};linecolor "blue";linestyle
1;pointstyle "point";linethickness 3;lineAttributes "Solid";var1range
"0,100";animationParamRange "290,350";num-x-gridlines 50;curveColor
"[flat::RGB:0x000000ff]";curveStyle "Line";animationStartTime "0";
animationEndTime "10"; animationFramesPerSecond
"10";animationVisibleBeforeStart FALSE;rangeset"XA";function
\TEXUX{$\allowbreak 8.\,\allowbreak 314\frac{1}{V}290$};linecolor
"red";linestyle 2;pointstyle "point";linethickness 1;lineAttributes
"Dash";var1range "0,40";animationParamRange "290,350";num-x-gridlines
100;curveColor "[flat::RGB:0x00ff0000]";curveStyle "Line";function
\TEXUX{$8.\,\allowbreak 314\frac{1}{V}100$};linecolor "red";linestyle
2;pointstyle "point";linethickness 1;lineAttributes "Dash";var1range
"0,40";animationParamRange "290,350";num-x-gridlines 100;curveColor
"[flat::RGB:0x00ff0000]";curveStyle "Line";VCamFile
'PU11AB00.xvz';valid_file "T";tempfilename
'PQXXR1ZF.wmf';tempfile-properties "XPR";}}And further remember that $\Delta
E_{int}$ does not depend on path. So we could find $\Delta E_{int}$ for an
adiabatic path, and we could find $\Delta E_{int}$ for a isochoric path, and
as long as the $\Delta T$ was the same for both paths, the $\Delta E_{int}$
would be the same as well! Also remember that for an isochoric path there is
no work done. So 
\[
Q_{isochoric}=\Delta E_{int} 
\]%
And now we know an equation for $Q_{isochoric}$!%
\[
Q_{isochoric}=nC_{V}\Delta T 
\]%
So if we know $T_{i}$ and $T_{f},$ then we could find $\Delta E_{int}$ by
finding the equivalent $Q_{isochoric}$ for the same two temperatures.%
\[
\Delta E_{int}=nC_{V}\Delta T 
\]%
This is tremendous! We have an equation for $\Delta E_{int}$, and $w=\Delta
E_{int},$ so with $Q=0$ we have a complete set.

But it turns out that we can add two more equations to our adiabatic set
that will really help. The first is a relationship between pressures and
volumes along an adiabatic path. Let's obtain this equation.

We now know for adiabatic processes we can say 
\[
\Delta E_{int}=nC_{V}\Delta T 
\]%
A very small change in the internal energy would be 
\[
dE=nC_{V}dT 
\]%
Now for our adiabatic we previously noted that there is not energy transfer
by heat done, so 
\begin{eqnarray*}
\Delta E_{int} &=&0+w \\
&=&\int Pd%
%TCIMACRO{\TeXButton{V}{\ooalign{\hfil$V$\hfil\cr\kern0.1em--\hfil\cr}}}%
%BeginExpansion
\ooalign{\hfil$V$\hfil\cr\kern0.1em--\hfil\cr}%
%EndExpansion
\end{eqnarray*}%
Then a very small amount of internal energy change, $dE_{int}$ would be 
\[
dE_{int}=-Pd%
%TCIMACRO{\TeXButton{V}{\ooalign{\hfil$V$\hfil\cr\kern0.1em--\hfil\cr}}}%
%BeginExpansion
\ooalign{\hfil$V$\hfil\cr\kern0.1em--\hfil\cr}%
%EndExpansion
\]%
Let's set these two expressions for $dE_{int}$ equal to each other. 
\[
-Pd%
%TCIMACRO{\TeXButton{V}{\ooalign{\hfil$V$\hfil\cr\kern0.1em--\hfil\cr}}}%
%BeginExpansion
\ooalign{\hfil$V$\hfil\cr\kern0.1em--\hfil\cr}%
%EndExpansion
=nC_{V}dT 
\]%
or%
\[
-\frac{Pd%
%TCIMACRO{\TeXButton{V}{\ooalign{\hfil$V$\hfil\cr\kern0.1em--\hfil\cr}}}%
%BeginExpansion
\ooalign{\hfil$V$\hfil\cr\kern0.1em--\hfil\cr}%
%EndExpansion
}{nC_{V}}=dT 
\]

Now we know the idea gas law%
\[
P%
%TCIMACRO{\TeXButton{V}{\ooalign{\hfil$V$\hfil\cr\kern0.1em--\hfil\cr}}}%
%BeginExpansion
\ooalign{\hfil$V$\hfil\cr\kern0.1em--\hfil\cr}%
%EndExpansion
=nRT 
\]%
if we take a total differential for a sample of $n$ moles of a gas%
\[
Pd%
%TCIMACRO{\TeXButton{V}{\ooalign{\hfil$V$\hfil\cr\kern0.1em--\hfil\cr}}}%
%BeginExpansion
\ooalign{\hfil$V$\hfil\cr\kern0.1em--\hfil\cr}%
%EndExpansion
+%
%TCIMACRO{\TeXButton{V}{\ooalign{\hfil$V$\hfil\cr\kern0.1em--\hfil\cr}}}%
%BeginExpansion
\ooalign{\hfil$V$\hfil\cr\kern0.1em--\hfil\cr}%
%EndExpansion
dP=nRdT 
\]%
we can substitute in our expression we have for an adiabatic process for $dT$%
\[
Pd%
%TCIMACRO{\TeXButton{V}{\ooalign{\hfil$V$\hfil\cr\kern0.1em--\hfil\cr}}}%
%BeginExpansion
\ooalign{\hfil$V$\hfil\cr\kern0.1em--\hfil\cr}%
%EndExpansion
+%
%TCIMACRO{\TeXButton{V}{\ooalign{\hfil$V$\hfil\cr\kern0.1em--\hfil\cr}}}%
%BeginExpansion
\ooalign{\hfil$V$\hfil\cr\kern0.1em--\hfil\cr}%
%EndExpansion
dP=-nR\frac{Pd%
%TCIMACRO{\TeXButton{V}{\ooalign{\hfil$V$\hfil\cr\kern0.1em--\hfil\cr}}}%
%BeginExpansion
\ooalign{\hfil$V$\hfil\cr\kern0.1em--\hfil\cr}%
%EndExpansion
}{nC_{V}} 
\]%
or%
\[
Pd%
%TCIMACRO{\TeXButton{V}{\ooalign{\hfil$V$\hfil\cr\kern0.1em--\hfil\cr}}}%
%BeginExpansion
\ooalign{\hfil$V$\hfil\cr\kern0.1em--\hfil\cr}%
%EndExpansion
+%
%TCIMACRO{\TeXButton{V}{\ooalign{\hfil$V$\hfil\cr\kern0.1em--\hfil\cr}}}%
%BeginExpansion
\ooalign{\hfil$V$\hfil\cr\kern0.1em--\hfil\cr}%
%EndExpansion
dP=-R\frac{Pd%
%TCIMACRO{\TeXButton{V}{\ooalign{\hfil$V$\hfil\cr\kern0.1em--\hfil\cr}}}%
%BeginExpansion
\ooalign{\hfil$V$\hfil\cr\kern0.1em--\hfil\cr}%
%EndExpansion
}{C_{V}} 
\]

We can use our relationship between $C_{V}$ and $C_{P}$ that we found in the
last section%
\[
C_{V}=C_{P}-R 
\]%
so%
\[
R=C_{P}-C_{V} 
\]%
and let's substitute this in for our $R$ in our total differential equation%
\[
Pd%
%TCIMACRO{\TeXButton{V}{\ooalign{\hfil$V$\hfil\cr\kern0.1em--\hfil\cr}}}%
%BeginExpansion
\ooalign{\hfil$V$\hfil\cr\kern0.1em--\hfil\cr}%
%EndExpansion
+%
%TCIMACRO{\TeXButton{V}{\ooalign{\hfil$V$\hfil\cr\kern0.1em--\hfil\cr}}}%
%BeginExpansion
\ooalign{\hfil$V$\hfil\cr\kern0.1em--\hfil\cr}%
%EndExpansion
dP=-\frac{C_{P}-C_{V}}{C_{V}}Pd%
%TCIMACRO{\TeXButton{V}{\ooalign{\hfil$V$\hfil\cr\kern0.1em--\hfil\cr}}}%
%BeginExpansion
\ooalign{\hfil$V$\hfil\cr\kern0.1em--\hfil\cr}%
%EndExpansion
\]%
or%
\[
Pd%
%TCIMACRO{\TeXButton{V}{\ooalign{\hfil$V$\hfil\cr\kern0.1em--\hfil\cr}}}%
%BeginExpansion
\ooalign{\hfil$V$\hfil\cr\kern0.1em--\hfil\cr}%
%EndExpansion
+%
%TCIMACRO{\TeXButton{V}{\ooalign{\hfil$V$\hfil\cr\kern0.1em--\hfil\cr}}}%
%BeginExpansion
\ooalign{\hfil$V$\hfil\cr\kern0.1em--\hfil\cr}%
%EndExpansion
dP=\frac{C_{V}-C_{P}}{C_{V}}Pd%
%TCIMACRO{\TeXButton{V}{\ooalign{\hfil$V$\hfil\cr\kern0.1em--\hfil\cr}}}%
%BeginExpansion
\ooalign{\hfil$V$\hfil\cr\kern0.1em--\hfil\cr}%
%EndExpansion
\]%
Now let's divide through by $P%
%TCIMACRO{\TeXButton{V}{\ooalign{\hfil$V$\hfil\cr\kern0.1em--\hfil\cr}}}%
%BeginExpansion
\ooalign{\hfil$V$\hfil\cr\kern0.1em--\hfil\cr}%
%EndExpansion
.$ It's not apparent that this will help, but it does so 
\[
\frac{Pd%
%TCIMACRO{\TeXButton{V}{\ooalign{\hfil$V$\hfil\cr\kern0.1em--\hfil\cr}}}%
%BeginExpansion
\ooalign{\hfil$V$\hfil\cr\kern0.1em--\hfil\cr}%
%EndExpansion
}{P%
%TCIMACRO{\TeXButton{V}{\ooalign{\hfil$V$\hfil\cr\kern0.1em--\hfil\cr}}}%
%BeginExpansion
\ooalign{\hfil$V$\hfil\cr\kern0.1em--\hfil\cr}%
%EndExpansion
}+\frac{%
%TCIMACRO{\TeXButton{V}{\ooalign{\hfil$V$\hfil\cr\kern0.1em--\hfil\cr}}}%
%BeginExpansion
\ooalign{\hfil$V$\hfil\cr\kern0.1em--\hfil\cr}%
%EndExpansion
dP}{P%
%TCIMACRO{\TeXButton{V}{\ooalign{\hfil$V$\hfil\cr\kern0.1em--\hfil\cr}}}%
%BeginExpansion
\ooalign{\hfil$V$\hfil\cr\kern0.1em--\hfil\cr}%
%EndExpansion
}=\frac{C_{V}-C_{P}}{C_{V}}\frac{Pd%
%TCIMACRO{\TeXButton{V}{\ooalign{\hfil$V$\hfil\cr\kern0.1em--\hfil\cr}}}%
%BeginExpansion
\ooalign{\hfil$V$\hfil\cr\kern0.1em--\hfil\cr}%
%EndExpansion
}{P%
%TCIMACRO{\TeXButton{V}{\ooalign{\hfil$V$\hfil\cr\kern0.1em--\hfil\cr}}}%
%BeginExpansion
\ooalign{\hfil$V$\hfil\cr\kern0.1em--\hfil\cr}%
%EndExpansion
} 
\]%
and canceling the extra $V$'s and $P$'s gives%
\[
\frac{d%
%TCIMACRO{\TeXButton{V}{\ooalign{\hfil$V$\hfil\cr\kern0.1em--\hfil\cr}}}%
%BeginExpansion
\ooalign{\hfil$V$\hfil\cr\kern0.1em--\hfil\cr}%
%EndExpansion
}{%
%TCIMACRO{\TeXButton{V}{\ooalign{\hfil$V$\hfil\cr\kern0.1em--\hfil\cr}}}%
%BeginExpansion
\ooalign{\hfil$V$\hfil\cr\kern0.1em--\hfil\cr}%
%EndExpansion
}+\frac{dP}{P}=\frac{C_{V}-C_{P}}{C_{V}}\frac{d%
%TCIMACRO{\TeXButton{V}{\ooalign{\hfil$V$\hfil\cr\kern0.1em--\hfil\cr}}}%
%BeginExpansion
\ooalign{\hfil$V$\hfil\cr\kern0.1em--\hfil\cr}%
%EndExpansion
}{%
%TCIMACRO{\TeXButton{V}{\ooalign{\hfil$V$\hfil\cr\kern0.1em--\hfil\cr}}}%
%BeginExpansion
\ooalign{\hfil$V$\hfil\cr\kern0.1em--\hfil\cr}%
%EndExpansion
} 
\]%
rearranging terms gives%
\begin{eqnarray*}
\frac{dP}{P} &=&\left( \frac{C_{V}-C_{P}}{C_{V}}-1\right) \frac{d%
%TCIMACRO{\TeXButton{V}{\ooalign{\hfil$V$\hfil\cr\kern0.1em--\hfil\cr}}}%
%BeginExpansion
\ooalign{\hfil$V$\hfil\cr\kern0.1em--\hfil\cr}%
%EndExpansion
}{%
%TCIMACRO{\TeXButton{V}{\ooalign{\hfil$V$\hfil\cr\kern0.1em--\hfil\cr}}}%
%BeginExpansion
\ooalign{\hfil$V$\hfil\cr\kern0.1em--\hfil\cr}%
%EndExpansion
} \\
\frac{dP}{P} &=&\left( \frac{C_{V}-C_{P}}{C_{V}}-\frac{C_{V}}{C_{V}}\right) 
\frac{d%
%TCIMACRO{\TeXButton{V}{\ooalign{\hfil$V$\hfil\cr\kern0.1em--\hfil\cr}}}%
%BeginExpansion
\ooalign{\hfil$V$\hfil\cr\kern0.1em--\hfil\cr}%
%EndExpansion
}{%
%TCIMACRO{\TeXButton{V}{\ooalign{\hfil$V$\hfil\cr\kern0.1em--\hfil\cr}}}%
%BeginExpansion
\ooalign{\hfil$V$\hfil\cr\kern0.1em--\hfil\cr}%
%EndExpansion
} \\
\frac{dP}{P} &=&\left( -\frac{C_{P}}{C_{V}}\right) \frac{d%
%TCIMACRO{\TeXButton{V}{\ooalign{\hfil$V$\hfil\cr\kern0.1em--\hfil\cr}}}%
%BeginExpansion
\ooalign{\hfil$V$\hfil\cr\kern0.1em--\hfil\cr}%
%EndExpansion
}{%
%TCIMACRO{\TeXButton{V}{\ooalign{\hfil$V$\hfil\cr\kern0.1em--\hfil\cr}}}%
%BeginExpansion
\ooalign{\hfil$V$\hfil\cr\kern0.1em--\hfil\cr}%
%EndExpansion
}
\end{eqnarray*}%
Lets define a new quantity 
\begin{equation}
\gamma =\frac{C_{P}}{C_{V}}
\end{equation}%
so we have%
\[
\frac{dP}{P}=-\gamma \frac{d%
%TCIMACRO{\TeXButton{V}{\ooalign{\hfil$V$\hfil\cr\kern0.1em--\hfil\cr}}}%
%BeginExpansion
\ooalign{\hfil$V$\hfil\cr\kern0.1em--\hfil\cr}%
%EndExpansion
}{%
%TCIMACRO{\TeXButton{V}{\ooalign{\hfil$V$\hfil\cr\kern0.1em--\hfil\cr}}}%
%BeginExpansion
\ooalign{\hfil$V$\hfil\cr\kern0.1em--\hfil\cr}%
%EndExpansion
} 
\]

Which is exciting. We are just down to $P$'s and $%
%TCIMACRO{\TeXButton{V}{\ooalign{\hfil$V$\hfil\cr\kern0.1em--\hfil\cr}}}%
%BeginExpansion
\ooalign{\hfil$V$\hfil\cr\kern0.1em--\hfil\cr}%
%EndExpansion
$'s and a gamma. We may be able to get an expression of $P$ as a function of 
$%
%TCIMACRO{\TeXButton{V}{\ooalign{\hfil$V$\hfil\cr\kern0.1em--\hfil\cr}}}%
%BeginExpansion
\ooalign{\hfil$V$\hfil\cr\kern0.1em--\hfil\cr}%
%EndExpansion
$ yet! We can integrate this last equation%
\begin{eqnarray*}
\int \frac{dP}{P} &=&-\gamma \int \frac{d%
%TCIMACRO{\TeXButton{V}{\ooalign{\hfil$V$\hfil\cr\kern0.1em--\hfil\cr}}}%
%BeginExpansion
\ooalign{\hfil$V$\hfil\cr\kern0.1em--\hfil\cr}%
%EndExpansion
}{%
%TCIMACRO{\TeXButton{V}{\ooalign{\hfil$V$\hfil\cr\kern0.1em--\hfil\cr}}}%
%BeginExpansion
\ooalign{\hfil$V$\hfil\cr\kern0.1em--\hfil\cr}%
%EndExpansion
} \\
\ln \left( P\right) +\zeta _{P} &=&-\gamma \ln \left( 
%TCIMACRO{\TeXButton{V}{\ooalign{\hfil$V$\hfil\cr\kern0.1em--\hfil\cr}}}%
%BeginExpansion
\ooalign{\hfil$V$\hfil\cr\kern0.1em--\hfil\cr}%
%EndExpansion
\right) +\zeta _{V}
\end{eqnarray*}%
where $\zeta _{P}$ and $\zeta _{V}$ are the constants of integration. We can
write this as%
\begin{eqnarray*}
\ln \left( P\right) +\gamma \ln \left( 
%TCIMACRO{\TeXButton{V}{\ooalign{\hfil$V$\hfil\cr\kern0.1em--\hfil\cr}}}%
%BeginExpansion
\ooalign{\hfil$V$\hfil\cr\kern0.1em--\hfil\cr}%
%EndExpansion
\right) &=&-\zeta _{P}+\zeta _{V} \\
&=&\zeta
\end{eqnarray*}%
where $\zeta =\zeta _{V}-\zeta _{P}$ is a constant. Knowing a little bit
about logs, we can write this as 
\[
\ln \left( P%
%TCIMACRO{\TeXButton{V}{\ooalign{\hfil$V$\hfil\cr\kern0.1em--\hfil\cr}}}%
%BeginExpansion
\ooalign{\hfil$V$\hfil\cr\kern0.1em--\hfil\cr}%
%EndExpansion
^{\gamma }\right) =\zeta 
\]%
or we can exponentiate both sides 
\begin{equation}
P%
%TCIMACRO{\TeXButton{V}{\ooalign{\hfil$V$\hfil\cr\kern0.1em--\hfil\cr}}}%
%BeginExpansion
\ooalign{\hfil$V$\hfil\cr\kern0.1em--\hfil\cr}%
%EndExpansion
^{\gamma }=e^{\zeta }
\end{equation}%
%TCIMACRO{%
%\TeXButton{Question 123.13.3}{\marginpar {
%\hspace{-0.5in}
%\begin{minipage}[t]{1in}
%\small{Question 123.13.3}
%\end{minipage}
%}}}%
%BeginExpansion
\marginpar {
\hspace{-0.5in}
\begin{minipage}[t]{1in}
\small{Question 123.13.3}
\end{minipage}
}%
%EndExpansion
where $e^{\zeta }$ is still just a constant.

Note that we have done what we set out to do (sort of). We have an
expression of how $P$ varies with $%
%TCIMACRO{\TeXButton{V}{\ooalign{\hfil$V$\hfil\cr\kern0.1em--\hfil\cr}}}%
%BeginExpansion
\ooalign{\hfil$V$\hfil\cr\kern0.1em--\hfil\cr}%
%EndExpansion
.$ 
\[
P=\frac{e^{\zeta }}{%
%TCIMACRO{\TeXButton{V}{\ooalign{\hfil$V$\hfil\cr\kern0.1em--\hfil\cr}}}%
%BeginExpansion
\ooalign{\hfil$V$\hfil\cr\kern0.1em--\hfil\cr}%
%EndExpansion
^{\gamma }} 
\]%
We could put this into our work equation and integrate to find the work done
in an adiabatic process!%
\begin{eqnarray*}
\Delta E_{int} &=&w \\
&=&-\int \frac{e^{\zeta }}{%
%TCIMACRO{\TeXButton{V}{\ooalign{\hfil$V$\hfil\cr\kern0.1em--\hfil\cr}}}%
%BeginExpansion
\ooalign{\hfil$V$\hfil\cr\kern0.1em--\hfil\cr}%
%EndExpansion
^{\gamma }}d%
%TCIMACRO{\TeXButton{V}{\ooalign{\hfil$V$\hfil\cr\kern0.1em--\hfil\cr}}}%
%BeginExpansion
\ooalign{\hfil$V$\hfil\cr\kern0.1em--\hfil\cr}%
%EndExpansion
\end{eqnarray*}

But also note that we have a great conservation equation for an adiabatic
process.

\[
P%
%TCIMACRO{\TeXButton{V}{\ooalign{\hfil$V$\hfil\cr\kern0.1em--\hfil\cr}}}%
%BeginExpansion
\ooalign{\hfil$V$\hfil\cr\kern0.1em--\hfil\cr}%
%EndExpansion
^{\gamma }=\text{constant} 
\]%
or 
\begin{equation}
P_{i}%
%TCIMACRO{\TeXButton{V}{\ooalign{\hfil$V$\hfil\cr\kern0.1em--\hfil\cr}}}%
%BeginExpansion
\ooalign{\hfil$V$\hfil\cr\kern0.1em--\hfil\cr}%
%EndExpansion
_{i}^{\gamma }=P_{f}%
%TCIMACRO{\TeXButton{V}{\ooalign{\hfil$V$\hfil\cr\kern0.1em--\hfil\cr}}}%
%BeginExpansion
\ooalign{\hfil$V$\hfil\cr\kern0.1em--\hfil\cr}%
%EndExpansion
_{f}^{\gamma }
\end{equation}%
and we know that conservation equations come in very handy in solving
problems! Let's add this last equation to our set of equations for adiabatic
processes.

We can get a second equation for adiabatic processes relating the volume and
temperature. If we know an initial state 
\[
P_{i}%
%TCIMACRO{\TeXButton{V}{\ooalign{\hfil$V$\hfil\cr\kern0.1em--\hfil\cr}}}%
%BeginExpansion
\ooalign{\hfil$V$\hfil\cr\kern0.1em--\hfil\cr}%
%EndExpansion
_{i}=nRT_{i} 
\]%
we can use our adiabatic conservation equation from above to write%
\[
P_{f}%
%TCIMACRO{\TeXButton{V}{\ooalign{\hfil$V$\hfil\cr\kern0.1em--\hfil\cr}}}%
%BeginExpansion
\ooalign{\hfil$V$\hfil\cr\kern0.1em--\hfil\cr}%
%EndExpansion
_{f}^{\gamma }=P_{i}%
%TCIMACRO{\TeXButton{V}{\ooalign{\hfil$V$\hfil\cr\kern0.1em--\hfil\cr}}}%
%BeginExpansion
\ooalign{\hfil$V$\hfil\cr\kern0.1em--\hfil\cr}%
%EndExpansion
_{i}^{\gamma } 
\]%
we can solve the ideal gas law for $P_{i}$ and 
\[
P_{i}=\frac{nRT_{i}}{%
%TCIMACRO{\TeXButton{V}{\ooalign{\hfil$V$\hfil\cr\kern0.1em--\hfil\cr}}}%
%BeginExpansion
\ooalign{\hfil$V$\hfil\cr\kern0.1em--\hfil\cr}%
%EndExpansion
_{i}} 
\]%
and use this in our equation 
\[
\frac{nRT_{f}}{%
%TCIMACRO{\TeXButton{V}{\ooalign{\hfil$V$\hfil\cr\kern0.1em--\hfil\cr}}}%
%BeginExpansion
\ooalign{\hfil$V$\hfil\cr\kern0.1em--\hfil\cr}%
%EndExpansion
_{f}}%
%TCIMACRO{\TeXButton{V}{\ooalign{\hfil$V$\hfil\cr\kern0.1em--\hfil\cr}}}%
%BeginExpansion
\ooalign{\hfil$V$\hfil\cr\kern0.1em--\hfil\cr}%
%EndExpansion
_{f}^{\gamma }=\frac{nRT_{i}}{%
%TCIMACRO{\TeXButton{V}{\ooalign{\hfil$V$\hfil\cr\kern0.1em--\hfil\cr}}}%
%BeginExpansion
\ooalign{\hfil$V$\hfil\cr\kern0.1em--\hfil\cr}%
%EndExpansion
_{i}}%
%TCIMACRO{\TeXButton{V}{\ooalign{\hfil$V$\hfil\cr\kern0.1em--\hfil\cr}}}%
%BeginExpansion
\ooalign{\hfil$V$\hfil\cr\kern0.1em--\hfil\cr}%
%EndExpansion
_{i}^{\gamma } 
\]%
to yield%
\[
\frac{T_{f}}{%
%TCIMACRO{\TeXButton{V}{\ooalign{\hfil$V$\hfil\cr\kern0.1em--\hfil\cr}}}%
%BeginExpansion
\ooalign{\hfil$V$\hfil\cr\kern0.1em--\hfil\cr}%
%EndExpansion
_{f}}%
%TCIMACRO{\TeXButton{V}{\ooalign{\hfil$V$\hfil\cr\kern0.1em--\hfil\cr}}}%
%BeginExpansion
\ooalign{\hfil$V$\hfil\cr\kern0.1em--\hfil\cr}%
%EndExpansion
_{f}^{\gamma }=\frac{T_{i}}{%
%TCIMACRO{\TeXButton{V}{\ooalign{\hfil$V$\hfil\cr\kern0.1em--\hfil\cr}}}%
%BeginExpansion
\ooalign{\hfil$V$\hfil\cr\kern0.1em--\hfil\cr}%
%EndExpansion
_{i}}%
%TCIMACRO{\TeXButton{V}{\ooalign{\hfil$V$\hfil\cr\kern0.1em--\hfil\cr}}}%
%BeginExpansion
\ooalign{\hfil$V$\hfil\cr\kern0.1em--\hfil\cr}%
%EndExpansion
_{i}^{\gamma } 
\]%
or%
\begin{equation}
T_{f}%
%TCIMACRO{\TeXButton{V}{\ooalign{\hfil$V$\hfil\cr\kern0.1em--\hfil\cr}}}%
%BeginExpansion
\ooalign{\hfil$V$\hfil\cr\kern0.1em--\hfil\cr}%
%EndExpansion
_{f}^{\gamma -1}=T_{i}%
%TCIMACRO{\TeXButton{V}{\ooalign{\hfil$V$\hfil\cr\kern0.1em--\hfil\cr}}}%
%BeginExpansion
\ooalign{\hfil$V$\hfil\cr\kern0.1em--\hfil\cr}%
%EndExpansion
_{i}^{\gamma -1}
\end{equation}%
This is another conservation equation for adiabatic processes! We will add
this to our equation set for adiabatic processes

These equations are great shortcuts for problem solving \emph{but remember
they are only valid for adiabatic processes!}

Notice that we never did complete the integral for work%
\[
\Delta E_{int}=w=-\int \frac{e^{\zeta }}{%
%TCIMACRO{\TeXButton{V}{\ooalign{\hfil$V$\hfil\cr\kern0.1em--\hfil\cr}}}%
%BeginExpansion
\ooalign{\hfil$V$\hfil\cr\kern0.1em--\hfil\cr}%
%EndExpansion
^{\gamma }}d%
%TCIMACRO{\TeXButton{V}{\ooalign{\hfil$V$\hfil\cr\kern0.1em--\hfil\cr}}}%
%BeginExpansion
\ooalign{\hfil$V$\hfil\cr\kern0.1em--\hfil\cr}%
%EndExpansion
\]%
We totally could at this point, but we already know the answer to be 
\[
\Delta E_{int}=nC_{V}\Delta T 
\]%
where the $\Delta T$ is the same as it would be for our actual adiabatic
process. Since we know this and it is easy to calculate, we won't often do
the integral

\subsection{Update for Adiabatic processes}

\FRAME{dtbpFX}{2.2087in}{1.4719in}{0pt}{}{}{Plot}{\special{language
"Scientific Word";type "MAPLEPLOT";width 2.2087in;height 1.4719in;depth
0pt;display "USEDEF";plot_snapshots TRUE;mustRecompute FALSE;lastEngine
"MuPAD";animated TRUE;animationStartTime "0"; animationEndTime "10";
animationFramesPerSecond "10";xmin "0";xmax "100";animationParamMin
"290";animationParamMax "350";xviewmin "0";xviewmax "40";yviewmin
"0";yviewmax "400";viewset"XY";rangeset"X";plottype 4;labeloverrides
3;x-label "V(m^3)";y-label "P(Pa)";axesFont "Times New
Roman,12,0000000000,useDefault,normal";numpoints 100;plotstyle
"patch";axesstyle "normal";axestips FALSE;xis \TEXUX{V};animationParam
\TEXUX{t};var1name \TEXUX{$V$};animationParamList \TEXUX{$t$};function
\TEXUX{$\allowbreak 7500\frac{1}{V^{1.67}}$};linecolor "blue";linestyle
1;pointstyle "point";linethickness 3;lineAttributes "Solid";var1range
"0,100";animationParamRange "290,350";num-x-gridlines 50;curveColor
"[flat::RGB:0x000000ff]";curveStyle "Line";animationStartTime "0";
animationEndTime "10"; animationFramesPerSecond
"10";animationVisibleBeforeStart FALSE;rangeset"XA";function
\TEXUX{$\allowbreak 8.\,\allowbreak 314\frac{1}{V}290$};linecolor
"red";linestyle 2;pointstyle "point";linethickness 1;lineAttributes
"Dash";var1range "0,40";animationParamRange "290,350";num-x-gridlines
100;curveColor "[flat::RGB:0x00ff0000]";curveStyle "Line";function
\TEXUX{$8.\,\allowbreak 314\frac{1}{V}100$};linecolor "red";linestyle
2;pointstyle "point";linethickness 1;lineAttributes "Dash";var1range
"0,40";animationParamRange "290,350";num-x-gridlines 100;curveColor
"[flat::RGB:0x00ff0000]";curveStyle "Line";VCamFile
'PU11IA01.xvz';valid_file "T";tempfilename
'PQXXR1ZG.wmf';tempfile-properties "XPR";}}Recall that for adiabatic
processes we know that 
\[
Q=0 
\]%
so that 
\[
\Delta E_{int}=w 
\]%
and we found that%
\[
\Delta E_{int}=nC_{V}\Delta T 
\]%
we can add to these equations%
\[
P_{f}%
%TCIMACRO{\TeXButton{V}{\ooalign{\hfil$V$\hfil\cr\kern0.1em--\hfil\cr}}}%
%BeginExpansion
\ooalign{\hfil$V$\hfil\cr\kern0.1em--\hfil\cr}%
%EndExpansion
_{f}^{\gamma }=P_{i}%
%TCIMACRO{\TeXButton{V}{\ooalign{\hfil$V$\hfil\cr\kern0.1em--\hfil\cr}}}%
%BeginExpansion
\ooalign{\hfil$V$\hfil\cr\kern0.1em--\hfil\cr}%
%EndExpansion
_{i}^{\gamma } 
\]%
and 
\[
T_{f}%
%TCIMACRO{\TeXButton{V}{\ooalign{\hfil$V$\hfil\cr\kern0.1em--\hfil\cr}}}%
%BeginExpansion
\ooalign{\hfil$V$\hfil\cr\kern0.1em--\hfil\cr}%
%EndExpansion
_{f}^{\gamma -1}=T_{i}%
%TCIMACRO{\TeXButton{V}{\ooalign{\hfil$V$\hfil\cr\kern0.1em--\hfil\cr}}}%
%BeginExpansion
\ooalign{\hfil$V$\hfil\cr\kern0.1em--\hfil\cr}%
%EndExpansion
_{i}^{\gamma -1} 
\]%
where 
\[
\gamma =\frac{C_{P}}{C_{V}} 
\]%
This will get us closer to our goal of finding $Q,$ $w,$ and $\Delta E_{int}$
for this process.

\section{Heat Transfer Mechanisms}

%TCIMACRO{%
%\TeXButton{Question 123.13.4}{\marginpar {
%\hspace{-0.5in}
%\begin{minipage}[t]{1in}
%\small{Question 123.13.3}
%\end{minipage}
%}}}%
%BeginExpansion
\marginpar {
\hspace{-0.5in}
\begin{minipage}[t]{1in}
\small{Question 123.13.3}
\end{minipage}
}%
%EndExpansion
The change in internal energy is the sum of the energy from each type of
energy transfer%
\[
\Delta E_{int}=Q+W 
\]%
Work we know a lot about, but let's become more familiar with $Q.$ We may
talk about different ways to transfer energy, then%
\begin{equation}
\Delta E_{int}=\sum_{i}(\text{Transfer mechanism)}_{i}
\end{equation}%
where work is one of those energy transfer mechanisms and there may be many
different $Q$ transfer types. In what follows, we will talk about a few.

\subsection{Thermal Conduction}

This is what we have been calling heat up to this point. We define it a
little more carefully as the exchange of kinetic energy between microscopic
particles (molecules and atoms) through direct contact.

When we add energy by heat to part of a solid, the atoms are locally
displaced from their equilibrium locations. These atoms knock into their
neighbors, and cause them to be displaced (they increase their kinetic
energy). This process continues through the whole solid.

The rate of heat transfer will depend on the properties of the atoms that
make up the substance.\FRAME{dtbpF}{1.9132in}{3.0592in}{0pt}{}{}{Figure}{%
\special{language "Scientific Word";type "GRAPHIC";maintain-aspect-ratio
TRUE;display "USEDEF";valid_file "T";width 1.9132in;height 3.0592in;depth
0pt;original-width 1.9212in;original-height 3.0894in;cropleft "0";croptop
"1";cropright "1";cropbottom "0";tempfilename
'PQXXR1ZH.wmf';tempfile-properties "XPR";}}We can envision a piece of solid
as in the figure. It has thickness $\Delta x$ and cross-sectional area $A.$
If one face of the slab is at $T_{c}$ and the other at $T_{h}.$ We find that
energy $Q$ transfers in a time interval $\Delta t$ from the hotter face to
the colder one. The rate 
\[
\mathcal{P=}\frac{E}{\Delta t}=\frac{Q}{\Delta t} 
\]%
is proportional to $A.$ and $\Delta T.$%
\[
\mathcal{P}=\frac{Q}{\Delta t}\varpropto A\frac{\Delta T}{\Delta L} 
\]%
We can take $\Delta L$ very small%
\begin{equation}
\mathcal{P}=k_{therm}A\left\vert \frac{dT}{dL}\right\vert
\end{equation}%
where we now have another constant $k$! This $k_{therm}$ is the \emph{%
thermal conductivity} of the material and we define $\left\vert \frac{dT}{dL}%
\right\vert $ as the \emph{temperature gradient.}

A gradient is a rate of change, so this gradient is the rate of change of
the temperature with respect to position.

For a one dimensional problem, we take a rod. If we assume that $k_{therm}$
is not temperature dependent and the rod is uniform, then%
\[
\left\vert \frac{dT}{dL}\right\vert =\frac{T_{h}-T_{c}}{\Delta L} 
\]%
and 
\begin{eqnarray}
\mathcal{P}_{rod} &\mathcal{=}&\frac{Q}{\Delta t}=k_{therm}A\frac{T_{h}-T_{c}%
}{\Delta L} \\
&=&\frac{A\left( T_{h}-T_{c}\right) }{\frac{\Delta L}{k_{therm}}}  \nonumber
\end{eqnarray}

If we have a complicated multipart rod with $n$ different substances we
would have%
\begin{equation}
\mathcal{P}=\frac{A\left( T_{h}-T_{c}\right) }{\dsum\limits_{i}^{n}\frac{%
\Delta L_{i}}{k_{therm_{i}}}}
\end{equation}%
The amount of energy transferred by thermal conduction is 
\[
Q=\mathcal{P}\Delta t=\frac{A\left( T_{h}-T_{c}\right) \Delta t}{%
\dsum\limits_{i}^{n}\frac{\Delta L_{i}}{k_{therm_{i}}}} 
\]

All this wasn't too bad mathematically. But the building industry does some
simplifications to our thermal conduction equation. We define a term%
\begin{equation}
R=\frac{\Delta L}{k}
\end{equation}%
as the \textquotedblleft R-value\textquotedblright\ of a material. Then a
wall or multipart solid would have%
\begin{equation}
\mathcal{P=}\frac{Q}{\Delta t}=\frac{A\left( T_{h}-T_{c}\right) }{%
\dsum\limits_{i}^{n}R_{i}}
\end{equation}%
so that 
\[
Q_{conduction}=\frac{A\left( T_{h}-T_{c}\right) \Delta t}{%
\dsum\limits_{i}^{n}R_{i}} 
\]%
These R-values are used in home insulation and heating design and are
usually in English units

\subsection{Convection}

This is a fundamentally new type of energy transport in our study of
thermodynamics, although we are all familiar with it. Convection is the
transfer of energy by the movement of the atoms or molecules, themselves.
Because the group of moving atoms or molecules has internal energy, moving
these molecules also moves the energy. So convection is an energy transfer.
This is like circulation of air in your home. Your student apartment might
have base-board heading. But your parent's home would likely use forced
convection where a fan blows the air. \FRAME{dtbpF}{2.4365in}{2.5173in}{0pt}{%
}{}{Figure}{\special{language "Scientific Word";type
"GRAPHIC";maintain-aspect-ratio TRUE;display "USEDEF";valid_file "T";width
2.4365in;height 2.5173in;depth 0pt;original-width 2.4543in;original-height
2.5368in;cropleft "0";croptop "1";cropright "1";cropbottom "0";tempfilename
'PQXXR1ZI.wmf';tempfile-properties "XPR";}}For several decades houses have
been made with the idea that you can heat or cool a house with passive
convection. Here is an example.\FRAME{dtbpF}{5.5832in}{1.7374in}{0in}{}{}{%
Figure}{\special{language "Scientific Word";type
"GRAPHIC";maintain-aspect-ratio TRUE;display "USEDEF";valid_file "T";width
5.5832in;height 1.7374in;depth 0in;original-width 5.5253in;original-height
1.7002in;cropleft "0";croptop "1";cropright "1";cropbottom "0";tempfilename
'PU138I01.wmf';tempfile-properties "XPR";}}This house tries to control the
thermal conduction mechanisms to keep the interior of the house at nearly
the same temperature year round.

\subsection{Radiation}

Light is a form of energy transfer. You will learn more about this in PH220,
but for now, what we need to know is that light is a form of wave, and as we
know waves carry energy with them. So, light is an energy transfer. We call
this energy removal or delivery by light \textquotedblleft
radiation.\textquotedblright\ This is different than the use of the word
\textquotedblleft radiation\textquotedblright\ in nuclear physics.

All objects radiate energy according to Stefan's law%
\begin{equation}
\mathcal{P=}\frac{Q}{\Delta t}=\sigma AeT^{4}
\end{equation}%
The parts of this law are Stephen's constant $\sigma =5.6696\times 10^{-8}%
\frac{\unit{W}}{\unit{m}^{2}\unit{K}}$, the factor, $A,$ is the surface
area, $e$ is the emissivity of the material. This $e$ tells us how like a
perfect absorber (called a black body) it is, and $T$ is the temperature.
Note there are other forms for these! But this is the form we will use.

The emissivity is a value from $0$ to $1.$ If $e=0$ then the material is a
perfect reflector. This would be a perfect mirror. If the emissivity is $1,$
then it is a perfect absorber. This is more like blacktop in sunlight. The
blacktop warms up, even in the winter, because it absorbs the sunlight. Of
course, nothing has an emissivity of exactly $0$ or exactly $1.$ Real
objects are somewhere in the middle.

It may seem strange that \textquotedblleft black\textquotedblright\ objects
glow, but they do. Think again of that blacktop. The ice on the blacktop
melts, because the blacktop not only absorbs the radiation from the sun, it
also emits radiation.

An extreme example is the Sun, itself. Light that hits the sun from outside
will scatter off the gasses in the sun. It takes a very long time for the
light to reemerge, if it ever does. Most likely it will be absorbed by the
gasses. So the Sun is considered a \textquotedblleft black
body!\textquotedblright\ But it glows, so it does not look black.

We can explain many common experiences using the idea of emissivity. Take
this leaf for example. Why is it sinking into the snow?

\FRAME{dhF}{1.6527in}{1.9908in}{0pt}{}{}{Figure}{\special{language
"Scientific Word";type "GRAPHIC";maintain-aspect-ratio TRUE;display
"USEDEF";valid_file "T";width 1.6527in;height 1.9908in;depth
0pt;original-width 5.3938in;original-height 6.5086in;cropleft "0";croptop
"1";cropright "1";cropbottom "0";tempfilename
'PQXXR1ZK.wmf';tempfile-properties "XPR";}}The leaf has a different
emissivity than the snow, so it absorbs more radiation from the sun, the
temperature of the leaf rises and it melts the snow around it.

With this understanding, we can see that all objects absorb radiation as
well as radiate. If an object is at temperature $T$ and it's surroundings
are at temperature $T_{o}.$ then the net rate of energy gained or lost is 
\begin{equation}
\mathcal{P\mathcal{=}}\frac{Q}{\Delta t}\mathcal{=}\sigma Ae\left(
T^{4}-T_{o}^{4}\right)
\end{equation}%
and the energy transfer would be 
\[
Q_{radiation}=\frac{Q}{\Delta t}\mathcal{=}\sigma Ae\left(
T^{4}-T_{o}^{4}\right) \Delta t 
\]

\subsection{Multiple transfer mechanisms}

As we said before, we might have many different energy transfer mechanisms.
We could have all of these (and more).%
\[
\Delta E_{int}=Q_{conduction}+Q_{convection}+Q_{radiation}+w 
\]

Let's think about an example that uses radiation, convection, and insulation
to keep something cold. It is called a dewar flask.

\FRAME{fhFU}{3.3696in}{1.8369in}{0pt}{\Qcb{You might see our liquid nitrogen
dewar in demonstrations. The flask is pictured on the left, and a schematic
diagram of how a dewar works is given on the right.}}{}{Figure}{\special%
{language "Scientific Word";type "GRAPHIC";maintain-aspect-ratio
TRUE;display "USEDEF";valid_file "T";width 3.3696in;height 1.8369in;depth
0pt;original-width 3.4051in;original-height 1.8432in;cropleft "0";croptop
"1";cropright "1";cropbottom "0";tempfilename
'PQXXR1ZL.wmf';tempfile-properties "XPR";}}This is the same idea as a
thermos bottle, only better made (and more expensive). The flask has doubled
walls with the space between evacuated to reduce convection and eliminate
conduction. The walls are silvered to reduce heat transfer by radiation. The
cap has a long piece of insulation attached to it to prevent conduction
through the flask opening. New designs use multi-layer films which make
their surfaces super Reflective.

\chapter{Microscopic View}

So far, as we have considered the motion of atoms with thermal energy we
have been able to treat whole moles of a gas at a time. But you might guess
that we are sort of taking averages of the motions of the atoms or
molecules. Let's make this more explicit in this lecture.

%TCIMACRO{%
%\TeXButton{Fundamental Concepts}{\hspace{-1.3in}{\Large Fundamental Concepts\vspace{0.25in}}}}%
%BeginExpansion
\hspace{-1.3in}{\Large Fundamental Concepts\vspace{0.25in}}%
%EndExpansion

\begin{itemize}
\item In a gas, the molecules have different speeds

\item The distribution of speeds is given by the Boltzmann distribution law

\item the root mean square speed $\left( v_{rms}=\sqrt{\overline{v^{2}}}%
\right) $is a better measure of the speed of gas molecules than the average
velocity

\item Pressure for an ideal gas is given by $P=\frac{2}{3}\frac{N}{%
%TCIMACRO{\TeXButton{V}{\ooalign{\hfil$V$\hfil\cr\kern0.1em--\hfil\cr}}}%
%BeginExpansion
\ooalign{\hfil$V$\hfil\cr\kern0.1em--\hfil\cr}%
%EndExpansion
}\left( \frac{1}{2}m\overline{v^{2}}\right) $
\end{itemize}

\section{Molecular Model of an Ideal Gas}

We have hinted that the internal energy of a system must be the energy
associated with the atoms and molecules that make the system. It's time to
make that association in our mental model more concrete, and more
mathematical. We will be able to see how the thermodynamic laws we have
studied are generated by the basic laws of motion. But we will still limit
our study to ideal gases in this class. So let's state some assumptions that
follow from the ideal gas approximation.%
\[
\begin{tabular}{|c|}
\hline
{\large Ideal Gas Model} \\ \hline
The number of molecules in the gas is very large, \\ \hline
The average separation between molecules is large compared the their
dimensions \\ \hline
The molecules obey Newton's laws of motion, \\ \hline
On the whole the molecules move randomly \\ \hline
The molecules interact only by short-range forces during elastic collisions
\\ \hline
The molecules make elastic collisions with the walls \\ \hline
The gas under consideration is a pure substance; that is; all molecules are
identical. \\ \hline
\end{tabular}%
\]

We will often say \textquotedblleft molecule\textquotedblright\ but for an
ideal gas atoms and molecules act alike.%
%TCIMACRO{%
%\TeXButton{Question 123.14.1}{\marginpar {
%\hspace{-0.5in}
%\begin{minipage}[t]{1in}
%\small{Question 123.14.1}
%\end{minipage}
%}}}%
%BeginExpansion
\marginpar {
\hspace{-0.5in}
\begin{minipage}[t]{1in}
\small{Question 123.14.1}
\end{minipage}
}%
%EndExpansion

\section{Speed of molecules}

The internal energy of a system might consist of nuclear internal energy, or
chemical internal energy, but let's reserve those internal energies for
PH279. For us, internal energy is mostly due to the motion of the molecules
in the system. That motion has kinetic energy associated with it. But not a
kinetic energy of the whole system moving together. It is a kinetic energy
of the parts of the system, the molecules. Think of driving a car. We could
describe the kinetic energy of the whole car. That is the kinetic energy we
learned about in PH121. But we could also talk about the kinetic energy of
one of the pistons in the engine. These two kinetic energies would be
different! Just like this, the kinetic energy of the molecules will be
different than the kinetic energy of the system as a whole. \FRAME{dtbpF}{%
3.5453in}{2.3629in}{0pt}{}{}{Figure}{\special{language "Scientific
Word";type "GRAPHIC";maintain-aspect-ratio TRUE;display "USEDEF";valid_file
"T";width 3.5453in;height 2.3629in;depth 0pt;original-width
3.5843in;original-height 2.3789in;cropleft "0";croptop "1";cropright
"1";cropbottom "0";tempfilename 'PQXXR1ZM.wmf';tempfile-properties "XPR";}}%
To find this kinetic energy of the molecules, we would need to know how fast
the molecules are going in an ideal gas at a particular temperate and
pressure in a particular volume.

The average velocity is not helpful, because with our ideal gas model, the
molecules move randomly. So we would guess that on the whole their
velocities would cancel. If this is not true, we call the situation wind.
But if there is no over-all movement, the average velocity, $\mathbf{\bar{v}}%
=0.$ But we know there must be movement because the molecules have some
internal energy. The kinetic energy depends on the speed of the molecules.
But not all molecules have the same velocity. So when we say
\textquotedblleft speed of the molecules\textquotedblright\ we have to be
specific, what speed? The average? The speed the most particles have?

The average velocity won't do. \FRAME{dtbpF}{3.0725in}{2.1988in}{0in}{}{}{%
Figure}{\special{language "Scientific Word";type
"GRAPHIC";maintain-aspect-ratio TRUE;display "USEDEF";valid_file "T";width
3.0725in;height 2.1988in;depth 0in;original-width 3.1027in;original-height
2.2121in;cropleft "0";croptop "1";cropright "1";cropbottom "0";tempfilename
'PQXXR1ZN.wmf';tempfile-properties "XPR";}}

The molecule motions have random directions. So the average velocity is
zero. Some velocities are in negative directions and some are in positive
directions, so the velocities cancel out. But if we square the velocities
they are all positive. Then we could find the average of the squared
velocities and that would be better. But a velocity squared is not a
velocity. So to get back to velocity, let's take a square root.

\[
v_{rms}=\sqrt{\overline{v^{2}}} 
\]%
This quantity is like an average, and gives a representative value that is
near the most probable speeds.

Using this, we can get an expression that relates the temperature of our gas
to the speed of the molecules. But it we need to know something about how
many molecules have what speed. Having the \textquotedblleft
average\textquotedblright\ is not enough. To find the \emph{distribution} of
speeds, let's remember the idea of number density%
\[
n_{V}=\frac{\#\text{ of molecules }}{%
%TCIMACRO{\TeXButton{V}{\ooalign{\hfil$V$\hfil\cr\kern0.1em--\hfil\cr}}}%
%BeginExpansion
\ooalign{\hfil$V$\hfil\cr\kern0.1em--\hfil\cr}%
%EndExpansion
} 
\]%
We can express the number of molecules this way assuming we know the volume.
But we want the speed of the molecules, and we have learned that concept of
energy usually can get us a speed if we don't care to know the direction.
Knowing the direction for every air molecule in a room would not be helpful
in most cases, so we can be content with just knowing the speed. So lets
find a sort of \textquotedblleft energy density,\textquotedblright\ that is,
the density of molecules that have energy between two amounts of energy,
say, $E_{1}$ and $E_{2}$. We want $E_{1}$ and $E_{2}$ to be quite close
together. So let's let $E_{1}=$ $E$ and $E_{2}=E+\Delta E$ where $\Delta E$
is a small amount of energy. If you take quantum mechanics, or statistical
mechanics, you will likely see this quantity quite a lot. Then the number of
molecules with a particular energy between $E_{1}$ and $E_{2}$ could be
written as%
\begin{equation}
n_{V}\left( E\right) dE=\frac{\#\text{ of molecules with energy between }E%
\text{ and }E+\Delta E}{%
%TCIMACRO{\TeXButton{V}{\ooalign{\hfil$V$\hfil\cr\kern0.1em--\hfil\cr}}}%
%BeginExpansion
\ooalign{\hfil$V$\hfil\cr\kern0.1em--\hfil\cr}%
%EndExpansion
}
\end{equation}

This is called a \emph{distribution function}. We find distribution
functions in statistics. They are associated with probabilities. The
standard \textquotedblleft bell curve\textquotedblright\ used sometimes in
grading is a distribution function. It tells the total number of students
that got a particular number of points in a class.

What we need is the probability that the molecules will have a particular
energy (or speed, since this is kinetic energy). A function that gives the
amount of molecules that have a particular amount of energy also is called a 
$distribution$ function. The distribution function that we will use is
written symbolically in this cryptic fashion, $n_{V}\left( E\right) dE.$ It
is the number of molecules with a particular energy divided by the total
number of molecules.

We won't derive an equation for this quantity, instead we will borrow a
result from our junior level thermal physics class.%
\begin{equation}
n_{V}\left( E\right) =n_{o}e^{-\frac{E}{k_{B}T}}
\end{equation}%
where $n_{o}$ is the number of molecules per unit volume having energy
between $E=0$ and $E=dE.$ This distribution function is called the \emph{%
Boltzmann distribution law.} It tells us that the probability of finding the
molecules in a particular energy state varies exponentially as the negative
of the energy divided by $k_{B}T.$

\section{Distribution of Molecular Speeds}

%TCIMACRO{%
%\TeXButton{Question 123.14.2}{\marginpar {
%\hspace{-0.5in}
%\begin{minipage}[t]{1in}
%\small{Question 123.14.2}
%\end{minipage}
%}}}%
%BeginExpansion
\marginpar {
\hspace{-0.5in}
\begin{minipage}[t]{1in}
\small{Question 123.14.2}
\end{minipage}
}%
%EndExpansion
Since there is a distribution of energies, we expect our gas molecules to
have a distribution of velocities. That is, the molecules in the gas do not
all go the same speed. Again we won't derive this (at least not in this
class!), but the distribution should depend on temperature, $T,$ since we
know the internal energy is tied to temperature. The distribution is as
follows:%
\begin{equation}
N_{v}=4\pi N\left( \frac{m}{2\pi k_{B}T}\right) ^{\frac{3}{2}}v^{2}e^{-\frac{%
mv^{2}}{2k_{B}T}}
\end{equation}

where $m$ is the mass of the molecule.

If there are $dN$ molecules with speeds between $v$ and $v+dv$ then%
\begin{equation}
dN=N_{v}dv
\end{equation}%
so there should be 
\begin{equation}
N=\int_{0}^{\infty }N_{v}dv
\end{equation}%
total molecules.

The kinetic energy of the molecules is hiding in the exponent, so we could
write this as 
\begin{equation}
N_{v}=4\pi N\left( \frac{m}{2\pi k_{B}T}\right) ^{\frac{3}{2}}v^{2}e^{-\frac{%
K}{k_{B}T}}
\end{equation}

If we plot $N_{v}$ vs. $v$ we get the figure below. The number of molecules
with speeds between $v$ and $v+dv$ is the area under the blue curve. The
peak of the curve tells us the most probable speed, that is, the speed the
most molecules have, $v_{mp}$. The curve is not symmetric, so the most
probable speed is not the average speed, $\bar{v}$. There is also our new
speed estimate marked $v_{rms}$.\FRAME{dtbpF}{3.3998in}{3.3546in}{0pt}{}{}{%
Figure}{\special{language "Scientific Word";type
"GRAPHIC";maintain-aspect-ratio TRUE;display "USEDEF";valid_file "T";width
3.3998in;height 3.3546in;depth 0pt;original-width 3.4353in;original-height
3.3892in;cropleft "0";croptop "1";cropright "1";cropbottom "0";tempfilename
'PQXXR1ZO.wmf';tempfile-properties "XPR";}}

If we plot $N_{v}$ for different temperatures, we observe that the peak
shifts, and the curve broadens\FRAME{dtbpFU}{3.3244in}{2.449in}{0pt}{\Qcb{%
Temperature dependence of the Maxwell-Boltsman distribution (Image in the
Public Domain couracy Fred Stober)}}{}{Figure}{\special{language "Scientific
Word";type "GRAPHIC";maintain-aspect-ratio TRUE;display "USEDEF";valid_file
"T";width 3.3244in;height 2.449in;depth 0pt;original-width
3.359in;original-height 2.4667in;cropleft "0";croptop "1";cropright
"1";cropbottom "0";tempfilename 'PQXXR1ZP.wmf';tempfile-properties "XPR";}}A
motivated student could now find the most probably speed by finding the
maximum of $N_{v}.$ To do this, we take a derivative

\begin{eqnarray*}
\frac{dN_{v}}{dv} &=&\frac{d}{dv}\left( 4\pi N\left( \frac{m}{2\pi k_{B}T}%
\right) ^{\frac{3}{2}}v^{2}e^{-\frac{mv^{2}}{2k_{B}T}}\right) \\
&=&4\pi N\left( \frac{m}{2\pi k_{B}T}\right) ^{\frac{3}{2}}\frac{d}{dv}%
\left( v^{2}e^{-\frac{m}{2k_{B}T}v^{2}}\right) \\
&=&4\pi N\left( \frac{m}{2\pi k_{B}T}\right) ^{\frac{3}{2}}\left( -2\frac{m}{%
2k_{B}T}v^{3}e^{-\frac{m}{2k_{B}T}v^{2}}+2ve^{-\frac{m}{2k_{B}T}v^{2}}\right)
\\
&=&\allowbreak -2ve^{-\frac{mv^{2}}{2k_{B}T}}\left( \frac{mv^{2}}{2k_{B}T}%
-1\right) 4\pi N\left( \frac{m}{2\pi k_{B}T}\right) ^{\frac{3}{2}}
\end{eqnarray*}

set this equal to zero%
\begin{eqnarray*}
0 &=&\allowbreak -2ve^{-\frac{mv^{2}}{2k_{B}T}}\left( \frac{mv^{2}}{2k_{B}T}%
-1\right) 4\pi N\left( \frac{m}{2\pi k_{B}T}\right) ^{\frac{3}{2}} \\
\frac{mv^{2}}{2k_{B}T} &=&1 \\
v_{mp} &=&\sqrt{\frac{2k_{B}T}{m}}
\end{eqnarray*}%
\begin{equation}
v_{mp}=\sqrt{\frac{2k_{B}T}{m}}
\end{equation}%
and this is great! We have related the temperature, $T$, to the most
probable speed of the molecules. But it is more convenient to use $v_{rms,}$
so let's see if we can modify this expression to be in terms of $v_{rms}.$

The average value of $v^{n}$ is given by%
\[
\overline{v^{n}}=\frac{1}{N}\int_{0}^{\infty }v^{n}N_{v}dv 
\]%
This motivated student could also use this to find the average speed (not
the average velocity, which is zero). He or she will want the average value
of $v^{1}$ 
\begin{eqnarray*}
\overline{v^{1}} &=&\frac{1}{N}\int_{0}^{\infty }v^{1}4\pi N\left( \frac{m}{%
2\pi k_{B}T}\right) ^{\frac{3}{2}}v^{2}e^{-\frac{mv^{2}}{2k_{B}T}}dv \\
&=&4\pi \left( \frac{m}{2\pi k_{B}T}\right) ^{\frac{3}{2}}\int_{0}^{\infty
}v^{3}e^{-\frac{mv^{2}}{2k_{B}T}}dv \\
&=&4\pi \left( \frac{m}{2\pi k_{B}T}\right) ^{\frac{3}{2}}\left. \left( -%
\frac{1}{2\left( \frac{m}{2k_{B}T}\right) ^{2}}\left( e^{-\frac{mv^{2}}{%
2k_{B}T}}+\frac{m}{2k_{B}T}v^{2}e^{-\frac{mv^{2}}{2k_{B}T}}\right) \right)
\right\vert _{0}^{\infty } \\
&=&4\pi \left( \frac{m}{2\pi k_{B}T}\right) ^{\frac{3}{2}}\left( -\frac{1}{%
2\left( \frac{m}{2k_{B}T}\right) ^{2}}\right) \left( 0-\left( 1+0\right)
\right) \\
&=&2\pi \left( \frac{m}{2\pi k_{B}T}\right) ^{\frac{3}{2}}\left( \frac{1}{%
\left( \frac{\pi m}{2\pi k_{B}T}\right) ^{2}}\right) \\
&=&2\pi \left( \frac{m}{2\pi k_{B}T}\right) ^{\frac{3}{2}}\left( \left( 
\frac{m}{2\pi k_{B}T}\right) ^{-\frac{4}{2}}\pi ^{-2}\right) \\
&=&2\pi \left( \frac{m}{2\pi k_{B}T}\right) ^{=\frac{1}{2}}\left( \pi
^{-2}\right) \\
&=&2\left( \frac{2\pi k_{B}T}{m}\right) ^{\frac{1}{2}}\left( \pi ^{-1}\right)
\\
&=&\sqrt{\frac{8k_{B}T}{\pi m}}
\end{eqnarray*}%
so%
\[
\bar{v}=\sqrt{\frac{8k_{B}T}{\pi m}} 
\]%
which is also great, but not what we wanted. But it is close. This time
let's find the average value of $v^{2}.$ We hinted that the root mean
squared value would be useful. that is, $\overline{v^{2}}=$ $v_{rms}$. We
can find this like we found the average velocity

\begin{eqnarray*}
\overline{v^{2}} &=&\frac{1}{N}\int_{0}^{\infty }v^{2}4\pi N\left( \frac{m}{%
2\pi k_{B}T}\right) ^{\frac{3}{2}}v^{2}e^{-\frac{mv^{2}}{2k_{B}T}}dv \\
&=&4\pi \left( \frac{m}{2\pi k_{B}T}\right) ^{\frac{3}{2}}\int_{0}^{\infty
}v^{4}e^{-\frac{mv^{2}}{2k_{B}T}}dv
\end{eqnarray*}%
But the math is a bit more cumbersome. Let%
\[
a=\frac{m}{2k_{B}T} 
\]%
and let then%
\begin{eqnarray*}
\int x^{4}e^{-ax^{2}}dx &=&-\left. \frac{1}{8a^{\frac{5}{2}}}\left( 6\sqrt{a}%
xe^{-ax^{2}}-3\sqrt{\pi }\func{erf}\left( \sqrt{a}x\right) +4a^{\frac{3}{2}%
}x^{3}e^{-ax^{2}}\right) \allowbreak \right\vert _{0}^{\infty } \\
&=&-\frac{1}{8a^{\frac{5}{2}}}\left( -3\sqrt{\pi }\right)
\end{eqnarray*}

The quantity $\func{erf}\left( \sqrt{a}x\right) $ is called the
\textquotedblleft error function.\textquotedblright\ If you study this
function in a good mathematical book on integration you will find that 
\[
\func{erf}\left( \sqrt{a}\infty \right) =\frac{2}{\sqrt{\pi }}%
\int_{0}^{\infty }e^{-t^{2}}dt=\allowbreak 1 
\]

so%
\begin{equation}
\overline{v^{2}}=4\pi \left( \frac{m}{2\pi k_{B}T}\right) ^{\frac{3}{2}%
}\left( -\frac{1}{8\left( \frac{m}{2k_{B}T}\right) ^{\frac{5}{2}}}\left( -3%
\sqrt{\pi }\right) \right)  \nonumber
\end{equation}%
or 
\[
\overline{v^{2}}=3\frac{T}{m}k_{B} 
\]%
now recognize that $v_{rms}=\sqrt{\overline{v^{2}}}$ so%
\begin{equation}
v_{rms}=\sqrt{\frac{3k_{B}T}{m}}
\end{equation}%
and this is what we wanted. We have an expression that relates the
temperature of the gas to the $rms$ speed of the molecules of the gas.
Returning to $v_{rms}$ we see that, indeed, $v_{rms}$ is close to the
average and most probable speeds. It will show up in the next topic. So we
will need to recognize it.

%TCIMACRO{%
%\TeXButton{Question 123.14.3}{\marginpar {
%\hspace{-0.5in}
%\begin{minipage}[t]{1in}
%\small{Question 123.14.3}
%\end{minipage}
%}}}%
%BeginExpansion
\marginpar {
\hspace{-0.5in}
\begin{minipage}[t]{1in}
\small{Question 123.14.3}
\end{minipage}
}%
%EndExpansion
That we have some speeds higher than others explains why things do not
evaporate or all boil away at once.%
\[
\begin{tabular}{|c|c|c|}
\hline
& \textbf{A few rms Speeds for Gasses} &  \\ \hline
Gas & Molar Mass ($\unit{kg}/\unit{mol}$) & $v_{rms}$ at $20\unit{%
%TCIMACRO{\U{2103}}%
%BeginExpansion
{}^{\circ}{\rm C}%
%EndExpansion
}$ $\left( \unit{m}/\unit{s}\right) $ \\ \hline
H$_{2}$ & $2.02\times 10^{-3}$ & $1902$ \\ \hline
He & $4.0\times 10^{-3}$ & $1352$ \\ \hline
H$_{2}$O & $18\times 10^{-3}$ & $637$ \\ \hline
N$_{2}$ & $28\times 10^{-3}$ & $511$ \\ \hline
O$_{2}$ & $28\times 10^{-3}$ & $478$ \\ \hline
CO$_{2}$ & $44\times 10^{-3}$ & $408$ \\ \hline
\end{tabular}%
\]

\section{Mean Free Path}

%TCIMACRO{%
%\TeXButton{Question 123.14.4}{\marginpar {
%\hspace{-0.5in}
%\begin{minipage}[t]{1in}
%\small{Question 123.14.4}
%\end{minipage}
%}}}%
%BeginExpansion
\marginpar {
\hspace{-0.5in}
\begin{minipage}[t]{1in}
\small{Question 123.14.4}
\end{minipage}
}%
%EndExpansion
We started this chapter by saying how molecules interact with the walls of a
box. But the molecules really do bounce off each other. The average distance
a molecule travels before a molecule-molecule collision is called the \emph{%
mean free path}.\FRAME{fhF}{3.109in}{1.5584in}{0pt}{}{}{Figure}{\special%
{language "Scientific Word";type "GRAPHIC";maintain-aspect-ratio
TRUE;display "USEDEF";valid_file "T";width 3.109in;height 1.5584in;depth
0pt;original-width 11.1353in;original-height 5.5659in;cropleft "0";croptop
"1";cropright "1";cropbottom "0";tempfilename
'PQXXR1ZQ.wmf';tempfile-properties "XPR";}}We could give this a symbol,
let's re-use the Greek letter $\lambda $%
\begin{eqnarray*}
\lambda &=&\frac{\text{Length traveled}}{\text{Number of collisions}} \\
&=&\frac{L}{N_{\func{col}}}
\end{eqnarray*}%
This will give the average length before a collision happens.

The average length traveled is easy%
\[
L=\bar{v}\Delta t 
\]

But to predict the number of collisions is harder. To do this problem, let's
play a geometrical trick. First let's assume the molecules are spheres with
diameter $d.$ \FRAME{fhF}{4.3171in}{2.162in}{0pt}{}{}{Figure}{\special%
{language "Scientific Word";type "GRAPHIC";maintain-aspect-ratio
TRUE;display "USEDEF";valid_file "T";width 4.3171in;height 2.162in;depth
0pt;original-width 12.0252in;original-height 6.0096in;cropleft "0";croptop
"1";cropright "1";cropbottom "0";tempfilename
'PQXXR1ZR.wmf';tempfile-properties "XPR";}}We see that a collision does not
happen unless the distance between the molecules is less than or equal to $%
2d.$ We can model this interaction as a large particle of size $2d$ and many
point particles. We let the large particle move in a straight line to create
a cylindrical path. \FRAME{dtbpF}{3.1453in}{2.4578in}{0pt}{}{}{Figure}{%
\special{language "Scientific Word";type "GRAPHIC";maintain-aspect-ratio
TRUE;display "USEDEF";valid_file "T";width 3.1453in;height 2.4578in;depth
0pt;original-width 3.0995in;original-height 2.4163in;cropleft "0";croptop
"1";cropright "1";cropbottom "0";tempfilename
'PQXXR1ZS.wmf';tempfile-properties "XPR";}}All point particles in the
cylinder will collide with our large particle. The volume of such a big
cylindrical path is%
\[
%TCIMACRO{\TeXButton{V}{\ooalign{\hfil$V$\hfil\cr\kern0.1em--\hfil\cr}}}%
%BeginExpansion
\ooalign{\hfil$V$\hfil\cr\kern0.1em--\hfil\cr}%
%EndExpansion
=(\pi d^{2})L 
\]%
but the length $L$ of the cylinder will be 
\[
L=\bar{v}\Delta t 
\]%
so the volume will be%
\begin{equation}
%TCIMACRO{\TeXButton{V}{\ooalign{\hfil$V$\hfil\cr\kern0.1em--\hfil\cr}}}%
%BeginExpansion
\ooalign{\hfil$V$\hfil\cr\kern0.1em--\hfil\cr}%
%EndExpansion
=\pi d^{2}\bar{v}\Delta t
\end{equation}%
then, if $n_{V}$ is the number of molecules per unit volume then%
\begin{equation}
N_{\func{col}}=\pi d^{2}\bar{v}\Delta tn_{V}
\end{equation}%
is the number of point molecules in the cylinder.

The mean free path is the average distance traveled in a time interval $%
\Delta t$ divided by the number of collisions that occur in $\Delta t$%
\begin{eqnarray}
\lambda &=&\frac{\bar{v}\Delta t}{\pi d^{2}\bar{v}\Delta tn_{V}}  \nonumber
\\
&=&\frac{1}{\pi d^{2}n_{V}}
\end{eqnarray}%
The frequency of collisions is 
\begin{equation}
f=\pi d^{2}\bar{v}n_{V}
\end{equation}%
and the inverse of the frequency (like a period) is the mean free time.

It turns out that our simple model is off by a factor of $\sqrt{2}$ because
we assumed the molecules are stationary.%
\begin{equation}
\lambda =\frac{1}{\sqrt{2}\pi d^{2}n_{V}}
\end{equation}%
\begin{eqnarray}
f &=&\sqrt{2}\pi d^{2}\bar{v}n_{V} \\
&=&\frac{\bar{v}}{l}
\end{eqnarray}

\section{Microscopic Definition of Pressure}

Armed with a better understanding of the motions of gas particles, we want
to express macroscopic quantities in terms of microscopic effects. The
easiest to start with is pressure

\FRAME{dtbpF}{2.2911in}{2.2538in}{0in}{}{}{Figure}{\special{language
"Scientific Word";type "GRAPHIC";maintain-aspect-ratio TRUE;display
"USEDEF";valid_file "T";width 2.2911in;height 2.2538in;depth
0in;original-width 2.3062in;original-height 2.2689in;cropleft "0";croptop
"1";cropright "1";cropbottom "0";tempfilename
'PQXXR1ZT.wmf';tempfile-properties "XPR";}}Picture a particle in a cubical
box of side length $d.$

%TCIMACRO{%
%\TeXButton{Question 123.14.5}{\marginpar {
%\hspace{-0.5in}
%\begin{minipage}[t]{1in}
%\small{Question 123.14.5}
%\end{minipage}
%}}}%
%BeginExpansion
\marginpar {
\hspace{-0.5in}
\begin{minipage}[t]{1in}
\small{Question 123.14.5}
\end{minipage}
}%
%EndExpansion
As the particle approaches the wall of the box we may say that it has
velocity $v_{x}.$ The particle impacts the wall and bounces off again. Now
it travels with $-v_{x}$.

\FRAME{dtbpF}{1.7011in}{2.2494in}{0pt}{}{}{Figure}{\special{language
"Scientific Word";type "GRAPHIC";maintain-aspect-ratio TRUE;display
"USEDEF";valid_file "T";width 1.7011in;height 2.2494in;depth
0pt;original-width 2.9499in;original-height 3.9081in;cropleft "0";croptop
"1";cropright "1";cropbottom "0";tempfilename
'PQXXR1ZU.wmf';tempfile-properties "XPR";}}

We can think of why this would be. We know momentum is conserved. Before the
collision we may take the velocity of the wall to be zero, and we may
imagine the wall to have a very large mass (which it does, compared to the
particle!). If we call the particle mass $m$ and the wall mass $M$ then%
\begin{eqnarray*}
p_{i} &=&Mv_{wall_{i}}+mv_{x_{i}} \\
&=&mv_{x_{i}}
\end{eqnarray*}

but the wall will not move with any speed after the collision so $%
v_{wall_{f}}=0$ also. Then%
\begin{eqnarray*}
p_{f} &=&Mv_{wall_{f}}+mv_{xf} \\
&=&-mv_{x_{f}}
\end{eqnarray*}%
then the magnitude of the velocities must be equal and the change in
momentum must be 
\begin{eqnarray*}
\Delta p &=&mv_{x_{i}}-\left( -mv_{x_{f}}\right) \\
&=&2mv_{x_{i}}
\end{eqnarray*}

%TCIMACRO{%
%\TeXButton{Question 123.14.6}{\marginpar {
%\hspace{-0.5in}
%\begin{minipage}[t]{1in}
%\small{Question 123.14.6}
%\end{minipage}
%}}}%
%BeginExpansion
\marginpar {
\hspace{-0.5in}
\begin{minipage}[t]{1in}
\small{Question 123.14.6}
\end{minipage}
}%
%EndExpansion
Recall that the impulse from the collision of the particle on the wall would
be 
\begin{eqnarray*}
\bar{F}_{i,\text{on molecule}}\Delta t_{\text{collision}} &=&\Delta p_{x_{i}}
\\
&=&-2mv_{x_{i}}
\end{eqnarray*}%
This is the impulse momentum theorem from PH121. The time of the collision
is very short. This force is the force of the wall on the particle.

Now consider that if a particle bounces off one wall in the $x$-direction,
it must travel to the opposite wall and back before it can bounce off the
same wall again. That is, the time between bounces is roughly%
\begin{equation}
\Delta t=\frac{2d}{v_{x_{i}}}
\end{equation}

Averages are funny things, it is perfectly legal to redefine the time of our
average force to be this travel time $\Delta t.$ What does this mean? In the
first graph of the next figure we have the full definition of impulse 
\begin{equation}
\Delta p=\int_{t_{i}}^{t_{f}}\mathbf{F}dt
\end{equation}%
which says the impulse is the area under the $F$ vs. $t$ graph. \FRAME{dtbpF%
}{4.2471in}{2.8426in}{0pt}{}{}{Figure}{\special{language "Scientific
Word";type "GRAPHIC";maintain-aspect-ratio TRUE;display "USEDEF";valid_file
"T";width 4.2471in;height 2.8426in;depth 0pt;original-width
8.0869in;original-height 5.4008in;cropleft "0";croptop "1";cropright
"1";cropbottom "0";tempfilename 'PQXXR1ZV.wmf';tempfile-properties "XPR";}}%
The average force $\bar{F}$ is pictured in the second graph. It has the same
area as the first graph, but is flat, representing the average force (red
dotted line). The final graph has a $\bar{F}$ averaged over a much larger $%
\Delta t.$ The area is the same, but the magnitude of the average force is
much smaller.

This may seem useless, but at any rate, we can do this. So let's take our
average of $F$ over the time 
\[
\Delta t=\frac{2d}{v_{x_{i}}} 
\]%
which is the time it takes our particle to travel away from our wall, and
bounce back to the wall. We know that sometime within this $\Delta t$ the
collision with our wall actually occurs. The change in momentum is still 
\[
\bar{F}\Delta t=-2mv_{x_{i}} 
\]%
so we can write%
\begin{eqnarray*}
\bar{F}\frac{2d}{v_{x_{i}}} &=&-2mv_{x_{i}} \\
\bar{F} &=&-\frac{mv_{x_{i}}^{2}}{d}
\end{eqnarray*}%
So far our force has been the force of the wall on the particle, but we know
by Newton's third law that the particle force must be equal and opposite. 
\[
\bar{F}_{i,\text{on molecule}}=-\bar{F}_{i} 
\]%
then%
\[
\bar{F}_{i}=\frac{mv_{x_{i}}^{2}}{d} 
\]%
for one particle (molecule or atom).

For many particles%
\begin{eqnarray*}
\bar{F} &=&\dsum\limits_{i=1}^{N}\frac{mv_{x_{i}}^{2}}{d} \\
&=&\frac{m}{d}\dsum\limits_{i=1}^{N}v_{x_{i}}^{2}
\end{eqnarray*}%
We have found the average force on the container wall!

%TCIMACRO{%
%\TeXButton{Question 123.14.7}{\marginpar {
%\hspace{-0.5in}
%\begin{minipage}[t]{1in}
%\small{Question 123.14.7}
%\end{minipage}
%}}}%
%BeginExpansion
\marginpar {
\hspace{-0.5in}
\begin{minipage}[t]{1in}
\small{Question 123.14.7}
\end{minipage}
}%
%EndExpansion
Now remember how to take an average%
\[
\bar{x}=\frac{1}{N}\dsum\limits_{i=1}^{N}x_{i} 
\]%
so if we were to calculate the average squared velocity%
\[
\overline{v_{x_{i}}^{2}}=\frac{1}{N}\dsum\limits_{i=1}^{N}v_{x_{i}}^{2} 
\]%
We almost have this in our force equation. Let's put in the missing parts%
\begin{eqnarray*}
\bar{F} &=&\frac{m}{d}\frac{N}{N}\dsum\limits_{i=1}^{N}v_{x_{i}}^{2} \\
&=&\frac{m}{d}N\overline{v_{x_{i}}^{2}}
\end{eqnarray*}

Going back to a single molecule, we remember that%
\[
v_{i}^{2}=v_{x_{i}}^{2}+v_{y_{i}}^{2}+v_{z_{i}}^{2} 
\]

then if we take the averages of each piece like we did with $v_{x_{i}}^{2}$ 
\[
\overline{v^{2}}=\overline{v_{x}^{2}}+\overline{v_{y}^{2}}+\overline{%
v_{z}^{2}} 
\]

There is no real reason to prefer the $x$-direction, we could have chosen
the $y$-direction and had the same value. Then%
\[
v_{x_{i}}^{2}=v_{y_{i}}^{2}=v_{z_{i}}^{2} 
\]%
%TCIMACRO{%
%\TeXButton{Question 123.14.8}{\marginpar {
%\hspace{-0.5in}
%\begin{minipage}[t]{1in}
%\small{Question 123.14.8}
%\end{minipage}
%}}}%
%BeginExpansion
\marginpar {
\hspace{-0.5in}
\begin{minipage}[t]{1in}
\small{Question 123.14.8}
\end{minipage}
}%
%EndExpansion
and%
\[
\overline{v^{2}}=3v_{x_{i}}^{2} 
\]%
So our force on the wall is%
\begin{eqnarray*}
\bar{F} &=&\frac{m}{d}N\overline{v_{x_{i}}^{2}} \\
&=&\frac{m}{d}N\frac{1}{3}\overline{v^{2}} \\
&=&\frac{mN}{3d}\overline{v^{2}} \\
&=&\frac{N}{3d}\left( m\overline{v^{2}}\right)
\end{eqnarray*}%
%TCIMACRO{%
%\TeXButton{Question 123.14.9}{\marginpar {
%\hspace{-0.5in}
%\begin{minipage}[t]{1in}
%\small{Question 123.14.9}
%\end{minipage}
%}}}%
%BeginExpansion
\marginpar {
\hspace{-0.5in}
\begin{minipage}[t]{1in}
\small{Question 123.14.9}
\end{minipage}
}%
%EndExpansion
Finally we can find the pressure. We know%
\begin{eqnarray*}
P &=&\frac{F}{A} \\
&=&\frac{F}{d^{2}} \\
&=&\frac{1}{d^{2}}\frac{N}{3d}\left( m\overline{v^{2}}\right) \\
&=&\frac{1}{d^{3}}\frac{N}{3}\left( m\overline{v^{2}}\right) \\
&=&\frac{1}{3}\frac{N}{%
%TCIMACRO{\TeXButton{V}{\ooalign{\hfil$V$\hfil\cr\kern0.1em--\hfil\cr}}}%
%BeginExpansion
\ooalign{\hfil$V$\hfil\cr\kern0.1em--\hfil\cr}%
%EndExpansion
}\left( m\overline{v^{2}}\right) \\
&=&\frac{2}{3}\frac{N}{%
%TCIMACRO{\TeXButton{V}{\ooalign{\hfil$V$\hfil\cr\kern0.1em--\hfil\cr}}}%
%BeginExpansion
\ooalign{\hfil$V$\hfil\cr\kern0.1em--\hfil\cr}%
%EndExpansion
}\left( \frac{1}{2}m\overline{v^{2}}\right)
\end{eqnarray*}

\begin{equation}
P=\frac{2}{3}\frac{N}{%
%TCIMACRO{\TeXButton{V}{\ooalign{\hfil$V$\hfil\cr\kern0.1em--\hfil\cr}}}%
%BeginExpansion
\ooalign{\hfil$V$\hfil\cr\kern0.1em--\hfil\cr}%
%EndExpansion
}\left( \frac{1}{2}m\overline{v^{2}}\right)
\end{equation}%
and we see that the pressure of the gas is proportional to the number of
molecules per unit volume and to the average kinetic energy of the
molecules! In fact this shows some of the things we have learned. \ For
example, $P$ is inversely proportional to $%
%TCIMACRO{\TeXButton{V}{\ooalign{\hfil$V$\hfil\cr\kern0.1em--\hfil\cr}}}%
%BeginExpansion
\ooalign{\hfil$V$\hfil\cr\kern0.1em--\hfil\cr}%
%EndExpansion
$!

We already knew most of this. But now we have a mathematical form so we can
do calculations! We can do this for our other thermodynamic quantities. We
will start by finally defining temperature in our next lecture.

\chapter{What is Temperature? and Values for $C_{V}$ and $C_{P}$}

Finally we can define temperature in terms of the basic physics of the atoms
and molecules that make up the substance.

%TCIMACRO{%
%\TeXButton{Fundamental Concepts}{\hspace{-1.3in}{\Large Fundamental Concepts\vspace{0.25in}}}}%
%BeginExpansion
\hspace{-1.3in}{\Large Fundamental Concepts\vspace{0.25in}}%
%EndExpansion

\begin{itemize}
\item Temperature is related directly to the average kinetic energy of the
molecules of a gas

\item We call a way the molecule can move a \emph{degree of freedom.}

\item For an ideal gas $C_{V}=\frac{3}{2}R$ and $C_{P}=\frac{5}{2}R$

\item For an ideal gas $E_{int}=\frac{3}{2}k_{B}T$
\end{itemize}

\section{Temperature}

%TCIMACRO{%
%\TeXButton{Question 123.15.1}{\marginpar {
%\hspace{-0.5in}
%\begin{minipage}[t]{1in}
%\small{Question 123.15.1}
%\end{minipage}
%}}}%
%BeginExpansion
\marginpar {
\hspace{-0.5in}
\begin{minipage}[t]{1in}
\small{Question 123.15.1}
\end{minipage}
}%
%EndExpansion
%TCIMACRO{%
%\TeXButton{Question 123.15.2}{\marginpar {
%\hspace{-0.5in}
%\begin{minipage}[t]{1in}
%\small{Question 123.15.2}
%\end{minipage}
%}}}%
%BeginExpansion
\marginpar {
\hspace{-0.5in}
\begin{minipage}[t]{1in}
\small{Question 123.15.2}
\end{minipage}
}%
%EndExpansion
%TCIMACRO{%
%\TeXButton{Equation on Board}{\marginpar {
%\hspace{-0.5in}
%\begin{minipage}[t]{1in}
%\small{Equation on Board}
%\end{minipage}
%}}}%
%BeginExpansion
\marginpar {
\hspace{-0.5in}
\begin{minipage}[t]{1in}
\small{Equation on Board}
\end{minipage}
}%
%EndExpansion
Last lecture we found that%
\[
P=\frac{2}{3}\frac{N}{%
%TCIMACRO{\TeXButton{V}{\ooalign{\hfil$V$\hfil\cr\kern0.1em--\hfil\cr}}}%
%BeginExpansion
\ooalign{\hfil$V$\hfil\cr\kern0.1em--\hfil\cr}%
%EndExpansion
}\left( \frac{1}{2}m\overline{v^{2}}\right) 
\]%
We can rearrange this to look a little like the ideal gas law. 
\[
P%
%TCIMACRO{\TeXButton{V}{\ooalign{\hfil$V$\hfil\cr\kern0.1em--\hfil\cr}}}%
%BeginExpansion
\ooalign{\hfil$V$\hfil\cr\kern0.1em--\hfil\cr}%
%EndExpansion
=\frac{2}{3}N\left( \frac{1}{2}m\overline{v^{2}}\right) 
\]

But temperature does not appear in this equation. Let's recall that we can
write the ideal gas law as 
\[
P%
%TCIMACRO{\TeXButton{V}{\ooalign{\hfil$V$\hfil\cr\kern0.1em--\hfil\cr}}}%
%BeginExpansion
\ooalign{\hfil$V$\hfil\cr\kern0.1em--\hfil\cr}%
%EndExpansion
=Nk_{B}T 
\]%
If we set these two equations equal to each other%
\[
Nk_{B}T=\frac{2}{3}N\left( \frac{1}{2}m\overline{v^{2}}\right) 
\]%
then%
\begin{equation}
T=\frac{2}{3}\frac{1}{k_{B}}\left( \frac{1}{2}m\overline{v^{2}}\right)
\end{equation}%
This is fantastic! We have finally defined temperature. Temperature is
related directly to the average kinetic energy of the molecules! The average
kinetic energy is 
\[
\bar{K}_{mol}=\frac{1}{2}m\overline{v^{2}}=\frac{3}{2}k_{B}T 
\]%
Note that in each direction we should have 
\[
\bar{K}_{x}=\frac{1}{2}m\overline{v_{x}^{2}}=\frac{1}{2}k_{B}T 
\]%
so%
\[
\bar{K}_{y}=\frac{1}{2}k_{B}T 
\]%
and 
\[
\bar{K}_{z}=\frac{1}{2}k_{B}T 
\]%
So in each direction the molecules have 
\begin{equation}
\bar{K}_{i}=\frac{1}{2}k_{B}T
\end{equation}%
We call a way the molecule can move a \emph{degree of freedom. }Our
molecules can move in the $x,$ $y,$ and $z,$ direction, so our molecules
have three degrees of freedom. Each degree of freedom contributes $\frac{1}{2%
}k_{B}T$ worth of energy.

Of course, we only allow translational energy in our formulation for ideal
gasses, but for more complex molecules we could have rotational energy
(another way to move) and vibrational energy (yet another way to move) etc.
And each way the molecule can move is another degree of freedom would
contribute $\frac{1}{2}k_{B}T.$ But remember, these more complex molecules
would not be ideal gasses.

For the collection of $N$ molecules, we have all together 
\begin{eqnarray}
K &=&N\left( \frac{3}{2}k_{B}T\right) \\
&=&\frac{3}{2}Nk_{B}T \\
&=&\frac{3}{2}nRT  \nonumber
\end{eqnarray}%
and for our ideal gas we see that the internal energy depends only on the
temperature.

This is only true under our assumption that the molecules or atoms are
structureless. So this works well for actual monotonic gases.

%TCIMACRO{%
%\TeXButton{Question 123.15.3}{\marginpar {
%\hspace{-0.5in}
%\begin{minipage}[t]{1in}
%\small{Question 123.15.3}
%\end{minipage}
%}}}%
%BeginExpansion
\marginpar {
\hspace{-0.5in}
\begin{minipage}[t]{1in}
\small{Question 123.15.3}
\end{minipage}
}%
%EndExpansion
Let's do a problem. We found earlier by borrowing results from Statistical
Mechanics that 
\[
v_{rms}=\sqrt{\frac{3k_{B}T}{m}} 
\]%
Let's show this from knowing that each degree of freedom contributes $\frac{1%
}{2}k_{B}T.$ We just found that 
\[
\bar{K}=\frac{K}{N}=\frac{3}{2}k_{B}T 
\]%
and we know that 
\[
\bar{K}=\frac{1}{2}m\overline{v^{2}} 
\]%
Setting these equal gives%
\begin{eqnarray*}
\frac{3}{2}k_{B}T &=&\frac{1}{2}m\overline{v^{2}} \\
3Nk_{B}T &=&m\overline{v^{2}} \\
\frac{3Nk_{B}T}{m} &=&\overline{v^{2}}
\end{eqnarray*}%
so

\[
v_{rms}=\sqrt{\overline{v^{2}}}=\sqrt{\frac{3k_{B}T}{m}} 
\]%
which is just what we found before, but this was a much easier and more
understandable way to solve the problem. This shows the power of the simple
idea of the ideal gas approximation.

The rms speed tells us that if the molecule or atom is tiny (has little
mass) it must move very fast. If it is larger, it will move more slowly. 
%TCIMACRO{%
%\TeXButton{Question 123.15.4}{\marginpar {
%\hspace{-0.5in}
%\begin{minipage}[t]{1in}
%\small{Question 123.15.4}
%\end{minipage}
%}}}%
%BeginExpansion
\marginpar {
\hspace{-0.5in}
\begin{minipage}[t]{1in}
\small{Question 123.15.4}
\end{minipage}
}%
%EndExpansion

\section{$C_{V}$ and $C_{P}$ values}

In the problems we have done so far, we had to look up the $C_{V}$ and $%
C_{P}.$ But they must come from somewhere. We should discover why they are
what they are. To do this, remember for an ideal gas, we only have
translational kinetic energy, so it must be true that 
\[
E_{int}=K_{tran}=\frac{3}{2}nRT 
\]

Now let's use our special processes. If we take the system through a process
at constant volume,%
\[
Q=\Delta E_{int} 
\]%
and%
\[
Q=nC_{V}\Delta T 
\]%
so%
\[
\Delta E_{int}=nC_{V}\Delta T 
\]%
We can write this as 
\begin{eqnarray*}
E_{int_{f}}-E_{int_{i}} &=&nC_{V}\left( T_{f}-T_{i}\right) \\
&=&nC_{V}T_{f}-nC_{V}T_{i}
\end{eqnarray*}%
this strongly suggests that if $C_{V}$ is constant%
\[
E_{int}=nC_{V}T 
\]

We can solve for $C_{V}$%
\[
C_{V}=\frac{1}{n}\frac{\Delta E_{int}}{\Delta T} 
\]%
and for very small changes in temperature%
\[
C_{V}=\frac{1}{n}\frac{dE_{int}}{dT} 
\]%
But for an ideal gas the internal energy is kinetic energy so we know that 
\[
E_{int}=\frac{3}{2}nRT 
\]%
So we can easily find $\frac{dE_{int}}{dT}$%
\[
\frac{dE_{int}}{dT}=\frac{3}{2}nR 
\]%
Then our value for $C_{V}$ is%
\[
C_{V}=\frac{1}{n}\frac{dE_{int}}{dT} 
\]
\begin{eqnarray*}
C_{V} &=&\frac{1}{n}\frac{3}{2}nR \\
&=&\frac{3}{2}R
\end{eqnarray*}%
Knowing $R=8.314\frac{\unit{J}}{\unit{mol}\unit{K}},$ the numerical value is%
\begin{eqnarray*}
C_{V} &=&\frac{3}{2}8.314\frac{\unit{J}}{\unit{mol}\unit{K}} \\
&=&12.\,\allowbreak 471\frac{\unit{J}}{\unit{mol}\unit{K}}
\end{eqnarray*}%
for all monotonic gasses. It turns out that this is a very good
approximation.

Finding $C_{P}$ is easy because we know that 
\[
C_{P}=C_{V}+R 
\]%
so 
\[
C_{P}=\frac{5}{2}R 
\]

For adiabatic processes we need the ratio of $C_{P}$ to $C_{V}$ or $\gamma .$
We now know that 
\[
C_{V}=\frac{3}{2}R 
\]%
\[
C_{P}=\frac{5}{2}R 
\]%
and we may take the ratio of these values%
\begin{eqnarray*}
\gamma &=&\frac{C_{P}}{C_{V}} \\
&=&\frac{\frac{5}{2}R}{\frac{3}{2}R} \\
&=&\frac{5}{3}
\end{eqnarray*}

This works well for monotonic gasses, but fails badly for more complex
gasses. So we see our ideal gas formulation starts to break down with more
complex gasses at this point. To go farther, we would need to include the
rotational and vibrational energy of the molecules. Still, we have come a
long ways with our simple ideal gas assumptions!

Note that for solids and liquids $\Delta 
%TCIMACRO{\TeXButton{V}{\ooalign{\hfil$V$\hfil\cr\kern0.1em--\hfil\cr}}}%
%BeginExpansion
\ooalign{\hfil$V$\hfil\cr\kern0.1em--\hfil\cr}%
%EndExpansion
$ is very small so very little work is done. This means that $C_{P}\approx
C_{V}$ which is why we we could get away with only one value for the molar
heat capacity in the tables for solids and liquids.

\section{The Equipartiction of Energy}

We can extend our ideal gas model a little by using what we know about
degrees of freedom. 
%TCIMACRO{%
%\TeXButton{Movie of Equapartition}{\marginpar {
%\hspace{-0.5in}
%\begin{minipage}[t]{1in}
%\small{Movie of Equapartition}
%\end{minipage}
%}} }%
%BeginExpansion
\marginpar {
\hspace{-0.5in}
\begin{minipage}[t]{1in}
\small{Movie of Equapartition}
\end{minipage}
}
%EndExpansion
Remember that we found that for each degree of freedom the internal energy
was 
\begin{equation}
E_{i}=\frac{1}{2}k_{B}T
\end{equation}%
we found for an ideal monotonic gas that the internal energy was 
\begin{equation}
E_{int}=3E_{i}=\frac{3}{2}k_{B}T
\end{equation}

\FRAME{dtbpF}{1.7167in}{1.6838in}{0pt}{}{}{Figure}{\special{language
"Scientific Word";type "GRAPHIC";maintain-aspect-ratio TRUE;display
"USEDEF";valid_file "T";width 1.7167in;height 1.6838in;depth
0pt;original-width 3.5864in;original-height 3.5172in;cropleft "0";croptop
"1";cropright "1";cropbottom "0";tempfilename
'PQXXR1ZW.wmf';tempfile-properties "XPR";}}where each $E_{i}$ came from a
translational degree of freedom. But a diatomic molecule has several more
degrees of freedom. It can rotate about any of the axes.\FRAME{dtbpF}{%
1.7383in}{1.7045in}{0pt}{}{}{Figure}{\special{language "Scientific
Word";type "GRAPHIC";maintain-aspect-ratio TRUE;display "USEDEF";valid_file
"T";width 1.7383in;height 1.7045in;depth 0pt;original-width
3.5864in;original-height 3.5172in;cropleft "0";croptop "1";cropright
"1";cropbottom "0";tempfilename 'PQXXR1ZX.wmf';tempfile-properties "XPR";}}

Here rotation about the $y$ axis does not contribute significantly because
the moment of inertia of a sphere (we will take the atom to be roughly
spherical) about it's axis is 
\begin{equation}
\mathbb{I}=\frac{2}{5}mr^{2}
\end{equation}%
where $m$ is the mass of the atom. Most of the mass is centered in the
nucleus (proton mass $=1.67\times 10^{-27}\unit{kg},$ electron mass $%
=9.11\times 10^{-31}\unit{kg})$, which has a radius of about $r=1.7\times
10^{-5}\unit{%
%TCIMACRO{\U{212b}}%
%BeginExpansion
\text{\AA}%
%EndExpansion
}$ is the radius of the atom. The moment of inertia for rotation about the
center of the two mass system is 
\begin{equation}
\mathbb{I}=\sum_{i}m_{i}R_{i}^{2}
\end{equation}%
where $R$ is the distance from the center of mass. For diatomic hydrogen, $R=%
\frac{1}{2}1.06\unit{%
%TCIMACRO{\U{212b}}%
%BeginExpansion
\text{\AA}%
%EndExpansion
}$ so we can see that the rotation about the $y$-axis is not very important,
so we are left with three translational and two rotational degrees of
freedom. This gives%
\begin{equation}
E_{int}=\left( \frac{3}{2}k_{B}T\right) _{trans}+\left( \frac{2}{2}%
k_{B}T\right) _{rot}=\frac{5}{2}k_{B}T
\end{equation}%
Writing this in molar terms%
\begin{equation}
E_{int}=\frac{5}{2}nRT
\end{equation}%
then 
\begin{equation}
C_{V}=\frac{1}{n}\frac{dE_{int}}{dT}=\frac{5}{2}R
\end{equation}%
and 
\begin{equation}
C_{P}=C_{V}+R
\end{equation}%
gives 
\begin{equation}
C_{P}=\frac{7}{2}R
\end{equation}

\subsection{Measurements of $C_{V}$ and $C_{P}$ at STP agree with these
values.}

BUT WAIT, we did not include vibration! The atoms are bond together with an
electrical attraction that acts quite like a spring force. So vibration
along the axis is possible and we need to add in one more degree of freedom.
We also have potential energy involved for a spring force, so we expect an
additional degree of freedom for vibration. \FRAME{dtbpF}{1.5298in}{1.4996in%
}{0pt}{}{}{Figure}{\special{language "Scientific Word";type
"GRAPHIC";maintain-aspect-ratio TRUE;display "USEDEF";valid_file "T";width
1.5298in;height 1.4996in;depth 0pt;original-width 3.5864in;original-height
3.5172in;cropleft "0";croptop "1";cropright "1";cropbottom "0";tempfilename
'PQXXR1ZY.wmf';tempfile-properties "XPR";}}When we add all these up, we get 
\begin{equation}
E_{int}=\left( \frac{3}{2}k_{B}T\right) _{trans}+\left( \frac{2}{2}%
k_{B}T\right) _{rot}+\left( \frac{2}{2}k_{B}T\right) _{vib}=\frac{7}{2}k_{B}T
\end{equation}%
which gives 
\begin{equation}
C_{V}=\frac{7}{2}R
\end{equation}%
and 
\begin{equation}
C_{P}=\frac{9}{2}R
\end{equation}

We should pause to ask, what values do we use? for diatomic gasses, is $%
C_{V}=\frac{7}{2}R$ all the time?

It turns out that when energy is added to a collection of molecules, it does
not pick randomly from the degrees of freedom. \FRAME{dhF}{3.2569in}{2.4422in%
}{0in}{}{}{Figure}{\special{language "Scientific Word";type
"GRAPHIC";maintain-aspect-ratio TRUE;display "USEDEF";valid_file "T";width
3.2569in;height 2.4422in;depth 0in;original-width 3.2119in;original-height
2.4007in;cropleft "0";croptop "1";cropright "1";cropbottom "0";tempfilename
'PQXXR1ZZ.wmf';tempfile-properties "XPR";}}

At low temperature, the translational degrees of freedom are preferred.
Then, as the temperature rises, the rotational degrees of freedom are
filled. Finally the vibrational degrees of freedom are used. The figure
shows this relationship for diatomic Hydrogen. Note that there are plateaus
at each of our values of $C_{V}$ that we found ($\frac{3}{2}R,\frac{5}{2}R,%
\frac{7}{2}R$).

We have talked about the simplest of molecules. If we had a more complex
molecule, there would be more complex rotational and vibrational degrees of
freedom. That is why for larger molecules the values of $C_{V}$ in table in
the book don't follow a simple fraction of $R.$

\section{Quantization}

We have come to the end of what classical theory can explain. We know that
different degrees of freedom are filled in groups, but we don't know why. We
will only hint at the reason, but it has to do with wave motion that we will
soon study!\FRAME{dhF}{3.7178in}{3.8787in}{0in}{}{}{Figure}{\special%
{language "Scientific Word";type "GRAPHIC";maintain-aspect-ratio
TRUE;display "USEDEF";valid_file "T";width 3.7178in;height 3.8787in;depth
0in;original-width 3.6703in;original-height 3.8311in;cropleft "0";croptop
"1";cropright "1";cropbottom "0";tempfilename
'PQXXR100.wmf';tempfile-properties "XPR";}}

Quantum theory tells us that atoms can be described as waves under boundary
conditions. Like a string with fixed ends, these atomic waves have quantized
frequencies. The energy of the molecule is proportional to the frequency in
quantum theory, so quantum mechanics tells us that the energy of the
molecule will be quantized. The figure shows an energy level diagram for a
molecule. Higher energy is in the $y$ direction. The vibrational and
rotational states are marked. (the maroon arrows show transitions from
higher to lower states. These transitions create radiation).

The lowest allowed state is called the \emph{ground state}. It is analogous
to the fundamental frequency (something we will study soon!). At low
temperatures the molecule only has enough energy to populate the lower
states, but when the temperature rises, there is enough energy transfer due
to collisions to push molecules into the rotational states. With higher
temperatures, you will see the collisions transfer molecules to the higher
vibrational states.

\section{Molar specific heat of solids}

Let's do a problem. Let's find $C_{V}$ for a solid, and compare it to the
specific heat of elemental solids.

We can view the atoms of solids as having a structure of springs and atoms.

\FRAME{dtbpF}{2.623in}{2.629in}{0pt}{}{}{Figure}{\special{language
"Scientific Word";type "GRAPHIC";maintain-aspect-ratio TRUE;display
"USEDEF";valid_file "T";width 2.623in;height 2.629in;depth
0pt;original-width 4.8611in;original-height 4.8732in;cropleft "0";croptop
"1";cropright "1";cropbottom "0";tempfilename
'PQXXR101.wmf';tempfile-properties "XPR";}}We can see that we should have
three translational degrees of freedom for each atom. We also have three
degrees of freedom associated with the potential energy from the spring-like
bonding forces. Recall that%
\begin{equation}
E_{x}=\frac{1}{2}mv_{x}^{2}-\frac{1}{2}kx^{2}
\end{equation}%
and we have and $E_{y}$ and $E_{z}$ as well.

From equipartition of energy we have 
\begin{equation}
E_{i}=6\left( \frac{1}{2}k_{B}T\right)
\end{equation}%
per atom! So we have%
\begin{eqnarray}
E_{int} &=&3Nk_{B}T \\
&=&3nRT
\end{eqnarray}%
and then for solids 
\begin{equation}
C_{V}=\frac{1}{n}\frac{dE_{int}}{dT}=3R
\end{equation}%
numerically this is 
\begin{eqnarray*}
C_{V} &=&3\times 8.314\frac{\unit{J}}{\unit{mol}\unit{K}} \\
&=&24.\,\allowbreak 942\frac{\unit{J}}{\unit{mol}\unit{K}}
\end{eqnarray*}%
This is the \emph{DuLong-Petit law}. Here is our table of specific heat
values again.%
\[
\begin{tabular}{|l|l|l|}
\hline
\textbf{Substance} & $c\left( \frac{\unit{J}}{\unit{kg}\unit{K}}\right) $ & $%
C_{V}\left( \frac{\unit{J}}{\unit{mol}\unit{K}}\right) $ \\ \hline
Aluminum & 900 & 24.3 \\ \hline
Copper & 385 & 24.4 \\ \hline
Iron & 449 & 25.1 \\ \hline
Gold & 129 & 25.4 \\ \hline
Lead & 128 & 26.5 \\ \hline
Ice & 2090 & 37.6 \\ \hline
Mercury & 140 & 28.1 \\ \hline
Water & 4190 & 75.4 \\ \hline
\end{tabular}%
\]%
we see that our simple analysis did not do too bad! For elemental solids we
get about the right number.

This law also breaks down for very cold temperatures where we need even more
quantum mechanics to explain what happens. Einstein and Debye both
contributed to explaining this departure, but their work is beyond this
freshman course. This is a topic for a solid state physics course which is
an elective that you might consider if you are interested is seeing a more
complete solution.\FRAME{dhFU}{2.5829in}{2.2166in}{0pt}{\Qcb{Debye and
Einstein models. (Image in the Public Domain courtesy Fr\'{e}d\'{e}ric Perez)%
}}{}{Figure}{\special{language "Scientific Word";type
"GRAPHIC";maintain-aspect-ratio TRUE;display "USEDEF";valid_file "T";width
2.5829in;height 2.2166in;depth 0pt;original-width 9.7222in;original-height
8.3333in;cropleft "0";croptop "1";cropright "1";cropbottom "0";tempfilename
'PQXXR102.bmp';tempfile-properties "XPR";}}

This almost completes our study of ideal gasses. Next lecture we will use
our microscopic view to explain energy transfer, and we will tackle the
second law of thermodynamics. Following that we will use what we know to
build idealized engines and refrigerators.

\chapter{Microscopic Energy Transfer Considerations}

%TCIMACRO{%
%\TeXButton{Fundamental Concepts}{\hspace{-1.3in}{\Large Fundamental Concepts\vspace{0.25in}}}}%
%BeginExpansion
\hspace{-1.3in}{\Large Fundamental Concepts\vspace{0.25in}}%
%EndExpansion

\begin{enumerate}
\item Microscopic idea of Energy Transfer

\item Microscopic idea of disorder

\item Entropy
\end{enumerate}

\section{Microscopic View of Energy Transfer}

%TCIMACRO{%
%\TeXButton{Question 123.16.1}{\marginpar {
%\hspace{-0.5in}
%\begin{minipage}[t]{1in}
%\small{Question 123.16.1}
%\end{minipage}
%}}}%
%BeginExpansion
\marginpar {
\hspace{-0.5in}
\begin{minipage}[t]{1in}
\small{Question 123.16.1}
\end{minipage}
}%
%EndExpansion
Suppose we have a chamber divided into two volumes. The volumes are
separated by a membrane. If they have different temperatures, we expect that
energy will transfer, but how?

Think of a gym. In our gyms, we have large nets that divide courts to keep
balls from passing from one court to another. Now think of millions of
basket balls in each side of the gym. The basket balls all are moving. Some
of the basket balls will strike the net. If a ball on the other side strikes
the net at the same place at the same time, the collision will transfer
energy. This would be a conservation of momentum and energy problem The
slower ball would gain energy, the faster ball would loose energy. \FRAME{dhF%
}{2.9914in}{2.2329in}{0pt}{}{}{Figure}{\special{language "Scientific
Word";type "GRAPHIC";maintain-aspect-ratio TRUE;display "USEDEF";valid_file
"T";width 2.9914in;height 2.2329in;depth 0pt;original-width
2.9473in;original-height 2.1923in;cropleft "0";croptop "1";cropright
"1";cropbottom "0";tempfilename 'PQXXR103.wmf';tempfile-properties "XPR";}}

This is analogous to our gasses in chambers. The molecules, like the
basketballs, transfer energy by collision.

The internal energy for each side of the chamber depends on both $N$ and $T$

\begin{eqnarray*}
E_{1i} &=&\frac{3}{2}N_{1}k_{B}T_{1i} \\
E_{2i} &=&\frac{3}{2}N_{2}k_{B}T_{2i}
\end{eqnarray*}

The total energy is 
\[
E_{tot}=E_{1i}+E_{2i} 
\]%
if our container does not change volume (no work done) and is insolated ($%
Q=0 $) then $E_{tot}$ will not change. But eventually the two sides come to
the same temperature%
\[
T_{f}=T_{1f}=T_{2f} 
\]%
this happens when the average energy of the molecules (or basketballs) are
the same. On average, neither side gains energy due to the collisions on the
membrane (net). We can rewrite our energy equation for the final condition
as 
\[
\bar{E}_{tot}=\bar{E}_{1i}=\bar{E}_{2f} 
\]%
because the average energies will all be the same. Using 
\begin{eqnarray*}
E_{1f} &=&N_{1}\frac{3}{2}k_{B}T_{1f}=N_{1}\bar{E}_{1} \\
E_{2f} &=&N_{2}\frac{3}{2}k_{B}T_{2f}=N_{2}\bar{E}_{2}
\end{eqnarray*}%
we can rewrite this as an energy per molecule%
\[
\frac{E_{tot}}{N_{1}+N_{2}}=\frac{E_{1f}}{N_{1}}=\frac{E_{2f}}{N_{2}} 
\]%
which gives a set of expressions for the final energy of each side%
\begin{eqnarray*}
E_{1f} &=&\frac{N_{1}E_{tot}}{N_{1}+N_{2}} \\
E_{2f} &=&\frac{N_{2}E_{tot}}{N_{1}+N_{2}}
\end{eqnarray*}%
and we can verify that 
\begin{eqnarray*}
E_{tot} &=&E_{1f}+E_{2f} \\
&=&\frac{N_{1}E_{tot}}{N_{1}+N_{2}}+\frac{N_{2}E_{tot}}{N_{1}+N_{2}} \\
&=&E_{tot}
\end{eqnarray*}

Let's take what we have learned and mathematically expressed as macroscopic
formula and show it at a microscopic level as a problem example.

Let's show that if $Q_{tot}=0,$ and no work is done, then%
\[
Q_{1}=-Q_{2} 
\]%
This is a calorimetry problem.

Start with the first law, 
\begin{eqnarray*}
Q_{1} &=&\Delta E_{1}=E_{1f}-E_{1i} \\
Q_{2} &=&\Delta E_{2}=E_{2f}-E_{2i}
\end{eqnarray*}%
since no work is done. Let's take $Q_{1}$ first 
\[
Q_{1}=E_{1f}-E_{1i} 
\]%
since%
\[
E_{1f}=\frac{N_{1}E_{tot}}{N_{1}+N_{2}} 
\]
and%
\[
E_{tot}=E_{1i}+E_{2i} 
\]%
then 
\[
Q_{1}=\frac{N_{1}\left( E_{1i}+E_{2i}\right) }{N_{1}+N_{2}}-E_{1i} 
\]%
Then, applying some algebra, 
\begin{eqnarray*}
Q_{1} &=&\frac{N_{1}\left( E_{1i}+E_{2i}\right) }{N_{1}+N_{2}}-E_{1i}\frac{%
N_{1}+N_{2}}{N_{1}+N_{2}} \\
&=&\frac{N_{1}E_{1i}+N_{1}E_{2i}-E_{1i}N_{1}-E_{1i}N_{2}}{N_{1}+N_{2}} \\
&=&\frac{N_{1}E_{2i}-E_{1i}N_{2}}{N_{1}+N_{2}}
\end{eqnarray*}%
Expanding this out gives%
\begin{eqnarray*}
Q_{1} &=&\frac{N_{1}E_{2i}}{N_{1}+N_{2}}-\frac{E_{1i}N_{2}}{N_{1}+N_{2}} \\
&=&\frac{N_{1}E_{2i}}{N_{1}+N_{2}}-\frac{E_{1i}N_{2}}{N_{1}+N_{2}}+\frac{%
N_{2}E_{2i}}{N_{1}+N_{2}}-\frac{N_{2}E_{2i}}{N_{1}+N_{2}} \\
&=&\frac{N_{1}E_{2i}}{N_{1}+N_{2}}+\frac{N_{2}E_{2i}}{N_{1}+N_{2}}-\frac{%
E_{1i}N_{2}}{N_{1}+N_{2}}-\frac{N_{2}E_{2i}}{N_{1}+N_{2}} \\
&=&\frac{N_{1}E_{2i}+N_{2}E_{2i}}{N_{1}+N_{2}}-\frac{E_{1i}N_{2}}{N_{1}+N_{2}%
}-\frac{N_{2}E_{2i}}{N_{1}+N_{2}}
\end{eqnarray*}%
The first term is just $E_{2i},$ and using $E_{tot}=E_{1i}+E_{2i}$ again we
see 
\begin{eqnarray*}
Q_{1} &=&E_{2i}-\left( \frac{\left( E_{1i}N_{2}+N_{2}E_{2i}\right) }{%
N_{1}+N_{2}}\right) \\
&=&E_{2i}-\left( \frac{N_{2}\left( E_{1i}+E_{2i}\right) }{N_{1}+N_{2}}\right)
\\
&=&E_{2i}-\left( \frac{N_{2}\left( E_{tot}\right) }{N_{1}+N_{2}}\right) \\
&=&E_{2i}-E_{2f} \\
&=&-Q_{2}
\end{eqnarray*}%
which is what we expect. We have shown our calorimetry equation using the
ideas of microscopic energy transport for ideal gasses$.$

\section{The arrow of time.}

Why does time go only one way? The answer to this question may surprise you. 
\emph{We don't know.} There is nothing in our laws of motion that tells us
that time should only go one way. Take motion under constant acceleration.%
\[
y_{f}=y_{i}+v_{o}\Delta t+\frac{1}{2}a\Delta t^{2} 
\]%
and let's start with $t_{i}=0$ so 
\[
y_{f}=y_{i}+v_{o}t_{f}+\frac{1}{2}at_{f}^{2} 
\]%
this equation describes the motion of an object under constant acceleration,
like that of gravity near the Earth's surface. But nothing in the equation
tells us that time must take on ever grater more positive values.

The truth is that there is only one physical law that tells us that time
goes only one way, and that law is an empirical law that says essentially
\textquotedblleft time only goes one way.\textquotedblright\ We experience
time going one way, so we developed a law that says so. This is not a
terribly convincing argument. But this is the state of physics.

Some processes (really all processes) can't actually run forward or
backward. Idealized processes can be run in reverse, but not real process.
We call processes that can only work one direction \emph{irreversible
processes}. If we shot a movie of an irreversible process and ran the movie
backwards, it would be funny, because we would know that the reverse process
can't happen.

Let's take an example. Suppose our process is dropping a marker on the
floor. Because the marker bounces on the floor are not perfectly elastic,
the potential energy of the marker is converted into heat energy. If we
added the lost energy back to the marker on the floor, it won't bounce up to
a height for us to catch. This process of dropping the marker is
irreversible.

To make this idea of irreversible processes and the arrow of time sound a
little better, let's study thermodynamic situations and see that at least
the arrow of time make sense.

\section{Disorder}

Suppose we have a system that has particles and places for those particles
to be placed. We can start with a small space (see figure below). If I have
one particle moving randomly, and divide my space into two compartments,
then I have a probability of $\frac{1}{2}$ that the particle will be in the
left compartment.\FRAME{dhF}{3.8199in}{0.4532in}{0pt}{}{}{Figure}{\special%
{language "Scientific Word";type "GRAPHIC";maintain-aspect-ratio
TRUE;display "USEDEF";valid_file "T";width 3.8199in;height 0.4532in;depth
0pt;original-width 7.5766in;original-height 0.8743in;cropleft "0";croptop
"1";cropright "1";cropbottom "0";tempfilename
'PQXXR104.wmf';tempfile-properties "XPR";}}Now if I have two particles, the
probability that two particles will be in the left compartment is $\left( 
\frac{1}{2}\right) ^{2}$ or $\frac{1}{4}.$ \FRAME{dhF}{2.8928in}{0.8233in}{%
0pt}{}{}{Figure}{\special{language "Scientific Word";type
"GRAPHIC";maintain-aspect-ratio TRUE;display "USEDEF";valid_file "T";width
2.8928in;height 0.8233in;depth 0pt;original-width 2.8496in;original-height
0.7904in;cropleft "0";croptop "1";cropright "1";cropbottom "0";tempfilename
'PQXXR105.wmf';tempfile-properties "XPR";}}I can keep adding particles. For
three particles the probability of being in the left side is $\left( \frac{1%
}{2}\right) ^{3}$ or $\frac{1}{8}.$ \FRAME{dhF}{2.7769in}{1.6042in}{0pt}{}{}{%
Figure}{\special{language "Scientific Word";type
"GRAPHIC";maintain-aspect-ratio TRUE;display "USEDEF";valid_file "T";width
2.7769in;height 1.6042in;depth 0pt;original-width 4.1295in;original-height
2.373in;cropleft "0";croptop "1";cropright "1";cropbottom "0";tempfilename
'PQXXR106.wmf';tempfile-properties "XPR";}}For $50$ particles it is 
\[
\left( \frac{1}{2}\right) ^{50}=\allowbreak \frac{1}{1125\,899\,\allowbreak
906\,842\,624} 
\]

We can extend this thinking to divisions of particles. Suppose I have $100$
particles. We know that they will have a distribution of speeds
(Maxwell-Boltzman distribution). What is the probability that the $50$
fastest particles will be in the left side and the $50$ slowest will be on
the right side? This is the product of the last result with itself 
\begin{eqnarray*}
\left( \frac{1}{2}\right) ^{50}\left( \frac{1}{2}\right) ^{50} &=& \\
&=&\allowbreak \frac{1}{1267\,\allowbreak 650\,600\,228\,\allowbreak
229\,401\,496\,\allowbreak 703\,205\,376} \\
&=&\allowbreak 7.\,\allowbreak 888\,6\times 10^{-31}
\end{eqnarray*}

%TCIMACRO{%
%\TeXButton{BYU 50 particle demonstration}{\marginpar {
%\hspace{-0.5in}
%\begin{minipage}[t]{1in}
%\small{BYU 50 particle demonstrationn}
%\end{minipage}
%}}}%
%BeginExpansion
\marginpar {
\hspace{-0.5in}
\begin{minipage}[t]{1in}
\small{BYU 50 particle demonstrationn}
\end{minipage}
}%
%EndExpansion
This is not very probable. And we are only working with $100$ particles and
two sides of a box!

The point to this exercise is to realize there are very may $\emph{states}$
or ways to place particles in a divided box. Although each configuration is
equally probable, suppose we want a particular outcome, say, the fastest $50$
molecules on one side and the slowest $50$ molecules on the other. There are
many many more ways to $\emph{not}$ have this configuration that there are
to have this configuration.

Each configuration of molecules and locations is called a \emph{microstate}.
There are a huge number of microstates in our example.

The collection of microstates is called a \emph{macrostate}. There is more
than one configuration of particles in our fast-slow division that places
all the fast on one side and all the slow on another side. But the
collection of all microstates that make up this division would be a
macrostate. For example, we could have an even distribution of particles
(within their fast/slow groups) on each side, or we could have all the
fastest particles in each group on the top, and the slowest on the bottom
within each major fast or slow side of the room. These two microstates sill
preserve the division between the $50$ fastest on the left and the slowest $%
50$ on the right. They just rearrange the molecules within their groups.

\section{Extension to large spaces and many particles}

Now let's take a room full of molecules. We count the number of microstates
by considering the number of places we can put a molecule. Each molecule has
a volume $%
%TCIMACRO{\TeXButton{V}{\ooalign{\hfil$V$\hfil\cr\kern0.1em--\hfil\cr}}}%
%BeginExpansion
\ooalign{\hfil$V$\hfil\cr\kern0.1em--\hfil\cr}%
%EndExpansion
_{m}.$ So there are 
\begin{equation}
\emph{w=}\frac{%
%TCIMACRO{\TeXButton{V}{\ooalign{\hfil$V$\hfil\cr\kern0.1em--\hfil\cr}}}%
%BeginExpansion
\ooalign{\hfil$V$\hfil\cr\kern0.1em--\hfil\cr}%
%EndExpansion
}{%
%TCIMACRO{\TeXButton{V}{\ooalign{\hfil$V$\hfil\cr\kern0.1em--\hfil\cr}}}%
%BeginExpansion
\ooalign{\hfil$V$\hfil\cr\kern0.1em--\hfil\cr}%
%EndExpansion
_{m}}
\end{equation}%
places to put the molecule that are distinct. Now suppose we have $N$
molecules. If we ignore the probability that we might have two molecules in
one space (which is different than the case above!) we have 
\begin{equation}
\left( \emph{w}\right) ^{N}=\left( \frac{%
%TCIMACRO{\TeXButton{V}{\ooalign{\hfil$V$\hfil\cr\kern0.1em--\hfil\cr}}}%
%BeginExpansion
\ooalign{\hfil$V$\hfil\cr\kern0.1em--\hfil\cr}%
%EndExpansion
}{%
%TCIMACRO{\TeXButton{V}{\ooalign{\hfil$V$\hfil\cr\kern0.1em--\hfil\cr}}}%
%BeginExpansion
\ooalign{\hfil$V$\hfil\cr\kern0.1em--\hfil\cr}%
%EndExpansion
_{m}}\right) ^{N}
\end{equation}%
ways to place these molecules among our $\emph{w}$ states. I'll call this 
\begin{equation}
\emph{W}_{i}=\left( \frac{%
%TCIMACRO{\TeXButton{V}{\ooalign{\hfil$V$\hfil\cr\kern0.1em--\hfil\cr}}}%
%BeginExpansion
\ooalign{\hfil$V$\hfil\cr\kern0.1em--\hfil\cr}%
%EndExpansion
_{i}}{%
%TCIMACRO{\TeXButton{V}{\ooalign{\hfil$V$\hfil\cr\kern0.1em--\hfil\cr}}}%
%BeginExpansion
\ooalign{\hfil$V$\hfil\cr\kern0.1em--\hfil\cr}%
%EndExpansion
_{m}}\right) ^{N}
\end{equation}%
where $%
%TCIMACRO{\TeXButton{V}{\ooalign{\hfil$V$\hfil\cr\kern0.1em--\hfil\cr}}}%
%BeginExpansion
\ooalign{\hfil$V$\hfil\cr\kern0.1em--\hfil\cr}%
%EndExpansion
_{i}$ is a particular volume (we will let it change to $%
%TCIMACRO{\TeXButton{V}{\ooalign{\hfil$V$\hfil\cr\kern0.1em--\hfil\cr}}}%
%BeginExpansion
\ooalign{\hfil$V$\hfil\cr\kern0.1em--\hfil\cr}%
%EndExpansion
_{f}$ in just a minute).

\section{A Measure of Disorder}

It would be convenient to find a way to measure the disorder. Then we could
see how a thermodynamic process affects the amount of disorder. The
traditional way to do this is to define a quantity%
\begin{equation}
S\equiv k_{B}\ln \left( W\right)
\end{equation}%
This has the property that $S$ increases as the number of states, $W,$
increases.

Let's take an example, our old friend, the adiabatic free expansion.

\FRAME{dtbpF}{2.335in}{1.7487in}{0pt}{}{}{Figure}{\special{language
"Scientific Word";type "GRAPHIC";maintain-aspect-ratio TRUE;display
"USEDEF";valid_file "T";width 2.335in;height 1.7487in;depth
0pt;original-width 3.3589in;original-height 2.5088in;cropleft "0";croptop
"1";cropright "1";cropbottom "0";tempfilename
'PRX6EQ0F.wmf';tempfile-properties "XPR";}}

For this case we start with all the atoms on one side of a membrane. No
energy transfers out by heat. No work is done, we just puncture the membrane
in the center and let the gas flow from one side to the other. How does the
disorder change in this process? Consider the number of states.%
\[
\emph{W}_{i}=\left( \frac{%
%TCIMACRO{\TeXButton{V}{\ooalign{\hfil$V$\hfil\cr\kern0.1em--\hfil\cr}}}%
%BeginExpansion
\ooalign{\hfil$V$\hfil\cr\kern0.1em--\hfil\cr}%
%EndExpansion
_{i}}{%
%TCIMACRO{\TeXButton{V}{\ooalign{\hfil$V$\hfil\cr\kern0.1em--\hfil\cr}}}%
%BeginExpansion
\ooalign{\hfil$V$\hfil\cr\kern0.1em--\hfil\cr}%
%EndExpansion
_{m}}\right) ^{N} 
\]%
\[
\emph{W}_{f}=\left( \frac{%
%TCIMACRO{\TeXButton{V}{\ooalign{\hfil$V$\hfil\cr\kern0.1em--\hfil\cr}}}%
%BeginExpansion
\ooalign{\hfil$V$\hfil\cr\kern0.1em--\hfil\cr}%
%EndExpansion
_{f}}{%
%TCIMACRO{\TeXButton{V}{\ooalign{\hfil$V$\hfil\cr\kern0.1em--\hfil\cr}}}%
%BeginExpansion
\ooalign{\hfil$V$\hfil\cr\kern0.1em--\hfil\cr}%
%EndExpansion
_{m}}\right) ^{N} 
\]

We can write our disorder measure as 
\[
S_{i}\equiv k_{B}\ln \left( \emph{W}_{i}\right) 
\]%
and 
\[
S_{f}\equiv k_{B}\ln \left( \emph{W}_{f}\right) 
\]%
then the change is order is 
\begin{eqnarray*}
\Delta S &=&S_{f}-S_{i} \\
&=&k_{B}\ln \left( \emph{W}_{f}\right) -k_{B}\ln \left( \emph{W}_{i}\right)
\\
&=&k_{B}\ln \left( \frac{\emph{W}_{f}}{\emph{W}_{i}}\right)
\end{eqnarray*}%
If we look at the ratio of the number of states%
\begin{eqnarray*}
\frac{\emph{W}_{f}}{\emph{W}_{i}} &=&\frac{\left( \frac{%
%TCIMACRO{\TeXButton{V}{\ooalign{\hfil$V$\hfil\cr\kern0.1em--\hfil\cr}}}%
%BeginExpansion
\ooalign{\hfil$V$\hfil\cr\kern0.1em--\hfil\cr}%
%EndExpansion
_{f}}{%
%TCIMACRO{\TeXButton{V}{\ooalign{\hfil$V$\hfil\cr\kern0.1em--\hfil\cr}}}%
%BeginExpansion
\ooalign{\hfil$V$\hfil\cr\kern0.1em--\hfil\cr}%
%EndExpansion
_{m}}\right) ^{N}}{\left( \frac{%
%TCIMACRO{\TeXButton{V}{\ooalign{\hfil$V$\hfil\cr\kern0.1em--\hfil\cr}}}%
%BeginExpansion
\ooalign{\hfil$V$\hfil\cr\kern0.1em--\hfil\cr}%
%EndExpansion
_{i}}{%
%TCIMACRO{\TeXButton{V}{\ooalign{\hfil$V$\hfil\cr\kern0.1em--\hfil\cr}}}%
%BeginExpansion
\ooalign{\hfil$V$\hfil\cr\kern0.1em--\hfil\cr}%
%EndExpansion
_{m}}\right) ^{N}} \\
&=&\left( \frac{%
%TCIMACRO{\TeXButton{V}{\ooalign{\hfil$V$\hfil\cr\kern0.1em--\hfil\cr}}}%
%BeginExpansion
\ooalign{\hfil$V$\hfil\cr\kern0.1em--\hfil\cr}%
%EndExpansion
_{f}}{%
%TCIMACRO{\TeXButton{V}{\ooalign{\hfil$V$\hfil\cr\kern0.1em--\hfil\cr}}}%
%BeginExpansion
\ooalign{\hfil$V$\hfil\cr\kern0.1em--\hfil\cr}%
%EndExpansion
_{i}}\right) ^{N}
\end{eqnarray*}%
so 
\begin{eqnarray}
\Delta S &=&k_{B}\ln \left( \left( \frac{%
%TCIMACRO{\TeXButton{V}{\ooalign{\hfil$V$\hfil\cr\kern0.1em--\hfil\cr}}}%
%BeginExpansion
\ooalign{\hfil$V$\hfil\cr\kern0.1em--\hfil\cr}%
%EndExpansion
_{f}}{%
%TCIMACRO{\TeXButton{V}{\ooalign{\hfil$V$\hfil\cr\kern0.1em--\hfil\cr}}}%
%BeginExpansion
\ooalign{\hfil$V$\hfil\cr\kern0.1em--\hfil\cr}%
%EndExpansion
_{i}}\right) ^{N}\right)  \nonumber \\
&=&Nk_{B}\ln \left( \left( \frac{%
%TCIMACRO{\TeXButton{V}{\ooalign{\hfil$V$\hfil\cr\kern0.1em--\hfil\cr}}}%
%BeginExpansion
\ooalign{\hfil$V$\hfil\cr\kern0.1em--\hfil\cr}%
%EndExpansion
_{f}}{%
%TCIMACRO{\TeXButton{V}{\ooalign{\hfil$V$\hfil\cr\kern0.1em--\hfil\cr}}}%
%BeginExpansion
\ooalign{\hfil$V$\hfil\cr\kern0.1em--\hfil\cr}%
%EndExpansion
_{i}}\right) \right)
\end{eqnarray}%
We should ask if the order increased or decreased. Since $%
%TCIMACRO{\TeXButton{V}{\ooalign{\hfil$V$\hfil\cr\kern0.1em--\hfil\cr}}}%
%BeginExpansion
\ooalign{\hfil$V$\hfil\cr\kern0.1em--\hfil\cr}%
%EndExpansion
_{f}$ is larger for our case than $%
%TCIMACRO{\TeXButton{V}{\ooalign{\hfil$V$\hfil\cr\kern0.1em--\hfil\cr}}}%
%BeginExpansion
\ooalign{\hfil$V$\hfil\cr\kern0.1em--\hfil\cr}%
%EndExpansion
_{i},$ then $\Delta S$ will be a positive number. This means the disorder
has increased in this case.

It is a good question to ask, will $\Delta S$ ever be negative? In other
words, will the gas ever line up on only one side every again? If the answer
is no, then this is an irreversible process. We can guess that this process
is irreversible. We would have to do work to pump the gas back into one side.

\section{Definition of entropy}

%TCIMACRO{%
%\TeXButton{Question 123.16.1}{\marginpar {
%\hspace{-0.5in}
%\begin{minipage}[t]{1in}
%\small{Question 123.16.1}
%\end{minipage}
%}}}%
%BeginExpansion
\marginpar {
\hspace{-0.5in}
\begin{minipage}[t]{1in}
\small{Question 123.16.1}
\end{minipage}
}%
%EndExpansion
We have learned to use one state variable, $\Delta E_{int}.$ The trick is to
realize that $\Delta E_{int}$ only depends on $\Delta T$ and use any easy
path to calculate it's value. That is why we can use 
\[
\Delta E_{int}=nC_{V}\Delta T 
\]%
no matter what process we have. State variables, like $\Delta E_{int}$ give
us powerful ways to calculate what will happen in a system.

%TCIMACRO{%
%\TeXButton{Question 123.16.2}{\marginpar {
%\hspace{-0.5in}
%\begin{minipage}[t]{1in}
%\small{Question 123.16.2}
%\end{minipage}
%}}}%
%BeginExpansion
\marginpar {
\hspace{-0.5in}
\begin{minipage}[t]{1in}
\small{Question 123.16.2}
\end{minipage}
}%
%EndExpansion
There is another state variable in Thermodynamics, \emph{change in entropy}, 
$\Delta S.$ We will again find that we can calculate $\Delta S$ by any path
because it depends only on the starting and ending points. And from our
previous discussion we recognize that this new state variable, it is just a
change in our disorder parameter! 
\[
\Delta S=k_{B}\ln \left( \frac{\emph{W}_{f}}{\emph{W}_{i}}\right) 
\]%
Note that this is a difference between two disorder parameters. We will call
our disorder parameter the \emph{entropy} of the system. 
\begin{equation}
S\equiv k_{B}\ln \left( W\right)
\end{equation}%
Entropy was originally defined at the macroscopic level. Statistical
Mechanics (the microscopic study of thermodynamics) gives us our modern
definition we have followed. We can state this definition as

\begin{definition}
Entropy is a measure of the amount of disorder in a system and its
environment.
\end{definition}

\subsection{Entropy and the Second Law}

In our example, because, $%
%TCIMACRO{\TeXButton{V}{\ooalign{\hfil$V$\hfil\cr\kern0.1em--\hfil\cr}}}%
%BeginExpansion
\ooalign{\hfil$V$\hfil\cr\kern0.1em--\hfil\cr}%
%EndExpansion
_{f}>%
%TCIMACRO{\TeXButton{V}{\ooalign{\hfil$V$\hfil\cr\kern0.1em--\hfil\cr}}}%
%BeginExpansion
\ooalign{\hfil$V$\hfil\cr\kern0.1em--\hfil\cr}%
%EndExpansion
_{i}$ we found that $\Delta S$ will be positive. This implies that the
disorder has increased. We can see that this is true, because there are now
molecules distributed among many more states (position, occupation
combinations) than before.

We will find this to be true in general, for irreversible processes (all
natural process that are recognized). The total entropy of an isolated
system that undergoes a change cannot decrease.

What about non-isolated systems? We must then consider the entropy of the
system \emph{and} its surroundings. We again find that entropy of a system 
\emph{and} its surroundings cannot decrease. But the entropy of a system can
decrease if the entropy of the surroundings increases more so the total
effect is an increase in entropy.

These statements constitute the most commonly seen form of the second law of
thermodynamics.

\subsection{Macroscopic view}

We don't really want to count the microstates of a process and look for
changes in their number to do calculations. It would get tedious very fast!
And so far our definition of $\Delta S$ was done only for a free expansion.
We expect that in some way disorder would increase if we transferred in
energy by heat, or did some kind of work on the system. We need a
macroscopic version of our definition of entropy.

We will give a definition and then show that it works. The change in entropy
is given by%
\begin{equation}
\Delta S=\frac{\Delta Q_{r}}{T}
\end{equation}%
where $Q_{r}$ is the energy transferred by heat through a reversible path
and $T$ is the absolute temperature ($\unit{K}$) of the system. The
assumption of a single temperature is because we have assumed $\Delta Q_{r}$
is small. But why do we use a reversible path?

Suppose we have an irreversible process. It might be hard to compute $Q.$
But remember that, like $\Delta E_{int},$ $\Delta S$ does not depend on
path,! it depends only on the initial and final states. So we may calculate
it using any convenient path, and a reversible path is more convenient.

Note that macroscopically we have not defined entropy but the change in
entropy, $\Delta S$. This is the meaningful quantity (like $\Delta E_{int})$
for thermodynamic calculations.

Let's do an example: What is the change in entropy when a block of ice
melts? We will use 
\[
\Delta S=\int \frac{dQ_{r}}{T} 
\]%
Note that while the ice is melting, we will have ice water at a constant $%
T_{melt}=273\unit{K}.$ Then our integral is 
\[
\Delta S=\frac{1}{T_{melt}}\int dQ_{r} 
\]%
and the integral is not too hard%
\[
\Delta S=\frac{Q_{r}}{T_{melt}} 
\]%
and for melting we know what $Q_{r}$ could be. We can freeze water as well
as melt ice. So we could model this as a reversible process. The amount of
energy transfer needed to melt the ice is just 
\[
Q=mL 
\]%
where $m$ is the mass of the ice and $L$ is the latent heat of fusion for
water. So 
\[
\Delta S=\frac{mL}{T_{melt}} 
\]

This wasn't too hard!

Let's do another example to tie our macroscopic $\Delta S$ to our
microscopic $\Delta S.$ Let's find the change in entropy for a gas brought
through an isothermal expansion. Again we have%
\[
\Delta S=\int_{%
%TCIMACRO{\TeXButton{V}{\ooalign{\hfil$V$\hfil\cr\kern0.1em--\hfil\cr}}}%
%BeginExpansion
\ooalign{\hfil$V$\hfil\cr\kern0.1em--\hfil\cr}%
%EndExpansion
_{i}}^{%
%TCIMACRO{\TeXButton{V}{\ooalign{\hfil$V$\hfil\cr\kern0.1em--\hfil\cr}}}%
%BeginExpansion
\ooalign{\hfil$V$\hfil\cr\kern0.1em--\hfil\cr}%
%EndExpansion
_{f}}\frac{dQ_{r}}{T} 
\]%
and since the process is isothermal, $T$ is constant%
\[
\Delta S=\frac{1}{T}\int_{%
%TCIMACRO{\TeXButton{V}{\ooalign{\hfil$V$\hfil\cr\kern0.1em--\hfil\cr}}}%
%BeginExpansion
\ooalign{\hfil$V$\hfil\cr\kern0.1em--\hfil\cr}%
%EndExpansion
_{i}}^{%
%TCIMACRO{\TeXButton{V}{\ooalign{\hfil$V$\hfil\cr\kern0.1em--\hfil\cr}}}%
%BeginExpansion
\ooalign{\hfil$V$\hfil\cr\kern0.1em--\hfil\cr}%
%EndExpansion
_{f}}dQ_{r} 
\]%
and the integral would seem easy. It is just $Q,$ but it is a particular $Q$
for an isothermal process. For an isothermal process we know%
\[
\Delta E_{int}=0 
\]%
and 
\[
\Delta E_{int}=Q+w 
\]%
so 
\[
Q=-w 
\]%
and further we know that for an isothermal process 
\begin{eqnarray*}
w &=&-\int Pd%
%TCIMACRO{\TeXButton{V}{\ooalign{\hfil$V$\hfil\cr\kern0.1em--\hfil\cr}} }%
%BeginExpansion
\ooalign{\hfil$V$\hfil\cr\kern0.1em--\hfil\cr}
%EndExpansion
\\
&=&-nRT\ln \left( \frac{%
%TCIMACRO{\TeXButton{V}{\ooalign{\hfil$V$\hfil\cr\kern0.1em--\hfil\cr}}}%
%BeginExpansion
\ooalign{\hfil$V$\hfil\cr\kern0.1em--\hfil\cr}%
%EndExpansion
_{f}}{%
%TCIMACRO{\TeXButton{V}{\ooalign{\hfil$V$\hfil\cr\kern0.1em--\hfil\cr}}}%
%BeginExpansion
\ooalign{\hfil$V$\hfil\cr\kern0.1em--\hfil\cr}%
%EndExpansion
_{i}}\right)
\end{eqnarray*}%
so the $Q_{r}$ for our process could be just 
\[
Q=nRT\ln \left( \frac{%
%TCIMACRO{\TeXButton{V}{\ooalign{\hfil$V$\hfil\cr\kern0.1em--\hfil\cr}}}%
%BeginExpansion
\ooalign{\hfil$V$\hfil\cr\kern0.1em--\hfil\cr}%
%EndExpansion
_{f}}{%
%TCIMACRO{\TeXButton{V}{\ooalign{\hfil$V$\hfil\cr\kern0.1em--\hfil\cr}}}%
%BeginExpansion
\ooalign{\hfil$V$\hfil\cr\kern0.1em--\hfil\cr}%
%EndExpansion
_{i}}\right) 
\]%
Then 
\begin{eqnarray*}
\Delta S &=&\frac{1}{T}nRT\ln \left( \frac{%
%TCIMACRO{\TeXButton{V}{\ooalign{\hfil$V$\hfil\cr\kern0.1em--\hfil\cr}}}%
%BeginExpansion
\ooalign{\hfil$V$\hfil\cr\kern0.1em--\hfil\cr}%
%EndExpansion
_{f}}{%
%TCIMACRO{\TeXButton{V}{\ooalign{\hfil$V$\hfil\cr\kern0.1em--\hfil\cr}}}%
%BeginExpansion
\ooalign{\hfil$V$\hfil\cr\kern0.1em--\hfil\cr}%
%EndExpansion
_{i}}\right) \\
&=&nR\ln \left( \frac{%
%TCIMACRO{\TeXButton{V}{\ooalign{\hfil$V$\hfil\cr\kern0.1em--\hfil\cr}}}%
%BeginExpansion
\ooalign{\hfil$V$\hfil\cr\kern0.1em--\hfil\cr}%
%EndExpansion
_{f}}{%
%TCIMACRO{\TeXButton{V}{\ooalign{\hfil$V$\hfil\cr\kern0.1em--\hfil\cr}}}%
%BeginExpansion
\ooalign{\hfil$V$\hfil\cr\kern0.1em--\hfil\cr}%
%EndExpansion
_{i}}\right)
\end{eqnarray*}%
and remember that $nR=Nk_{B}$ so 
\begin{eqnarray*}
\Delta S &=& \\
&=&Nk_{B}\ln \left( \frac{%
%TCIMACRO{\TeXButton{V}{\ooalign{\hfil$V$\hfil\cr\kern0.1em--\hfil\cr}}}%
%BeginExpansion
\ooalign{\hfil$V$\hfil\cr\kern0.1em--\hfil\cr}%
%EndExpansion
_{f}}{%
%TCIMACRO{\TeXButton{V}{\ooalign{\hfil$V$\hfil\cr\kern0.1em--\hfil\cr}}}%
%BeginExpansion
\ooalign{\hfil$V$\hfil\cr\kern0.1em--\hfil\cr}%
%EndExpansion
_{i}}\right)
\end{eqnarray*}%
which is just what we found from our microscopic view of entropy change! Now
we realize that when we let the gas flow from one side of our divide box to
the other, we did this isothermally. And that is true, the gas did not
change the average kinetic energy per molecule as it flowed from one side of
the box to the other. Since our system was isolated, $Q=0$ and $W=0$ for the
whole system, so $\Delta E_{int}=0$ and the system did not change
temperature. But surely something changed! And that something was $\Delta S!$

\subsection{Heat death of the Universe}

We should ask the question, is the universe an isolated system? We don't
know the answer to this for sure\footnote{%
After all, the \textquotedblleft Flash\textquotedblright\ and his colleagues
seem to be able to come and go from different \textquotedblleft universes
all the time.\textquotedblright}, but we believe this is true that the
universe is isolated.\footnote{%
Dispite what the television shows seem to say.} That means that the change
in entropy of the universe will always be positive. The universe is
\textquotedblleft running down.\textquotedblright\ The ultimate end of this
increase in energy would mean that all mater in the universe will be spread
out and cold. This is called the \emph{heat death of the universe}.

Note that this is a little different than the eschatological view that we
have in the Church of Jesus Christ of Latter Day Saints.

Throughout my career, many colleagues and acquaintances have said it is
foolish to believe in a church and a resurrection because the second law of
thermodynamics tells us that the universe will die in a cold blur. How do we
respond?

I can only give my opinion. But I think it is worth discussing, so here is a
disclaimer for what follows.

\begin{Note}
WARNING: This is not an official statement of doctrine This is an answer to
a question that many scientists must face from colleagues who will question
their faith using the principals of thermodynamics. Official statements by
the General Authorities supersede any opinions expressed here
\end{Note}

\subsection{The First Law and LDS\ Thought\protect\footnote{%
Comparing physical theory to LDS\ thought has it's dangers, I am aware. But
as scientists, we can't escape thinking about this. Just as a caution, John
A. Widsoe wrote a book called \emph{Joseph Smith as Scientist} in which
Widsoe tried very hard to show that LDS\ though is in harmony with Universal
Ether Theory. But we now know Universal Ether Theory is not correct, so
being in harmony with it is meaningless. So we should be cautious!}}

I suppose we should ask, do we believe in the first law of thermodynamics.
If this law does not work for us, we have no reason to expect the second law
to work. The first law is basically conservation of energy. Here are some
quotes to consider.

\begin{quote}
And there stood one among them that was like unto God, and he said unto
those who were with him: We will go down, for there is space there, and we
will take of these materials, and we will make an earth whereon these may
dwell; (Abraham 3:24 )

You ask the learned doctors why the say the world was made out of nothing;
and they will answer, `Doesn't the Bible say He created the world?' And they
infer, from the word create, that it must have been made out of nothing.
Now, the word create came from the word baurau which does not mean create
out of nothing; it means to organize; the same as a man would organize
materials and build a ship. (Joseph Smith, King Follett Discourse)
\end{quote}

The idea of the first law of thermodynamics seems in harmony with LDS
thought.

So let's take on the Second Law.

\subsection{The Second Law and LDS\ Thought}

Again here are some quotes

\begin{quote}
Wherefore, the first judgment which came upon man must needs have remained
to an endless duration. And if so, this flesh must have laid down to rot and
to crumble to its mother earth, to rise no more. (2 Nephi 9:7)

And our spirits must have become like unto [Satan], and we become devils,
angels to a devil, to be shut out from the presence of our God, and to
remain with the father of lies, in misery, like unto himself. (2 Nephi 9:9)
\end{quote}

We seem to accept the concept of the second law as well! --This will be a
surprise to our non-member friends.

But then how can we expect eternal life and exaltation? Does something
counter the second law (a Third or Fourth Law of Thermodynamics)?

\subsubsection{A Third Law of Thermodynamics?}

Again let's go to the scriptures. Jacob describes another force not
recognized by current scientific theory that counteracts this tendency to
decay

\begin{quote}
Wherefore, it must needs be an infinite atonement---save it should be an
infinite atonement this corruption could not put on incorruption. Wherefore,
the first judgment which came upon man must needs have remained to an
endless duration. And if so, this flesh must have laid down to rot and to
crumble to its mother earth, to rise no more. (2 Nephi 9:7-8)

O how great the goodness of our God, who prepareth a way for our escape from
the grasp of this awful monster; yea, that monster, death and hell, which I
call the death of the body, and also the death of the spirit. (2 Nephi 9:10)
\end{quote}

Atonement means to bring together. This is an apt description of a process
or law that would counter or balance the second law\footnote{%
Nibley, The meaning of the temple, reprinted in the Collected Works of Hugh
Nibley: volume 12, Deseret Book, Salt Lake City, UT, 1992}. Since the
atonement can only be accessed through Christ, it is not likely to be
discovered in a laboratory, but is a very real, physical, process. (see for
example Luke 24:39)

This is not doctrine, but is a way to think about the second law until we
get direct revelation on the subject. We really should not be worried about
a conflict between thermodynamics and our doctrine. I don't believe one
exists as you can see.

\chapter{Producing Useful Work}

So far we have talked about doing work on gas in cylinders. That may not
seem all that useful. But if we think that car engines have gas in cylinders
with pistons, then it may seem like this could be important.

%TCIMACRO{%
%\TeXButton{Fundamental Concepts}{\hspace{-1.3in}{\Large Fundamental Concepts\vspace{0.25in}}}}%
%BeginExpansion
\hspace{-1.3in}{\Large Fundamental Concepts\vspace{0.25in}}%
%EndExpansion

\begin{itemize}
\item $W_{\text{useful }}=-W_{\text{on the gas}}$
\end{itemize}

\section{Useful work.}

%TCIMACRO{%
%\TeXButton{Question 123.17.1}{\marginpar {
%\hspace{-0.5in}
%\begin{minipage}[t]{1in}
%\small{Question 123.17.1}
%\end{minipage}
%}}}%
%BeginExpansion
\marginpar {
\hspace{-0.5in}
\begin{minipage}[t]{1in}
\small{Question 123.17.1}
\end{minipage}
}%
%EndExpansion
Suppose that we want to produce useful work from a thermodynamic device. For
building such a device we have a nomenclature issue.\FRAME{dhF}{4.8775in}{%
1.8273in}{0pt}{}{}{Figure}{\special{language "Scientific Word";type
"GRAPHIC";maintain-aspect-ratio TRUE;display "USEDEF";valid_file "T";width
4.8775in;height 1.8273in;depth 0pt;original-width 4.8239in;original-height
1.7902in;cropleft "0";croptop "1";cropright "1";cropbottom "0";tempfilename
'PQXXR107.wmf';tempfile-properties "XPR";}}Consider our piston. If we add
energy by heat, and the piston is free to move, then we do an amount of work
on the gas 
\[
w=-P\left( 
%TCIMACRO{\TeXButton{V}{\ooalign{\hfil$V$\hfil\cr\kern0.1em--\hfil\cr}}}%
%BeginExpansion
\ooalign{\hfil$V$\hfil\cr\kern0.1em--\hfil\cr}%
%EndExpansion
_{f}-%
%TCIMACRO{\TeXButton{V}{\ooalign{\hfil$V$\hfil\cr\kern0.1em--\hfil\cr}}}%
%BeginExpansion
\ooalign{\hfil$V$\hfil\cr\kern0.1em--\hfil\cr}%
%EndExpansion
_{i}\right) 
\]%
\emph{due to}This is a negative amount of work, since $%
%TCIMACRO{\TeXButton{V}{\ooalign{\hfil$V$\hfil\cr\kern0.1em--\hfil\cr}}}%
%BeginExpansion
\ooalign{\hfil$V$\hfil\cr\kern0.1em--\hfil\cr}%
%EndExpansion
_{f}$ is lager than $%
%TCIMACRO{\TeXButton{V}{\ooalign{\hfil$V$\hfil\cr\kern0.1em--\hfil\cr}}}%
%BeginExpansion
\ooalign{\hfil$V$\hfil\cr\kern0.1em--\hfil\cr}%
%EndExpansion
_{i}.$ But think of a steam engine. We want the force by the gas on the
piston to win, because this pushing of the piston is what we can use to push
the wheel of a steam engine, or a turbine, etc. That is negative work on the
gas. And it is inconvenient to have the useful work be negative. Let's
define the useful work done by a thermodynamic device as the negative of
work done on the gas.%
\[
w_{\text{useful }}=-w 
\]

But, $w_{\text{useful }}$is kind of a awkward name. Usually a machine that
does useful work is called an engine, so let's call our useful work $W_{eng}$
for \textquotedblleft work done by an engine.\textquotedblright

\[
w_{eng}=-w 
\]

\section{Reservoirs}

%TCIMACRO{%
%\TeXButton{Question 123.17.2}{\marginpar {
%\hspace{-0.5in}
%\begin{minipage}[t]{1in}
%\small{Question 123.17.2}
%\end{minipage}
%}}}%
%BeginExpansion
\marginpar {
\hspace{-0.5in}
\begin{minipage}[t]{1in}
\small{Question 123.17.2}
\end{minipage}
}%
%EndExpansion
Think of a water reservoir. In the western US we know what these are. It is
a large man-made body of water that is designed to provide irrigation water
for farms and homes.

Think of what happens when you water your lawn with water from the
reservoir. Some water does leave from the reservoir, but compared to the
whole reservoir of water, one lawn watering makes very little difference.
Often there is more water coming into the reservoir, so really there is
negligible water loss from your individual lawn.

A heat reservoir is like this. A hot reservoir is a source of thermal energy
that can be tapped without causing any measurable change in temperature. We
can also have a cold reservoir that has a large lack of thermal energy. A
small amount of thermal energy transferred to this cold reservoir won't
measurably change its temperature

The next figure is called an \emph{Energy Transfer Diagram}.

\FRAME{dhF}{1.1917in}{1.6873in}{0pt}{}{}{Figure}{\special{language
"Scientific Word";type "GRAPHIC";maintain-aspect-ratio TRUE;display
"USEDEF";valid_file "T";width 1.1917in;height 1.6873in;depth
0pt;original-width 1.8351in;original-height 2.6091in;cropleft "0";croptop
"1";cropright "1";cropbottom "0";tempfilename
'PQXXR108.wmf';tempfile-properties "XPR";}}It shows energy moving by heat
from a hot reservoir to a cold reservoir.

%TCIMACRO{%
%\TeXButton{Question 123.17.3}{\marginpar {
%\hspace{-0.5in}
%\begin{minipage}[t]{1in}
%\small{Question 123.17.3}
%\end{minipage}
%}}}%
%BeginExpansion
\marginpar {
\hspace{-0.5in}
\begin{minipage}[t]{1in}
\small{Question 123.17.3}
\end{minipage}
}%
%EndExpansion
Using our understanding of the second law of thermodynamics, we can see that
an amount of energy, $Q_{h}$, can flow from the hot reservoir and an amount
of energy$,$ $Q_{c},$ can flow by heat to the cold reservoir. We expect that
if there is no other mechanism for dissipating the energy, then $%
Q_{c}=Q_{h}. $ We will define both $Q_{h}$ and $Q_{c}$ as positive
quantities. If they are negative, we will explicitly write a negative sign.

Look at this energy transfer diagram.\FRAME{dhF}{2.1525in}{2.1482in}{0in}{}{%
}{Figure}{\special{language "Scientific Word";type
"GRAPHIC";maintain-aspect-ratio TRUE;display "USEDEF";valid_file "T";width
2.1525in;height 2.1482in;depth 0in;original-width 2.1136in;original-height
2.1093in;cropleft "0";croptop "1";cropright "1";cropbottom "0";tempfilename
'PQXXR109.wmf';tempfile-properties "XPR";}}

This diagram says that energy flows from the cold reservoir to the hot
reservoir. From the second law of thermodynamics we would say that this
won't happen. This would increase the order of the cold reservoir
spontaneously. This does not happen.

%TCIMACRO{%
%\TeXButton{Question 123.17.4}{\marginpar {
%\hspace{-0.5in}
%\begin{minipage}[t]{1in}
%\small{Question 123.17.4}
%\end{minipage}
%}}}%
%BeginExpansion
\marginpar {
\hspace{-0.5in}
\begin{minipage}[t]{1in}
\small{Question 123.17.4}
\end{minipage}
}%
%EndExpansion
Here is another energy transfer diagram\FRAME{dhF}{1.5316in}{1.9493in}{0pt}{%
}{}{Figure}{\special{language "Scientific Word";type
"GRAPHIC";maintain-aspect-ratio TRUE;display "USEDEF";valid_file "T";width
1.5316in;height 1.9493in;depth 0pt;original-width 2.6974in;original-height
3.442in;cropleft "0";croptop "1";cropright "1";cropbottom "0";tempfilename
'PQXXR10A.wmf';tempfile-properties "XPR";}}This diagram says that we do work
that is directly converted into heat. This is a common experience. Rub your
hands together. Your hands warm up. The work has created more internal
energy, But that energy quickly dissipates into the room The room
temperature is not affected by this increase in internal energy of the air.
The air in the room is a cold reservoir.

Here is another diagram.\FRAME{dhF}{1.9666in}{1.9631in}{0pt}{}{}{Figure}{%
\special{language "Scientific Word";type "GRAPHIC";maintain-aspect-ratio
TRUE;display "USEDEF";valid_file "T";width 1.9666in;height 1.9631in;depth
0pt;original-width 2.4613in;original-height 2.4569in;cropleft "0";croptop
"1";cropright "1";cropbottom "0";tempfilename
'PQXXR10B.wmf';tempfile-properties "XPR";}}This diagram says that energy
from the hot reservoir is converted directly into useful work. This is not
something we expect to happen. And our second law of thermodynamics again
would prohibit this situation.

This would be a perfect engine, converting all the energy transferred from
the hot reservoir into useful work. In fact, this idea could give us a
perfectly good restatement of the second law of thermodynamics

\begin{Note}
Perfect engines are not possible
\end{Note}

\section{Heat Engines}

Using our energy transfer diagram, let's introduce a more practical engine.
Here is the diagram. We have added the engine explicitly on the diagram. 
\FRAME{dhF}{1.5454in}{1.8697in}{0pt}{}{}{Figure}{\special{language
"Scientific Word";type "GRAPHIC";maintain-aspect-ratio TRUE;display
"USEDEF";valid_file "T";width 1.5454in;height 1.8697in;depth
0pt;original-width 2.655in;original-height 3.2197in;cropleft "0";croptop
"1";cropright "1";cropbottom "0";tempfilename
'PQXXR10C.wmf';tempfile-properties "XPR";}}The engine takes energy from the
heat reservoir and turns it into useful work. But there is some energy that
is waisted and this energy ends up in the cold reservoir. The amount of
energy that becomes work must be 
\[
W_{\text{engine }}=Q_{h}-Q_{c} 
\]%
We call $Q_{h}-Q_{c}$ the net heat%
\[
Q_{net}=Q_{h}-Q_{c} 
\]%
Notice that $Q_{h}$ is positive \emph{for the engine} and $Q_{c}$ is
negative \emph{for the engine.}

A useful heat engine is a device that takes in energy by heat and operates
in a cycle, If it did not periodically return to its original state, then it
could not keep going. The basics of the cycle are as follows

\begin{enumerate}
\item The working substance absorbs energy by heat from the high temperature
reservoir

\item Work is done by the engine

\item Energy is expelled by heat to a low temperature reservoir
\end{enumerate}

\subsection{Example: Steam engine}

\FRAME{dhF}{2.5988in}{1.6414in}{0pt}{}{}{Figure}{\special{language
"Scientific Word";type "GRAPHIC";maintain-aspect-ratio TRUE;display
"USEDEF";valid_file "T";width 2.5988in;height 1.6414in;depth
0pt;original-width 10.5239in;original-height 6.6348in;cropleft "0";croptop
"1";cropright "1";cropbottom "0";tempfilename
'PQXXR10D.wmf';tempfile-properties "XPR";}}

In a steam engine

\begin{enumerate}
\item Water in the boiler absorbs energy from burning fuel (hot reservoir)
and evaporates to steam

\item The steam does work by expanding against a piston

\item The steam cools and condenses releasing heat to the outside air (cold.
reservoir) , and the liquid water returns to the boiler.
\end{enumerate}

Since the engine goes through a cyclical process, 
\[
\Delta E_{int}=0 
\]

Its initial and final internal energies are the same, so 
\begin{equation}
Q_{net}=W_{eng}
\end{equation}

The work done by the engine equals the \emph{net }energy absorbed by the
engine. The work is equal to the area enclosed by the curve of the PV
diagram.\FRAME{dhF}{2.1785in}{1.5601in}{0pt}{}{}{Figure}{\special{language
"Scientific Word";type "GRAPHIC";maintain-aspect-ratio TRUE;display
"USEDEF";valid_file "T";width 2.1785in;height 1.5601in;depth
0pt;original-width 4.0041in;original-height 2.8591in;cropleft "0";croptop
"1";cropright "1";cropbottom "0";tempfilename
'PQXXR10E.wmf';tempfile-properties "XPR";}}

Thermal efficiency is defined as the ratio of the net work done by the
engine during one cycle to the energy input at the higher temperature

\begin{eqnarray}
\eta &=&\frac{W_{eng}}{\left\vert Q_{h}\right\vert } \\
&=&\frac{Q_{net}}{Q_{h}} \\
&=&\frac{\left\vert Q_{h}\right\vert -\left\vert Q_{c}\right\vert }{Q_{h}} \\
&=&1-\frac{\left\vert Q_{c}\right\vert }{\left\vert Q_{h}\right\vert }
\end{eqnarray}%
We can think of this as 
\begin{equation}
\eta =\frac{\text{what you gain in work}}{\text{what you gave in energy by
heat}}
\end{equation}

For a car engine 
\begin{equation}
\eta =20\%
\end{equation}%
is a very good number (most student and faculty cars are far less efficient
than this).

%TCIMACRO{%
%\TeXButton{Question 123.17.5}{\marginpar {
%\hspace{-0.5in}
%\begin{minipage}[t]{1in}
%\small{Question 123.17.5}
%\end{minipage}
%}}}%
%BeginExpansion
\marginpar {
\hspace{-0.5in}
\begin{minipage}[t]{1in}
\small{Question 123.17.5}
\end{minipage}
}%
%EndExpansion
%TCIMACRO{%
%\TeXButton{Question 123.17.6}{\marginpar {
%\hspace{-0.5in}
%\begin{minipage}[t]{1in}
%\small{Question 123.17.6}
%\end{minipage}
%}}}%
%BeginExpansion
\marginpar {
\hspace{-0.5in}
\begin{minipage}[t]{1in}
\small{Question 123.17.6}
\end{minipage}
}%
%EndExpansion

Suppose that we have a $100\%$ efficient engine. We know that this is not
possible, but consider what this would mean. 
\begin{eqnarray}
\eta &=&1-\frac{\left\vert Q_{c}\right\vert }{\left\vert Q_{h}\right\vert }
\\
&=&1
\end{eqnarray}%
this can only happen when $\left\vert Q_{c}\right\vert =0$ which means that
no energy is transferred out of the engine by heat -- which is not really
possible. The fact that real engines have efficiencies much lower than $1$
leads to another statement of the second law of thermodynamics.

\FRAME{dhF}{2.9352in}{2.93in}{0in}{}{}{Figure}{\special{language "Scientific
Word";type "GRAPHIC";maintain-aspect-ratio TRUE;display "USEDEF";valid_file
"T";width 2.9352in;height 2.93in;depth 0in;original-width
2.8911in;original-height 2.8859in;cropleft "0";croptop "1";cropright
"1";cropbottom "0";tempfilename 'PQXXR10F.wmf';tempfile-properties "XPR";}}

\emph{It is impossible to construct a heat engine that, operating in a
cycle, inputs energy by heat from a hot reservoir and converts the energy
entirely into useful work.}

This is called the Kelvin-Plank form of the second law, and it means that 
\begin{equation}
W_{eng}<\left\vert Q_{h}\right\vert
\end{equation}%
so the device represented by our drawing above is impossible to achieve.

\section{Heat Pumps and Refrigerators}

Let's look at a special form of a heat engine, a head pump.%
%TCIMACRO{%
%\TeXButton{Question 123.17.7}{\marginpar {
%\hspace{-0.5in}
%\begin{minipage}[t]{1in}
%\small{Question 123.17.7}
%\end{minipage}
%}}}%
%BeginExpansion
\marginpar {
\hspace{-0.5in}
\begin{minipage}[t]{1in}
\small{Question 123.17.7}
\end{minipage}
}%
%EndExpansion
.

\FRAME{dhF}{2.041in}{2.6507in}{0pt}{}{}{Figure}{\special{language
"Scientific Word";type "GRAPHIC";maintain-aspect-ratio TRUE;display
"USEDEF";valid_file "T";width 2.041in;height 2.6507in;depth
0pt;original-width 2.002in;original-height 2.6091in;cropleft "0";croptop
"1";cropright "1";cropbottom "0";tempfilename
'PQXXR10G.wmf';tempfile-properties "XPR";}}A heat pump is an engine run in
reverse. We wish to pump heat from a cold reservoir to the heat reservoir.
To do this we must do work \emph{on }the heat pump. This is not a natural
process, so by adding in work we are adding energy to make it happen.

A window air conditioner unit is a good example of a heat pump. It removes
energy by heat from the inside of your apartment and transfers that energy
to the outside of your apparent by heat.process. To make this happen, you
need mechanical device that does work

A refrigerator is another example of a heat pump. It transfers energy by
heat from the cold interior to the warmer environment of your apartment
kitchen. It would be great if we could transfer heat from our cold reservoir
(room of a house or freezer compartment of a fridge) to a hot reservoir
without doing work, but this is not possible. This fact was discovered by
Clausius and lead to the Clausius statement of the second law--\emph{It is
impossible to construct a cyclical machine whose sole effect is to transfer
energy continuously by heat from one object to another object at a higher
temperature without the input of energy by work.}

Or more simply, as we have already said: Energy does not transfer
spontaneously by heat from a cold object to a hot object.

\subsection{Coefficient of Performance}

How well a heat pump works can be described by the \emph{coefficient of
performance} (COP).%
\begin{eqnarray}
COP_{h} &=&\frac{\text{energy transferred at high temperature}}{\text{work
done on heat pump}} \\
&=&\frac{\left\vert Q_{h}\right\vert }{W}
\end{eqnarray}

$COP$ is similar to efficiency. Think of our air conditioner. $Q_{h}$ is
typically higher than $W.$ That is, we usually transfer more energy by heat
to the outside world that we provide in work. The outside part of the
airconditioner does indeed get hotter than the outside air temperature. That
is why energy will transfer by heat. Values of $COP$ are generally greater
than $1,$ though it is possible for them to be less than $1.$ Of course we
would like the $COP$ to be as high as possible

For refrigeration, the important thing is how much energy we transfer out of
the cool area, $Q_{c}$, so for cooling the $COP$ is 
\begin{equation}
COP_{c}=\frac{\left\vert Q_{c}\right\vert }{W}
\end{equation}%
A good refrigerator should have a high COP, Typical values are 5 or 6.

\chapter{Ideal Gas Heat Engines}

So far we have talked about idealized heat engines and refrigerators. We
should try a somewhat realistic engine. Our goal in such a problem is to
find for each process the values of $\Delta E_{int},$ $Q,$ $W,$ and $%
W_{eng}, $ $Q_{h}$, $Q_{c}$. We also want to find the ideal gas state
variables $P,$ $%
%TCIMACRO{\TeXButton{V}{\ooalign{\hfil$V$\hfil\cr\kern0.1em--\hfil\cr}}}%
%BeginExpansion
\ooalign{\hfil$V$\hfil\cr\kern0.1em--\hfil\cr}%
%EndExpansion
,$ $n,$ and $T$ at the transition points between processes.

%TCIMACRO{%
%\TeXButton{Fundamental Concepts}{\hspace{-1.3in}{\Large Fundamental Concepts\vspace{0.25in}}}}%
%BeginExpansion
\hspace{-1.3in}{\Large Fundamental Concepts\vspace{0.25in}}%
%EndExpansion

\begin{itemize}
\item To describe a heat engine cycle you need to find $P,$ $%
%TCIMACRO{\TeXButton{V}{\ooalign{\hfil$V$\hfil\cr\kern0.1em--\hfil\cr}}}%
%BeginExpansion
\ooalign{\hfil$V$\hfil\cr\kern0.1em--\hfil\cr}%
%EndExpansion
,$ $n,$ $T,$ $\Delta E_{int},$ $Q,$ $W,$ for each part of the cycle and $%
W_{eng},$ $Q_{h}$, $Q_{c}$ for the entire cycle.

\item The Otto Cycle
\end{itemize}

\section{An example: The Otto Cycle}

\FRAME{dtbpF}{4.8888in}{4.0075in}{0in}{}{}{Figure}{\special{language
"Scientific Word";type "GRAPHIC";maintain-aspect-ratio TRUE;display
"USEDEF";valid_file "T";width 4.8888in;height 4.0075in;depth
0in;original-width 4.8352in;original-height 3.9583in;cropleft "0";croptop
"1";cropright "1";cropbottom "0";tempfilename
'PQXXR10H.wmf';tempfile-properties "XPR";}}

The $P%
%TCIMACRO{\TeXButton{V}{\ooalign{\hfil$V$\hfil\cr\kern0.1em--\hfil\cr}}}%
%BeginExpansion
\ooalign{\hfil$V$\hfil\cr\kern0.1em--\hfil\cr}%
%EndExpansion
\ $diagram above shows the Otto cycle. This cycle is approximately what
happens in a gasoline engine, like those in cars or, more like what we find
in lawn mowers.

For our engine, suppose the compression ratio is 
\[
\frac{%
%TCIMACRO{\TeXButton{V}{\ooalign{\hfil$V$\hfil\cr\kern0.1em--\hfil\cr}}}%
%BeginExpansion
\ooalign{\hfil$V$\hfil\cr\kern0.1em--\hfil\cr}%
%EndExpansion
_{A}}{%
%TCIMACRO{\TeXButton{V}{\ooalign{\hfil$V$\hfil\cr\kern0.1em--\hfil\cr}}}%
%BeginExpansion
\ooalign{\hfil$V$\hfil\cr\kern0.1em--\hfil\cr}%
%EndExpansion
_{B}}=8.00 
\]%
Let's take 
\begin{eqnarray*}
%TCIMACRO{\TeXButton{V}{\ooalign{\hfil$V$\hfil\cr\kern0.1em--\hfil\cr}}}%
%BeginExpansion
\ooalign{\hfil$V$\hfil\cr\kern0.1em--\hfil\cr}%
%EndExpansion
_{A} &=&500\unit{cm}^{3}=\allowbreak 0.000\,5\unit{m}^{3} \\
P_{A} &=&100\unit{kPa} \\
T_{A} &=&293\unit{K}
\end{eqnarray*}%
then we can find $%
%TCIMACRO{\TeXButton{V}{\ooalign{\hfil$V$\hfil\cr\kern0.1em--\hfil\cr}}}%
%BeginExpansion
\ooalign{\hfil$V$\hfil\cr\kern0.1em--\hfil\cr}%
%EndExpansion
_{B}$ 
\[
%TCIMACRO{\TeXButton{V}{\ooalign{\hfil$V$\hfil\cr\kern0.1em--\hfil\cr}}}%
%BeginExpansion
\ooalign{\hfil$V$\hfil\cr\kern0.1em--\hfil\cr}%
%EndExpansion
_{B}=\frac{%
%TCIMACRO{\TeXButton{V}{\ooalign{\hfil$V$\hfil\cr\kern0.1em--\hfil\cr}}}%
%BeginExpansion
\ooalign{\hfil$V$\hfil\cr\kern0.1em--\hfil\cr}%
%EndExpansion
_{A}}{8}=\allowbreak 62.\,\allowbreak 5\unit{cm}^{3}=\allowbreak
6.\,\allowbreak 25\times 10^{-5}\unit{m}^{3} 
\]%
and we will need to measure the temperature at $C$%
\[
T_{C}=1023\unit{K} 
\]%
and the intake temperature 
\[
T_{I}=273\unit{K} 
\]%
(which probably tells us that this must be a road trip in Rexburg, not
Phoenix). Further suppose our engine uses a diatomic ideal gas with 
\begin{eqnarray*}
C_{V} &=&\frac{5}{2}R \\
R &=&8.314\frac{\unit{J}}{\unit{mol}\unit{K}}
\end{eqnarray*}%
so 
\[
\gamma =1.4 
\]%
Let's follow the process through to see what the state variables are at each
point on the PV\ diagram, and what $Q_{h}$ and $Q_{c}$ are for each process,
and what the work, $W,$ is and what the change in internal energy will be, $%
\Delta E_{int}.$ Finally, we will find the efficiency of the engine.\FRAME{%
dtbpF}{5.1906in}{3.0156in}{0pt}{}{}{Figure}{\special{language "Scientific
Word";type "GRAPHIC";maintain-aspect-ratio TRUE;display "USEDEF";valid_file
"T";width 5.1906in;height 3.0156in;depth 0pt;original-width
5.1353in;original-height 2.9715in;cropleft "0";croptop "1";cropright
"1";cropbottom "0";tempfilename 'PQXXR10I.wmf';tempfile-properties "XPR";}}

Process $A\rightarrow B$ is an adiabatic compression. The piston moves
upward, compressing the fuel-air mixture. This happens rapidly, so there is
no time for energy to leave by heat, $Q=0.$ In this process $\Delta
E_{int}=W.$

we can use 
\[
P_{i}%
%TCIMACRO{\TeXButton{V}{\ooalign{\hfil$V$\hfil\cr\kern0.1em--\hfil\cr}}}%
%BeginExpansion
\ooalign{\hfil$V$\hfil\cr\kern0.1em--\hfil\cr}%
%EndExpansion
_{i}^{\gamma }=P_{f}%
%TCIMACRO{\TeXButton{V}{\ooalign{\hfil$V$\hfil\cr\kern0.1em--\hfil\cr}}}%
%BeginExpansion
\ooalign{\hfil$V$\hfil\cr\kern0.1em--\hfil\cr}%
%EndExpansion
_{f}^{\gamma } 
\]%
to find the final pressure for this process at $B$%
\[
P_{B}=P_{A}\frac{%
%TCIMACRO{\TeXButton{V}{\ooalign{\hfil$V$\hfil\cr\kern0.1em--\hfil\cr}}}%
%BeginExpansion
\ooalign{\hfil$V$\hfil\cr\kern0.1em--\hfil\cr}%
%EndExpansion
_{A}^{\gamma }}{%
%TCIMACRO{\TeXButton{V}{\ooalign{\hfil$V$\hfil\cr\kern0.1em--\hfil\cr}}}%
%BeginExpansion
\ooalign{\hfil$V$\hfil\cr\kern0.1em--\hfil\cr}%
%EndExpansion
_{B}^{\gamma }} 
\]

We can find the work from 
\begin{eqnarray*}
W_{AB} &=&-\int_{%
%TCIMACRO{\TeXButton{V}{\ooalign{\hfil$V$\hfil\cr\kern0.1em--\hfil\cr}}}%
%BeginExpansion
\ooalign{\hfil$V$\hfil\cr\kern0.1em--\hfil\cr}%
%EndExpansion
_{A}}^{%
%TCIMACRO{\TeXButton{V}{\ooalign{\hfil$V$\hfil\cr\kern0.1em--\hfil\cr}}}%
%BeginExpansion
\ooalign{\hfil$V$\hfil\cr\kern0.1em--\hfil\cr}%
%EndExpansion
_{B}}Pd%
%TCIMACRO{\TeXButton{V}{\ooalign{\hfil$V$\hfil\cr\kern0.1em--\hfil\cr}} }%
%BeginExpansion
\ooalign{\hfil$V$\hfil\cr\kern0.1em--\hfil\cr}
%EndExpansion
\\
&=&-\int_{%
%TCIMACRO{\TeXButton{V}{\ooalign{\hfil$V$\hfil\cr\kern0.1em--\hfil\cr}}}%
%BeginExpansion
\ooalign{\hfil$V$\hfil\cr\kern0.1em--\hfil\cr}%
%EndExpansion
_{A}}^{%
%TCIMACRO{\TeXButton{V}{\ooalign{\hfil$V$\hfil\cr\kern0.1em--\hfil\cr}}}%
%BeginExpansion
\ooalign{\hfil$V$\hfil\cr\kern0.1em--\hfil\cr}%
%EndExpansion
_{B}}P_{A}\frac{%
%TCIMACRO{\TeXButton{V}{\ooalign{\hfil$V$\hfil\cr\kern0.1em--\hfil\cr}}}%
%BeginExpansion
\ooalign{\hfil$V$\hfil\cr\kern0.1em--\hfil\cr}%
%EndExpansion
_{A}^{\gamma }}{%
%TCIMACRO{\TeXButton{V}{\ooalign{\hfil$V$\hfil\cr\kern0.1em--\hfil\cr}}}%
%BeginExpansion
\ooalign{\hfil$V$\hfil\cr\kern0.1em--\hfil\cr}%
%EndExpansion
^{\gamma }}d%
%TCIMACRO{\TeXButton{V}{\ooalign{\hfil$V$\hfil\cr\kern0.1em--\hfil\cr}} }%
%BeginExpansion
\ooalign{\hfil$V$\hfil\cr\kern0.1em--\hfil\cr}
%EndExpansion
\\
&=&-P_{A}%
%TCIMACRO{\TeXButton{V}{\ooalign{\hfil$V$\hfil\cr\kern0.1em--\hfil\cr}}}%
%BeginExpansion
\ooalign{\hfil$V$\hfil\cr\kern0.1em--\hfil\cr}%
%EndExpansion
_{A}^{\gamma }\int_{%
%TCIMACRO{\TeXButton{V}{\ooalign{\hfil$V$\hfil\cr\kern0.1em--\hfil\cr}}}%
%BeginExpansion
\ooalign{\hfil$V$\hfil\cr\kern0.1em--\hfil\cr}%
%EndExpansion
_{A}}^{%
%TCIMACRO{\TeXButton{V}{\ooalign{\hfil$V$\hfil\cr\kern0.1em--\hfil\cr}}}%
%BeginExpansion
\ooalign{\hfil$V$\hfil\cr\kern0.1em--\hfil\cr}%
%EndExpansion
_{B}}\frac{1}{%
%TCIMACRO{\TeXButton{V}{\ooalign{\hfil$V$\hfil\cr\kern0.1em--\hfil\cr}}}%
%BeginExpansion
\ooalign{\hfil$V$\hfil\cr\kern0.1em--\hfil\cr}%
%EndExpansion
^{\gamma }}d%
%TCIMACRO{\TeXButton{V}{\ooalign{\hfil$V$\hfil\cr\kern0.1em--\hfil\cr}} }%
%BeginExpansion
\ooalign{\hfil$V$\hfil\cr\kern0.1em--\hfil\cr}
%EndExpansion
\\
&=&-P_{A}^{\gamma }%
%TCIMACRO{\TeXButton{V}{\ooalign{\hfil$V$\hfil\cr\kern0.1em--\hfil\cr}}}%
%BeginExpansion
\ooalign{\hfil$V$\hfil\cr\kern0.1em--\hfil\cr}%
%EndExpansion
_{A}^{\gamma }\left( \frac{%
%TCIMACRO{\TeXButton{V}{\ooalign{\hfil$V$\hfil\cr\kern0.1em--\hfil\cr}}}%
%BeginExpansion
\ooalign{\hfil$V$\hfil\cr\kern0.1em--\hfil\cr}%
%EndExpansion
^{1-\gamma }}{1-\gamma }\right\vert _{%
%TCIMACRO{\TeXButton{V}{\ooalign{\hfil$V$\hfil\cr\kern0.1em--\hfil\cr}}}%
%BeginExpansion
\ooalign{\hfil$V$\hfil\cr\kern0.1em--\hfil\cr}%
%EndExpansion
_{A}}^{%
%TCIMACRO{\TeXButton{V}{\ooalign{\hfil$V$\hfil\cr\kern0.1em--\hfil\cr}}}%
%BeginExpansion
\ooalign{\hfil$V$\hfil\cr\kern0.1em--\hfil\cr}%
%EndExpansion
_{B}} \\
&=&-\frac{P_{A}%
%TCIMACRO{\TeXButton{V}{\ooalign{\hfil$V$\hfil\cr\kern0.1em--\hfil\cr}}}%
%BeginExpansion
\ooalign{\hfil$V$\hfil\cr\kern0.1em--\hfil\cr}%
%EndExpansion
_{A}^{\gamma }}{1-\gamma }\left( 
%TCIMACRO{\TeXButton{V}{\ooalign{\hfil$V$\hfil\cr\kern0.1em--\hfil\cr}}}%
%BeginExpansion
\ooalign{\hfil$V$\hfil\cr\kern0.1em--\hfil\cr}%
%EndExpansion
_{B}^{1-\gamma }-%
%TCIMACRO{\TeXButton{V}{\ooalign{\hfil$V$\hfil\cr\kern0.1em--\hfil\cr}}}%
%BeginExpansion
\ooalign{\hfil$V$\hfil\cr\kern0.1em--\hfil\cr}%
%EndExpansion
_{A}^{1-\gamma }\right)
\end{eqnarray*}%
We can rearrange this into a more convenient formula%
\begin{eqnarray*}
W_{AB} &=&\frac{P_{A}%
%TCIMACRO{\TeXButton{V}{\ooalign{\hfil$V$\hfil\cr\kern0.1em--\hfil\cr}}}%
%BeginExpansion
\ooalign{\hfil$V$\hfil\cr\kern0.1em--\hfil\cr}%
%EndExpansion
_{A}^{\gamma }}{\gamma -1}\left( \frac{1}{%
%TCIMACRO{\TeXButton{V}{\ooalign{\hfil$V$\hfil\cr\kern0.1em--\hfil\cr}}}%
%BeginExpansion
\ooalign{\hfil$V$\hfil\cr\kern0.1em--\hfil\cr}%
%EndExpansion
_{B}^{\gamma -1}}-\frac{1}{%
%TCIMACRO{\TeXButton{V}{\ooalign{\hfil$V$\hfil\cr\kern0.1em--\hfil\cr}}}%
%BeginExpansion
\ooalign{\hfil$V$\hfil\cr\kern0.1em--\hfil\cr}%
%EndExpansion
_{A}^{\gamma -1}}\right) \\
&=&\frac{1}{\gamma -1}\left( \frac{P_{A}%
%TCIMACRO{\TeXButton{V}{\ooalign{\hfil$V$\hfil\cr\kern0.1em--\hfil\cr}}}%
%BeginExpansion
\ooalign{\hfil$V$\hfil\cr\kern0.1em--\hfil\cr}%
%EndExpansion
_{A}^{\gamma }}{%
%TCIMACRO{\TeXButton{V}{\ooalign{\hfil$V$\hfil\cr\kern0.1em--\hfil\cr}}}%
%BeginExpansion
\ooalign{\hfil$V$\hfil\cr\kern0.1em--\hfil\cr}%
%EndExpansion
_{B}^{\gamma -1}}-\frac{P_{A}%
%TCIMACRO{\TeXButton{V}{\ooalign{\hfil$V$\hfil\cr\kern0.1em--\hfil\cr}}}%
%BeginExpansion
\ooalign{\hfil$V$\hfil\cr\kern0.1em--\hfil\cr}%
%EndExpansion
_{A}^{\gamma }}{%
%TCIMACRO{\TeXButton{V}{\ooalign{\hfil$V$\hfil\cr\kern0.1em--\hfil\cr}}}%
%BeginExpansion
\ooalign{\hfil$V$\hfil\cr\kern0.1em--\hfil\cr}%
%EndExpansion
_{A}^{\gamma -1}}\right) \\
&=&\frac{1}{\gamma -1}\left( \frac{P_{A}%
%TCIMACRO{\TeXButton{V}{\ooalign{\hfil$V$\hfil\cr\kern0.1em--\hfil\cr}}}%
%BeginExpansion
\ooalign{\hfil$V$\hfil\cr\kern0.1em--\hfil\cr}%
%EndExpansion
_{A}^{\gamma }}{%
%TCIMACRO{\TeXButton{V}{\ooalign{\hfil$V$\hfil\cr\kern0.1em--\hfil\cr}}}%
%BeginExpansion
\ooalign{\hfil$V$\hfil\cr\kern0.1em--\hfil\cr}%
%EndExpansion
_{B}^{\gamma }%
%TCIMACRO{\TeXButton{V}{\ooalign{\hfil$V$\hfil\cr\kern0.1em--\hfil\cr}}}%
%BeginExpansion
\ooalign{\hfil$V$\hfil\cr\kern0.1em--\hfil\cr}%
%EndExpansion
_{B}^{-1}}-\frac{P_{A}%
%TCIMACRO{\TeXButton{V}{\ooalign{\hfil$V$\hfil\cr\kern0.1em--\hfil\cr}}}%
%BeginExpansion
\ooalign{\hfil$V$\hfil\cr\kern0.1em--\hfil\cr}%
%EndExpansion
_{A}^{\gamma }}{%
%TCIMACRO{\TeXButton{V}{\ooalign{\hfil$V$\hfil\cr\kern0.1em--\hfil\cr}}}%
%BeginExpansion
\ooalign{\hfil$V$\hfil\cr\kern0.1em--\hfil\cr}%
%EndExpansion
_{A}^{\gamma -1}}\right) \\
&=&\frac{1}{\gamma -1}\left( P_{B}%
%TCIMACRO{\TeXButton{V}{\ooalign{\hfil$V$\hfil\cr\kern0.1em--\hfil\cr}}}%
%BeginExpansion
\ooalign{\hfil$V$\hfil\cr\kern0.1em--\hfil\cr}%
%EndExpansion
_{B}-P_{A}%
%TCIMACRO{\TeXButton{V}{\ooalign{\hfil$V$\hfil\cr\kern0.1em--\hfil\cr}}}%
%BeginExpansion
\ooalign{\hfil$V$\hfil\cr\kern0.1em--\hfil\cr}%
%EndExpansion
_{A}\right)
\end{eqnarray*}%
which we can generalize into another basic equation for adiabatic processes.%
\[
W_{AB}=\left( \frac{1}{\gamma -1}\right) \left( P_{f}%
%TCIMACRO{\TeXButton{V}{\ooalign{\hfil$V$\hfil\cr\kern0.1em--\hfil\cr}}}%
%BeginExpansion
\ooalign{\hfil$V$\hfil\cr\kern0.1em--\hfil\cr}%
%EndExpansion
_{f}-P_{i}%
%TCIMACRO{\TeXButton{V}{\ooalign{\hfil$V$\hfil\cr\kern0.1em--\hfil\cr}}}%
%BeginExpansion
\ooalign{\hfil$V$\hfil\cr\kern0.1em--\hfil\cr}%
%EndExpansion
_{i}\right) 
\]%
then for process $A\rightarrow B$ we have%
\begin{eqnarray*}
W_{AB} &=&\left( \frac{1}{\gamma -1}\right) \left( P_{B}%
%TCIMACRO{\TeXButton{V}{\ooalign{\hfil$V$\hfil\cr\kern0.1em--\hfil\cr}}}%
%BeginExpansion
\ooalign{\hfil$V$\hfil\cr\kern0.1em--\hfil\cr}%
%EndExpansion
_{B}-P_{A}%
%TCIMACRO{\TeXButton{V}{\ooalign{\hfil$V$\hfil\cr\kern0.1em--\hfil\cr}}}%
%BeginExpansion
\ooalign{\hfil$V$\hfil\cr\kern0.1em--\hfil\cr}%
%EndExpansion
_{A}\right) \\
&=&\left( \frac{1}{\gamma -1}\right) \left( P_{A}\frac{%
%TCIMACRO{\TeXButton{V}{\ooalign{\hfil$V$\hfil\cr\kern0.1em--\hfil\cr}}}%
%BeginExpansion
\ooalign{\hfil$V$\hfil\cr\kern0.1em--\hfil\cr}%
%EndExpansion
_{A}^{\gamma }}{%
%TCIMACRO{\TeXButton{V}{\ooalign{\hfil$V$\hfil\cr\kern0.1em--\hfil\cr}}}%
%BeginExpansion
\ooalign{\hfil$V$\hfil\cr\kern0.1em--\hfil\cr}%
%EndExpansion
_{B}^{\gamma }}%
%TCIMACRO{\TeXButton{V}{\ooalign{\hfil$V$\hfil\cr\kern0.1em--\hfil\cr}}}%
%BeginExpansion
\ooalign{\hfil$V$\hfil\cr\kern0.1em--\hfil\cr}%
%EndExpansion
_{B}-P_{A}%
%TCIMACRO{\TeXButton{V}{\ooalign{\hfil$V$\hfil\cr\kern0.1em--\hfil\cr}}}%
%BeginExpansion
\ooalign{\hfil$V$\hfil\cr\kern0.1em--\hfil\cr}%
%EndExpansion
_{A}\right) \\
&=&\left( \frac{1}{\gamma -1}\right) \left( P_{A}\frac{%
%TCIMACRO{\TeXButton{V}{\ooalign{\hfil$V$\hfil\cr\kern0.1em--\hfil\cr}}}%
%BeginExpansion
\ooalign{\hfil$V$\hfil\cr\kern0.1em--\hfil\cr}%
%EndExpansion
_{A}^{\gamma }}{\left( \frac{%
%TCIMACRO{\TeXButton{V}{\ooalign{\hfil$V$\hfil\cr\kern0.1em--\hfil\cr}}}%
%BeginExpansion
\ooalign{\hfil$V$\hfil\cr\kern0.1em--\hfil\cr}%
%EndExpansion
_{A}}{8}\right) ^{\gamma }}%
%TCIMACRO{\TeXButton{V}{\ooalign{\hfil$V$\hfil\cr\kern0.1em--\hfil\cr}}}%
%BeginExpansion
\ooalign{\hfil$V$\hfil\cr\kern0.1em--\hfil\cr}%
%EndExpansion
_{B}-P_{A}%
%TCIMACRO{\TeXButton{V}{\ooalign{\hfil$V$\hfil\cr\kern0.1em--\hfil\cr}}}%
%BeginExpansion
\ooalign{\hfil$V$\hfil\cr\kern0.1em--\hfil\cr}%
%EndExpansion
_{A}\right) \\
&=&\left( \frac{1}{\gamma -1}\right) \left( 8^{\gamma }P_{A}\left( \frac{%
%TCIMACRO{\TeXButton{V}{\ooalign{\hfil$V$\hfil\cr\kern0.1em--\hfil\cr}}}%
%BeginExpansion
\ooalign{\hfil$V$\hfil\cr\kern0.1em--\hfil\cr}%
%EndExpansion
_{A}}{8}\right) -P_{A}%
%TCIMACRO{\TeXButton{V}{\ooalign{\hfil$V$\hfil\cr\kern0.1em--\hfil\cr}}}%
%BeginExpansion
\ooalign{\hfil$V$\hfil\cr\kern0.1em--\hfil\cr}%
%EndExpansion
_{A}\right) \\
&=&\left( \frac{%
%TCIMACRO{\TeXButton{V}{\ooalign{\hfil$V$\hfil\cr\kern0.1em--\hfil\cr}}}%
%BeginExpansion
\ooalign{\hfil$V$\hfil\cr\kern0.1em--\hfil\cr}%
%EndExpansion
_{A}}{\gamma -1}\right) \left( 8^{\gamma -1}P_{A}-P_{A}\right) \\
&=&\left( \frac{%
%TCIMACRO{\TeXButton{V}{\ooalign{\hfil$V$\hfil\cr\kern0.1em--\hfil\cr}}}%
%BeginExpansion
\ooalign{\hfil$V$\hfil\cr\kern0.1em--\hfil\cr}%
%EndExpansion
_{A}P_{A}}{\gamma -1}\right) \left( 8^{\gamma -1}-1\right)
\end{eqnarray*}%
so the work is 
\begin{eqnarray*}
W_{AB} &=&\left( \frac{%
%TCIMACRO{\TeXButton{V}{\ooalign{\hfil$V$\hfil\cr\kern0.1em--\hfil\cr}}}%
%BeginExpansion
\ooalign{\hfil$V$\hfil\cr\kern0.1em--\hfil\cr}%
%EndExpansion
_{A}P_{A}}{\gamma -1}\right) \left( 8^{\gamma -1}-1\right) \\
&=&\left( \frac{\left( \allowbreak 0.000\,5\unit{m}^{3}\right) \left( 100%
\unit{kPa}\right) }{1.4-1}\right) \left( 8^{1.4-1}-1\right) \\
&=&\allowbreak 162.\,\allowbreak 17\unit{J}
\end{eqnarray*}

But this is the work done on the gas, the useful work is the negative of this%
\[
W_{eng_{AB}}=-\allowbreak 162.\,\allowbreak 17\unit{J} 
\]%
and the change in internal energy is 
\[
\Delta E_{AB}=\allowbreak 162.\,\allowbreak 17\unit{J} 
\]%
and we can find 
\[
E_{int}=\frac{5}{2}nRT 
\]%
but we need $n.$ We can get it from what we know at $A$%
\begin{eqnarray*}
n &=&\frac{P_{A}%
%TCIMACRO{\TeXButton{V}{\ooalign{\hfil$V$\hfil\cr\kern0.1em--\hfil\cr}}}%
%BeginExpansion
\ooalign{\hfil$V$\hfil\cr\kern0.1em--\hfil\cr}%
%EndExpansion
_{A}}{RT_{A}} \\
&=&\frac{\left( 100\unit{kPa}\right) \left( \allowbreak 0.000\,5\unit{m}%
^{3}\right) }{\left( 8.314\frac{\unit{J}}{\unit{mol}\unit{K}}\right) \left(
293\unit{K}\right) } \\
&=&\allowbreak 2.\,\allowbreak 052\,5\times 10^{-2}\unit{mol}
\end{eqnarray*}%
then%
\begin{eqnarray*}
E_{A} &=&\frac{5}{2}nRT_{A} \\
&=&\frac{5}{2}\left( \allowbreak 2.\,\allowbreak 052\,5\times 10^{-2}\unit{%
mol}\right) \left( 8.314\frac{\unit{J}}{\unit{mol}\unit{K}}\right) \left( 293%
\unit{K}\right) \\
&=&\allowbreak 125.\,\allowbreak 00\unit{J}
\end{eqnarray*}

We used 
\[
P_{B}=P_{A}\frac{%
%TCIMACRO{\TeXButton{V}{\ooalign{\hfil$V$\hfil\cr\kern0.1em--\hfil\cr}}}%
%BeginExpansion
\ooalign{\hfil$V$\hfil\cr\kern0.1em--\hfil\cr}%
%EndExpansion
_{A}^{\gamma }}{%
%TCIMACRO{\TeXButton{V}{\ooalign{\hfil$V$\hfil\cr\kern0.1em--\hfil\cr}}}%
%BeginExpansion
\ooalign{\hfil$V$\hfil\cr\kern0.1em--\hfil\cr}%
%EndExpansion
_{B}^{\gamma }} 
\]%
but we should find a numeric answer for $P_{B}$%
\begin{eqnarray*}
P_{B} &=&P_{A}\frac{%
%TCIMACRO{\TeXButton{V}{\ooalign{\hfil$V$\hfil\cr\kern0.1em--\hfil\cr}}}%
%BeginExpansion
\ooalign{\hfil$V$\hfil\cr\kern0.1em--\hfil\cr}%
%EndExpansion
_{A}^{\gamma }}{\left( \frac{%
%TCIMACRO{\TeXButton{V}{\ooalign{\hfil$V$\hfil\cr\kern0.1em--\hfil\cr}}}%
%BeginExpansion
\ooalign{\hfil$V$\hfil\cr\kern0.1em--\hfil\cr}%
%EndExpansion
_{A}}{8}\right) ^{\gamma }} \\
&=&P_{A}\frac{%
%TCIMACRO{\TeXButton{V}{\ooalign{\hfil$V$\hfil\cr\kern0.1em--\hfil\cr}}}%
%BeginExpansion
\ooalign{\hfil$V$\hfil\cr\kern0.1em--\hfil\cr}%
%EndExpansion
_{A}^{\gamma }}{\frac{%
%TCIMACRO{\TeXButton{V}{\ooalign{\hfil$V$\hfil\cr\kern0.1em--\hfil\cr}}}%
%BeginExpansion
\ooalign{\hfil$V$\hfil\cr\kern0.1em--\hfil\cr}%
%EndExpansion
_{A}^{\gamma }}{8^{\gamma }}} \\
&=&8^{\gamma }P_{A} \\
&=&8^{1.4}\left( 100\unit{kPa}\right) \\
&=&\allowbreak 1837.\,\allowbreak 9\unit{kPa}
\end{eqnarray*}%
and finally, from the ideal gas law 
\begin{eqnarray*}
T_{B} &=&\frac{P_{B}%
%TCIMACRO{\TeXButton{V}{\ooalign{\hfil$V$\hfil\cr\kern0.1em--\hfil\cr}}}%
%BeginExpansion
\ooalign{\hfil$V$\hfil\cr\kern0.1em--\hfil\cr}%
%EndExpansion
_{B}}{nR} \\
&=&\frac{\left( 1837.\,\allowbreak 9\unit{kPa}\right) \left( 6.\,25\times
10^{-5}\unit{m}^{3}\right) }{\left( 2.\,052\,5\times 10^{-2}\unit{mol}%
\right) \left( 8.314\frac{\unit{J}}{\unit{mol}\unit{K}}\right) } \\
&=&673.\,\allowbreak 15\unit{K}
\end{eqnarray*}%
at last we can find $E_{B}$%
\begin{eqnarray*}
E_{B} &=&\frac{5}{2}nRT_{B} \\
&=&\frac{5}{2}\left( \allowbreak 2.\,\allowbreak 052\,5\times 10^{-2}\unit{%
mol}\right) \left( 8.314\frac{\unit{J}}{\unit{mol}\unit{K}}\right) \left(
673.\,\allowbreak 15\unit{K}\right) \\
&=&\allowbreak 287.\,\allowbreak 17\unit{J}
\end{eqnarray*}%
what we have learned is summarized in this table

\[
\begin{tabular}{|c|c|c|c|c|c|c|}
\hline
\textbf{State} & $\mathbf{T}\left( \unit{K}\right) $ & $\mathbf{P}\left( 
\unit{kPa}\right) $ & $%
%TCIMACRO{\TeXButton{V}{\ooalign{\hfil$V$\hfil\cr\kern0.1em--\hfil\cr}}}%
%BeginExpansion
\ooalign{\hfil$V$\hfil\cr\kern0.1em--\hfil\cr}%
%EndExpansion
\left( \unit{m}^{3}\right) $ & $\mathbf{n}\left( \unit{mol}\right) $ & $%
\mathbf{E}_{int}\left( \unit{J}\right) $ & \textbf{--} \\ \hline
$\mathbf{A}$ & $293$ & $100$ & $0.000\,5$ & $2.\,052\,5\times 10^{-2}$ & $%
\allowbreak 125.\,\allowbreak 00$ & \textbf{--} \\ \hline
$\mathbf{B}$ & $673.\,\allowbreak 15$ & $\allowbreak 1837.\,\allowbreak 9$ & 
$6.\,25\times 10^{-5}$ & $2.\,052\,5\times 10^{-2}$ & $287.\,\allowbreak 17$
& \textbf{--} \\ \hline
\textbf{Process} & $\mathbf{Q}\left( \unit{J}\right) $ & $\mathbf{W}%
_{int}\left( \unit{J}\right) $ & $\mathbf{\Delta E}_{int}\left( \unit{J}%
\right) $ & $\mathbf{W}_{eng}\left( \unit{J}\right) $ & $\mathbf{Q}_{h}$ & $%
\mathbf{Q}_{c}$ \\ \hline
$\mathbf{AB}$ & $0$ & $\allowbreak 162.\,\allowbreak 17\unit{J}$ & $%
\allowbreak 162.\,\allowbreak 17\unit{J}$ & $-\allowbreak 162.\,\allowbreak
17\unit{J}$ & $\mathbf{0}$ & $\mathbf{0}$ \\ \hline
\end{tabular}%
\]

That was just one process!

\FRAME{dtbpF}{3.3304in}{2.9568in}{0in}{}{}{Figure}{\special{language
"Scientific Word";type "GRAPHIC";maintain-aspect-ratio TRUE;display
"USEDEF";valid_file "T";width 3.3304in;height 2.9568in;depth
0in;original-width 3.2846in;original-height 2.9136in;cropleft "0";croptop
"1";cropright "1";cropbottom "0";tempfilename
'PQXXR10J.wmf';tempfile-properties "XPR";}}

Let's take on the next process $B\rightarrow C.$ This is where the spark
plug ignites the air-fuel mixture created by the fuel injectors. A great
deal of energy is released in a hurry. The temperature jumps up and the
pressure does too, but the volume is the same because this process happens
in a flash--literally!

So this must be an isochoric process. We expect $W_{BC}=0,$ so $\Delta
E_{BC}=Q_{BC}.$ The pressure increases, so we expect $Q_{BC}$ to be
positive. This must be part of $Q_{h}$! We know the temperature at $C$ and
we know the volume so we can find the final pressure%
\begin{eqnarray*}
P_{C} &=&\frac{nRT_{C}}{%
%TCIMACRO{\TeXButton{V}{\ooalign{\hfil$V$\hfil\cr\kern0.1em--\hfil\cr}}}%
%BeginExpansion
\ooalign{\hfil$V$\hfil\cr\kern0.1em--\hfil\cr}%
%EndExpansion
_{C}} \\
&=&\frac{\left( 2.\,052\,5\times 10^{-2}\right) \left( 8.314\frac{\unit{J}}{%
\unit{mol}\unit{K}}\right) \left( 1023\right) }{6.\,25\times 10^{-5}\unit{m}%
^{3}} \\
&=&2.\,\allowbreak 793\,1\times 10^{6}\unit{Pa} \\
&=&2793.\,\allowbreak 1\unit{kPa}
\end{eqnarray*}%
We can find the internal energy at $C$ now%
\begin{eqnarray*}
E_{C} &=&\frac{5}{2}nRT_{C} \\
&=&\frac{5}{2}\left( \allowbreak 2.\,\allowbreak 052\,5\times 10^{-2}\unit{%
mol}\right) \left( 8.314\frac{\unit{J}}{\unit{mol}\unit{K}}\right) \left(
1023\unit{K}\right) \\
&=&\allowbreak 436.\,\allowbreak 42\unit{J}
\end{eqnarray*}%
We can use 
\begin{eqnarray*}
Q &=&nC_{V}\Delta T \\
&=&n\frac{5}{2}R\left( T_{C}-T_{B}\right)
\end{eqnarray*}%
For the process $BC$ to find the energy transfer by heat%
\begin{eqnarray*}
Q_{BC} &=&\left( 2.\,052\,5\times 10^{-2}\unit{mol}\right) \frac{5}{2}\left(
8.314\frac{\unit{J}}{\unit{mol}\unit{K}}\right) \left( 1023\unit{K}%
-673.\,\allowbreak 15\unit{K}\right) \\
&=&149.\,\allowbreak 25\unit{J}
\end{eqnarray*}%
We already mentioned that $Q_{BC}=\Delta E_{int_{BC}}$ and $W_{BC}=0,$ so we
can complete our table of state variables and process variables for this
process. 
\[
\begin{tabular}{|c|c|c|c|c|c|c|}
\hline
\textbf{State} & $\mathbf{T}\left( \unit{K}\right) $ & $\mathbf{P}\left( 
\unit{kPa}\right) $ & $%
%TCIMACRO{\TeXButton{V}{\ooalign{\hfil$V$\hfil\cr\kern0.1em--\hfil\cr}}}%
%BeginExpansion
\ooalign{\hfil$V$\hfil\cr\kern0.1em--\hfil\cr}%
%EndExpansion
\left( \unit{m}^{3}\right) $ & $\mathbf{n}\left( \unit{mol}\right) $ & $%
\mathbf{E}_{int}\left( \unit{J}\right) $ & \textbf{--} \\ \hline
$\mathbf{B}$ & $673.\,\allowbreak 15$ & $\allowbreak 1837.\,\allowbreak 9$ & 
$6.\,25\times 10^{-5}$ & $2.\,052\,5\times 10^{-2}$ & $287.\,\allowbreak 17$
& \textbf{--} \\ \hline
$\mathbf{C}$ & $1023$ & $2793.\,\allowbreak 1$ & $6.\,25\times 10^{-5}$ & $%
2.\,052\,5\times 10^{-2}$ & $436.\,\allowbreak 42$ & \textbf{--} \\ \hline
\textbf{Process} & $\mathbf{Q}\left( \unit{J}\right) $ & $\mathbf{W}%
_{int}\left( \unit{J}\right) $ & $\mathbf{\Delta E}_{int}\left( \unit{J}%
\right) $ & $\mathbf{W}_{eng}\left( \unit{J}\right) $ & $\mathbf{Q}_{h}$ & $%
Q_{c}$ \\ \hline
$\mathbf{BC}$ & $149.\,\allowbreak 25$ & $0$ & $149.\,\allowbreak 25$ & $%
\mathbf{0}$ & $149.\,\allowbreak 25$ & $0$ \\ \hline
\end{tabular}%
\]

\FRAME{dtbpF}{5.4812in}{3.211in}{0pt}{}{}{Figure}{\special{language
"Scientific Word";type "GRAPHIC";maintain-aspect-ratio TRUE;display
"USEDEF";valid_file "T";width 5.4812in;height 3.211in;depth
0pt;original-width 6.3278in;original-height 3.6962in;cropleft "0";croptop
"1";cropright "1";cropbottom "0";tempfilename
'PQXXR10K.wmf';tempfile-properties "XPR";}}

The next process is $C\rightarrow D.$ This is another adiabatic process,
this time an expansion. This is sometimes called the \textquotedblleft power
stroke\textquotedblright\ because the hot, high pressure gas pushes the
piston down. This push is what makes the car go. We know that $Q_{CD}=0$ so $%
\Delta E_{CD}=W_{CD}$

To find the ideal gas state variables at $D$ let's start with 
\[
P_{i}%
%TCIMACRO{\TeXButton{V}{\ooalign{\hfil$V$\hfil\cr\kern0.1em--\hfil\cr}}}%
%BeginExpansion
\ooalign{\hfil$V$\hfil\cr\kern0.1em--\hfil\cr}%
%EndExpansion
_{i}^{\gamma }=P_{f}%
%TCIMACRO{\TeXButton{V}{\ooalign{\hfil$V$\hfil\cr\kern0.1em--\hfil\cr}}}%
%BeginExpansion
\ooalign{\hfil$V$\hfil\cr\kern0.1em--\hfil\cr}%
%EndExpansion
_{f}^{\gamma } 
\]%
again, then 
\[
P_{D}=P_{C}\frac{%
%TCIMACRO{\TeXButton{V}{\ooalign{\hfil$V$\hfil\cr\kern0.1em--\hfil\cr}}}%
%BeginExpansion
\ooalign{\hfil$V$\hfil\cr\kern0.1em--\hfil\cr}%
%EndExpansion
_{C}^{\gamma }}{%
%TCIMACRO{\TeXButton{V}{\ooalign{\hfil$V$\hfil\cr\kern0.1em--\hfil\cr}}}%
%BeginExpansion
\ooalign{\hfil$V$\hfil\cr\kern0.1em--\hfil\cr}%
%EndExpansion
_{D}^{\gamma }} 
\]%
but from our PV\ diagram we see $%
%TCIMACRO{\TeXButton{V}{\ooalign{\hfil$V$\hfil\cr\kern0.1em--\hfil\cr}}}%
%BeginExpansion
\ooalign{\hfil$V$\hfil\cr\kern0.1em--\hfil\cr}%
%EndExpansion
_{C}=%
%TCIMACRO{\TeXButton{V}{\ooalign{\hfil$V$\hfil\cr\kern0.1em--\hfil\cr}}}%
%BeginExpansion
\ooalign{\hfil$V$\hfil\cr\kern0.1em--\hfil\cr}%
%EndExpansion
_{B}$ and $%
%TCIMACRO{\TeXButton{V}{\ooalign{\hfil$V$\hfil\cr\kern0.1em--\hfil\cr}}}%
%BeginExpansion
\ooalign{\hfil$V$\hfil\cr\kern0.1em--\hfil\cr}%
%EndExpansion
_{D}=%
%TCIMACRO{\TeXButton{V}{\ooalign{\hfil$V$\hfil\cr\kern0.1em--\hfil\cr}}}%
%BeginExpansion
\ooalign{\hfil$V$\hfil\cr\kern0.1em--\hfil\cr}%
%EndExpansion
_{A}$. This makes sense because the cylinder and piston system maximum and
minimum volumes don't change. So we can write%
\[
P_{D}=P_{C}\frac{%
%TCIMACRO{\TeXButton{V}{\ooalign{\hfil$V$\hfil\cr\kern0.1em--\hfil\cr}}}%
%BeginExpansion
\ooalign{\hfil$V$\hfil\cr\kern0.1em--\hfil\cr}%
%EndExpansion
_{B}^{\gamma }}{%
%TCIMACRO{\TeXButton{V}{\ooalign{\hfil$V$\hfil\cr\kern0.1em--\hfil\cr}}}%
%BeginExpansion
\ooalign{\hfil$V$\hfil\cr\kern0.1em--\hfil\cr}%
%EndExpansion
_{A}^{\gamma }}=8^{-\gamma }P_{C} 
\]%
and knowing this we can find the temperature at $D$ using the ideal gas law%
\[
T_{D}=\frac{P_{D}%
%TCIMACRO{\TeXButton{V}{\ooalign{\hfil$V$\hfil\cr\kern0.1em--\hfil\cr}}}%
%BeginExpansion
\ooalign{\hfil$V$\hfil\cr\kern0.1em--\hfil\cr}%
%EndExpansion
_{D}}{nR} 
\]%
let's numerically calculate $P_{D}$ and $T_{D}$ at this point.%
\begin{eqnarray*}
P_{D} &=&8^{-1.4}\left( 2793.\,\allowbreak 1\right) \\
&=&151.\,\allowbreak 97\unit{kPa}
\end{eqnarray*}%
and 
\begin{eqnarray*}
T_{D} &=&\frac{\left( 151.\,\allowbreak 97\unit{kPa}\right) \left( 0.000\,5%
\unit{m}^{3}\right) }{\left( 2.\,052\,5\times 10^{-2}\unit{mol}\right)
\left( 8.314\frac{\unit{J}}{\unit{mol}\unit{K}}\right) } \\
&=&445.\,\allowbreak 28\unit{K}
\end{eqnarray*}

and while we are at it we can complete our state $D$ by calculating $E_{D}$%
\begin{eqnarray*}
E_{D} &=&\frac{5}{2}nRT_{D} \\
&=&\frac{5}{2}\left( \allowbreak 2.\,\allowbreak 052\,5\times 10^{-2}\unit{%
mol}\right) \left( 8.314\frac{\unit{J}}{\unit{mol}\unit{K}}\right) \left(
445.\,\allowbreak 28\unit{K}\right) \\
&=&189.\,\allowbreak 96\unit{J}
\end{eqnarray*}%
Now for the work done, we again use our new adiabatic equation 
\begin{eqnarray*}
W_{CD} &=&\left( \frac{1}{\gamma -1}\right) \left( P_{D}%
%TCIMACRO{\TeXButton{V}{\ooalign{\hfil$V$\hfil\cr\kern0.1em--\hfil\cr}}}%
%BeginExpansion
\ooalign{\hfil$V$\hfil\cr\kern0.1em--\hfil\cr}%
%EndExpansion
_{D}-P_{C}%
%TCIMACRO{\TeXButton{V}{\ooalign{\hfil$V$\hfil\cr\kern0.1em--\hfil\cr}}}%
%BeginExpansion
\ooalign{\hfil$V$\hfil\cr\kern0.1em--\hfil\cr}%
%EndExpansion
_{C}\right) \\
&=&\left( \frac{1}{\gamma -1}\right) \left( 8^{-\gamma }P_{C}%
%TCIMACRO{\TeXButton{V}{\ooalign{\hfil$V$\hfil\cr\kern0.1em--\hfil\cr}}}%
%BeginExpansion
\ooalign{\hfil$V$\hfil\cr\kern0.1em--\hfil\cr}%
%EndExpansion
_{A}-P_{C}%
%TCIMACRO{\TeXButton{V}{\ooalign{\hfil$V$\hfil\cr\kern0.1em--\hfil\cr}}}%
%BeginExpansion
\ooalign{\hfil$V$\hfil\cr\kern0.1em--\hfil\cr}%
%EndExpansion
_{B}\right) \\
&=&\left( \frac{P_{C}}{\gamma -1}\right) \left( 8^{-\gamma }%
%TCIMACRO{\TeXButton{V}{\ooalign{\hfil$V$\hfil\cr\kern0.1em--\hfil\cr}}}%
%BeginExpansion
\ooalign{\hfil$V$\hfil\cr\kern0.1em--\hfil\cr}%
%EndExpansion
_{A}-%
%TCIMACRO{\TeXButton{V}{\ooalign{\hfil$V$\hfil\cr\kern0.1em--\hfil\cr}}}%
%BeginExpansion
\ooalign{\hfil$V$\hfil\cr\kern0.1em--\hfil\cr}%
%EndExpansion
_{B}\right)
\end{eqnarray*}%
all of which we know, so%
\begin{eqnarray*}
W_{CD} &=&\left( \frac{2793.1\unit{kPa}}{1.4-1}\right) \left( 8^{-1.4}\left(
0.000\,5\unit{m}^{3}\right) -6.\,25\times 10^{-5}\unit{m}^{3}\right) \\
&=&-246.\,\allowbreak 46\unit{J}
\end{eqnarray*}%
which completes our set for process $CD$ 
\[
\begin{tabular}{|c|c|c|c|c|c|c|}
\hline
\textbf{State} & $\mathbf{T}\left( \unit{K}\right) $ & $\mathbf{P}\left( 
\unit{kPa}\right) $ & $%
%TCIMACRO{\TeXButton{V}{\ooalign{\hfil$V$\hfil\cr\kern0.1em--\hfil\cr}}}%
%BeginExpansion
\ooalign{\hfil$V$\hfil\cr\kern0.1em--\hfil\cr}%
%EndExpansion
\left( \unit{m}^{3}\right) $ & $\mathbf{n}\left( \unit{mol}\right) $ & $%
\mathbf{E}_{int}\left( \unit{J}\right) $ & \textbf{--} \\ \hline
$\mathbf{C}$ & $1023$ & $2793.\,\allowbreak 1$ & $6.\,25\times 10^{-5}$ & $%
2.\,052\,5\times 10^{-2}$ & $436.\,\allowbreak 42$ & \textbf{--} \\ \hline
$\mathbf{D}$ & $445.\,\allowbreak 28$ & $151.\,\allowbreak 97$ & $0.000\,5$
& $2.\,052\,5\times 10^{-2}$ & $189.\,\allowbreak 96\unit{J}$ & \textbf{--}
\\ \hline
\textbf{Process} & $\mathbf{Q}\left( \unit{J}\right) $ & $\mathbf{W}%
_{int}\left( \unit{J}\right) $ & $\mathbf{\Delta E}_{int}\left( \unit{J}%
\right) $ & $\mathbf{W}_{eng}\left( \unit{J}\right) $ & $\mathbf{Q}_{h}$ & $%
\mathbf{Q}_{c}$ \\ \hline
$\mathbf{CD}$ & $0$ & $-246.\,\allowbreak 46$ & $-246.\,\allowbreak 46$ & $%
246.\,\allowbreak 46$ & $\mathbf{0}$ & $\mathbf{0}$ \\ \hline
\end{tabular}%
\]

We are nearing the end of the cycle, In process $D\rightarrow A$ a valve
opens letting the pressure drop quickly. \FRAME{dtbpF}{3.0891in}{2.5927in}{%
0pt}{}{}{Figure}{\special{language "Scientific Word";type
"GRAPHIC";maintain-aspect-ratio TRUE;display "USEDEF";valid_file "T";width
3.0891in;height 2.5927in;depth 0pt;original-width 3.6123in;original-height
3.0268in;cropleft "0";croptop "1";cropright "1";cropbottom "0";tempfilename
'PQXXR10L.wmf';tempfile-properties "XPR";}}Since this happens quickly, the
piston does not have time to change position, so the volume does not change.
Process $DA$ is (almost) isochoric. We already know $P_{A},$ $%
%TCIMACRO{\TeXButton{V}{\ooalign{\hfil$V$\hfil\cr\kern0.1em--\hfil\cr}}}%
%BeginExpansion
\ooalign{\hfil$V$\hfil\cr\kern0.1em--\hfil\cr}%
%EndExpansion
_{A},$ $n,$ and $T_{A}$. But we need to find the process work, heat and
change in internal energy. Since the volume does not change, $W_{DA}=0$ and $%
\Delta E_{DA}=Q_{DA}$ which we can find using 
\begin{eqnarray*}
Q_{DA} &=&nC_{V}\Delta T \\
&=&n\frac{5}{2}R\left( T_{A}-T_{D}\right)
\end{eqnarray*}%
which gives%
\begin{eqnarray*}
Q_{DA} &=&\left( 2.\,052\,5\times 10^{-2}\unit{mol}\right) \frac{5}{2}\left(
8.314\frac{\unit{J}}{\unit{mol}\unit{K}}\right) \left( 293\unit{K}%
-445.\,\allowbreak 28\unit{K}\right) \\
&=&-64.\,\allowbreak 964\unit{J}
\end{eqnarray*}%
which tells us that we have lost energy by heat. So $Q_{DA}$ must contribute
to $Q_{c}.$ 
\[
\begin{tabular}{|c|c|c|c|c|c|c|}
\hline
\textbf{State} & $\mathbf{T}\left( \unit{K}\right) $ & $\mathbf{P}\left( 
\unit{kPa}\right) $ & $%
%TCIMACRO{\TeXButton{V}{\ooalign{\hfil$V$\hfil\cr\kern0.1em--\hfil\cr}}}%
%BeginExpansion
\ooalign{\hfil$V$\hfil\cr\kern0.1em--\hfil\cr}%
%EndExpansion
\left( \unit{m}^{3}\right) $ & $\mathbf{n}\left( \unit{mol}\right) $ & $%
\mathbf{E}_{int}\left( \unit{J}\right) $ & \textbf{--} \\ \hline
$\mathbf{D}$ & $445.\,\allowbreak 28$ & $151.\,\allowbreak 97$ & $0.000\,5$
& $2.\,052\,5\times 10^{-2}$ & $189.\,\allowbreak 96\unit{J}$ & \textbf{--}
\\ \hline
$\mathbf{A}$ & $293$ & $100$ & $0.000\,5$ & $2.\,052\,5\times 10^{-2}$ & $%
\allowbreak 125.\,\allowbreak 00$ & \textbf{--} \\ \hline
\textbf{Process} & $\mathbf{Q}\left( \unit{J}\right) $ & $\mathbf{W}%
_{int}\left( \unit{J}\right) $ & $\mathbf{\Delta E}_{int}\left( \unit{J}%
\right) $ & $\mathbf{W}_{eng}\left( \unit{J}\right) $ & $\mathbf{Q}_{h}$ & $%
\mathbf{Q}_{c}$ \\ \hline
$\mathbf{DA}$ & $-64.\,\allowbreak 964$ & $0$ & $-64.\,\allowbreak 964$ & $0$
& $0$ & $64.\,\allowbreak 964$ \\ \hline
\end{tabular}%
\]

\FRAME{dhF}{3.2569in}{2.8323in}{0pt}{}{}{Figure}{\special{language
"Scientific Word";type "GRAPHIC";maintain-aspect-ratio TRUE;display
"USEDEF";valid_file "T";width 3.2569in;height 2.8323in;depth
0pt;original-width 3.2119in;original-height 2.789in;cropleft "0";croptop
"1";cropright "1";cropbottom "0";tempfilename
'PQXXR10M.wmf';tempfile-properties "XPR";}}

But we are not done. The opening of the valve let the spent fuel leave as
exhaust. But we can help remove the exhaust by pushing the piston up. This
is done at constant pressure, because the valve is open to the exhaust
system air pressure. We will lose gas by venting it. We can find the new
number of moles using the ideal gas law%
\[
P_{I}%
%TCIMACRO{\TeXButton{V}{\ooalign{\hfil$V$\hfil\cr\kern0.1em--\hfil\cr}}}%
%BeginExpansion
\ooalign{\hfil$V$\hfil\cr\kern0.1em--\hfil\cr}%
%EndExpansion
_{I}=nRT 
\]%
since $P_{I}=P_{A}$ and 
\[
\frac{%
%TCIMACRO{\TeXButton{V}{\ooalign{\hfil$V$\hfil\cr\kern0.1em--\hfil\cr}}}%
%BeginExpansion
\ooalign{\hfil$V$\hfil\cr\kern0.1em--\hfil\cr}%
%EndExpansion
_{A}}{%
%TCIMACRO{\TeXButton{V}{\ooalign{\hfil$V$\hfil\cr\kern0.1em--\hfil\cr}}}%
%BeginExpansion
\ooalign{\hfil$V$\hfil\cr\kern0.1em--\hfil\cr}%
%EndExpansion
_{I}}=8.00 
\]%
$%
%TCIMACRO{\TeXButton{V}{\ooalign{\hfil$V$\hfil\cr\kern0.1em--\hfil\cr}}}%
%BeginExpansion
\ooalign{\hfil$V$\hfil\cr\kern0.1em--\hfil\cr}%
%EndExpansion
_{I}=8/%
%TCIMACRO{\TeXButton{V}{\ooalign{\hfil$V$\hfil\cr\kern0.1em--\hfil\cr}}}%
%BeginExpansion
\ooalign{\hfil$V$\hfil\cr\kern0.1em--\hfil\cr}%
%EndExpansion
_{A}$ so we can guess that we should have%
\[
\frac{P_{A}%
%TCIMACRO{\TeXButton{V}{\ooalign{\hfil$V$\hfil\cr\kern0.1em--\hfil\cr}}}%
%BeginExpansion
\ooalign{\hfil$V$\hfil\cr\kern0.1em--\hfil\cr}%
%EndExpansion
_{A}}{P_{I}%
%TCIMACRO{\TeXButton{V}{\ooalign{\hfil$V$\hfil\cr\kern0.1em--\hfil\cr}}}%
%BeginExpansion
\ooalign{\hfil$V$\hfil\cr\kern0.1em--\hfil\cr}%
%EndExpansion
_{I}}=\frac{n_{A}RT_{A}}{n_{I}RT_{I}} 
\]%
\[
8=\frac{n_{A}T_{A}}{n_{I}T_{I}} 
\]%
so%
\[
n_{I}=\frac{n_{A}T_{A}}{8T_{I}} 
\]%
or%
\begin{eqnarray*}
n_{I} &=&\frac{\left( 2.\,052\,5\times 10^{-2}\unit{mol}\right) \left( 293%
\unit{K}\right) }{8\left( 273\unit{K}\right) } \\
&=&2.\,\allowbreak 753\,6\times 10^{-3}\unit{mol}
\end{eqnarray*}%
and then $E_{I}$ will be 
\begin{eqnarray*}
E_{D} &=&\frac{5}{2}n_{I}RT_{I} \\
&=&\frac{5}{2}\left( 2.\,\allowbreak 753\,6\times 10^{-3}\unit{mol}\right)
\left( 8.314\frac{\unit{J}}{\unit{mol}\unit{K}}\right) \left( 273\unit{K}%
\right) \\
&=&\allowbreak 15.\,\allowbreak 625\unit{J}
\end{eqnarray*}

and the work done for an isobaric process is 
\[
W_{AI}=P_{A}\left( 
%TCIMACRO{\TeXButton{V}{\ooalign{\hfil$V$\hfil\cr\kern0.1em--\hfil\cr}}}%
%BeginExpansion
\ooalign{\hfil$V$\hfil\cr\kern0.1em--\hfil\cr}%
%EndExpansion
_{I}-%
%TCIMACRO{\TeXButton{V}{\ooalign{\hfil$V$\hfil\cr\kern0.1em--\hfil\cr}}}%
%BeginExpansion
\ooalign{\hfil$V$\hfil\cr\kern0.1em--\hfil\cr}%
%EndExpansion
_{A}\right) 
\]%
but this assumed that $n$ did not change. Really we are just moving the gas,
so our work is 
\begin{eqnarray*}
W &=&F\Delta x \\
&\approx &0
\end{eqnarray*}%
because it takes very little force to push out the gas, so we can ignore $%
W_{AI}.$

Finding $Q_{AI}$ we would like to use 
\[
Q=nC_{P}\Delta T 
\]%
but we quickly run into trouble again because $n$ changed! So what has gone
wrong?

Well we are using convection to remove heat instead of conduction. We can
get around this by finding $\Delta E_{int}$ directly. 
\begin{eqnarray*}
\Delta E_{AI} &=&E_{I}-E_{A} \\
&=&15.\,\allowbreak 625\unit{J}-125.\,\allowbreak 00\unit{J} \\
&=&-109.\,\allowbreak 38\unit{J}
\end{eqnarray*}%
From this and the first law%
\[
\Delta E_{int}=Q+W 
\]%
we can find $Q$%
\begin{eqnarray*}
Q &\approx &\Delta E_{int} \\
&\approx &-109.\,\allowbreak 38\unit{J}
\end{eqnarray*}

Energy is leaving with the gas, so we seem to pick up a contribution to $%
Q_{c}.$ But really this is not part of our thermodynamic process, because we
mechanically removed the energy by removing the gas. Let's denote this by
putting our energy in parenthesis to remind us that this is a more
mechanical than thermal process.%
\[
\begin{tabular}{|c|c|c|c|c|c|c|}
\hline
\textbf{State} & $\mathbf{T}\left( \unit{K}\right) $ & $\mathbf{P}\left( 
\unit{kPa}\right) $ & $%
%TCIMACRO{\TeXButton{V}{\ooalign{\hfil$V$\hfil\cr\kern0.1em--\hfil\cr}}}%
%BeginExpansion
\ooalign{\hfil$V$\hfil\cr\kern0.1em--\hfil\cr}%
%EndExpansion
\left( \unit{m}^{3}\right) $ & $\mathbf{n}\left( \unit{mol}\right) $ & $%
\mathbf{E}_{int}\left( \unit{J}\right) $ & \textbf{--} \\ \hline
$\mathbf{A}$ & $293$ & $100$ & $0.000\,5$ & $2.\,052\,5\times 10^{-2}$ & $%
\allowbreak 125.\,\allowbreak 00$ & \textbf{--} \\ \hline
$\mathbf{I}$ & $273$ & $100$ & $6.\,25\times 10^{-5}$ & $2.\,\allowbreak
753\,6\times 10^{-3}$ & $\allowbreak 15.\,\allowbreak 625$ & \textbf{--} \\ 
\hline
\textbf{Process} & $\mathbf{Q}\left( \unit{J}\right) $ & $\mathbf{W}%
_{int}\left( \unit{J}\right) $ & $\mathbf{\Delta E}_{int}\left( \unit{J}%
\right) $ & $\mathbf{W}_{eng}\left( \unit{J}\right) $ & $\mathbf{Q}_{h}$ & $%
\mathbf{Q}_{c}$ \\ \hline
$\mathbf{AI}$ & $\left( -109.\,\allowbreak 38\right) $ & $\sim 0$ & $\left(
-109.\,\allowbreak 38\right) $ & $\sim 0$ & $0$ & $\left( 109.\,\allowbreak
38\right) $ \\ \hline
\end{tabular}%
\]%
$\allowbreak $\FRAME{dhF}{3.0606in}{3.2932in}{0pt}{}{}{Figure}{\special%
{language "Scientific Word";type "GRAPHIC";maintain-aspect-ratio
TRUE;display "USEDEF";valid_file "T";width 3.0606in;height 3.2932in;depth
0pt;original-width 3.0173in;original-height 3.2474in;cropleft "0";croptop
"1";cropright "1";cropbottom "0";tempfilename
'PQXXR20N.wmf';tempfile-properties "XPR";}}Now the exhaust valve closes, and
the intake valve opens. The new fuel mixture is brought in. We essentially
reverse process $AI$ in process $IA.$ 
\[
\begin{tabular}{|c|c|c|c|c|c|c|}
\hline
\textbf{State} & $\mathbf{T}\left( \unit{K}\right) $ & $\mathbf{P}\left( 
\unit{kPa}\right) $ & $%
%TCIMACRO{\TeXButton{V}{\ooalign{\hfil$V$\hfil\cr\kern0.1em--\hfil\cr}}}%
%BeginExpansion
\ooalign{\hfil$V$\hfil\cr\kern0.1em--\hfil\cr}%
%EndExpansion
\left( \unit{m}^{3}\right) $ & $\mathbf{n}\left( \unit{mol}\right) $ & $%
\mathbf{E}_{int}\left( \unit{J}\right) $ & \textbf{--} \\ \hline
$\mathbf{I}$ & $293$ & $100$ & $6.\,25\times 10^{-5}$ & $2.\,\allowbreak
565\,7\times 10^{-3}$ & $15.\,\allowbreak 625$ & \textbf{--} \\ \hline
$\mathbf{A}$ & $293$ & $100$ & $0.000\,5$ & $2.\,052\,5\times 10^{-2}$ & $%
\allowbreak 125.\,\allowbreak 00$ & \textbf{--} \\ \hline
\textbf{Process} & $\mathbf{Q}\left( \unit{J}\right) $ & $\mathbf{W}%
_{int}\left( \unit{J}\right) $ & $\mathbf{\Delta E}_{int}\left( \unit{J}%
\right) $ & $\mathbf{W}_{eng}\left( \unit{J}\right) $ & $\mathbf{Q}_{h}$ & $%
\mathbf{Q}_{c}$ \\ \hline
$\mathbf{IA}$ & $\left( 109.\,\allowbreak 38\right) $ & $\sim 0$ & $\left(
109.\,\allowbreak 38\right) $ & $\sim 0$ & $\left( 109.\,\allowbreak
38\right) $ & $0$ \\ \hline
\end{tabular}%
\]%
But there is a difference. The new gas has useful fuel. so we are ready to
complete the cycle again by compressing the gas, igniting it, and receiving
the power stroke and exhaust. Note that Processes $AI$ and $IA$ have a net
zero effect thermodynamically, but are very important to making your car or
lawn mower go!

We can plot what we have found. Note that since $n$ changed for the $AI-IA$
processes, it is not really possible to include it on the PV\ diagram. That
is why it has a dashed line. We know the volume and pressure along the path,
but the predicted temperature from the graph may not be right because $n$
changed too.\FRAME{dtbpFX}{4.5143in}{3.1211in}{0pt}{}{}{Plot}{\special%
{language "Scientific Word";type "MAPLEPLOT";width 4.5143in;height
3.1211in;depth 0pt;display "USEDEF";plot_snapshots TRUE;mustRecompute
FALSE;lastEngine "MuPAD";xmin "0";xmax "0.0005";xviewmin "0";xviewmax
"0.0005";yviewmin "10000";yviewmax "3.0E6";viewset"XY";rangeset"X";plottype
4;labeloverrides 3;x-label "V (m^3)";y-label "P (Pa)";axesFont "Times New
Roman,12,0000000000,useDefault,normal";numpoints 100;plotstyle
"patch";axesstyle "normal";axestips FALSE;xis \TEXUX{V};var1name
\TEXUX{$V$};function \TEXUX{$\allowbreak \frac{49.\,\allowbreak
999}{V}$};linecolor "red";linestyle 2;pointstyle "point";linethickness
1;lineAttributes "Dash";var1range "0,0.0005";num-x-gridlines 100;curveColor
"[flat::RGB:0x00ff0000]";curveStyle "Line";function
\TEXUX{$\frac{174.\,\allowbreak 57}{V}$};linecolor "red";linestyle
2;pointstyle "point";linethickness 1;lineAttributes "Dash";var1range
"0,0.0005";num-x-gridlines 100;curveColor
"[flat::RGB:0x00ff0000]";curveStyle "Line";function \TEXUX{$\left\{
\MATRIX{3,3}{c}\VR{,,l,,,}{,,l,,,}{,,l,,,}{,,,,,}\HR{,,,}\CELL{\frac{2.\,%
\allowbreak 390\,9}{V^{1.4}}}\CELL{\text{if}}\CELL{6.\,25\times
10^{-5}<V<0.000\,5}\CELL{0}\CELL{\text{if}}\CELL{V\geq
0.000\,5}\CELL{0}\CELL{\text{if}}\CELL{V\leq 6.\,25\times 10^{-5}}\right.
$};linecolor "blue";linestyle 1;pointstyle "point";linethickness
3;lineAttributes "Solid";var1range "0,0.0005";num-x-gridlines 100;curveColor
"[flat::RGB:0x000000ff]";curveStyle "Line";function
\TEXUX{$\MATRIX{2,5}{c}\VR{,,c,,,}{,,c,,,}{,,,,,}\HR{,,,,,}\CELL{0.000\,5}%
\CELL{100000}\CELL{6.\,25\times 10^{-5}}\CELL{1837\,900}\CELL{6.\,25\times
10^{-5}}\CELL{2793100}\CELL{0.000\,5}\CELL{151\,\allowbreak
970}\CELL{6.\,25\times 10^{-5}}\CELL{100000}$};linecolor "black";linestyle
2;pointplot TRUE;pointstyle "circle";linethickness 1;lineAttributes
"Dash";var1range "0,0.0005";num-x-gridlines 100;curveColor
"[flat::RGB:0000000000]";curveStyle "Point";function \TEXUX{$\left\{
\MATRIX{3,3}{c}\VR{,,l,,,}{,,l,,,}{,,l,,,}{,,,,,}\HR{,,,}\CELL{\frac{3.\,%
\allowbreak 633\,4}{V^{1.4}}}\CELL{\text{if}}\CELL{6.\,25\times
10^{-5}<V<0.000\,5}\CELL{0}\CELL{\text{if}}\CELL{V\geq
0.000\,5}\CELL{0}\CELL{\text{if}}\CELL{V\leq 6.\,25\times 10^{-5}}\right.
$};linecolor "blue";linestyle 1;pointstyle "point";linethickness
3;lineAttributes "Solid";var1range "0,0.0005";num-x-gridlines 100;curveColor
"[flat::RGB:0x000000ff]";curveStyle "Line";function
\TEXUX{$\MATRIX{2,2}{c}\VR{,,c,,,}{,,c,,,}{,,,,,}\HR{,,}\CELL{6.\,25\times
10^{-5}}\CELL{1837\,900}\CELL{6.\,25\times
10^{-5}}\CELL{2793100}$};linecolor "blue";linestyle 1;pointstyle
"point";linethickness 3;lineAttributes "Solid";var1range
"0,0.0005";num-x-gridlines 100;curveColor
"[flat::RGB:0x000000ff]";curveStyle "Line";function
\TEXUX{$\MATRIX{2,2}{c}\VR{,,c,,,}{,,c,,,}{,,,,,}\HR{,,}\CELL{0.000\,5}%
\CELL{100000}\CELL{0.000\,5}\CELL{151\,\allowbreak 970}$};linecolor
"blue";linestyle 1;pointstyle "point";linethickness 3;lineAttributes
"Solid";var1range "0,0.0005";num-x-gridlines 100;curveColor
"[flat::RGB:0x000000ff]";curveStyle "Line";function
\TEXUX{$\MATRIX{2,2}{c}\VR{,,c,,,}{,,c,,,}{,,,,,}\HR{,,}\CELL{6.\,25\times
10^{-5}}\CELL{100000}\CELL{0.000\,5}\CELL{100000}$};linecolor
"blue";linestyle 2;pointstyle "point";linethickness 3;lineAttributes
"Dash";var1range "0,0.0005";num-x-gridlines 100;curveColor
"[flat::RGB:0x000000ff]";curveStyle "Line";VCamFile
'PQXXR21Q.xvz';valid_file "T";tempfilename
'PQXXR20O.wmf';tempfile-properties "XPR";}}

Let's summarize what we found in large tables. First the ideal gas state
variables and $E_{int}$

\[
\begin{tabular}{|c|c|c|c|c|c|}
\hline
\textbf{State} & $\mathbf{T}\left( \unit{K}\right) $ & $\mathbf{P}\left( 
\unit{kPa}\right) $ & $%
%TCIMACRO{\TeXButton{V}{\ooalign{\hfil$V$\hfil\cr\kern0.1em--\hfil\cr}}}%
%BeginExpansion
\ooalign{\hfil$V$\hfil\cr\kern0.1em--\hfil\cr}%
%EndExpansion
\left( \unit{m}^{3}\right) $ & $\mathbf{n}\left( \unit{mol}\right) $ & $%
\mathbf{E}_{int}\left( \unit{J}\right) $ \\ \hline
$\mathbf{A}$ & $293$ & $100$ & $0.000\,5$ & $2.\,052\,5\times 10^{-2}$ & $%
\allowbreak 125.\,\allowbreak 00$ \\ \hline
$\mathbf{B}$ & $673.\,\allowbreak 15$ & $\allowbreak 1837.\,\allowbreak 9$ & 
$6.\,25\times 10^{-5}$ & $2.\,052\,5\times 10^{-2}$ & $287.\,\allowbreak 17$
\\ \hline
$\mathbf{C}$ & $1023$ & $2793.\,\allowbreak 1$ & $6.\,25\times 10^{-5}$ & $%
2.\,052\,5\times 10^{-2}$ & $436.\,\allowbreak 42$ \\ \hline
$\mathbf{D}$ & $445.\,\allowbreak 28$ & $151.\,\allowbreak 97$ & $0.000\,5$
& $2.\,052\,5\times 10^{-2}$ & $189.\,\allowbreak 96\unit{J}$ \\ \hline
$\mathbf{I}$ & $293$ & $100$ & $6.\,25\times 10^{-5}$ & $2.\,\allowbreak
565\,7\times 10^{-3}$ & $15.\,\allowbreak 625$ \\ \hline
\end{tabular}%
\]

and now the values of $Q,$ $W,$ and $E_{int}$ for the processes. We have
included $W_{eng},$ $Q_{h}$ and $Q_{c}$ as well%
\[
\begin{tabular}{|c|c|c|c|c|c|c|}
\hline
\textbf{Process} & $\mathbf{Q}\left( \unit{J}\right) $ & $\mathbf{W}%
_{int}\left( \unit{J}\right) $ & $\mathbf{\Delta E}_{int}\left( \unit{J}%
\right) $ & $\mathbf{W}_{eng}\left( \unit{J}\right) $ & $\mathbf{Q}_{h}$ & $%
\mathbf{Q}_{c}$ \\ \hline
$\mathbf{AB}$ & $0$ & $\allowbreak 162.\,\allowbreak 17$ & $\allowbreak
162.\,\allowbreak 17$ & $-\allowbreak 162.\,\allowbreak 17$ & $\mathbf{0}$ & 
$\mathbf{0}$ \\ \hline
$\mathbf{BC}$ & $149.\,\allowbreak 25$ & $0$ & $149.\,\allowbreak 25$ & $%
\mathbf{0}$ & $149.\,\allowbreak 25$ & $0$ \\ \hline
$\mathbf{CD}$ & $0$ & $-246.\,\allowbreak 46$ & $-246.\,\allowbreak 46$ & $%
246.\,\allowbreak 46$ & $\mathbf{0}$ & $\mathbf{0}$ \\ \hline
$\mathbf{DA}$ & $-64.\,\allowbreak 964$ & $0$ & $-64.\,\allowbreak 964$ & $0$
& $0$ & $64.\,\allowbreak 964$ \\ \hline
$\mathbf{AI}$ & $\left( -109.\,\allowbreak 38\right) $ & $\left( \sim
0\right) $ & $\left( -109.\,\allowbreak 38\right) $ & $\left( \sim 0\right) $
& $\left( 0\right) $ & $\left( 109.\,\allowbreak 38\right) $ \\ \hline
$\mathbf{IA}$ & $\left( 109.\,\allowbreak 38\right) $ & $\left( \sim
0\right) $ & $\left( 109.\,\allowbreak 38\right) $ & $\left( \sim 0\right) $
& $\left( 109.\,\allowbreak 38\right) $ & $\left( 0\right) $ \\ \hline
Whole Cycle & $84.\,\allowbreak 286$ & $-84.\,\allowbreak 29$ & $0$ & $%
\allowbreak 84.\,\allowbreak 29$ & $149.\,\allowbreak 25$ & $%
64.\,\allowbreak 964$ \\ \hline
\end{tabular}%
\]%
Note that the last row of the table is a total for the entire cycle. We see
that we have useful work of $84.3\unit{J}$ and that $Q_{h}=258.\,\allowbreak
63\unit{J}$ and $Q_{c}=174.\,\allowbreak 34.$ We can identify $Q_{h}$ and $%
Q_{c}$ on our graph. The useful work is the difference between the work done
in the two adiabatic processes. \FRAME{dtbpF}{4.6172in}{3.6927in}{0pt}{}{}{%
Figure}{\special{language "Scientific Word";type
"GRAPHIC";maintain-aspect-ratio TRUE;display "USEDEF";valid_file "T";width
4.6172in;height 3.6927in;depth 0pt;original-width 4.5653in;original-height
3.6452in;cropleft "0";croptop "1";cropright "1";cropbottom "0";tempfilename
'PQXXR20P.wmf';tempfile-properties "XPR";}}We can see that it takes some
work to make process $AB$ happen. That is why we need a starter motor to get
our engine started. But once the engine get's going the momentum from the
piston and crank shaft keep the engine running. But still process $AB$
requires energy input by work, so it is like a resistance for the cycle.

What is the efficiency of this cycle? We don't have a really good way with
our simple view of thermodynamics to accurately account for the change in
chemical potential energy involved with state $I.$ The energy transfer in
processes $AI$ and $IA$ are, in a sense, mechanical energy processes. So for
this class it might be more fair to say that processes $AI$ and $IA$ have no
net effect (except for refueling) and exclude their effect from $Q_{c}$ and $%
Q_{h}$ then ignoring the $I$ processes%
\begin{eqnarray*}
\eta &=&\frac{W_{eng}}{Q_{h}} \\
&=&\frac{84.\,\allowbreak 29}{149.\,\allowbreak 25} \\
&=&\allowbreak 0.564\,76
\end{eqnarray*}

For a gasoline engine, it is often quoted that the efficiency is%
\[
\eta _{otto}=1-\frac{1}{\left( \frac{%
%TCIMACRO{\TeXButton{V}{\ooalign{\hfil$V$\hfil\cr\kern0.1em--\hfil\cr}}}%
%BeginExpansion
\ooalign{\hfil$V$\hfil\cr\kern0.1em--\hfil\cr}%
%EndExpansion
_{A}}{%
%TCIMACRO{\TeXButton{V}{\ooalign{\hfil$V$\hfil\cr\kern0.1em--\hfil\cr}}}%
%BeginExpansion
\ooalign{\hfil$V$\hfil\cr\kern0.1em--\hfil\cr}%
%EndExpansion
_{B}}\right) ^{\gamma -1}} 
\]%
Let's test this. 
\begin{eqnarray*}
\eta _{otto} &=&1-\frac{1}{\left( \frac{0.000\,5}{6.\,25\times 10^{-5}}%
\right) ^{1.4-1}} \\
&=&0.564\,72\allowbreak
\end{eqnarray*}

But you might guess that this is very optimistic. Efficiencies for actual
engines come in at values around $20\%.$

\chapter{The Limit of Efficiency for Heat Engines}

We now know that we can design acutual engines that power cars and useful
machines by combining several of our special thermodynamic procces into
larger, more complex processes that form a cycle. You might guess that there
are more than just isochoric, isobaric, isothermal, and adiabatic processes
that exist. A good question is what processes should we combine to form our
engine? Past researchers have asked this question over and over and many new
engines have been developed. Of course we want our engines to be as
efficient as possible, and the second law of thermodynamics says the engines
can't be 100\%y efficient. But is there a \textquotedblleft most
efficient\textquotedblright\ engine, one that no other engine can beat in
efficiency? If there is, we could use this to judge new engine designs. Any
engine with a better efficiency would be a mistake that won't work, and that
approach this maximum efficiency would be a possible improvement in
technology. 

There is such a maximally efficient engine. We will study it in this lecture.

%TCIMACRO{%
%\TeXButton{Fundamental Concepts}{\hspace{-1.3in}{\Large Fundamental Concepts\vspace{0.25in}}}}%
%BeginExpansion
\hspace{-1.3in}{\Large Fundamental Concepts\vspace{0.25in}}%
%EndExpansion

\begin{itemize}
\item The Carnot cycle represents the maximum theoretical limit of
efficiency for a heat engine.

\item The efficiency of the Carnot cycle depends only on the two extreme
temperatures $\eta _{C}=1-\frac{T_{c}}{T_{h}}.$

\item An engine design that is more efficient that a Carnot engine must have
a fundamental flaw in it's calculations.
\end{itemize}

%TCIMACRO{%
%\TeXButton{Question 123.19.1}{\marginpar {
%\hspace{-0.5in}
%\begin{minipage}[t]{1in}
%\small{Question 123.19.1}
%\end{minipage}
%}}}%
%BeginExpansion
\marginpar {
\hspace{-0.5in}
\begin{minipage}[t]{1in}
\small{Question 123.19.1}
\end{minipage}
}%
%EndExpansion

Carnot was a French Engineer. He thought of an engine that would operate on
an ideal reversible cycle. He hit upon the most efficient cycle possible.
Sadly, it is not possible to create his engine in practice. Carnot's engine
assumed no friction and ideal thermodynamic processes. So why is it useful
to know about Carnot's engine design? The Carnot cycle provides us with a
useful upper limit for all practical engines. To tell if a design is
practical, we can compare it to the Carnot cycle. If our design looks like
it will outperform the Carnot engine, we know we have done something wrong.
Or conversely, if our system design requires an engine to be better than a
Carnot engine, we need to go back to the drawing board.

Knowing just this much we can state Carnot's theorem.

\begin{Note}
No real heat engine operating between two energy reservoirs can be more
efficient than a Carnot engine operating between the same two reservoirs
\end{Note}

\begin{Note}
All real engines are less efficient than a Carnot engine because they do not
operate through a reversible cycle. The efficiency of a real engine is
reduced by friction, energy losses through conduction, etc.
\end{Note}

\section{The Carnot Cycle}

Carnot found a set of special processes that when combined made a
theoretical heat engine. And he proved mathmatically that this particular
combination is theoretically the most efficient possible. Let's take a look
at this engine design.

The Carnot cycle is shown in the figure. It consists of the following parts:

\begin{enumerate}
\item Process $A\rightarrow B$ is an isothermal expansion

\item Process $B\rightarrow C$ is an adiabatic expansion

\item Process $C\rightarrow D$ is an isothermal contraction

\item Process $D\rightarrow A$ is an adiabatic compression
\end{enumerate}

\FRAME{dhF}{3.774in}{2.623in}{0pt}{}{}{Figure}{\special{language "Scientific
Word";type "GRAPHIC";maintain-aspect-ratio TRUE;display "USEDEF";valid_file
"T";width 3.774in;height 2.623in;depth 0pt;original-width
3.7256in;original-height 2.5815in;cropleft "0";croptop "1";cropright
"1";cropbottom "0";tempfilename 'PQXXR20Q.wmf';tempfile-properties "XPR";}}

Let's take an example. Let's choose a Carnot engine that operates between
the same two extreme temperatures as our Otto cycle example. 
\begin{eqnarray*}
T_{c} &=&273\unit{K} \\
T_{h} &=&1023\unit{K}
\end{eqnarray*}%
Let's suppose the carnot engine is made from a piston with a monotonic idea
gas. Let's choose the minimum and maximum volumes to be 
\begin{eqnarray*}
%TCIMACRO{\TeXButton{V}{\ooalign{\hfil$V$\hfil\cr\kern0.1em--\hfil\cr}}}%
%BeginExpansion
\ooalign{\hfil$V$\hfil\cr\kern0.1em--\hfil\cr}%
%EndExpansion
_{A} &=&0.001\unit{m}^{3} \\
%TCIMACRO{\TeXButton{V}{\ooalign{\hfil$V$\hfil\cr\kern0.1em--\hfil\cr}}}%
%BeginExpansion
\ooalign{\hfil$V$\hfil\cr\kern0.1em--\hfil\cr}%
%EndExpansion
_{C} &=&0.015\unit{m}^{3}
\end{eqnarray*}

and let's have 
\[
n=1\unit{mol} 
\]%
of gas. For these numbers, our graph looks like this:\FRAME{dtbpF}{4.0269in}{%
2.7612in}{0pt}{}{}{Figure}{\special{language "Scientific Word";type
"GRAPHIC";maintain-aspect-ratio TRUE;display "USEDEF";valid_file "T";width
4.0269in;height 2.7612in;depth 0pt;original-width 4.0748in;original-height
2.7851in;cropleft "0";croptop "1";cropright "1";cropbottom "0";tempfilename
'PQXXR20R.wmf';tempfile-properties "XPR";}}Our job is to find $P,$ $%
%TCIMACRO{\TeXButton{V}{\ooalign{\hfil$V$\hfil\cr\kern0.1em--\hfil\cr}}}%
%BeginExpansion
\ooalign{\hfil$V$\hfil\cr\kern0.1em--\hfil\cr}%
%EndExpansion
,$ $T,$ $W$, $Q,$ and $\Delta E_{int}$ for each part of the cycle and $Q_{h}$%
, $Q_{c},$ and $W_{eng}$ for the whole cycle.

Let's take each process separately like we did for the Otto cycle.

\section{Process $A\rightarrow B$ isothermal expansion}

The gas is placed in contact with the high temperature reservoir, $T_{h}.$
The gas absorbs heat $|Q_{h}|.$ The gas does work $W_{AB}$ in raising the
piston\FRAME{dtbpF}{4.9644in}{3.3288in}{0in}{}{}{Figure}{\special{language
"Scientific Word";type "GRAPHIC";maintain-aspect-ratio TRUE;display
"USEDEF";valid_file "T";width 4.9644in;height 3.3288in;depth
0in;original-width 5.0301in;original-height 3.3634in;cropleft "0";croptop
"1";cropright "1";cropbottom "0";tempfilename
'PQXXR20S.wmf';tempfile-properties "XPR";}}

Remember for isothermal processes we can write%
\[
\frac{P_{i}%
%TCIMACRO{\TeXButton{V}{\ooalign{\hfil$V$\hfil\cr\kern0.1em--\hfil\cr}}}%
%BeginExpansion
\ooalign{\hfil$V$\hfil\cr\kern0.1em--\hfil\cr}%
%EndExpansion
_{i}}{P_{f}%
%TCIMACRO{\TeXButton{V}{\ooalign{\hfil$V$\hfil\cr\kern0.1em--\hfil\cr}}}%
%BeginExpansion
\ooalign{\hfil$V$\hfil\cr\kern0.1em--\hfil\cr}%
%EndExpansion
_{f}}=\frac{nRT_{i}}{nRT_{i}} 
\]%
because the temperature does not change, so%
\[
P_{A}%
%TCIMACRO{\TeXButton{V}{\ooalign{\hfil$V$\hfil\cr\kern0.1em--\hfil\cr}}}%
%BeginExpansion
\ooalign{\hfil$V$\hfil\cr\kern0.1em--\hfil\cr}%
%EndExpansion
_{A}=P_{B}%
%TCIMACRO{\TeXButton{V}{\ooalign{\hfil$V$\hfil\cr\kern0.1em--\hfil\cr}}}%
%BeginExpansion
\ooalign{\hfil$V$\hfil\cr\kern0.1em--\hfil\cr}%
%EndExpansion
_{B} 
\]%
and we know for an isothermal process that 
\[
W_{AB}=nRT_{A}\ln \left( \frac{%
%TCIMACRO{\TeXButton{V}{\ooalign{\hfil$V$\hfil\cr\kern0.1em--\hfil\cr}}}%
%BeginExpansion
\ooalign{\hfil$V$\hfil\cr\kern0.1em--\hfil\cr}%
%EndExpansion
_{A}}{%
%TCIMACRO{\TeXButton{V}{\ooalign{\hfil$V$\hfil\cr\kern0.1em--\hfil\cr}}}%
%BeginExpansion
\ooalign{\hfil$V$\hfil\cr\kern0.1em--\hfil\cr}%
%EndExpansion
_{B}}\right) 
\]

and that%
\[
\Delta E_{int}=0 
\]%
so%
\[
\left\vert Q_{AB}\right\vert =\left\vert -W_{AB}\right\vert 
\]%
and we can identify this as $\left\vert Q_{h}\right\vert $ and $T_{A}=T_{h}$

We can complete state $A$ using the ideal gas law%
\begin{eqnarray*}
P_{A} &=&\frac{nRT_{h}}{%
%TCIMACRO{\TeXButton{V}{\ooalign{\hfil$V$\hfil\cr\kern0.1em--\hfil\cr}}}%
%BeginExpansion
\ooalign{\hfil$V$\hfil\cr\kern0.1em--\hfil\cr}%
%EndExpansion
_{A}}=\frac{\left( 1\unit{mol}\right) \left( 8.314\frac{\unit{J}}{\unit{mol}%
\unit{K}}\right) \left( 1023\unit{K}\right) }{0.001\unit{m}^{3}} \\
&=&8.\,\allowbreak 505\,2\times 10^{6}\unit{Pa}
\end{eqnarray*}

To complete state $B$ we need the volume and the pressure. At point $B$ both
the adiabat and the isotherm have the same pressure and volume. So we know 
\[
P_{B}=\frac{nRT_{h}}{%
%TCIMACRO{\TeXButton{V}{\ooalign{\hfil$V$\hfil\cr\kern0.1em--\hfil\cr}}}%
%BeginExpansion
\ooalign{\hfil$V$\hfil\cr\kern0.1em--\hfil\cr}%
%EndExpansion
_{B}} 
\]%
and 
\[
P_{B}=\frac{K_{BC}}{%
%TCIMACRO{\TeXButton{V}{\ooalign{\hfil$V$\hfil\cr\kern0.1em--\hfil\cr}}}%
%BeginExpansion
\ooalign{\hfil$V$\hfil\cr\kern0.1em--\hfil\cr}%
%EndExpansion
_{B}^{\gamma }} 
\]%
where the constant $K_{BC}=P_{B}%
%TCIMACRO{\TeXButton{V}{\ooalign{\hfil$V$\hfil\cr\kern0.1em--\hfil\cr}}}%
%BeginExpansion
\ooalign{\hfil$V$\hfil\cr\kern0.1em--\hfil\cr}%
%EndExpansion
_{B}^{\gamma }$ from our adiabatic equations. But we don't know $K_{BC}$, $%
P_{B},$ or $%
%TCIMACRO{\TeXButton{V}{\ooalign{\hfil$V$\hfil\cr\kern0.1em--\hfil\cr}}}%
%BeginExpansion
\ooalign{\hfil$V$\hfil\cr\kern0.1em--\hfil\cr}%
%EndExpansion
_{B}$

We do know that at $C$ 
\begin{eqnarray*}
P_{C} &=&\frac{nRT_{c}}{%
%TCIMACRO{\TeXButton{V}{\ooalign{\hfil$V$\hfil\cr\kern0.1em--\hfil\cr}}}%
%BeginExpansion
\ooalign{\hfil$V$\hfil\cr\kern0.1em--\hfil\cr}%
%EndExpansion
_{C}} \\
P_{C} &=&\frac{K_{BC}}{%
%TCIMACRO{\TeXButton{V}{\ooalign{\hfil$V$\hfil\cr\kern0.1em--\hfil\cr}}}%
%BeginExpansion
\ooalign{\hfil$V$\hfil\cr\kern0.1em--\hfil\cr}%
%EndExpansion
_{c}^{\gamma }}
\end{eqnarray*}%
and at point $C$ and we do know $%
%TCIMACRO{\TeXButton{V}{\ooalign{\hfil$V$\hfil\cr\kern0.1em--\hfil\cr}}}%
%BeginExpansion
\ooalign{\hfil$V$\hfil\cr\kern0.1em--\hfil\cr}%
%EndExpansion
_{C,}$ so we can find $K_{BC}$%
\[
\frac{nRT_{c}}{%
%TCIMACRO{\TeXButton{V}{\ooalign{\hfil$V$\hfil\cr\kern0.1em--\hfil\cr}}}%
%BeginExpansion
\ooalign{\hfil$V$\hfil\cr\kern0.1em--\hfil\cr}%
%EndExpansion
_{C}}=\frac{K_{BC}}{%
%TCIMACRO{\TeXButton{V}{\ooalign{\hfil$V$\hfil\cr\kern0.1em--\hfil\cr}}}%
%BeginExpansion
\ooalign{\hfil$V$\hfil\cr\kern0.1em--\hfil\cr}%
%EndExpansion
_{C}^{\gamma }} 
\]%
\begin{eqnarray*}
K_{BC} &=&\frac{nRT_{c}}{%
%TCIMACRO{\TeXButton{V}{\ooalign{\hfil$V$\hfil\cr\kern0.1em--\hfil\cr}}}%
%BeginExpansion
\ooalign{\hfil$V$\hfil\cr\kern0.1em--\hfil\cr}%
%EndExpansion
_{C}}%
%TCIMACRO{\TeXButton{V}{\ooalign{\hfil$V$\hfil\cr\kern0.1em--\hfil\cr}}}%
%BeginExpansion
\ooalign{\hfil$V$\hfil\cr\kern0.1em--\hfil\cr}%
%EndExpansion
_{C}^{\gamma } \\
&=&nRT_{c}%
%TCIMACRO{\TeXButton{V}{\ooalign{\hfil$V$\hfil\cr\kern0.1em--\hfil\cr}}}%
%BeginExpansion
\ooalign{\hfil$V$\hfil\cr\kern0.1em--\hfil\cr}%
%EndExpansion
_{C}^{\gamma -1}
\end{eqnarray*}%
so%
\begin{eqnarray*}
K_{BC} &=&nRT_{c}%
%TCIMACRO{\TeXButton{V}{\ooalign{\hfil$V$\hfil\cr\kern0.1em--\hfil\cr}}}%
%BeginExpansion
\ooalign{\hfil$V$\hfil\cr\kern0.1em--\hfil\cr}%
%EndExpansion
_{C}^{\gamma -1}=\left( 1\unit{mol}\right) \left( 8.314\frac{\unit{J}}{\unit{%
mol}\unit{K}}\right) \left( 273\unit{K}\right) \left( 0.015\unit{m}%
^{3}\right) ^{\frac{2}{3}} \\
&=&\allowbreak 138.\,\allowbreak 05\unit{J}\unit{m}^{2}
\end{eqnarray*}%
then moving back to point $B$ we can set the two equations for pressure
equal 
\begin{eqnarray*}
P_{B} &=&\frac{nRT_{h}}{%
%TCIMACRO{\TeXButton{V}{\ooalign{\hfil$V$\hfil\cr\kern0.1em--\hfil\cr}}}%
%BeginExpansion
\ooalign{\hfil$V$\hfil\cr\kern0.1em--\hfil\cr}%
%EndExpansion
_{B}} \\
P_{B} &=&\frac{K_{BC}}{%
%TCIMACRO{\TeXButton{V}{\ooalign{\hfil$V$\hfil\cr\kern0.1em--\hfil\cr}}}%
%BeginExpansion
\ooalign{\hfil$V$\hfil\cr\kern0.1em--\hfil\cr}%
%EndExpansion
_{B}^{\gamma }}
\end{eqnarray*}%
to get 
\[
\frac{nRT_{h}}{%
%TCIMACRO{\TeXButton{V}{\ooalign{\hfil$V$\hfil\cr\kern0.1em--\hfil\cr}}}%
%BeginExpansion
\ooalign{\hfil$V$\hfil\cr\kern0.1em--\hfil\cr}%
%EndExpansion
_{B}}=\frac{K_{BC}}{%
%TCIMACRO{\TeXButton{V}{\ooalign{\hfil$V$\hfil\cr\kern0.1em--\hfil\cr}}}%
%BeginExpansion
\ooalign{\hfil$V$\hfil\cr\kern0.1em--\hfil\cr}%
%EndExpansion
_{B}^{\gamma }} 
\]%
then 
\[
\frac{%
%TCIMACRO{\TeXButton{V}{\ooalign{\hfil$V$\hfil\cr\kern0.1em--\hfil\cr}}}%
%BeginExpansion
\ooalign{\hfil$V$\hfil\cr\kern0.1em--\hfil\cr}%
%EndExpansion
_{B}^{\gamma }}{%
%TCIMACRO{\TeXButton{V}{\ooalign{\hfil$V$\hfil\cr\kern0.1em--\hfil\cr}}}%
%BeginExpansion
\ooalign{\hfil$V$\hfil\cr\kern0.1em--\hfil\cr}%
%EndExpansion
_{B}}=\frac{K_{BC}}{nRT_{h}} 
\]%
\[
%TCIMACRO{\TeXButton{V}{\ooalign{\hfil$V$\hfil\cr\kern0.1em--\hfil\cr}}}%
%BeginExpansion
\ooalign{\hfil$V$\hfil\cr\kern0.1em--\hfil\cr}%
%EndExpansion
_{B}^{\gamma -1}=\frac{K_{BC}}{nRT_{h}} 
\]%
which gives us an awkward 
\[
%TCIMACRO{\TeXButton{V}{\ooalign{\hfil$V$\hfil\cr\kern0.1em--\hfil\cr}}}%
%BeginExpansion
\ooalign{\hfil$V$\hfil\cr\kern0.1em--\hfil\cr}%
%EndExpansion
_{B}=\sqrt[\gamma -1]{\frac{K_{BC}}{nRT_{h}}} 
\]%
but it works.%
\begin{eqnarray*}
%TCIMACRO{\TeXButton{V}{\ooalign{\hfil$V$\hfil\cr\kern0.1em--\hfil\cr}}}%
%BeginExpansion
\ooalign{\hfil$V$\hfil\cr\kern0.1em--\hfil\cr}%
%EndExpansion
_{B} &=&\sqrt[\frac{2}{3}]{\frac{138.\,\allowbreak 05\unit{J}\unit{m}^{2}}{%
\left( 1\unit{mol}\right) \left( 8.314\frac{\unit{J}}{\unit{mol}\unit{K}}%
\right) \left( 1023\unit{K}\right) }} \\
&=&\allowbreak 2.\,\allowbreak 067\,9\times 10^{-3}\unit{m}^{3}
\end{eqnarray*}%
and we are back to the ideal gas law to find the pressure%
\begin{eqnarray*}
P_{B} &=&\frac{nRT_{B}}{%
%TCIMACRO{\TeXButton{V}{\ooalign{\hfil$V$\hfil\cr\kern0.1em--\hfil\cr}}}%
%BeginExpansion
\ooalign{\hfil$V$\hfil\cr\kern0.1em--\hfil\cr}%
%EndExpansion
_{B}} \\
&=&\frac{\left( 1\unit{mol}\right) \left( 8.314\frac{\unit{J}}{\unit{mol}%
\unit{K}}\right) \left( 1023\unit{K}\right) }{\allowbreak 2.\,\allowbreak
067\,9\times 10^{-3}\unit{m}^{3}} \\
&=&\allowbreak 4.\,\allowbreak 113\,0\times 10^{6}\unit{Pa}
\end{eqnarray*}%
Armed with all of this, we can compute the work and heat.%
\[
W_{AB}=nRT_{h}\ln \left( \frac{%
%TCIMACRO{\TeXButton{V}{\ooalign{\hfil$V$\hfil\cr\kern0.1em--\hfil\cr}}}%
%BeginExpansion
\ooalign{\hfil$V$\hfil\cr\kern0.1em--\hfil\cr}%
%EndExpansion
_{A}}{%
%TCIMACRO{\TeXButton{V}{\ooalign{\hfil$V$\hfil\cr\kern0.1em--\hfil\cr}}}%
%BeginExpansion
\ooalign{\hfil$V$\hfil\cr\kern0.1em--\hfil\cr}%
%EndExpansion
_{B}}\right) 
\]%
\begin{eqnarray*}
W_{AB} &=&\left( 1\unit{mol}\right) \left( 8.314\frac{\unit{J}}{\unit{mol}%
\unit{K}}\right) \left( \allowbreak 1023\unit{K}\right) \ln \left( \frac{%
0.001\unit{m}^{3}}{\allowbreak 2.\,\allowbreak 067\,9\times 10^{-3}\unit{m}%
^{3}}\right) \\
&=&\allowbreak -6179.\,\allowbreak 3\unit{J}
\end{eqnarray*}

We can find

\[
\begin{tabular}{|c|c|c|c|c|c|c|}
\hline
\textbf{State} & $\mathbf{T}\left( \unit{K}\right) $ & $\mathbf{P}\left( 
\unit{Pa}\right) $ & $%
%TCIMACRO{\TeXButton{V}{\ooalign{\hfil$V$\hfil\cr\kern0.1em--\hfil\cr}}}%
%BeginExpansion
\ooalign{\hfil$V$\hfil\cr\kern0.1em--\hfil\cr}%
%EndExpansion
\left( \unit{m}^{3}\right) $ & $\mathbf{n}\left( \unit{mol}\right) $ & 
\textbf{--} & \textbf{--} \\ \hline
$\mathbf{A}$ & $1023$ & $8.\,\allowbreak 505\,2\times 10^{6}$ & $0.001$ & $1$
& \textbf{--} & \textbf{--} \\ \hline
$\mathbf{B}$ & $1023$ & $\allowbreak 4.\,\allowbreak 113\,0\times 10^{6}$ & $%
2.\,\allowbreak 067\,9\times 10^{-3}$ & $1$ & \textbf{--} & \textbf{--} \\ 
\hline
\textbf{Process} & $\mathbf{Q}\left( \unit{J}\right) $ & $\mathbf{W}%
_{int}\left( \unit{J}\right) $ & $\mathbf{\Delta E}_{int}\left( \unit{J}%
\right) $ & $\mathbf{W}_{eng}\left( \unit{J}\right) $ & $\mathbf{Q}_{h}$ & $%
\mathbf{Q}_{c}$ \\ \hline
$\mathbf{AB}$ & $6179.\,\allowbreak 3$ & $-6179.\,\allowbreak 3$ & $0$ & $%
6179.\,\allowbreak 3$ & $6179.\,\allowbreak 3$ & $0$ \\ \hline
\end{tabular}%
\]

\section{Process $B\rightarrow C$ adiabatic expansion}

The base of the cylinder is replaced by a thermally nonconducting wall. No
energy enters or leaves the system by heat. 
\[
Q=0 
\]

The temperature falls from $T_{h}$ to $T_{c}.$ The gas does work $W_{BC}.$%
\FRAME{dtbpF}{5.3086in}{3.822in}{0in}{}{}{Figure}{\special{language
"Scientific Word";type "GRAPHIC";maintain-aspect-ratio TRUE;display
"USEDEF";valid_file "T";width 5.3086in;height 3.822in;depth
0in;original-width 5.3813in;original-height 3.8663in;cropleft "0";croptop
"1";cropright "1";cropbottom "0";tempfilename
'PQXXR20T.wmf';tempfile-properties "XPR";}}

Remember that for an adiabatic process 
\[
P_{A}%
%TCIMACRO{\TeXButton{V}{\ooalign{\hfil$V$\hfil\cr\kern0.1em--\hfil\cr}}}%
%BeginExpansion
\ooalign{\hfil$V$\hfil\cr\kern0.1em--\hfil\cr}%
%EndExpansion
_{A}^{\gamma }=P_{B}%
%TCIMACRO{\TeXButton{V}{\ooalign{\hfil$V$\hfil\cr\kern0.1em--\hfil\cr}}}%
%BeginExpansion
\ooalign{\hfil$V$\hfil\cr\kern0.1em--\hfil\cr}%
%EndExpansion
_{B}^{\gamma } 
\]%
and we usually find a more convenient path for calculating $\Delta E_{int}.$

\[
\Delta E_{int}=nC_{V}\Delta T 
\]%
We can find the pressure at $C$ using the ideal gas law%
\begin{eqnarray*}
P_{C} &=&\frac{nRT_{c}}{%
%TCIMACRO{\TeXButton{V}{\ooalign{\hfil$V$\hfil\cr\kern0.1em--\hfil\cr}}}%
%BeginExpansion
\ooalign{\hfil$V$\hfil\cr\kern0.1em--\hfil\cr}%
%EndExpansion
_{C}}=\frac{\left( 1\unit{mol}\right) \left( 8.314\frac{\unit{J}}{\unit{mol}%
\unit{K}}\right) \left( \allowbreak 273\unit{K}\right) }{0.015\unit{m}^{3}}
\\
&=&1.\,\allowbreak 513\,1\times 10^{5}\unit{Pa}
\end{eqnarray*}%
and%
\begin{eqnarray*}
\Delta E_{int} &=&\left( 1\unit{mol}\right) \frac{3}{2}\left( 8.314\frac{%
\unit{J}}{\unit{mol}\unit{K}}\right) \left( 273\unit{K}-1023\unit{K}\right)
\\
&=&-9353.\,\allowbreak 3\unit{J}
\end{eqnarray*}%
we can summarize all this.\bigskip 
\[
\begin{tabular}{|c|c|c|c|c|c|c|}
\hline
\textbf{State} & $\mathbf{T}\left( \unit{K}\right) $ & $\mathbf{P}\left( 
\unit{Pa}\right) $ & $%
%TCIMACRO{\TeXButton{V}{\ooalign{\hfil$V$\hfil\cr\kern0.1em--\hfil\cr}}}%
%BeginExpansion
\ooalign{\hfil$V$\hfil\cr\kern0.1em--\hfil\cr}%
%EndExpansion
\left( \unit{m}^{3}\right) $ & $\mathbf{n}\left( \unit{mol}\right) $ & 
\textbf{--} & \textbf{--} \\ \hline
$\mathbf{B}$ & $1023$ & $\allowbreak 4.\,\allowbreak 113\,0\times 10^{6}$ & $%
2.\,\allowbreak 067\,9\times 10^{-3}$ & $1$ & \textbf{--} & \textbf{--} \\ 
\hline
$\mathbf{C}$ & $273$ & $1.\,\allowbreak 513\,1\times 10^{5}$ & $0.015\unit{m}%
^{3}$ & $1$ & \textbf{--} & \textbf{--} \\ \hline
\textbf{Process} & $\mathbf{Q}\left( \unit{J}\right) $ & $\mathbf{W}%
_{int}\left( \unit{J}\right) $ & $\mathbf{\Delta E}_{int}\left( \unit{J}%
\right) $ & $\mathbf{W}_{eng}\left( \unit{J}\right) $ & $\mathbf{Q}_{h}$ & $%
\mathbf{Q}_{c}$ \\ \hline
$\mathbf{BC}$ & $0$ & $-9353.\,\allowbreak 3$ & $-9353.\,\allowbreak 3$ & $%
9353.\,\allowbreak 3$ & $0$ & $0$ \\ \hline
\end{tabular}%
\]

\section{Process $C\rightarrow D$ isothermal compression}

The gas is placed in contact with the cold temperature reservoir. The gas
expels energy $\left\vert Q_{c}\right\vert $ and work $W_{CD}$ is done on
the gas.\FRAME{dtbpF}{5.6518in}{3.9905in}{0pt}{}{}{Figure}{\special{language
"Scientific Word";type "GRAPHIC";maintain-aspect-ratio TRUE;display
"USEDEF";valid_file "T";width 5.6518in;height 3.9905in;depth
0pt;original-width 5.7317in;original-height 4.0375in;cropleft "0";croptop
"1";cropright "1";cropbottom "0";tempfilename
'PQXXR20U.wmf';tempfile-properties "XPR";}}The process is isothermal, so $%
T_{C}=T_{D}$ and $nRT_{C}=nRT_{D}$ so we have 
\begin{equation}
P_{C}%
%TCIMACRO{\TeXButton{V}{\ooalign{\hfil$V$\hfil\cr\kern0.1em--\hfil\cr}}}%
%BeginExpansion
\ooalign{\hfil$V$\hfil\cr\kern0.1em--\hfil\cr}%
%EndExpansion
_{C}=P_{D}%
%TCIMACRO{\TeXButton{V}{\ooalign{\hfil$V$\hfil\cr\kern0.1em--\hfil\cr}}}%
%BeginExpansion
\ooalign{\hfil$V$\hfil\cr\kern0.1em--\hfil\cr}%
%EndExpansion
_{D}
\end{equation}%
and 
\begin{equation}
W_{CD}=nRT_{c}\ln \left( \frac{%
%TCIMACRO{\TeXButton{V}{\ooalign{\hfil$V$\hfil\cr\kern0.1em--\hfil\cr}}}%
%BeginExpansion
\ooalign{\hfil$V$\hfil\cr\kern0.1em--\hfil\cr}%
%EndExpansion
_{C}}{%
%TCIMACRO{\TeXButton{V}{\ooalign{\hfil$V$\hfil\cr\kern0.1em--\hfil\cr}}}%
%BeginExpansion
\ooalign{\hfil$V$\hfil\cr\kern0.1em--\hfil\cr}%
%EndExpansion
_{D}}\right)
\end{equation}%
again%
\begin{equation}
\Delta E_{int}=0
\end{equation}%
so%
\begin{equation}
\left\vert Q_{CD}\right\vert =\left\vert -W_{CD}\right\vert
\end{equation}%
and we can identify this as $\left\vert Q_{c}\right\vert $

We again need to find $%
%TCIMACRO{\TeXButton{V}{\ooalign{\hfil$V$\hfil\cr\kern0.1em--\hfil\cr}}}%
%BeginExpansion
\ooalign{\hfil$V$\hfil\cr\kern0.1em--\hfil\cr}%
%EndExpansion
_{D}$ and we can use the same method as before, finding $%
%TCIMACRO{\TeXButton{V}{\ooalign{\hfil$V$\hfil\cr\kern0.1em--\hfil\cr}}}%
%BeginExpansion
\ooalign{\hfil$V$\hfil\cr\kern0.1em--\hfil\cr}%
%EndExpansion
_{B}.$ At position $D$ 
\[
P_{D}=\frac{nRT_{c}}{%
%TCIMACRO{\TeXButton{V}{\ooalign{\hfil$V$\hfil\cr\kern0.1em--\hfil\cr}}}%
%BeginExpansion
\ooalign{\hfil$V$\hfil\cr\kern0.1em--\hfil\cr}%
%EndExpansion
_{D}} 
\]%
and 
\[
P_{D}=\frac{K_{DA}}{%
%TCIMACRO{\TeXButton{V}{\ooalign{\hfil$V$\hfil\cr\kern0.1em--\hfil\cr}}}%
%BeginExpansion
\ooalign{\hfil$V$\hfil\cr\kern0.1em--\hfil\cr}%
%EndExpansion
_{D}^{\gamma }} 
\]%
but we don't know $K_{DA}$, $P_{D},$ or $%
%TCIMACRO{\TeXButton{V}{\ooalign{\hfil$V$\hfil\cr\kern0.1em--\hfil\cr}}}%
%BeginExpansion
\ooalign{\hfil$V$\hfil\cr\kern0.1em--\hfil\cr}%
%EndExpansion
_{D}$

But we also know that at $A$ 
\begin{eqnarray*}
P_{A} &=&\frac{nRT_{h}}{%
%TCIMACRO{\TeXButton{V}{\ooalign{\hfil$V$\hfil\cr\kern0.1em--\hfil\cr}}}%
%BeginExpansion
\ooalign{\hfil$V$\hfil\cr\kern0.1em--\hfil\cr}%
%EndExpansion
_{A}} \\
P_{A} &=&\frac{K_{DA}}{%
%TCIMACRO{\TeXButton{V}{\ooalign{\hfil$V$\hfil\cr\kern0.1em--\hfil\cr}}}%
%BeginExpansion
\ooalign{\hfil$V$\hfil\cr\kern0.1em--\hfil\cr}%
%EndExpansion
_{A}^{\gamma }}
\end{eqnarray*}%
and at point $A$ and we do know $%
%TCIMACRO{\TeXButton{V}{\ooalign{\hfil$V$\hfil\cr\kern0.1em--\hfil\cr}}}%
%BeginExpansion
\ooalign{\hfil$V$\hfil\cr\kern0.1em--\hfil\cr}%
%EndExpansion
_{A,}$ so we can find $K_{DA}$%
\[
\frac{nRT_{h}}{%
%TCIMACRO{\TeXButton{V}{\ooalign{\hfil$V$\hfil\cr\kern0.1em--\hfil\cr}}}%
%BeginExpansion
\ooalign{\hfil$V$\hfil\cr\kern0.1em--\hfil\cr}%
%EndExpansion
_{A}}=\frac{K_{DA}}{%
%TCIMACRO{\TeXButton{V}{\ooalign{\hfil$V$\hfil\cr\kern0.1em--\hfil\cr}}}%
%BeginExpansion
\ooalign{\hfil$V$\hfil\cr\kern0.1em--\hfil\cr}%
%EndExpansion
_{A}^{\gamma }} 
\]%
so%
\[
K_{DA}=nRT_{h}%
%TCIMACRO{\TeXButton{V}{\ooalign{\hfil$V$\hfil\cr\kern0.1em--\hfil\cr}}}%
%BeginExpansion
\ooalign{\hfil$V$\hfil\cr\kern0.1em--\hfil\cr}%
%EndExpansion
_{A}^{\gamma -1} 
\]%
or%
\begin{eqnarray*}
K_{DA} &=&\left( 1\unit{mol}\right) \left( 8.314\frac{\unit{J}}{\unit{mol}%
\unit{K}}\right) \left( 1023\unit{K}\right) \left( 0.001\unit{m}^{3}\right)
^{\frac{2}{3}} \\
&=&85.\,\allowbreak 052\unit{J}\unit{m}^{2}
\end{eqnarray*}%
then moving back to point $D$ we have 
\[
\frac{nRT_{c}}{%
%TCIMACRO{\TeXButton{V}{\ooalign{\hfil$V$\hfil\cr\kern0.1em--\hfil\cr}}}%
%BeginExpansion
\ooalign{\hfil$V$\hfil\cr\kern0.1em--\hfil\cr}%
%EndExpansion
_{D}}=\frac{K_{DA}}{%
%TCIMACRO{\TeXButton{V}{\ooalign{\hfil$V$\hfil\cr\kern0.1em--\hfil\cr}}}%
%BeginExpansion
\ooalign{\hfil$V$\hfil\cr\kern0.1em--\hfil\cr}%
%EndExpansion
_{D}^{\gamma }} 
\]%
and 
\[
%TCIMACRO{\TeXButton{V}{\ooalign{\hfil$V$\hfil\cr\kern0.1em--\hfil\cr}}}%
%BeginExpansion
\ooalign{\hfil$V$\hfil\cr\kern0.1em--\hfil\cr}%
%EndExpansion
_{D}=\sqrt[\gamma -1]{\frac{K_{DA}}{nRT_{c}}} 
\]%
numerically we get%
\begin{eqnarray*}
%TCIMACRO{\TeXButton{V}{\ooalign{\hfil$V$\hfil\cr\kern0.1em--\hfil\cr}}}%
%BeginExpansion
\ooalign{\hfil$V$\hfil\cr\kern0.1em--\hfil\cr}%
%EndExpansion
_{D} &=&\sqrt[\frac{2}{3}]{\frac{85.\,\allowbreak 052\unit{J}\unit{m}^{2}}{%
\left( 1\unit{mol}\right) \left( 8.314\frac{\unit{J}}{\unit{mol}\unit{K}}%
\right) \left( 273\unit{K}\right) }} \\
&=&\allowbreak 7.\,\allowbreak 253\,7\times 10^{-3}\unit{m}^{3}
\end{eqnarray*}%
And we are back to the ideal gas law to find the pressure%
\begin{eqnarray*}
P_{D} &=&\frac{\left( 1\unit{mol}\right) \left( 8.314\frac{\unit{J}}{\unit{%
mol}\unit{K}}\right) \left( \allowbreak 273\unit{K}\right) }{\allowbreak
7.\,\allowbreak 253\,7\times 10^{-3}\unit{m}^{3}} \\
&=&3.\,\allowbreak 129\,1\times 10^{5}\unit{Pa}
\end{eqnarray*}%
And so we can find the work as%
\begin{eqnarray*}
W_{CD} &=&\left( 1\unit{mol}\right) \left( 8.314\frac{\unit{J}}{\unit{mol}%
\unit{K}}\right) \left( \allowbreak 273\unit{K}\right) \ln \left( \frac{0.015%
\unit{m}^{3}}{7.\,\allowbreak 253\,7\times 10^{-3}\unit{m}^{3}}\right) \\
&=&\allowbreak 1649.0\unit{J}
\end{eqnarray*}%
and our summary is as follows

\[
\begin{tabular}{|c|c|c|c|c|c|c|}
\hline
\textbf{State} & $\mathbf{T}\left( \unit{K}\right) $ & $\mathbf{P}\left( 
\unit{Pa}\right) $ & $%
%TCIMACRO{\TeXButton{V}{\ooalign{\hfil$V$\hfil\cr\kern0.1em--\hfil\cr}}}%
%BeginExpansion
\ooalign{\hfil$V$\hfil\cr\kern0.1em--\hfil\cr}%
%EndExpansion
\left( \unit{m}^{3}\right) $ & $\mathbf{n}\left( \unit{mol}\right) $ & 
\textbf{--} & \textbf{--} \\ \hline
$\mathbf{C}$ & $273$ & $1.\,\allowbreak 513\,1\times 10^{5}$ & $0.015\unit{m}%
^{3}$ & $1$ & \textbf{--} & \textbf{--} \\ \hline
$\mathbf{D}$ & $273$ & $3.\,\allowbreak 129\,1\times 10^{5}$ & $%
7.\,\allowbreak 253\,7\times 10^{-3}$ & $1$ & \textbf{--} & \textbf{--} \\ 
\hline
\textbf{Process} & $\mathbf{Q}\left( \unit{J}\right) $ & $\mathbf{W}%
_{int}\left( \unit{J}\right) $ & $\mathbf{\Delta E}_{int}\left( \unit{J}%
\right) $ & $\mathbf{W}_{eng}\left( \unit{J}\right) $ & $\mathbf{Q}_{h}$ & $%
\mathbf{Q}_{c}$ \\ \hline
$\mathbf{CD}$ & $-1649.0$ & $1649.0$ & $0$ & $0$ & $0$ & $1649.0$ \\ \hline
\end{tabular}%
\]

\section{Process $D\rightarrow A$ adiabatic compression}

The gas is again placed against a thermally nonconducting wall so no heat is
exchanged with the surroundings. The temperature of the gas increases from $%
T_{c}$ to $T_{h}.$The work done on the gas is $W_{DA}$\FRAME{dtbpF}{5.2731in%
}{4.034in}{0pt}{}{}{Figure}{\special{language "Scientific Word";type
"GRAPHIC";maintain-aspect-ratio TRUE;display "USEDEF";valid_file "T";width
5.2731in;height 4.034in;depth 0pt;original-width 5.3458in;original-height
4.0819in;cropleft "0";croptop "1";cropright "1";cropbottom "0";tempfilename
'PQXXR20V.wmf';tempfile-properties "XPR";}}We know $Q=0$ and that $\Delta
E_{int}=W.$ Again we can use%
\[
\Delta E_{int}=nC_{V}\Delta T 
\]%
\begin{eqnarray*}
\Delta E_{int} &=&\left( 1\unit{mol}\right) \frac{3}{2}\left( 8.314\frac{%
\unit{J}}{\unit{mol}\unit{K}}\right) \left( 1023\unit{K}-273\unit{K}\right)
\\
&=&9353.\,\allowbreak 3\unit{J}
\end{eqnarray*}%
so then 
\[
\begin{tabular}{|c|c|c|c|c|c|c|}
\hline
\textbf{State} & $\mathbf{T}\left( \unit{K}\right) $ & $\mathbf{P}\left( 
\unit{Pa}\right) $ & $%
%TCIMACRO{\TeXButton{V}{\ooalign{\hfil$V$\hfil\cr\kern0.1em--\hfil\cr}}}%
%BeginExpansion
\ooalign{\hfil$V$\hfil\cr\kern0.1em--\hfil\cr}%
%EndExpansion
\left( \unit{m}^{3}\right) $ & $\mathbf{n}\left( \unit{mol}\right) $ & 
\textbf{--} & \textbf{--} \\ \hline
$\mathbf{D}$ & $273$ & $3.\,\allowbreak 129\,1\times 10^{5}$ & $%
7.\,\allowbreak 253\,7\times 10^{-3}$ & $1$ & \textbf{--} & \textbf{--} \\ 
\hline
$\mathbf{A}$ & $1023$ & $3.\,\allowbreak 1\times 10^{6}$ & $0.001$ & $1$ & 
\textbf{--} & \textbf{--} \\ \hline
\textbf{Process} & $\mathbf{Q}\left( \unit{J}\right) $ & $\mathbf{W}%
_{int}\left( \unit{J}\right) $ & $\mathbf{\Delta E}_{int}\left( \unit{J}%
\right) $ & $\mathbf{W}_{eng}\left( \unit{J}\right) $ & $\mathbf{Q}_{h}$ & $%
\mathbf{Q}_{c}$ \\ \hline
$\mathbf{DA}$ & $0$ & $9353.\,\allowbreak 3$ & $9353.\,\allowbreak 3$ & $%
-9353.\,\allowbreak 3$ & $0$ & $0$ \\ \hline
\end{tabular}%
\]%
All together we have%
\[
\begin{tabular}{|c|c|c|c|c|}
\hline
\textbf{State} & $\mathbf{T}\left( \unit{K}\right) $ & $\mathbf{P}\left( 
\unit{Pa}\right) $ & $%
%TCIMACRO{\TeXButton{V}{\ooalign{\hfil$V$\hfil\cr\kern0.1em--\hfil\cr}}}%
%BeginExpansion
\ooalign{\hfil$V$\hfil\cr\kern0.1em--\hfil\cr}%
%EndExpansion
\left( \unit{m}^{3}\right) $ & $\mathbf{n}\left( \unit{mol}\right) $ \\ 
\hline
$\mathbf{A}$ & $1023$ & $8.\,\allowbreak 505\,2\times 10^{6}$ & $0.001$ & $1$
\\ \hline
$\mathbf{B}$ & $1023$ & $\allowbreak 4.\,\allowbreak 113\,0\times 10^{6}$ & $%
2.\,\allowbreak 067\,9\times 10^{-3}$ & $1$ \\ \hline
$\mathbf{C}$ & $273$ & $1.\,\allowbreak 513\,1\times 10^{5}$ & $0.015$ & $1$
\\ \hline
$\mathbf{D}$ & $273$ & $3.\,\allowbreak 129\,1\times 10^{5}$ & $%
7.\,\allowbreak 253\,7\times 10^{-3}$ & $1$ \\ \hline
\end{tabular}%
\]

and our internal energy, heat, and work are%
\[
\begin{tabular}{|c|c|c|c|c|c|c|}
\hline
\textbf{Process} & $\mathbf{Q}\left( \unit{J}\right) $ & $\mathbf{W}%
_{int}\left( \unit{J}\right) $ & $\mathbf{\Delta E}_{int}\left( \unit{J}%
\right) $ & $\mathbf{W}_{eng}\left( \unit{J}\right) $ & $\mathbf{Q}_{h}$ & $%
\mathbf{Q}_{c}$ \\ \hline
$\mathbf{AB}$ & $6179.\,\allowbreak 3$ & $-6179.\,\allowbreak 3$ & $0$ & $%
6179.\,\allowbreak 3$ & $6179.\,\allowbreak 3$ & $0$ \\ \hline
$\mathbf{BC}$ & $0$ & $-9353.\,\allowbreak 3$ & $-9353.\,\allowbreak 3$ & $%
9353.\,\allowbreak 3$ & $0$ & $0$ \\ \hline
$\mathbf{CD}$ & $-1649.0$ & $1649.0$ & $0$ & $-1649.0$ & $0$ & $1649.0$ \\ 
\hline
$\mathbf{DA}$ & $0$ & $9353.\,\allowbreak 3$ & $9353.\,\allowbreak 3$ & $%
-9353.\,\allowbreak 3$ & $0$ & $0$ \\ \hline
Total & $4530.\,\allowbreak 3$ & $-4530.\,\allowbreak 3$ & $0$ & $%
4530.\,\allowbreak 3$ & $6179.\,\allowbreak 3$ & $1649.0$ \\ \hline
\end{tabular}%
\]

%TCIMACRO{%
%\TeXButton{Question 123.19.2}{\marginpar {
%\hspace{-0.5in}
%\begin{minipage}[t]{1in}
%\small{Question 123.19.2}
%\end{minipage}
%}} }%
%BeginExpansion
\marginpar {
\hspace{-0.5in}
\begin{minipage}[t]{1in}
\small{Question 123.19.2}
\end{minipage}
}
%EndExpansion
We can see that process $A\rightarrow B$ provides $Q_{h}$ and process $%
C\rightarrow D$ provides $Q_{C}.$Let's label this in our PV\ diagram. \FRAME{%
dhF}{3.2292in}{2.2468in}{0pt}{}{}{Figure}{\special{language "Scientific
Word";type "GRAPHIC";maintain-aspect-ratio TRUE;display "USEDEF";valid_file
"T";width 3.2292in;height 2.2468in;depth 0pt;original-width
3.1842in;original-height 2.207in;cropleft "0";croptop "1";cropright
"1";cropbottom "0";tempfilename 'PQXXR20W.wmf';tempfile-properties "XPR";}}

We can also find the efficiency of the Carnot cycle. The net work is $%
W_{ABCDA}$ and the efficiency is 
\[
\eta =\frac{W_{eng}}{\left\vert Q_{h}\right\vert }=1-\frac{\left\vert
Q_{c}\right\vert }{\left\vert Q_{h}\right\vert } 
\]

No heat is transferred during the adiabatic processes, so the only heat
transfer is only during the isothermal processes, we have%
\begin{eqnarray*}
\eta &=&1-\frac{\left\vert nRT_{c}\ln \left( \frac{%
%TCIMACRO{\TeXButton{V}{\ooalign{\hfil$V$\hfil\cr\kern0.1em--\hfil\cr}}}%
%BeginExpansion
\ooalign{\hfil$V$\hfil\cr\kern0.1em--\hfil\cr}%
%EndExpansion
_{C}}{%
%TCIMACRO{\TeXButton{V}{\ooalign{\hfil$V$\hfil\cr\kern0.1em--\hfil\cr}}}%
%BeginExpansion
\ooalign{\hfil$V$\hfil\cr\kern0.1em--\hfil\cr}%
%EndExpansion
_{D}}\right) \right\vert }{\left\vert nRT_{h}\ln \left( \frac{%
%TCIMACRO{\TeXButton{V}{\ooalign{\hfil$V$\hfil\cr\kern0.1em--\hfil\cr}}}%
%BeginExpansion
\ooalign{\hfil$V$\hfil\cr\kern0.1em--\hfil\cr}%
%EndExpansion
_{A}}{%
%TCIMACRO{\TeXButton{V}{\ooalign{\hfil$V$\hfil\cr\kern0.1em--\hfil\cr}}}%
%BeginExpansion
\ooalign{\hfil$V$\hfil\cr\kern0.1em--\hfil\cr}%
%EndExpansion
_{B}}\right) \right\vert } \\
&=&1-\frac{\left\vert T_{c}\ln \left( \frac{%
%TCIMACRO{\TeXButton{V}{\ooalign{\hfil$V$\hfil\cr\kern0.1em--\hfil\cr}}}%
%BeginExpansion
\ooalign{\hfil$V$\hfil\cr\kern0.1em--\hfil\cr}%
%EndExpansion
_{C}}{%
%TCIMACRO{\TeXButton{V}{\ooalign{\hfil$V$\hfil\cr\kern0.1em--\hfil\cr}}}%
%BeginExpansion
\ooalign{\hfil$V$\hfil\cr\kern0.1em--\hfil\cr}%
%EndExpansion
_{D}}\right) \right\vert }{\left\vert T_{h}\ln \left( \frac{%
%TCIMACRO{\TeXButton{V}{\ooalign{\hfil$V$\hfil\cr\kern0.1em--\hfil\cr}}}%
%BeginExpansion
\ooalign{\hfil$V$\hfil\cr\kern0.1em--\hfil\cr}%
%EndExpansion
_{A}}{%
%TCIMACRO{\TeXButton{V}{\ooalign{\hfil$V$\hfil\cr\kern0.1em--\hfil\cr}}}%
%BeginExpansion
\ooalign{\hfil$V$\hfil\cr\kern0.1em--\hfil\cr}%
%EndExpansion
_{B}}\right) \right\vert }
\end{eqnarray*}%
and using the adiabatic temperature-volume relationships%
\[
T_{h}%
%TCIMACRO{\TeXButton{V}{\ooalign{\hfil$V$\hfil\cr\kern0.1em--\hfil\cr}}}%
%BeginExpansion
\ooalign{\hfil$V$\hfil\cr\kern0.1em--\hfil\cr}%
%EndExpansion
_{A}^{\gamma -1}=T_{c}%
%TCIMACRO{\TeXButton{V}{\ooalign{\hfil$V$\hfil\cr\kern0.1em--\hfil\cr}}}%
%BeginExpansion
\ooalign{\hfil$V$\hfil\cr\kern0.1em--\hfil\cr}%
%EndExpansion
_{D}^{\gamma -1} 
\]%
and%
\[
T_{h}%
%TCIMACRO{\TeXButton{V}{\ooalign{\hfil$V$\hfil\cr\kern0.1em--\hfil\cr}}}%
%BeginExpansion
\ooalign{\hfil$V$\hfil\cr\kern0.1em--\hfil\cr}%
%EndExpansion
_{B}^{\gamma -1}=T_{c}%
%TCIMACRO{\TeXButton{V}{\ooalign{\hfil$V$\hfil\cr\kern0.1em--\hfil\cr}}}%
%BeginExpansion
\ooalign{\hfil$V$\hfil\cr\kern0.1em--\hfil\cr}%
%EndExpansion
_{C}^{\gamma -1} 
\]%
\[
\frac{T_{h}%
%TCIMACRO{\TeXButton{V}{\ooalign{\hfil$V$\hfil\cr\kern0.1em--\hfil\cr}}}%
%BeginExpansion
\ooalign{\hfil$V$\hfil\cr\kern0.1em--\hfil\cr}%
%EndExpansion
_{A}^{\gamma -1}}{T_{h}%
%TCIMACRO{\TeXButton{V}{\ooalign{\hfil$V$\hfil\cr\kern0.1em--\hfil\cr}}}%
%BeginExpansion
\ooalign{\hfil$V$\hfil\cr\kern0.1em--\hfil\cr}%
%EndExpansion
_{B}^{\gamma -1}}=\frac{T_{c}%
%TCIMACRO{\TeXButton{V}{\ooalign{\hfil$V$\hfil\cr\kern0.1em--\hfil\cr}}}%
%BeginExpansion
\ooalign{\hfil$V$\hfil\cr\kern0.1em--\hfil\cr}%
%EndExpansion
_{D}^{\gamma -1}}{T_{c}%
%TCIMACRO{\TeXButton{V}{\ooalign{\hfil$V$\hfil\cr\kern0.1em--\hfil\cr}}}%
%BeginExpansion
\ooalign{\hfil$V$\hfil\cr\kern0.1em--\hfil\cr}%
%EndExpansion
_{C}^{\gamma -1}} 
\]%
or%
\[
\frac{%
%TCIMACRO{\TeXButton{V}{\ooalign{\hfil$V$\hfil\cr\kern0.1em--\hfil\cr}}}%
%BeginExpansion
\ooalign{\hfil$V$\hfil\cr\kern0.1em--\hfil\cr}%
%EndExpansion
_{A}^{\gamma -1}}{%
%TCIMACRO{\TeXButton{V}{\ooalign{\hfil$V$\hfil\cr\kern0.1em--\hfil\cr}}}%
%BeginExpansion
\ooalign{\hfil$V$\hfil\cr\kern0.1em--\hfil\cr}%
%EndExpansion
_{B}^{\gamma -1}}=\frac{%
%TCIMACRO{\TeXButton{V}{\ooalign{\hfil$V$\hfil\cr\kern0.1em--\hfil\cr}}}%
%BeginExpansion
\ooalign{\hfil$V$\hfil\cr\kern0.1em--\hfil\cr}%
%EndExpansion
_{C}^{\gamma -1}}{%
%TCIMACRO{\TeXButton{V}{\ooalign{\hfil$V$\hfil\cr\kern0.1em--\hfil\cr}}}%
%BeginExpansion
\ooalign{\hfil$V$\hfil\cr\kern0.1em--\hfil\cr}%
%EndExpansion
_{D}^{\gamma -1}} 
\]%
which gives us%
\[
\sqrt[\gamma -1]{\frac{%
%TCIMACRO{\TeXButton{V}{\ooalign{\hfil$V$\hfil\cr\kern0.1em--\hfil\cr}}}%
%BeginExpansion
\ooalign{\hfil$V$\hfil\cr\kern0.1em--\hfil\cr}%
%EndExpansion
_{A}^{\gamma -1}}{%
%TCIMACRO{\TeXButton{V}{\ooalign{\hfil$V$\hfil\cr\kern0.1em--\hfil\cr}}}%
%BeginExpansion
\ooalign{\hfil$V$\hfil\cr\kern0.1em--\hfil\cr}%
%EndExpansion
_{B}^{\gamma -1}}}=\sqrt[\gamma -1]{\frac{%
%TCIMACRO{\TeXButton{V}{\ooalign{\hfil$V$\hfil\cr\kern0.1em--\hfil\cr}}}%
%BeginExpansion
\ooalign{\hfil$V$\hfil\cr\kern0.1em--\hfil\cr}%
%EndExpansion
_{C}^{\gamma -1}}{%
%TCIMACRO{\TeXButton{V}{\ooalign{\hfil$V$\hfil\cr\kern0.1em--\hfil\cr}}}%
%BeginExpansion
\ooalign{\hfil$V$\hfil\cr\kern0.1em--\hfil\cr}%
%EndExpansion
_{D}^{\gamma -1}}} 
\]%
then%
\[
\frac{%
%TCIMACRO{\TeXButton{V}{\ooalign{\hfil$V$\hfil\cr\kern0.1em--\hfil\cr}}}%
%BeginExpansion
\ooalign{\hfil$V$\hfil\cr\kern0.1em--\hfil\cr}%
%EndExpansion
_{A}}{%
%TCIMACRO{\TeXButton{V}{\ooalign{\hfil$V$\hfil\cr\kern0.1em--\hfil\cr}}}%
%BeginExpansion
\ooalign{\hfil$V$\hfil\cr\kern0.1em--\hfil\cr}%
%EndExpansion
_{B}}=\frac{%
%TCIMACRO{\TeXButton{V}{\ooalign{\hfil$V$\hfil\cr\kern0.1em--\hfil\cr}}}%
%BeginExpansion
\ooalign{\hfil$V$\hfil\cr\kern0.1em--\hfil\cr}%
%EndExpansion
_{D}}{%
%TCIMACRO{\TeXButton{V}{\ooalign{\hfil$V$\hfil\cr\kern0.1em--\hfil\cr}}}%
%BeginExpansion
\ooalign{\hfil$V$\hfil\cr\kern0.1em--\hfil\cr}%
%EndExpansion
_{C}} 
\]%
using this we find%
\begin{eqnarray*}
\eta &=&1-\frac{\left\vert T_{c}\ln \left( \frac{%
%TCIMACRO{\TeXButton{V}{\ooalign{\hfil$V$\hfil\cr\kern0.1em--\hfil\cr}}}%
%BeginExpansion
\ooalign{\hfil$V$\hfil\cr\kern0.1em--\hfil\cr}%
%EndExpansion
_{C}}{%
%TCIMACRO{\TeXButton{V}{\ooalign{\hfil$V$\hfil\cr\kern0.1em--\hfil\cr}}}%
%BeginExpansion
\ooalign{\hfil$V$\hfil\cr\kern0.1em--\hfil\cr}%
%EndExpansion
_{D}}\right) \right\vert }{\left\vert T_{h}\ln \left( \frac{%
%TCIMACRO{\TeXButton{V}{\ooalign{\hfil$V$\hfil\cr\kern0.1em--\hfil\cr}}}%
%BeginExpansion
\ooalign{\hfil$V$\hfil\cr\kern0.1em--\hfil\cr}%
%EndExpansion
_{D}}{%
%TCIMACRO{\TeXButton{V}{\ooalign{\hfil$V$\hfil\cr\kern0.1em--\hfil\cr}}}%
%BeginExpansion
\ooalign{\hfil$V$\hfil\cr\kern0.1em--\hfil\cr}%
%EndExpansion
_{C}}\right) \right\vert } \\
&=&1-\frac{\left\vert T_{c}\ln \left( \frac{%
%TCIMACRO{\TeXButton{V}{\ooalign{\hfil$V$\hfil\cr\kern0.1em--\hfil\cr}}}%
%BeginExpansion
\ooalign{\hfil$V$\hfil\cr\kern0.1em--\hfil\cr}%
%EndExpansion
_{C}}{%
%TCIMACRO{\TeXButton{V}{\ooalign{\hfil$V$\hfil\cr\kern0.1em--\hfil\cr}}}%
%BeginExpansion
\ooalign{\hfil$V$\hfil\cr\kern0.1em--\hfil\cr}%
%EndExpansion
_{D}}\right) \right\vert }{\left\vert -T_{h}\ln \left( \frac{%
%TCIMACRO{\TeXButton{V}{\ooalign{\hfil$V$\hfil\cr\kern0.1em--\hfil\cr}}}%
%BeginExpansion
\ooalign{\hfil$V$\hfil\cr\kern0.1em--\hfil\cr}%
%EndExpansion
_{C}}{%
%TCIMACRO{\TeXButton{V}{\ooalign{\hfil$V$\hfil\cr\kern0.1em--\hfil\cr}}}%
%BeginExpansion
\ooalign{\hfil$V$\hfil\cr\kern0.1em--\hfil\cr}%
%EndExpansion
_{D}}\right) \right\vert }
\end{eqnarray*}%
or, removing the minus sign because of the absolute values%
\[
\eta =1-\frac{T_{c}}{T_{h}} 
\]%
where the temperatures, of course, is in Kelvin. This means that for this
process%
\[
\frac{\left\vert Q_{c}\right\vert }{\left\vert Q_{h}\right\vert }=\frac{T_{c}%
}{T_{h}} 
\]%
and 
\begin{equation}
\eta _{C}=1-\frac{T_{c}}{T_{h}}
\end{equation}%
This is the general form for a Carnot cycle. Then for our example%
\begin{eqnarray*}
\eta _{C} &=&1-\frac{273\unit{K}}{1023\unit{K}} \\
&=&0.733\,14
\end{eqnarray*}

This is a great review of how to use our thermodynamic equations for special
process, even though we can't build a Carnot engine. But more importantly,
we can judge the efficacy of engine designs. For example, let's take our
Otto cycle from last time. Our design gave 
\begin{eqnarray*}
\eta _{otto} &=&1-\frac{1}{\left( \frac{0.000\,5}{6.\,25\times 10^{-5}}%
\right) ^{1.4-1}} \\
&=&0.564\,72\allowbreak 
\end{eqnarray*}%
So the Otto cycle example might be achievable. Of course, we can tell that
our Otto cycle is still an idealization. We recognize that we neglected all
friction, and assumed perfect adiabatic processes. But we can see that there
is some hope of making our Otto cycle-based engines more efficient. But no
Otto cycle that operates between the same $T_{c}=273\unit{K}$ and $T_{h}=1023%
\unit{K}$ can never be better than $73\%$ effective because that is the
maximum efficiency for a heat engine opperating between these two
temperatures.

We are done with our introductory study of thermodynamics. Many of you will
have junior level thermodynamics classes, so you are likely to see more of
this. But we can stop for now. For physics majors the next physics class is
likely to be PH220, electricity and magnetism. I would prefer to call this
class \textquotedblleft introduction to field theory\textquotedblright\
because we really study electric and magnetic fields for the first time. We
will take up the topic of wave motion at the end of PH220 to mathematically
describe light as electromagnetic waves. And then we will describe
everything using waves in PH279, \textquotedblleft modern
physics,\textquotedblright\ as our first introduction to quantum mechanics. 

I hope you have enjoyed learning about waves, optics, and thermodynamics. I
hope you will see the world a little differently (and think about the
processing of seeing the world!). For physics majors, there is still so many
cool things to learn. If you are leaving physics at this point, you can
continue to learn about the world and the universe by reading and watching
physics related material. If you are innovating new techniques or products,
you might even want to hire the services of a physicist to help with the
underlying principles for your new venture!

\label{AppendixMarker}

\QTP{Appendix}
Experiment 1

\section{Introduction}

We have learned that 
\[
\omega ^{2}=\frac{k}{m} 
\]%
and we know that 
\[
\omega =2\pi f 
\]%
and 
\[
T=\frac{1}{f} 
\]%
so we would predict that 
\begin{eqnarray}
T &=&\frac{1}{\frac{\omega }{2\pi }}=\frac{2\pi }{\omega }  \label{period} \\
&=&2\pi \sqrt{\frac{m}{k}}  \nonumber
\end{eqnarray}%
We would like to verify that this is true experimentally.

Unfortunately, springs don't come marked with their spring constants. You
will need to measure your spring constant. A way to do this is to hang your
spring motionless. When the mass is not moving Newton's second law tells us 
\[
\sum_{y}F=ma_{y}=F_{S}-F_{g} 
\]%
With no change in velocity, the acceleration must be zero 
\[
0=F_{S}-F_{g} 
\]%
so%
\begin{eqnarray*}
F_{S} &=&F_{g} \\
k\Delta y &=&mg
\end{eqnarray*}%
or%
\begin{equation}
k=\frac{mg}{\Delta y}  \label{konstant}
\end{equation}

We believe that the equation of motion for our mass-spring system would be 
\begin{eqnarray}
y\left( t\right) &=&y_{\max }\cos \left( \omega t+\phi _{o}\right)
\label{motion} \\
&=&y_{\max }\cos \left( \frac{2\pi }{T}t+\phi _{o}\right)  \nonumber
\end{eqnarray}

\section{Equipment}

We have springs, masses, and motion detectors. Your motion detector should
be placed below the mass as was demonstrated in class. Be careful to not
drop the mass on the motion detector.

The motion detectors connect to our laptop computers. Use the LoggerPro
program to collect data. Your lab instructor will help you set up the
program and show you how to collect data.

\subsection{Assignment}

\begin{enumerate}
\item Find your spring constant

\item Using equation (\ref{period}) predict the period of oscillation of
your mass and spring.

\item Suspend your mass from the spring and capture several cycles of the
motion using the motion detector.

\item Verify that equation (\ref{motion}) is a good model for our physical
situation by fitting a sinusoidal curve to the data (ask instructor for
help).

\item Using your curve fit equation, compare your predicted period to your
measured period. Do this by matching the fit equation to our theoretical
equation of motion 
\begin{eqnarray*}
Y &=&A\ast \cos (Bx+C)+D \\
y\left( t\right) &=&y_{\max }\cos \left( \frac{2\pi }{T}t+\phi _{o}\right)
\end{eqnarray*}%
we can see that 
\[
B=\frac{2\pi }{T} 
\]%
and solve for $T.$

\item Do your predicted and measured periods match?

\item If they don't match, consider your inputs in equation (\ref{period})
and try again.
\end{enumerate}

\QTP{contentssection}
Experiment 2

\section{Introduction}

\FRAME{dhF}{4.7668in}{2.0513in}{0pt}{}{}{Figure}{\special{language
"Scientific Word";type "GRAPHIC";maintain-aspect-ratio TRUE;display
"USEDEF";valid_file "T";width 4.7668in;height 2.0513in;depth
0pt;original-width 4.7124in;original-height 2.0124in;cropleft "0";croptop
"1";cropright "1";cropbottom "0";tempfilename
'PQXXR20X.wmf';tempfile-properties "XPR";}}We studied calorimetry earlier in
the semester. The idea behind calorimetry is just conservation of energy. A
calorimeter is supposed to be an insulated system, so energy can't leave by
heat.

Your job today is to measure the latent heat of fusion of ice/water.

Some helpful reminders are the specific heat equation%
\[
Q=m_{i}c_{i}\Delta T 
\]%
and the latent heat equation.%
\begin{eqnarray*}
Q &=&\pm L_{f}M\qquad \text{melt/freeze} \\
Q &=&\pm L_{v}M\qquad \text{boil/condense}
\end{eqnarray*}%
From our table we used in class, we know what the value should be.

\[
\begin{tabular}{|l|l|l|l|l|}
\hline
\textbf{Substance} & $T_{m}\left( \unit{%
%TCIMACRO{\U{2103}}%
%BeginExpansion
{}^{\circ}{\rm C}%
%EndExpansion
}\right) $ & $L_{f}\left( \frac{\unit{J}}{\unit{kg}}\right) $ & $T_{m}\left( 
\unit{%
%TCIMACRO{\U{2103}}%
%BeginExpansion
{}^{\circ}{\rm C}%
%EndExpansion
}\right) $ & $L_{v}\left( \frac{\unit{J}}{\unit{kg}}\right) $ \\ \hline
Nitrogen $\left( N_{2}\right) $ & $-210$ & $0.26\times 10^{5}$ & $-196$ & $%
1.99\times 10^{5}$ \\ \hline
Ethyl alcohol & $-114$ & $1.09\times 10^{5}$ & $78$ & $8.79\times 10^{5}$ \\ 
\hline
Mercury & $-39$ & $0.11\times 10^{5}$ & $357$ & $2.96\times 10^{5}$ \\ \hline
Water & $0$ & $3.33\times 10^{5}$ & $100$ & $22.6\times 10^{5}$ \\ \hline
Lead & $328$ & $0.25\times 10^{5}$ & $1750$ & $8.58\times 10^{5}$ \\ \hline
\end{tabular}%
\]%
Find the latent heat of fusion for water/ice and calculate a percent error
to see how well you did%
\[
\%Error=\frac{measured-acutal}{actual} 
\]

\section{Equipment}

I will have ice water available in a tub. I will also have calorimeter cups,
thermometers, and triple beam balances for finding mass. Water will be
available through our store room. Let me know if you need something else.

\section{Assignment}

Measure the latent heat of fusion of water.

\begin{enumerate}
\item Place some ice in water to form a slush (this will give you ice at $0%
\unit{%
%TCIMACRO{\U{2103}}%
%BeginExpansion
{}^{\circ}{\rm C}%
%EndExpansion
}$).

\item Place some water into the calorimeter cup, and measure its temperature
and mass.

\item Place a piece of ice into the water in the calorimeter. You will need
to find a way to measure its mass.

\item Measure the temperature change of the water after the ice has melted.

\item Calculate the latent heat, and compare to the accepted value.

\item Turn in your answer, $\%Error,$ and all your group names on a single
sheet of paper (the lab instructions sheet is fine)
\end{enumerate}

\QTP{Appendix}
Experiment 3

\section{Introduction}

A converging lens is one which is thicker at the center than at the edge,
and which converges incident parallel rays to a real focus on the opposite
side of the lens from the object. A diverging lens is thinner at the center
than at the edge and diverges the light from a virtual focus on the same
side of the lens as the object. The optic axis of a lens is a line drawn
through the center of the lens perpendicular to the face of the lens. The
principal focal point is a point on the principal axis through which
incident rays parallel to the principal axis pass, or appear to pass, after
refraction by the lens. The focal length, $f$, of the lens is the distance
from the optical center of the lens to the principal focal point. The
reciprocal of the focal length in meters is called the power of the lens and
is expressed in diopters (d).

\[
P=\frac{1}{f\left( \unit{m}\right) } 
\]

If $s$ is the distance of the object from the optical center of the lens, $%
s^{\prime }$ the distance of the image from the lens, and $f$ the focal
length, the thin lens equation is%
\begin{equation}
\frac{1}{s}+\frac{1}{s^{\prime }}=\frac{1}{f}  \label{lens}
\end{equation}

The magnification is given by

\begin{equation}
m=-\frac{s^{\prime }}{s}  \label{magnification}
\end{equation}

For the case where more than one lens is present, Equation (\ref%
{magnification}) is applied successively to each lens, treating the image
formed by one lens as the object for the next lens, keeping in mind that the
distances in Equation (\ref{magnification}) are measured with respect to
each individual lens, and not a single fixed point.

\section{Equipment}

We have $2\unit{m}$ sticks, lens and holders, paper screens, and light
sources.

\section{Assignment}

Verify the lens equation (\ref{lens}), and the image magnification equation (%
\ref{magnification}), for the lens with the longer focal length. I suggest
you use at least three object distances greater than two focal lengths. I
suggest you measure the image distance and the image height. I suggest you
also measure at least three object distances between one and two focal
lengths. Measure the image distance and the image height.

Plot your magnification results. Interpret the results, does the lens
equation work?

Compute the magnification in each case and verify that equation (\ref%
{magnification}) holds.

The lens equation is supposed to be good for virtual images as well as real
images. We need to verify the equation for this case as well. I suggest you
do this by measuring at least one object distance closer to the lens than
one focal length, locate and measure the image distance. Measure the image
height. Verify that equations (\ref{lens}) and (\ref{magnification}) are
valid for this case. Include this case on your graph.

\QTP{Appendix}
Problem Types

Simple Harmonic Motion (SHM)

\begin{eqnarray*}
x\left( t\right) &=&x_{\max }\cos \left( \omega t+\phi _{o}\right) \\
v\left( t\right) &=&-\omega x_{\max }\sin \left( \omega t+\phi _{o}\right) \\
a\left( t\right) &=&-\omega ^{2}x_{\max }\cos \left( \omega t+\phi
_{o}\right)
\end{eqnarray*}%
\[
\omega =2\pi f 
\]%
\[
f=\frac{1}{T} 
\]%
\[
v_{\max }=\omega x_{\max } 
\]%
\[
a_{\max }=\omega ^{2}x_{\max } 
\]

Energy in SHM%
\begin{eqnarray*}
K &=&\frac{1}{2}mv^{2} \\
&=&\frac{1}{2}kx_{\max }^{2}\sin ^{2}\left( \omega t+\phi \right)
\end{eqnarray*}

\begin{eqnarray*}
U_{s} &=&\frac{1}{2}kx^{2} \\
&=&\frac{1}{2}kx_{\max }^{2}\cos ^{2}\left( \omega t+\phi \right)
\end{eqnarray*}

Angular Frequency and oscillation

For a mass sprig system%
\[
\omega =\sqrt{\frac{k}{m}} 
\]

for a pendulum%
\[
\omega =\sqrt{\frac{g}{L}} 
\]%
for a physical pendulum 
\[
\omega =\sqrt{\frac{mgd}{\mathbb{I}}} 
\]

Damped oscillation

\[
x\left( t\right) =x_{\max }e^{-\frac{b}{2m}t}\cos \left( \omega t+\phi
_{o}\right) 
\]%
\[
\omega =\sqrt{\frac{k}{m}-\left( \frac{b}{2m}\right) ^{2}} 
\]%
\[
x\left( t\right) =x_{\max }e^{-\frac{b}{2m}t}\cos \left( \left( \sqrt{\frac{k%
}{m}-\left( \frac{b}{2m}\right) ^{2}}\right) t+\phi _{o}\right) 
\]

Forced oscillation

\[
\mathbf{F}_{d}=-b\mathbf{v} 
\]%
\[
F\left( t\right) =F_{o}\sin \left( \omega _{f}t\right) 
\]%
\[
\Sigma F=F_{o}\sin \left( \omega _{f}t\right) -kx-bv_{x}=ma 
\]

\[
x\left( t\right) =A\cos \left( \omega _{f}t+\phi \right) 
\]%
\[
A=\frac{\frac{F_{o}}{m}}{\sqrt{\left( \omega _{f}^{2}-\omega _{o}^{2}\right)
^{2}+\left( \frac{b\omega _{f}}{m}\right) ^{2}}} 
\]%
\[
\omega _{o}=\sqrt{\frac{k}{m}} 
\]

Waves

\[
y\left( t\right) =A\cos \left( ax-\omega t+\phi \right) 
\]%
\[
k=\frac{2\pi }{\lambda } 
\]%
\[
v=\frac{\lambda }{T} 
\]%
\[
v=\lambda f 
\]

Waves on strings

\[
v=\sqrt{\frac{T_{s}}{\mu }} 
\]%
\[
\mu =\frac{m}{L} 
\]

Sound waves

\begin{eqnarray*}
v_{sound} &=&v_{o}\sqrt{1+\frac{T_{c}}{T_{o}}} \\
v_{o} &=&331\frac{\unit{m}}{\unit{s}} \\
T_{o} &=&273
\end{eqnarray*}

\[
P\equiv \frac{F}{A} 
\]%
\[
s\left( x,t\right) =s_{\max }\cos \left( kx-\omega t+\phi _{o}\right) 
\]

Spherical waves

\[
y=A\left( r\right) \sin \left( kr-\omega t+\phi _{o}\right) 
\]

Light waves

\[
n=\frac{c}{v} 
\]%
\[
c=299792458\frac{\unit{m}}{\unit{s}} 
\]%
Power and Intensity%
\[
\mathcal{P}=\frac{\Delta E}{\Delta t} 
\]%
\[
\mathcal{I}\equiv \frac{\mathcal{P}}{A} 
\]

\[
A=4\pi r^{2} 
\]

Sound Intensity Level

\[
\beta =10\log _{10}\left( \frac{I}{I_{o}}\right) 
\]%
\[
I_{o}=1\times 10^{-12}\frac{\unit{W}}{\unit{m}^{2}} 
\]

Doppler Effect

\[
f^{\prime }=\frac{v+v_{d}}{v}f\text{ \qquad observer moving toward the source%
} 
\]%
\[
f^{\prime }=\frac{v-v_{d}}{v}f\text{ \qquad observer moving away from the
source} 
\]%
\[
f^{\prime }=\frac{v}{v-v_{s}}f\qquad \text{Source moving toward observer} 
\]%
\[
f^{\prime }=\frac{v}{v+v_{s}}f\qquad \text{Source moving away from observer} 
\]%
\[
f^{\prime }=\frac{v\pm v_{d}}{v\mp v_{s}}f 
\]

Superposition

\[
\sin a+\sin b=2\cos \left( \frac{a-b}{2}\right) \sin \left( \frac{a+b}{2}%
\right) 
\]%
\begin{eqnarray*}
y_{r} &=&y_{\max }\sin \left( kx-\omega t\right) +A\sin \left( kx-\omega
t+\phi _{o}\right) \\
&=&2y_{\max }\cos \left( \frac{\phi _{o}}{2}\right) \sin \left( kx-\omega t+%
\frac{\phi _{o}}{2}\right)
\end{eqnarray*}%
Standing waves%
\[
\sin \left( a\pm b\right) =\sin \left( a\right) \cos \left( b\right) \pm
\cos \left( a\right) \sin \left( b\right) 
\]%
\begin{eqnarray*}
y_{r} &=&y_{\max }\sin \left( kx-\omega t\right) +y_{\max }\sin \left(
kx+\omega t\right) \\
&=&\left( 2y_{\max }\sin \left( kx\right) \right) \cos \left( \omega t\right)
\end{eqnarray*}%
\[
x=n\frac{\lambda }{4} 
\]%
\begin{eqnarray*}
f_{n} &=&\frac{v}{\lambda _{n}}= \\
&=&\frac{n}{2L}v \\
&=&\frac{n}{2L}\sqrt{\frac{T}{\mu }}
\end{eqnarray*}

Standing waves in pipes

Open on both ends

\begin{eqnarray*}
\lambda _{n} &=&\frac{2}{n}L \\
f_{n} &=&n\frac{v}{2L}\qquad n=1,2,3,4\ldots
\end{eqnarray*}%
Closed on one end%
\begin{eqnarray*}
\lambda _{n} &=&\frac{4}{n}L \\
f_{n} &=&n\frac{v}{4L}\qquad n=1,3,5\ldots
\end{eqnarray*}

\QTP{Appendix}
References

This appendix contains the references.

\QTP{Appendix}
Index

This appendix contains the index.

\end{document}
